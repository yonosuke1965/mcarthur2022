\RequirePackage{plautopatch}
\documentclass[paper=a4,book,openany]{jlreq} 
\jlreqsetup{quote_beforeafter_space=.5\baselineskip}
\title{セックスの倫理学入門 \\ The Ethics of Sex: An Introduction (2022)}
\author{ニール・マッカーサー (Neil McArthur) \\ ChatGPT(アルト君) \& 江口某試訳}
\usepackage{makeidx}
\makeindex
 \usepackage{datetime2}
\usepackage{natbib}
\usepackage{pxrubrica}\rubysetup{J}
\newcommand{\ig}[1]{}           %無視させる。索引に載せるため。
 \usepackage[
 unicode,
 pdfusetitle,
 hidelinks,
 bookmarks=true,
   bookmarksopen=true,
   bookmarksdepth=3,
   bookmarksnumbered=true,
   bookmarksopen=true,
   ]{hyperref}
% \usepackage[hypens]{url}

\setcounter{tocdepth}{1}
\newenvironment{mylist}{\small \list{}{}{}   \itemindent=-1\zw}{\endlist}
\def\DDASH{―\kern-.5\zw―\kern-.5\zw―} % 線引き
\renewcommand{\bibname}{参考文献}
\renewcommand{\refname}{参考文献}

%%%%%%%%%%%%%%%%%%%%%%%%%%%%%%%%%%%%%%%%%%%%%%%%%%%%%%%%%%%%%%%%
\begin{document}
\frontmatter
\phantomsection\addcontentsline{toc}{chapter}{表紙}
\begin{center}
\vspace{3\zw}

{\Huge ニール・マッカーサー}

\vspace{1\zw}

{\Huge セックスの倫理学}

% \vspace{1\zw}
% {\huge ChatGPT&江口某試訳}
% \vspace{3\zw}
% \par{\Large
%       性的同意、ポルノ、セックスワーク、デートアプリ、\\ セックスロボット、非実在児童ポルノ、セックスドラッグ\\SM・ボンデージ、同性婚、多重婚\\
%       \vspace{1\zw}
%       フェミニズム、ジェンダー論、クィア論から強硬保守派まで\\十分目配りした豊かな洞察!
% }
   \end{center}

  \begin{center}
  \today\hspace{1\zw} \DTMcurrenttime 版
  \end{center}

%  \maketitle{}

\tableofcontents
\if0
\chapter*{謝辞}
\phantomsection\addcontentsline{toc}{chapter}{謝辞}

(短い謝辞だし、どうせなら出版から3年経過してトランプ返り咲いたのを受けて、見解を修正すべきところとか今注目すべきネタとかそういうのについての短い日本語版序文を書いてもらって、最後に謝辞つけるか。
)

\vspace{2\zw}
セックスの倫理学に取り組む大きな利点の一つは、人々はいつもこの話題について話したがっているということだ。
私の友人たちは長年にわたって意見や個人的な経験を共有してくれた。
それは私のこの領域の理解を豊かにしてくれた。
私は何年にもわたってセックス倫理の授業を担当してきたが、学生との無数の有益な議論からも多くを学んできた。

マーキー・ツイストは、セックステクノロジーに関する共同研究の素晴らしい協力者であり、常に私を支えてくれる友人でもある。
マリナ・アドシェイドとの共同研究は、同意に関する理解をはじめ、私たちがともに議論し執筆してきたさまざまなテーマへの理解を深めてくれた。
ジョン・ダナハーは本原稿の一部を快く読んでコメントを寄せてくれた。

私の子どもたちは、セックス・ロボットについてメディア上でおおっぴらに語る父親とともに暮らしてこなければならず、辛抱強くつきあってくれた。
そして、デネル・ジョンソンは、この本をようやく書き上げるまでの間、私の精神の安定を保ってくれた。
私はこの本を、彼女に愛と感謝を込めて捧げる。

\chapter*{日本語版への序文}
\phantomsection\addcontentsline{toc}{chapter}{日本語版への序文}

新しいテックネタが多いから書いてもらおう。
\fi
\mainmatter

\chapter*{序章}
\phantomsection
\addcontentsline{toc}{chapter}{序章}

セックス倫理学は刺激的な分野だ。
それは単に、その対象がセックスという、そもそも非常に刺激的なものだからではない{\DDASH}もちろん、それも一因ではある。
しかしむしろ、この分野の議論が緊急性を帯び、急速に展開しているからだ。
かつての私たちは性道徳の問題には簡単な答えがあると考えていたかもしれないが、今ではもはや誰もそんな考えを持っていない。
女性の権利や同性愛者の権利を求める運動、インターネットやスマートフォンといった新しい技術の発展、宗教観の変化{\DDASH}これらの要因をはじめとするさまざまな理由により、私たちは親や祖父母が生きた世界とは異なる世界に生きている。

これらの変化がいったい何を意味するのかを理解するのは大変だとあなたは思っているかもしれない。
だがあなたは一人ではない。
哲学の助けを借りよう。
人々は時としてこんな考えを持つことがある。
哲学者、特に倫理学者という連中は、権威ある偉そうな立場から、世間の人々にあれこれ考えたりおこなったりするべきだとかするべきでないと宣告することが自分の仕事だと思っているのだ、と。
しかし、それは哲学の最も悪しき形態であり、ここで私が目指すものではない。
哲学の最善の姿は、あらかじめ定められた答えを提供するのではなく、むしろ困難な問題について考える方法を示し、それによって私たちが自ら解決策を見出せるようにするものだ。
本書ではさまざまな意見を紹介し、そのそれぞれを支持する論証もおこなう。
だが本書は、セックスに関して、何が正しく何が間違っているのかという託宣を下すものではない。
私が何よりも関心を持っているのは、\ruby{開かれた}{オープンな}議論だ。
私は、セックス倫理学において最も重要で興味深い問題を特定し、それらを多角的に探究することを試みた。
私の目的は、議論と自己省察を促すことだ。

本書で扱う問題は、大きく二つのカテゴリーに分けられる。
一つは道徳的な問題であり、もう一つは政治的・法的な問題だ。
この二つのカテゴリーはけっして明確に分かれるものではない。
むしろ多くの点で重なり合い、本書で議論しているトピックスにはそれが切り離せない問題もある。
しかし、最初に両者の違いを概観しておくことは有益だ。

\ruby{道徳哲学}{モラルフィロソフィー}は、\ruby{正}{ライト}と\ruby{不正}{ロング}に関する自分の信念を、批判的かつ誠実に検討する意欲を持つ人々、そして証拠や論証に照らして自分の信念を修正することを厭わない人々のためにある。
何があっても自分の考えが変わることはないとあらかじめ凝り固まってしまっている人々や、どの疑問に対しても自分はすでに答えを知っていると思っている人々、あるいは、まったく利己的であったり気まぐれなやり方で、どんな行為をするのかを決める人々がいる。
哲学者はこうした人々に向けて語る言葉をあまり持っていない。
しかし、私たちのほとんどはそんな人間ではない。
私たちのほとんどは、善良な人間でありたいと願っている。
だが、それがどういうことなのかよくわからないと本気で感じることもしばしばだ。

私たち自身が道徳的問題に直面することもあれば、他の人々が直面することもある。
読者の中には、「セックスのような私的な事柄について、どうやって他人を\ruby{判断}{ジャッジ}できるというのか?」と疑問を抱く人もいるかもしれない。
しかし、道徳判断を下すということは、自分は誰かさんより立派だと上から目線で言い放つことを意味するわけではない。
むしろ、自分が同じ状況に置かれたらその人とは別のやり方で行動するかどうか、あるいは、程度の差こそあれ、ある人がある意思決定をしたということがその人を称賛する根拠になるかどうか、と考えることを意味している。
実際のところ、私たちは日常的に他者を判断している。
友人がたとえば、「上司と寝ている」「お金を払ってセックスワーカーとセックスをした」「他人の配偶者とインターネット上で性的なやりとりをした」と打ち明けた場合、私たちはどうしてもその人のふるまいについて何らかの判断を下さざるをえない。
そして、たとえ友人を\ruby{判断}{ジャッジ}したり責めたりしないという決断を下したとしても、それ自体が一つの道徳判断だ。
本書の目的は、人々がこうした難しい状況を十分慎重に熟考した上で判断を下せるよう手助けすることだ。
そしてそうした判断は、自分自身で納得し、他者と共有できるもののはずだ。

また、本書は道徳哲学の第二の問い、「どう生きることが最善か?」という問いにも取り組む。
ここでは、必ずしも「道徳的」問題とは見なされないテーマ、つまり、他人をどう扱うかというよりは、\ruby{私的}{プライベート}な選択に関わっている問題についても論じる。
哲学の目的の一つは、こうした私的な選択が真に自律的なものであることを保証する。
\ruby{個人的}{プライベート}な事柄、つまり当人の生活だけにかかわる事柄に関しては、かなりの自由が認められるべきだ{\DDASH}つまり、何者であれ、当人が意志決定しようとすることを邪魔すべきではない{\DDASH}ということには、ほとんどの人が同意している。
しかし、自由であることと自律的であることは同じではない。
両者は関連しているが、自律には自由以上のものが求められる。
\ruby{自由}{リバティ}とは、他人による制約なしに自らの望む行動をとる能力だ。
他方、\ruby{自律}{オートノミー}とは、選択可能な選択肢を合理的かつ批判的に吟味した上で、自らが望む人生を実現するための選択をおこなう能力だ。
そこで選ばれる生活は、長期的に見て最も\ruby{幸福}{ハピネス}をもたらす生き方だ。
自由は、他者が私たちの行動を妨げないことによって与えられるものだが、自律は主に自らの努力によって獲得されるものだ。

私たちの自由を制約せずとも、私たちの自律を制限するものは多くある。
家族からの期待、社会に根付いた偏見、自分自身の不安、そしておそらく最も強力な要因である習慣や怠惰{\DDASH}これらはすべて、私たちが本当に望む人生を生きることを妨げる要因となりえる。
本書では、読者が抱いているセックスに関する前提を批判的に検討し、それらが本当に妥当なものかどうかを問い直すことで、最終的に最も幸福をもたらす選択をおこなえるよう手助けしたいと考えている。

私はまた、政治的な問題や法的な問題についても論じる。
セクシュアリティをめぐる争いは、多くの場合、道徳的議論によってではなく、立法府や裁判所によって決着がつけられる。
私が関心を持っているのは、現在の法律や政策がどのようなものであるかということよりも、それらがどのようであるべきかという点だ。
私は、「私たちはどのような社会に生きたいのか、そしてそれをどのように実現することができるのか?」という問いを投げかけたい。
そのために、読者は自らを国の立法府を完全に掌握する大統領や首相、あるいは最高裁のキャスティングボート(最終決裁権)を握っている裁判官であると想像してみるとよい。

道徳的問題であれ、政治的問題であれ、私は「\ruby{未解決の問い}{オープン・クエスチョン}」{\DDASH}つまり、真に二つの陣営に分かれており、\ruby{分別のある}{リーズナブル}人々の間で意見が分かれているような問題{\DDASH}を取り上げようと試みた。
すべての問題がこの条件を満たすわけではない。
むかしむかし、人々は「特定の曜日〔安息日〕に配偶者とセックスをすることに問題はないのか」とか「オーラルセックスはそもそも許されるのか」といった議論をしていた。
しかし、これらの議論を今繰り返すのは馬鹿馬鹿しいだろう{\DDASH}もう誰もそんなことを気にしていないからだ。
今から 100 年後の人々には、本書で取り上げる議論の中にも同様に時代遅れに見えるものもあるだろう。
しかし本書では現時点において学者や学生、メディア関係者が最も関心を寄せており、それゆえ興味深く理性的な議論の基盤を形成しうる議論を扱うことにしている。

アカデミックな哲学になじみのある読者は、本書が通常の学術書や専門誌に掲載される論文とはやや異なる点を持つことに気づくだろう。
本書における引用や参考文献には、学術的な著作に加えて、メディアの記事やブログも含まれている。
これはいくつかの理由による。

第一に、すでに述べたように、セックス倫理学は急速に発展している分野だ。
\ruby{性科学}{セクソロジー}全般は確立された学問であり、多様な分野にわたる査読付き研究が豊富に存在する。
しかし、セックス倫理に特化して研究する哲学者の数は依然として限られている(以前よりは増えているが)。
本書で扱う問題の中には、学術誌でほとんど議論されていないもの、あるいはまったく議論されていないものもある。
哲学者による著作のみを参照していたのでは、取り上げるテーマや論点が限定されてしまう。
より幅広い情報源を活用することで、読者が関心を持ちそうな問題に焦点を当てることができる。
たとえ学術研究がまだ追いついていないとしても、こうしたアプローチは重要だ。

第二に、多様な情報源を取り入れることで、本書で扱う問題に直接関わり、影響を受けている人々の言葉や見解を反映することができる。
社会問題について哲学者がお互いとのみ対話し、お互いの著作のみを引用する時代は終わった{\DDASH}あるいは、終わるべきだ。
たとえば、ポルノグラフィやセックスワークについて論じる際には、これらの業界にいる人々が書き発言している言葉を無視すべきではない。
これは、引用した人々の意見に必ずしも同意しなければならないということではなく、またそうした人々の意見は一致していると言いたいのでもない。
だがとにかく、彼ら彼女らの声には少なくとも耳を傾けねばならない。

第三に、本書では可能な限り、仮想的な事例を構築するのではなく、実際の事例を用いて議論を展開している。
現代の倫理学になじんでいる読者は、倫理学者が思考実験{\DDASH}抽象的なシナリオを詳細に構築し、しばしば議論の対象となっている事柄とはかけ離れたもの{\DDASH}を好んで用いることを知っているだろう。
私自身も思考実験を完全に避けることはできないが、本書では可能な限り、実際に起こった出来事を用いて問題を明らかにするよう努めた。

いくつか注意点を述べておく。
第一に、本書は実践的倫理学の書だ。
私は本書が読者の生活に関連することを願っている。
だが、これはアドバイス本ではない。
私はカウンセラーの訓練を受けたわけではないし、またたとえば「恋人が大麻をやめないから別れるべきか」や「ポルノ映画の照明技師の仕事を引き受けるべきか」、あるいは「新しい恋人がすでに自分の両親と3Pセックスをしたことがあると知った場合にどうすればよいか」といった相談に答える立場にはない(そう、これらは実際にあった話だ\citep{savage21:_mum_dat})。

また、私は法律家でもない。
本書で扱う法的議論は、前述の通り規範的だ。
現実の法律や判例について論じることはあるが、私の関心は主に「法律は理想的にはどのようにあるべきか」という点にある。
現実の法制度が理想とはかけ離れていることは言うまでもない。

第二に、本書で考察する議論のなかに宗教に訴えるものは一切含まれていない。
もちろん、性道徳は世界の主要宗教にとって長らく重要な関心事であり、また多くの読者は、自分自身の宗教的信念に基づいた見解を持っているだろう。
私はそれを尊重する。
しかし、本書では、原則としてどの宗教の信者であっても受け入れ可能であるような議論に焦点を当てるように努めた。
多文化社会においては、他者が自分と同じ信念を共有しているとは限らないため、共通の基盤に基づいて道徳的議論をおこなう必要がある。

第三に、セックス倫理の領域を明確に区切ることは困難だ。
本書では、どのトピックを含め、どのトピックを除外するかについて難しい決断をしなければならなかった。
他の哲学者ならば異なる線引きをすることもあるだろう。
本書では、性的行動や親密な関係としての「セックス」に焦点を当てる。
結婚の定義を扱うのは、社会としてどのような性的関係を正当とみなすのかを決定する必要があるからだ。
ただし、ジェンダーアイデンティティに関する問題は扱っていない。
また、妊娠中絶、遺伝子工学、子供を持つことの道徳性など、生殖に関する問題も扱わない。
これらのトピックはセックスに無関係ではないが、あまりにも複雑であり、十分に論じるには独立した詳細な検討が必要だ。

本書内の各章や節の分け方も、必然的にある程度は恣意的な選択を反映している。
たとえば、BDSM(ボンデージ、支配と服従、サドマゾヒズム)に関する議論は「セックスと性的関係」に含めるべきなのか、それとも「同意」に関する章で論じるべきなのか。
本書が筋の通ったものになるように全力で構成を練ったし、読者に各トピック間のつながりを案内するためにクロスレファレンスを設けている。

第四に、本書では事例を選ぶにあたって、私がリベラル西洋地域と呼ぶ場所{\DDASH}主としてヨーロッパ、北米、オーストララシア〔オーストラリア、ニュージーランドなど〕{\DDASH}を中心に扱っている。
これは、これらの地域が優れているとか、特段注目に値すると示唆したいからではない。
本書の主な読者層がこれらの地域にいると想定していること、そしてセックス倫理に関する学術研究の大半がこれらの国々から発信され、そうした視点を反映していることが理由だ。
また、他の法域や文化については、信頼性と敬意をもって論じるには専門的な知識が必要であり、表面的または不正確な議論となるリスクを避けるようにした。

最後に、本書では特定の章に対するトリガーワーニング(事前内容警告)は設けていない。
ただし、性的暴行やヴァーチャル\ruby{小児性愛}{ペドフィリア}といった、一部の読者にとって動揺を引き起こす可能性のあるトピックを扱う。
私は慎重にこれらのテーマに取り組むよう努めた。
だが、こうした議論が苦痛を伴う可能性があると感じる読者は、目次を確認し、該当する部分を読み飛ばすことを推奨する。
本書の各章は独立して読めるよう構成されている。

本書は、まず歴史上の哲学者が採用してきたセックス倫理のさまざまなアプローチを概観することから始まる。
その後、現代のセックス倫理に関する多様なテーマを検討する。
ほとんどの節において、一つの問いを提示し、それに関する賛否双方の議論を展開する。
議論の順序には特に意図はなく、特定の立場を優先するものではない。
各節の最後には、読者が議論を深めるための質問と、さらに研究するための参考文献を提供する。
最後に、本書全体を通じて用いられる哲学的原則について論じる章を設けた。
読者は必要に応じて、その章を参照するとよいだろう。

\chapter{哲学史におけるセックス倫理学}

一つ質問させてほしい。
一日に何回、\ruby{不動産}{プロパティ}について考えるだろうか?さて、では、セックスについては何回考えるだろうか?もしあなたが家を探している最中でもなく、不動産弁護士でもないなら、おそらく不動産よりもセックスについて考える回数の方がはるかに多いはずだ。
哲学には長い歴史がある。
そしてその長い歴史の中で、哲学者たちは\ruby{所有物}{プロパティ}に関する問題について何千、いや何十万ページもの文章を書き残してきた。
しかし、セックスについて書いた哲学者は比較的少なく、しかも彼らの多くは、少なくとも21世紀以前においては、当時の伝統的な見解を擁護することに主眼を置いていた。

確かに、過去の哲学者たちは完全に自由に発言できたことなどまずなかった。
過去の哲学者の多くは、一定の検閲を課す政府の下で生きていた。
しかし、それと対照的に文学の歴史を見てみると、作家たちは常にセックスを明示的に扱い、既存の価値観に挑戦する作品を生み出してきた。
キリスト教が支配的であった中世の最盛期においてすら、チョーサーの『バースの女房の物語』には、性的快楽への愛と、自分の魅力が男性に及ぼす支配力について率直に語る女性が登場する。
同時代のボッカッチョの『デカメロン』もまた、姦通や婚前セックスを自由に扱っている。
17世紀には、ジョン・ロックが婚外セックスを禁じる法律を擁護する一方で、ロチェスター伯爵\ig{\footnote{2nd Earl of Rochester, 1647-1680、イングランドの貴族、宮廷詩人。
}}は、当時の社会基準から見ても露骨かつ道徳的にスキャンダラスな詩や戯曲を書いていた。
さらに興味深いのは、多くの哲学者は、ロックを含め、政治や宗教に関する既成概念に挑戦するために迫害、追放、さらには死をも覚悟したという点だ。
しかし、セックスについて書くために同じリスクを冒した哲学者はいなかった。

これは、哲学の歴史が本書にとって価値がないということではない。
むしろ反対に、現代のセックス倫理学を論じる哲学者の多くが過去の哲学者の思想に依拠している。
哲学の歴史は、本書で扱う問題を理解するための貴重な資料を提供してくれる。
現代の哲学者の中には、歴史上の哲学者による数少ないセックス倫理に関する論考を参照する者もいれば、過去の哲学者によって発展された主要な倫理学理論をセックス倫理学の問題に適用する者もいる。
本書では、これらの理論について随所で言及していく。

本章では、セックス倫理学に関する文献で最も頻繁に論じられる哲学者、あるいはこの主題に最も関連性のある哲学者を紹介し、以降の章の枠組みを提供することを目的とする。
古代ギリシャの哲学者から始め、近代までの流れをたどる。

\section{古代ギリシャと徳倫理学}

古代ギリシャの哲学者たちは、現代の多くの哲学者よりも広い視野から倫理を考察していた。
彼らにとって倫理とは、単に他者に対してどのように振る舞うべきかを探究するものではなく、いかに\ruby{幸福}{ハッピー}で健全な人生を追求できるかを問うものであった。
彼らは、誠実さや親切さのような他者志向の美徳と並んで、節制や自己鍛錬のような自己の発展に関わる美徳を培うべきだと考えていた。

古代ギリシャの性道徳に関する議論はこうした視点を反映している。
古代ギリシャの哲学者たちは、性行動が私たちの自己発展に及ぼす影響を考察した。
彼らは自然を信頼し、自然は基本的には(必ずしも誤りなくとは言えないが)幸福への確かな指針を示すものと考えた。
こうした理由から、彼らは欲望をごく自然な衝動として受け入れ、それを他の欲求と同様に満足させるに値するものと見なした。
しかし、彼らは自己制御を重視しており、それこそが自由人を奴隷と区別するものであり、善き人生を追求するために不可欠だと考えた。
彼らは性的欲望{\DDASH}ギリシャ語で「エロース」(\emph{eros}){\DDASH}が人間の自己制御にとって危険な脅威となりうることを懸念した。
エロースはしばしば病や狂気として描写され、それに支配されることで人生が破綻しかねないとされた。
一方で、ギリシャ人には「フィリア」(友愛)という別の概念もあった。
フィリアとは、穏やかで愛情深い愛であり、エロースによって圧倒されることなく経験できるものであった。
フィリアは人を支え、慰め、徳ある人生に容易に統合できるものとされた。

ソクラテス(c.470--c.399 BCE)は、若い頃に神殿を訪れ、神託を受けたとされる。
プラトン(c.428--c348 BCE)によれば、その神託はソクラテスに「彼よりも賢い者はいない」と告げた。
しかし、ソクラテスの弟子であるクセノポン(c.430--c.354 BCE)はこの話を異なる形で伝えている。
クセノポンによれば、神託はソクラテスが最も賢いのではなく、「最も自由な人間」だと告げたという。
その理由は、彼が最も自己制御に優れていたからであった。
ソクラテスは死刑を前にした際、このことを誇らしげに振り返り、弟子たちにこう問いかけた。
「私以上に肉体的欲望に支配されない者を知っているか?」、と\citep[16]{xenophon13:_apolog_socrat}。

ソクラテスの教えについて、クセノポンは「肉欲に関して、彼はこう言っていた。
「それには断固として近づくな。一度手を出すと、自己を制御するのは容易ではない」」\citep{xenophon13:_memor}と記している。
プラトンの対話篇において、ソクラテスは理性に支配される者と情欲に支配される者を区別し、それに対応する二種類の愛{\DDASH}情熱的な愛と理性的な愛{\DDASH}を区別している。
「我々の中には二つのものがあり、それが我々を導き、我々はそれに従う。
一つは快楽への生得的欲望であり、もう一つは最善を追求するという獲得された信念だ」と『パイドロス』の中で述べている(237d6--9)。

ソクラテスによれば、快楽への欲望に支配される者は、自己中心的で支配的な愛し方しかできない。
彼らは「狼が子羊を愛するように」愛するのだ(241c6--d1.)%\footnote{\emph{Phaedrus} 241c6--d1.} 。
しかし、それとは異なる、より哲学的な愛がある。
それは欲望に駆られたものではなく、相手の善を真に願うものだ(252d1--253c2)。
今日、「プラトニック・ラブ(プラトン的恋愛)」という言葉は、肉体的欲望を超越した深い愛情を指す言葉として広く用いられている。

プラトンの哲学は今なお研究され、高く評価されているが、彼の弟子であるアリストテレス(385--323 BCE)の著作は、現代の倫理思想、特にセックス倫理に関する議論にさらに大きな影響を与えている。
アリストテレスは、本書で扱うような問題について直接的にはほとんど言及していない。
しかし、彼の倫理思想は「徳倫理学」として知られる理論の中心的な基盤となっており、多くの人々がこれをセックス倫理の問題に応用している。
第一に、アリストテレスは、すべての主要なギリシャの思想家と同様に、倫理を広義に捉え、それを「いかに生きるべきか」「どのような人間を目指すべきか」を探究するものと考えている。
この視点をセックス倫理に適用すると、どのような\ruby{関係}{リレーションシップ}を築くかといった個人的な選択も、倫理的な考察や議論の対象に含まれることになる。

第二に、この広範な倫理的視点に従い、アリストテレスは特定の行為や行為の規則よりも、\ruby{性格}{キャラクター}の発展に焦点を当てる。
彼は倫理的な決定は複雑であり、一般的な原則が指針となることはあっても、適切な行為を固定された規則や正確な計算に還元することはできないと考える。
むしろ、正しく行動するためには、適切な\ruby{性向}{ディスポジション}を身につけることが重要であり、それによって個々の状況において、理性的な人々が適切である、または称賛すべきであると考える方法で対応できるようになる。

第三に、アリストテレスは、称賛に値する性向は習慣化によって発展すると考える。
すなわち、人は単に一般的な行動規範を考察したり定式化したりすることで美徳を身につけるのではなく、具体的な状況において美徳にかなった行為を実践することで美徳を培う。
このようにして、時間とともにそれが内面化され、習慣となる。
彼は『ニコマコス倫理学』の中で次のように述べている。

\begin{quote}
  性格の美徳は習慣から生じる……我々は技術を習得する際、学び終えた時に生み出さなければならないのと同じ成果を繰り返し生み出すことで学ぶ。
たとえば、大工になるには建築し、竪琴奏者になるには竪琴を弾くことで身につける。
同様に、正義の人になるには正義の行為をし、節度ある人になるには節度のある行為をし、勇敢な人になるには勇敢な行為をすることでそうなるのだ。
(『ニコマコス倫理学』1103a15--1103b)
\end{quote}

第四に、アリストテレスは、幸福は特定の基本的な人間の善を追求することから生じ、それらが集まって良い人生を構成すると考える。
美徳とは、これらの基本的な善を自分自身や他者の中で促進する傾向をもつ優れた性格特性のことだ。
アリストテレスは普遍的な人間の善の決定版リストを示すことはなく、それらはほとんどの人がすでに理解しており同意しているものだと考えているようだ。
しかし、そのリストに、友人、健康、名誉ある評判などが含まれることは明らかだ。
マーサ・ヌスバウムは、アリストテレスが、ある特定の時代において普遍的に正しい美徳のリストを発見したと主張することを意図的に避けたのだと指摘している。
彼女は次のように述べている。
「アリストテレスの美徳は、常に新しい状況や新しい証拠に照らして修正される余地を残している……すべての一般的な説明は暫定的に保持され、正しい決定の要約として、また新たな決定の指針として機能する」\citep[pp.259--260]{nussbaum93:_non_relat_virtues}。

第五に、アリストテレスは、あらゆる事柄において「中庸」を求めるべきだと主張する。
美徳とは、過度と不足の両方を避けることによって成り立つものだ。
彼は次のように述べている。
「あらゆる快楽に耽溺し、どんな快楽も避けようとしない人は放埒になる。
一方で、あらゆる快楽を遠ざける人は、無感覚な人間になってしまう」(『ニコマコス倫理学』II.ii)。
性的な事柄においても、快楽を完全に避けてしまうことなく、またその追求に没頭しすぎることなく、節度を保つことが求められる。

最後に、アリストテレスは、国家は人々の\ruby{性格}{キャラクター}形成に関与し、美徳の涵養を支援すべきだと考えた。
彼は、たとえば、夫婦に子供を持つことを義務づける法律や、家族の規模を制限する法律など、直接的な措置を検討している。
だが、彼の関心は主に間接的な方法に向けられていた。
たとえば、政府が子供たちに美徳を植え付け、よき市民としてのあり方を教える教育を提供するべきだと考えた。

アリストテレスの思想は、現代の倫理学において「徳倫理学」(virtue ethics)として知られるアプローチの一部として採用されている。
古代ギリシャの伝統を受け継ぎ、現代の徳倫理学者たちは、倫理を非常に広義に捉えている。
ジェームズ・キーナンは次のように述べている。
「徳倫理学は私たちの人生全体を包含する。
それは、人生のあらゆる瞬間を、徳を獲得し発展させる機会とみなす」\citep[p.185]{keenan05:_virtue}。

また、現代の徳倫理学者たちはアリストテレスにならい、性格の発展と、それを形成する習慣化の役割に焦点を当てる。
ギルバート・ハーマンは次のように述べる。

\begin{quote}
  道徳的な美徳とは、理想的に道徳的美徳のある人物がもつ強固な性格特性だ……道徳的選択の典型的な場面において、行為者は、美徳ある人物がその状況でとるであろう行動をすべきだ……目標は単に正しいことをすることではない。
正しい種類の人間になることだ。
すなわち、人は、美徳のある性格を発展させる必要がある。
\citep[][pp.119-120]{harman99:_virtue_ethic_charac_trait}
\end{quote}

セックス倫理に関しては、私たちの行動と性格の関係についての問題がしばしば議論の中心となる。
本書では、このような問いについてさまざまな論題で議論していくことになる。

一部の徳倫理学者は、アリストテレスの見解を採用し、行動や性格の極端な状態の間で「中庸」を求めるべきだと主張する。
そして、これには性的快楽の追求も含まれる。
ラジャ・ハルワニは、性的快楽は刹那的なものであるため、「私たちはセックスや性的活動を人生においてあまりに高い地位にまで引き上げるべきではない。
こうすることは誤りであり、とりわけ、私たちにより価値のあることを成し遂げる能力があるならばなおさらだ」と述べる\citep[p.183]{halwani10:_philos_love_sex_marriag}。

 ハルワニは、私たちの性的行動を、より価値のある他の活動の追求に与える影響によって評価している。
また多くの徳倫理学者は、アリストテレスの「完成主義」(perfectionist)的な政治観も共有している。
すなわち、彼らは、国家には、少なくともある程度、市民を美徳ある行動へと導く役割があると考えている。
ただし、これが必ずしも国家の強制力を用いることを意味するわけではない。
たとえば、公教育のような手段を用いて、市民を徳ある行動へと促すことができると考える者もいる。

\section{アウグスティヌス、アクィナス、そして自然法論}

中世において、スコラ哲学者たちはアリストテレスやストア派の思想を聖書の教えと調和させようと試みた。
その結果として生まれたのが、道徳哲学、特にセックス倫理の分野において持続的な影響を及ぼすこととなる統合的な理論だ。

キリスト教は古くから、人々のプライベートな性的行動を規制することに関心をもち、むしろそれに夢中だったとさえ言えるが、聖書そのものには性道徳に関する記述がほとんど見られない。
旧約聖書は主に、異教の性的な儀式や異邦人との結婚など、ユダヤ教を脅かす行為に関心を寄せている\footnote{結婚に宗教的な多様性をもたらすことの禁止は創世記34:15--16、士師記7:1--6に表れる。
性的儀式の禁止は、たとえば民数記25:1--9やミカ書1:5--7にある。
}。
新約聖書に記されたイエス\ig{イエス・キリスト}も、性的な事柄にはほとんど言及しておらず、しばしばキリスト教における性の抑圧の原因とされるパウロ\ig{聖パウロ}ですら、性的な問題について語ることは少なかった。
パウロは自分たちの生涯のうちに世界の終焉が訪れると考えていたため、結婚を推奨しなかった。
しかし、彼は「情欲に燃えるよりは、結婚した方が良い」と述べており、未練の残る欲望に苛まれるくらいならば結婚を選ぶべきだと認めている\footnote{第一コリント人への手紙 7: 9.}。

中世を通じて、教会は信徒の性的行動への関心を次第に強めていった。
結婚は教会の秘跡の一つとされ、教義においてセックスは結婚した夫婦の間に限るべきものとされた。
さらに、婚姻内であっても生殖を伴わない性的行為は禁じられ、宗教的祝祭日や聖日には性交を控えるべきとされ、また、遠縁の親族同士の結婚すら禁じられた。

キリスト教神学は、教会が生殖を目的とする結婚内のセックスに焦点を当てることを哲学的に正当化した。
この神学体系は多くの神学者によって発展したが、その中でも最も重要な二人が、聖アウグスティヌス(354--430)と聖トマス・アクィナス(1225--1274)だ。

アウグスティヌスの自伝によれば、彼は若き日には自由奔放に性的快楽を追求していた。
彼はかつて、「主よ、私を貞潔にしてください{\DDASH}ただし、今すぐにではなく」と祈ったことを記している。
彼は15年間、ある女性と婚外関係を持っていたが、母親はアウグスティヌスを説得してその関係を解消し、もっと社会的に尊敬される結婚をするよう命じた。
しかし、彼は結婚せずキリスト教に改宗し、神秘的な宗教体験を経て説教師となった。
最終的には北アフリカのヒッポレギウス(現在のアルジェリア北東部のアンナバ)の司教に任じられることとなる。

アウグスティヌスは晩年、神学的な問題について執筆し講演をおこなうようになると、性的欲望に対して極端な警戒心を抱くようになった。
彼はギリシャ哲学者と同様に、性的欲望が理性的な自己制御を脅かすものだと考えた。
初期キリスト教会の多くの思想家と同じく、彼は貞潔をキリスト教的生活の最高の形態と見なした(『神の国』, 14.16--28)。

\nocite{augustine98:_city_god_pagan}
\nocite{アウグスティヌス82岩波}
\nocite{アウグスティヌス76:告白岩波}

それでもなお、彼は結婚が神によって認められたものであると認識しており、これはアダムとイブのパートナーシップによって明らかだとした。
彼は結婚内での性的行為には一定の役割があると認め、結婚によって得られる三つの特定の善を挙げている。
それは、\emph{proles}(子供を作ること)、\emph{fides}(貞節)、\emph{sacramentum}(持続的な\ruby{誓約}{コミットメント})だ。
彼はしばしば生殖を最優先するが、配偶者間の友情の絆もまた非常に重要であることを明言している\citep{augustine98:_excel_marriag}。

アウグスティヌスの思想はカトリック教会において極めて大きな影響を持ち、後にマルティン・ルターをはじめとするプロテスタント神学者たちにも影響を与えた。
しかし、彼は体系的な思想家ではなく、セックスの倫理について徹底的に論じることはなかった。
中世キリスト教伝統において、セックスに関する最も重要な思想家は間違いなく聖トマス・アクィナスだ。
アクィナスはドミニコ会修道士であり、アリストテレス、聖書、初期教会の教父たちの思想を統合した神学体系を構築した。

アクィナスは、自然法論(Natural Law Theory)と呼ばれる哲学学派に最も重要な影響を与えている。
現代の哲学者の中にも、この理論を現代の倫理的問題に適用しようと試みる人々がいる。
アクィナスにならいつつ、この理論の支持者たちは、特定の神学的前提に依存しない形でその正当性を論証しようとする。
自然法論の基本的な考え方は、人間には共通の理性的能力が備わっており、道徳の真理は合理的な存在者が熟考することで認識可能だ、というものだ。
したがって、人々が道徳問題について十分に考察すれば、道徳が求めることを理解するようになるという。
アクィナスのようなキリスト教的自然法論の支持者たちは、神が摂理によって私たちに道徳的真理を見極める能力を与えていると信じており、そのため、自然法の命題は聖書の道徳的教えと一致すると考える。
しかし、彼らは自然法の命題の真偽が聖書の記述との一致によって決まるとは考えない。
むしろ彼らは、神の善と英知によって、自然法の真理は、正しく理解された神の啓示された文書〔聖書〕と一致するよう神によって定められていると信じている。

アクィナスはアリストテレスと同じく、理性によって識別される普遍的な人間の善が存在すると信じていた。
道徳的な行為とは、これらの普遍的な諸々の善を促進するものであり、非道徳的な行為とはそれらを阻害するものだ。
アクィナスは、アウグスティヌスや教会の公式教義を継承し、セックスは結婚という結びつきの中でのみおこなわれるべきだと主張する。
そして彼もまた、結婚を生殖の手段であると同時に友情の絆として捉えた。
彼は次のように述べている。
「夫と妻の間には最高の友情が存在する。
なぜなら、彼らは単に肉体的結合によって結びつくのではなく(それは獣たちの間にもある種の親和性を生じさせる)、家庭生活のあらゆる側面における共同作業の中で結びついているからだ」\citep[Bk.3 Pt. 2 Chap. 123]{aquinas55:_summa_gentil}。
結婚内のセックスは、生殖と交友という普遍的な善に奉仕するものだ。
しかし、婚外のセックスや結婚内であっても特定の種類のセックスは、これらの善に寄与しないため、道徳的に非難されるとアクィナスは考えた。

アクィナスは強迫的ともいえる目録作成者・分類者であり、さまざまな不道徳な性的行為を「淫蕩の罪」(sins of lechery)と呼んでランク付けしている。
これには、その比較的軽い罪としてマスターベーション、それより重いものとして順に異性愛者間の口淫や肛門性交、同性愛、そして最も重い性的罪とされる獣姦が含まれる。
以上のものは、セックスの自然な生殖目的に反するため、大罪 (mortal sins) だとされている。
また、第二のカテゴリーも存在している。
アクィナスは異性愛者間の近親姦、姦通、誘惑、強姦の四つをその例として挙げている。
これらは「自然な」性行為ではあるが、不道徳な目的のためにおこなわれるものであり、他者に害を及ぼすため不道徳とされる。
これらは大罪ではなく、小罪(venial sins)に分類され、神の意図に対する冒涜というよりも社会道徳に対する侮辱と考えられていた\citep[cf.][II.ii, Question 154]{aquinas20:_summa_theol}。

自然法の概念は長い間、道徳的・法的思考に浸透しており、婚外のあらゆる性的行為を禁止する正当化の根拠として頻繁に用いられてきた。
法は何世紀にもわたり、これらを「自然に反する罪」と定義してきた。
アラスカ州最高裁は1969年、州の反ソドミー法を評価する際、「自然法」という概念が法制度にどのような認識論的基盤を提供するのかを検討した。
裁判所は次のように述べている。
「この主題に関する多くの判例に見られる自然法の概念は、暗黙のうちに、理性によって導き出され、すべての常識を持つ人々に認識可能な確立された規範の観念を指している」(\emph{Harris v. State}, 457 P.2d 638, 1969)。
また、ジョージア州最高裁は1905年に、自然法について次のように述べている。
「自然法は直観的に認識されるものであり、その存在を証明する証人として良心が呼び出される」(\emph{Pavesich v. New England Life Ins. Co.}, pp.69--70.)。

こうした自然法の概念が法に組み込まれる際、それらが持つ正当性についての批判的検討はほとんどおこなわれなかった。
しかし、20世紀に入ると状況が変わり始めた。
オリヴァー・ウェンデル・ホームズ判事は1918年に発表した論文の中で、自然法の概念に対して本格的な批判を展開した。
ホームズ判事は次のように述べている。
「自然法を信じる法学者たちは、自分たちや隣人が親しみ受け入れてきたものを、すべての人々が受け入れるべきものとして無邪気に考えている状態にあるように思われる」\citep[p.40]{holmes18:_natur_law} 。

その後の数十年間、自然法論は性的行為に対する法的規制の正当化手段としてますます疑問視されるようになった。
1969年、アラスカ州最高裁は州の「自然に反する罪」法に対する異議申し立てを審理し、この法律を違憲と判断した。
裁判所は次のように問いかけた。
「私たちは自然法の内容をどのような知識の源から得るのか?」(\emph{Harris v. State},  p.23)。
裁判所はホームズ判事の論文を引用し、満足のいく回答は得られないと結論づけた。
1973年には、アイルランド最高裁が避妊に対する禁止法を審査し、次のように認定した。
「私たちのような多元的社会において、裁判所が憲法上の法原則として、自然法の異なる解釈のいずれかを選択することは許されない」(\emph{McGee v. Attorney General}, pp.318--319)。

現在では、自然法論は公的な言説においてはさほど目立つ存在ではなくなった。
しかし、それが完全に消えたわけではない。
たとえば、バージニア州司法長官を務めたケン・クチネリは、連邦最高裁が2003年に\emph{Lawrence v. Texas} 事件で反ソドミー法を違憲とした後も、州の反ソドミー法を擁護し続けた。
彼は2013年にメディアに対して次のように述べている。
「私の見解では、同性愛行為は……不正だ。
それは本質的に不正だ。
そして、自然法に基づく国においては、それを反映した政策を持つのが適切だ……同性愛行為は、自然法と整合しない」\citep{bump13:_virgin_is_retro}。

現代哲学には「新自然法論」(New Natural Law Theory)と呼ばれる学派があり、これはアクィナスの自然法論を現代の倫理的課題群、特にセックスの倫理に適用できる形に再構築しようとするものだ。
新自然法論の代表的な論者には、ジョン・フィニス、ジャーメイン・グリゼズ、ロバート・ジョージがいる。
彼らは、道徳的行為とは普遍的な人間の善を促進するものであるとするアクィナスの立場を踏襲している。
彼らの考えによれば、道徳的であるためには、セックスはこうした善を促進しなければならない。
すなわち、セックスは「生殖」または「男女間の生涯にわたる伴侶関係」のいずれかの善を支えるものでなければならないとする。
新自然法論者たちは、これを説明するために「生物学的\ruby{結合}{ユニオン}」というテクニカルな概念を用いる。
異性愛のカップルが性行為をおこなう際には、「彼らが共に形成する全体の共通の生物学的目的へと調整される」とされる\citep[p.25]{girgis12:_what_marriag}。
\ig{Ryan T. Anderson}

したがって、婚外の性行為は、異性愛・同性愛を問わず、このような結合を形成することはできない。
フィニスは、婚姻関係以外の性行為は「実際には相互マスターベーションの一種であり、あらゆるマスターベーションと同様に自己疎外的(self-alienating)であり、非人格化(depersonalizing)である」と述べている\citep[p.1066]{finnis94:_law_moral_sexual_orien}。

新自然法論者たちは、国家が人々の性行動を規制する役割を果たすべきだと考える。
ロバート・ジョージは次のように述べている。

\begin{quote}
健全な政治と良い法律とは、人々が道徳的に正しく価値のある生活を送ることを助けることに関心をもつ。
また、良き政治社会は、公権力の強制力を行使することにより、人々を悪徳の腐敗的影響からある程度保護することが正当化される。
\citep[p.20]{george93:_makin_men_moral}
\end{quote}

新自然法論のセックス倫理は厳格なものであり、同性愛のセックス、避妊、マスターベーションなど、一般的に広く受け入れられている多くの\ruby{実践慣行}{プラクティス}を不道徳とみなす。
そのため、提唱者たちは、自分たちの主張は宗教的な根拠には依拠していないと主張するものの、非キリスト教徒の哲学者の間ではほとんど支持を得ていない。
その概念枠組みが難解であり、立場の多くの点が極端であると考えられているためだ。
オックスフォード大学では、フィニスが同性愛者の権利に反対していることを理由に、学生の間で彼を教授職から排除しようとする動きさえあった\citep{benn19:_we_dont_think_john_finnis}。
しかし、新自然法論は保守派の間では依然として影響力を持ち続けている。
実際、\emph{The New York Times}は、新自然法論の代表的な哲学者であるロバート・ジョージを「この国で最も影響力のある保守的キリスト教思想家」と評し、「13世紀のカトリック哲学を実際の政治的影響力に変えた人物」と述べている\citep{kirkpatrick09:_conser_chris_big_think}。

\section{カントとカント主義}

20世紀には、ジョン・ロールズの著作の影響によって、インマヌエル・カント(Immanuel Kant, 1724--1804)の道徳理論への関心が再燃した。
カントの道徳および政治に関する思想は、現代の哲学者たちの間で非常に大きな影響力を持つようになった。
カントと功利主義者のジェレミー・ベンサムはほぼ同時代の人物であり(カントの方が年長だが、その道徳哲学の出版は後発である)、カントの道徳理論は功利主義としばしば対比される。
ベンサムの功利主義とは異なり、カントの理論は、「善」(good)よりも「正しさ」(right)を優先する。
カントは、私たちは普遍的な「格律」(maxim)、すなわち同様の状況にあるすべての人に適用できる一般的な規則に従って行動しなければならないと主張する。
カントによれば、真に道徳的な行為とは、ある特定の状況において最良の結果を生み出すように工夫することではなく(これは功利主義者の考え方である、本書1.4節参照) 、理性そのものが私たちに授けてくれる格律に従って行動することだ。

カントは、私たちが従うべき指針として「定言命法」(categorical imperative)を提示する。
その第一定式は次のように述べられる。
「ただ、汝の意志の格率が、常に同時に普遍的立法の原理として妥当しうるように行為せよ」\citep[4:421, p.34]{kant11:_groun_metap_moral}。

さらに、普遍的な格律は、一つの包括的な命令によって導かれなければならない。
それはすなわち、すべての人間を尊敬をもって扱うことだ。
カントは次のようにも述べている。
「汝自身の人格においてであれ、他のすべての人格においてであれ、人間性を常に同時に目的として扱い、けっして単なる手段としてのみ扱わないように行為せよ」\citep[4:429, p.41]{kant11:_groun_metap_moral}。
カントは、理性的存在としての人間の価値に対して深い敬意を抱いており、しばしば現代の人権思想の創始者の一人と見なされるがこれは適切だ。
カントによれば、私たちの最も根本的な権利は次のようなものだ。
すなわち、「他者の選択によって拘束されることのない自由{\DDASH}ただし、普遍的法則と両立し、他者の自由と両立する限りにおいての自由」\citep[6:237]{kant96:_metap_moral}への権利だ。
カントは、どれほどの社会的利益が生じるとしても、個人を犠牲にすることは正当化できないと考える。
また、彼は「モノ化」(objectification)の害について警告する。
モノ化とはすなわち、他者を固有の尊厳を持つ個人としてではなく、単なる道具として扱うことだ。
モノ化を回避するためには、私たちの行為によって影響を受ける人々がそれを理性的に同意できるか、少なくとも同意する可能性があるかを考慮しなければならない。
オノラ・オニールは次のように述べている。
「カントの見解において、他者を単なる手段として用いることとは、その他者自身が採用\kenten{しえない}格律に基づいて行為することだ\citep[p.138]{oneill89:_const_reason}。

モノ化は、私たちがその人の理性的同意の可能性を無視する時に生じる。
多くの場合、私たちは他者の実際の同意を得ることができ、またそうすべきだ。
ただし、その同意は自由意思に基づくものであり、判断能力がある状態で、なんらかの仕方で欺かれることなくおこなわれていなければならない。
ただし、たとえば心肺蘇生を施す際のように、相手が意識を失っている場合には、私たちはその人が同意すると合理的に想定できるような行動をとるべきであるような場合もある。
カントは、たとえ当事者が同意した場合でも、ある種の行為は不道徳だと考える。
彼は、奴隷になることや自分の臓器を売ることのような\ruby{貶める扱い}{デグレーション}には誰も同意することができないと考える。
また、こうした貶め的な行為に自ら従う場合、私たちは自らをモノ化してしまうとも主張する。
ただし、カントは「貶める扱い」とは何かについて詳細には説明していないが、いくつかの例を挙げている{\DDASH}そのなかにはセックスに関するものも含まれている。
カントは、結婚関係以外でのあらゆるセックスがそれに該当すると考えている。

カントのセックス倫理学への影響は複雑だ\citep[]{herman02:_could_it_be_worth_think}。
彼自身の性的問題に関する見解は非常に極端だ。
カントは、婚外のセックスを「自然に対する犯罪」とみなし、常に深く不道徳だと主張する。
実際、彼は非生殖的なセックス(マスターベーションを含む)は自殺よりも悪いとさえ述べている(このカントの見解についてはSoble, 2003\nocite{soble03:_kant_sexual_perver}を参照)。
カントは、セックスには目的があり、それは生殖だと考える。
そして、婚姻外のセックスは、身体的欲望を満たすこと以外に目的を持たないため、本質的に人間を\ruby{貶める}{デグレーディング}と考えた。
彼にとって、マスターベーションは、自己の身体を単なる欲望の手段とし、理性的存在としての自らの尊厳を損なうため、誤った行為とみなされた。

現代のカント主義者は、カントのセックス倫理学に関する具体的な主張を控え目なものにする傾向がある。
アレン・ウッドは「性道徳は、おそらくカントの見解を擁護するのに最も適さないテーマだ」と述べている\citep[p.224]{wood08:_kantian_ethic}。
その代わりに、彼らはカントのもう一つの重要な洞察{\DDASH}すなわち、私たちは自らの人生について自律的な決定を下す能力と権利をもつ{\DDASH}という点に焦点を当てる。
現代のカント主義者は、道徳と法は人々に可能な限り最大の自己決定権を与えることを目的とすべきであり、それは功利主義が掲げる「最大多数の最大幸福」という原則よりも優先されるべきだと考える。
カント的な自律の概念は、性的自由に関する最も重要な判決にも反映されている。
たとえば、アメリカ合衆国最高裁が妊娠中絶の合法性を支持した\emph{Planned Parenthood v. Casey} (1992)の判決では、(明示的ではないが)カント的な言語が性的および生殖の自由を正当化するために用いられた。

\begin{quote}
これらの問題は、人生において最も親密かつ個人的な選択にかかわるものであり、人格の尊厳と自律の中心にある選択であり、第14修正条項によって保護される自由の中核をなしている。
自由の核心にあるのは、人が自らの実存、意味、宇宙、そして人間の生命の神秘についての概念を、自分自身で定義する権利である。
これらの事柄に関する信念は、国家による強制の下で形成されたものであるならば、\ruby{人格}{パーソンフッド}の属性を定義するものとはなりえない。
\footnote{\emph{Planned Parenthood of Southeastern Pa. v. Casey}。裁判所はのちにこの一節を\emph{Lawrence v. Texas}判決で引用している。}
\end{quote}

現代のカント主義者はまた、カントからモノ化の危険性についての認識を受け継いでいる。
しかし、彼らの間では、何がモノ化に当たるのか、つまり、どのような扱いがあまりに\ruby{「貶め」的}{デグレーディング}であるためにけっして同意できないかについて意見が分かれている。
そのため、セックス倫理学に関する多くの問題についてはカント主義者の間でも対立が生じている。
一部のカント主義者は、カジュアルセックス、セックスワーク、ポルノグラフィなどを批判する。
これらは他者を自己の欲望を満たすための手段として利用するものであり、人間の尊厳を損なうからだ。
他方で、別の立場のカント主義者は、これらの行為は個人の自律の領域に属するものであり、外部からの干渉から保護されるべきだと擁護する。
この議論の対立は、本書で取り上げるいくつかの問題においても見られることになる。
\section{功利主義:ベンサムとミル}

功利主義は「最大多数の最大幸福」というシンプルなスローガンを掲げる哲学だ。
しかし皮肉なことに、近代功利主義の創始者とされるジェレミー・ベンサム(1748--1832)は、学問への厳格な献身で知られ、恋愛やセックスといった世俗的な楽しみを味わうことにはほとんど関心を示さなかったと言われている。
彼は結婚せず、相当な財産を持ちながらも質素な生活を送った。
しかし、彼の哲学は、古典的な道徳観やキリスト教的な「自制の美徳」に反論しつつ、快楽が重要であること、そして実際には快楽こそが価値の最終的な尺度であることを示そうとするものであった。
実際、彼は異なる行為がもたらす快楽の「快楽価」(hedonic value)を理論的には正確に測定できると考え、道徳的に最も優れた行為とは、関係する人々が経験する快楽の総量を最大化するものだと主張した。
この考えが、彼が最初に体系的な哲学として洗練させた「古典的功利主義」の基盤となった。

先に、カントが近代的な人権の創始者の一人とされることを述べた。
哲学者たちはしばしば、個人の権利を重視するカント主義的アプローチと、公的な\ruby{福祉}{ウェルフェア}を優先する功利主義的アプローチを対比する。
歴史的な正確性についての異論はあるものの、この対比は本書で扱う多くの問題を考える上で有用な枠組みを提供してくれる。
ベンサムは、人権を「竹馬に乗ったナンセンス」と痛烈に批判したことで有名だ。
彼は、明らかに皆の\ruby{利益}{グッド}となるような行為を私たちが採用することを、個人の権利の要求が妨げるべきではないと考えた。
このような基本的権利への抵抗ゆえに、功利主義者はしばしば、個人の自由を犠牲にして国家の広範な介入を正当化する思想だと非難されてきた。
しかし、功利主義者は一般に性道徳の問題に対してリベラルな態度をとり、人々の私生活への規制を制限しようとしてきた。

人間の最終的な善が快楽にあると考える者ならば、おそらく、セックスの価値について多くを語るべきであるはずだ。
なぜなら、セックスは人間が享受できる最も強烈な快楽の一つだからだ。
しかし、ベンサムは公刊された著書の中では、セックスについてほとんど言及していない。
彼は『政府論』の序文で読者に向けて次のような弁明の注意書きをしている。
「読者はおそらく、本文中に「性的」 (sexual)という単語が八〜十回ほど登場することに気づくだろう」と\citep[p.533]{bentham77:_commen_commen_fragm_gover}。
彼の公刊された著作全体を通じて、セックスやセクシュアリティに関する言及は約60回にすぎない(比較すると、\ruby{財産}{プロパティ}に関する言及は1100回以上にのぼる)。
しかし、この事態は単なる見落としではなく、意図的な自己抑制の結果であったことが判明している。
現代の編集者たちは、セックスと性的自由について数百ページに及ぶ彼の未公刊の手稿を発見している。
どうやら彼はそれを公表しないことを選択したらしい。
なぜなら、功利主義はすでに従来の道徳観に挑戦し、感覚的快楽の追求を賛美する思想として批判を浴びていたからだ。
彼は、異端的な性行動を公然と擁護することで、さらに批判の材料を提供してしまうことを避けたかったのかもしれない。

これらの未公刊の論文において、ベンサムは幸福におけるセックスの重要性を率直に述べている。
彼は、性的快楽を「人類にとって最大の、そしておそらく唯一の\ruby{本物}{リアル}の快楽である」と評している。
そして、成人同士の同意に基づくさまざまな形態のセックスを犯罪としていた当時のイギリスの法律を批判した。
彼は、いわゆる「不自然な」性行為について次のように述べる。
「私は長年、こうした行為が今日のヨーロッパ諸国のすべてでこれほど厳しく扱われることを正当化する十分な根拠を見つけようと苦心してきた。
しかし、功利の原則に照らして、そのような根拠を見出すことはできなかった」。
また彼は「あらゆる形態の性的満足のための包括的な自由」が人類にもたらす利益は計り知れないと言う。
そして、もし同意する成人がそのような自由を認められたならば、「一体どのような計算が、この新たに生み出される快楽の総量を測ることができるのだろうか?」と問いかけている。
彼は特に、同性愛者に対する厳しい処罰に憤慨していた。
もしベンサムがこれらの論文を公表していたならば、彼は性的自由と多様性の寛容を擁護する最も早い時期の、かつ最も徹底した論者の一人として歴史に名を刻んでいただろう。

ベンサムの死後、その弟子であるジョン・スチュアート・ミル(1806--1873)が、功利主義の推進と発展の役割を引き継いだ。
1859年、ミルはその後古典的名著となる『自由論』を発表した。
この書は、国民が性生活も含む自分の人生を自由に選択できるべきだと考える人々にとって中心的なテキストとなった。
ミルは次のように主張する。
「いかなる人の行為のうち、社会に対して責任を負うべき部分は、他者に関わる部分のみである。
自己にのみ関わる部分については、彼の独立は本来的に絶対的である。
自己自身について、自己の身体と精神について、個人は主権者である」\citep[p.13]{mill15:_liber_utilit_other_essay}。
ミルは、私たちの道徳的判断と法の双方を導くべき格調高い指針を提示しており、それはのちにミルの「危害原則」(harm principle)として知られるようになった。
彼はこう述べている。
「文明社会の一員に対して、その意志に反して正当に権力が行使されうる唯一の目的は、他者への危害を防ぐことだ」\citep[p.13]{mill15:_liber_utilit_other_essay}。

ミルの『自由論』は、国家の権力に対する個人の自由を雄弁に擁護するものだ。
しかし、彼はベンサムと同様、この自由を絶対的な権利という観点から考えるべきではないことを明確にしている。
彼は次のように述べる。
「私は、抽象的な権利という概念を、功利から独立したものとして扱うことで、私の議論に利点をもたらすことを放棄する。
私は、すべての倫理的問題における究極的な基準として功利を考える」\citep[p.14]{mill15:_liber_utilit_other_essay}。
しかし、ミルは権利の概念を全面的に拒絶するわけではない。
むしろ、彼はそれを功利主義的な観点から再定義し、権利をそれに対応する社会的義務の産物として捉えている。
彼は次のように述べる。
「私たちがある事柄を人の権利と呼ぶとき、それは、その人がその権利を保持することを社会に対して正当に要求できることを意味する。
その権利は、法律の力によって、あるいは教育や世論の力によって保護されるべきものだ」\citep[p.166]{mill15:_liber_utilit_other_essay}。

ミルの私生活は、ヴィクトリア朝時代のイギリスの基準からすれば、やや型破りなものであった。
彼は、既婚女性であるハリエット・テイラーと長年にわたる関係を持っていた(おそらく純粋にプラトニックな関係であったと思われる)。
彼女の夫の死後、二人は正式に結婚した。
ミルは私的には性的自由を擁護しており、日記の中で「性的関係に関して人は何を自由におこなえるかということは重要な問題ではなく、完全に私的な問題と見なされるべきだ。
それは彼ら自身以外の誰にも関係がない」と記している\citep[p.664]{mill88:diary}。
しかし、彼が公刊した著作では、彼の自由に関する見解が人々の私的な性的行動を規制する法にどのような影響を持つのかについては明示されていない。
『自由論』の中で、ミルは「私通(fornication)[未婚者同士の合意に基づくセックス]……は容認されねばならない」と一言述べているが、その立場を詳しく説明したり擁護したりはしていない。
また、売春については違法とされてよいと述べている\citep[p.96]{mill15:_liber_utilit_other_essay}。

ミル自身がこの問題について沈黙を守ったにもかかわらず、彼の危害原則は、道徳の名の下に私的行動を規制する法律に反対するリベラル派によってしばしば引用されてきた。
裁判所もまた、彼の議論をしばしば引用している。
たとえば、アラスカ州最高裁は、1969年の \emph{Harris v. State} 判決において、州の反ソドミー法を違憲とする判断を下した際にミルの議論を参照している(ただし、ミル自身が避けた「基本的権利」という概念も援用している)。
その判決文には次のように記されている。

\begin{quote}
各種の性的行動を規制する法律に対する批判の要点は、それらの法律が社会を危害から保護するために必要な範囲を超えており、個人の自由に介入しているという点にある。
人間には基本的権利として自由が認められるべきであり、純粋に宗教的な信念を、世俗的・社会的・経済的・政府的秩序を守るという実証的価値を欠いた形で、刑法を通じて他者に押し付けるのは不適切である。
これらの主張の多くは、ジョン・スチュアート・ミルの基本的な学説の一つを反映している。
[ここで『自由論』の一節を引用している。
](\emph{Harris v. State}, p.23)

\end{quote}

現代の功利主義者は、ベンサムの考え方に従い、最も正しい行動とは人々の総体的な幸福を最大化するものだと主張する。
ただし、すべての功利主義者が快楽を価値の最終的な尺度としているわけではない。
たとえば、一部の功利主義者は幸福を欲求の充足という観点で測る。
しかし、功利主義者全般に共通するのは、最終的に人々をできる限り幸福にすることを目指すべきだという見解だ。
功利主義者は一般的に性的自由を支持する傾向がある。
なぜなら、そうした自由を求める人々にとっての利益が、社会全体にとってのコストを上回ることが多いからだ。
しかし、多くのケースでは、この計算は単純ではない。
本書で扱うさまざまな問題において、この点が明らかになるだろう。
たとえば、セックスワークをめぐる議論では、一部の功利主義者は、その害があまりにも大きいため、セックスワークを違法とすべきだと考える。
一方で、別の功利主義者は逆の立場をとるのだ。

\section{フェミニズム}

セックス倫理学が長らく多くの哲学者によって周辺的な主題と見なされてきたことはすでに述べた。
しかし、ここには重要な例外がある。
フェミニスト哲学者は、この問題を長年にわたり探究の中心に据えてきた。
「個人的なことは政治的なことだ」という有名なフェミニストのスローガンがあるように、フェミニスト思想家たちは、本書で扱うさまざまな問題{\DDASH}結婚や一夫一妻制からセックスワークやポルノグラフィに至るまで{\DDASH}を精緻に分析してきた。

フェミニスト哲学は、カント主義や功利主義のような単一の体系的な理論ではない。
この用語は、むしろ広範な視点の集合体を指す。
すべてのフェミニスト哲学者に共通する点として、次のようなものが挙げられる。
すなわち、私たちが生きる社会は体系的に男性を女性よりも優遇しているという確信、社会現象を家父長制というレンズを通して理解するという姿勢、ジェンダー不平等は不正であり是正されるべきだという信念、そしてジェンダー不平等の撤廃を中心的な課題とする決意だ。
それ以上の点においては、多様な意見やアプローチが存在する。

フェミニズムは、縦軸では歴史的な「波」によって、横軸ではリベラルフェミニズムやラディカルフェミニズムなどの異なるアプローチによって分類することができる。
第一波フェミニズムは、19世紀から20世紀初頭にかけて、女性の基本的な法的権利(財産所有権や参政権など)の獲得を目指して戦った。
第一波フェミニズムの重要な思想家であるメアリ・ウルストンクラフト(1759--1797)は、『女性の権利の擁護』(1792)\nocite{ウルストンクラーフト女性の権利}
の中で、女性の男性への依存は当時の社会環境の産物であり、本質的な性質ではないと主張した。
彼女は、女性が教育を受ける機会や市民的・政治的権利を与えられれば、自律的で独立した生活を送ることが可能だと論じた\citep{wollstonecraft93:_vindic_right_woman}。
第一波フェミニズムの政治的な綱領は、1848年のセネカ・フォールズ会議で明確に示された。
この会議では、知識人や活動家の女性たちがニューヨーク州北部に集まり、「女性の社会的、民事的、宗教的な状況と権利」を討論し、参政権を含む男性と平等な権利を求める決議を採択した\citep{parker08:_senec_falls_conven}。

1960年代に登場した第二波フェミニズムは、運動の焦点を拡大し、ジェンダーの不平等をより包括的に捉えようとした。
彼女たちは、教育や職場における女性への障壁の撤廃、政治における女性の代表性の向上、家庭内暴力や性的暴行から女性をよりよく保護するための法改正など、幅広い改革のために戦った。
また、女性の機会を制限し、人生の可能性を形作る非公式な社会的ダイナミクスについても精査し始めた。
ベティ・フリーダン、グロリア・スタイネム、ジャーメイン・グリアといった作家たちは、家父長制が女性の社会的地位だけでなく、彼女たち自身の自己認識や人生設計にどのような影響を与えるのかについて、鋭い分析を提供した。

第二波フェミニズムの登場以来数十年にわたって、フェミニスト哲学者たちは家父長制を分析するために多様な視点を発展させてきた。
その中でも最も著名なものが、リベラルフェミニズムとラディカルフェミニズムだ。
リベラルフェミニストたちは、既存の社会的・法的構造の改革を通じて、女性が道徳的に平等な立場を得ることを目指す。
彼女たちは自由を中心的な政治的価値とし、\ruby{公正さ}{フェアネス}や基本的権利の概念に訴える。
リベラルフェミニストは、女性が私生活や職業生活において男性と同じ選択肢を持ち、既存の政治制度によって完全に代表されることを確保すべきだと考える。
スーザン・ウェンデルは、リベラルフェミニズムについて次のように述べている。

\begin{quote}
リベラルフェミニストたちは、女性の人間としての価値は、男性や子供の\ruby{福祉}{ウェルビーイング}にとっての手段ではなく、男性の価値と等しいものだと主張し、その公的・私的な承認を求めてきた。
これには、女性の自由やプライバシーの尊重が含まれる。
リベラルフェミニストたちは、常に女性の法的権利の平等を推進してきたが、近年では性別に基づく事実上の差別の撤廃を求め、その目的のために国家の関与を訴えている。
彼女たちは、伝統的なリベラルな考えに則り、社会改革と人間的な充足にとっての教育の力と重要性を信じており、メアリ・ウルストンクラフト以来、少女や女性が少年や男性と同等の教育を受けることを求めてきた。
\citep[p.66]{wendell87:_qualif_defen_liber_femin}
\end{quote}

しかし、リベラルフェミニズムは、哲学的および実践的な観点から、他のフェミニストたちから批判を受けることもある。
哲学的には、リベラルフェミニストが古典的リベラリズムの前提をある程度受け入れていることに疑問が呈されてきた。
その中でも「抽象的個人主義」の概念が問題視されることが多い。
アリソン・ジャガーは、抽象的個人主義を次のように特徴づけている。
「論理的に(経験的ではないにせよ)、人間個人は社会的文脈の外部に存在しうる。
個人の本質的な特徴、欲求や関心、能力や願望は、社会的文脈とは無関係に与えられるものであり、その文脈によって生み出されたり根本的に変容させられるものとは考えられていない\citep[pp.28-29]{jaggar83:_femin_polit_human_natur}。
この見方によれば、個人は孤立し自己利益を追求する存在とされ、(少なくとも理想的には)合理性によって動機づけられている。
しかし、多くのフェミニストは、こうした抽象的個人主義はフェミニスト的意識を損なうと考える。
フェミニストたちは、私たちはそれぞれの個人が埋め込まれた文脈を理解する必要があり、家父長制に抵抗するためには女性たちの広範な連帯を確立すべきだと主張する。
また、批判者たちは、リベラリズムが前提とする人間は合理的で自己利益を追求する存在だという考え方そのものが家父長制的であり、人間は本性として身体的・感情的側面をもち、また家族や社会生活へ埋め込まれているものであることを無視していると批判する。
実践的には、リベラルフェミニズムが既存の社会制度を活用して女性の地位を向上させることが可能だと主張する点が、あまりに漸進的すぎる、あるいは\ruby{素朴}{ナイーブ}すぎるとみなされてきた。

ラディカルフェミニズムという術語は、リベラルフェミニズムとは異なり、既存の政治構造が女性の地位向上にどの程度活用できるかについて異なる見解を持つ立場を指す。
ラディカルフェミニストたちは、現在の社会のあらゆる側面が家父長制によって形成されており、ジェンダーに基づくあらゆる区別を廃止するためには、それを徹底的に変革する必要があると考える。
ラディカルフェミニストは、女性が生活の中で、特にセックスに関して、どの程度有意味な自律性を行使できるかについてリベラルフェミニストと意見を異にする。
キャサリン・マッキノンは次のように述べる。
「女性にはほとんど選択の余地がなく、女性の役割を自由に選択するような人物になるしかない」\citep[p.124]{mackinnon89:_towar_femin_theor_of_state}。
同様に、アンドレア・ドウォーキンは次のように主張する。
\ig{Andrea Dworkin}

\begin{quote}
  女性は従順であるよう育てられる。
女性らしさを規定するすべてのルール{\DDASH}服装、行動、態度{\DDASH}は本質的に\ruby{精神}{スピリット}を打ち砕くものだ。
女性は男性を必要とするよう訓練される。
それは性的にではなく、形而上的な意味でである。
女性は満たされるべき「空虚」として育てられる。
女性は男性を恐れ、男性を喜ばせる必要があることを知り、より裕福でより強い男性の助けなしには生存できないと理解するよう仕向けられる。
\citep[p.81]{dworkin83:_righ_wing_women}
\end{quote}

ラディカルフェミニズムは、「セクシュアリティがジェンダー不平等の要」だと考える。
キャサリン・マッキノンによれば、ラディカルフェミニスト理論は、
\begin{quote}
セクシュアリティを男性権力の社会的構築物として捉える。
すなわちセクシュアリティは、男性によって定義され、女性に対して強制され、ジェンダーの意味を構成するものだ。
こうしたアプローチは、フェミニズムの中心を、女性の男性に対する従属という観点に置き、セックス{\DDASH}すなわち支配と服従のセクシュアリティ{\DDASH}を、この従属のプロセスにおいて根本的な要素と位置づける。
あるレベルでセックスがこのプロセスにおいて決定的なものなのだ。
\citep[p.128]{mackinnon89:_towar_femin_theor_of_state}
\end{quote}

こうした視点を持つラディカルフェミニストのなかには、家父長制という大きな文脈の中では、女性が男性とのセックスに真に自由に同意することは不可能だと考える者もいる。
彼女たちは、ジェンダー間の権力格差のために、家父長制社会における異性愛セックスは常に何らかの形で強制的であり、搾取的だと主張する。
アンドレア・ドウォーキンは異性愛セックスを「それは、男性の女性への軽蔑を純粋で、無菌的で、形式的に表現したものに他ならない」とする\citep[p.138]{dworkin87:_inter}。
\ig{Andrea Dworkin}こうした見解を持つラディカルフェミニストのなかには、政治的レズビアニズムを提唱する者や、女性が男性社会から分離することを求める者もいる。

ラディカルフェミニストたちは、国家を家父長制システムの重要な一部とみなす。
リベラルフェミニストとは異なり、彼女たちは既存の制度を改革することで女性の真の平等が達成できるとは考えない。
マッキノンは次のように宣言する。
「国家は、フェミニズムの意味において「男性的」である。
すなわち、法は女性を、男性が女性を見るように見、男性が女性を扱うように扱うのだ。
リベラルな国家は、男性というジェンダーの利益のために、強制的かつ権威的に社会秩序を構成する」\citep[pp.161--162]{mackinnon89:_towar_femin_theor_of_state}。
しかし、ラディカルフェミニストたちは、国家の権力をフェミニストの目的に利用できるなら、それを活用すべきだとも考える。
その一例が、マッキノンとドウォーキンによって提案されたポルノグラフィ規制のモデル条例だ。
彼女たちは、ポルノグラフィが及ぼす危害に対してその製作者に民事責任を負わせる立法を提案した。

リベラルフェミニズムとラディカルフェミニズムは、第二波フェミニズムの二つの主要な潮流だが、それだけがフェミニスト思想のすべてではない。
フェミニズムには、マルクス主義的、ポスト構造主義的、精神分析的アプローチをとる思想家も存在する。
また、フェミニズムの異なる潮流の間に明確な区別を設けることが必ずしも容易だとは限らず、また有益だとも言えない。
たとえば、最も著名なラディカルフェミニスト哲学者の一人であるマッキノンは、マルクス主義の概念を自由に引用している。
彼女は、女性の抑圧を資本主義社会における労働者階級の搾取と類似し、並行したものとして理解すべきだと考える。
彼女によれば、女性の抑圧は、資本主義における労働者の搾取と同様に、体系的かつ構造的に組み込まれたものであり、単なる個別の問題ではない\citep[p.515]{mackinnon82:_femin_marx_meth_stat}。

1990年代には第三波フェミニズムが登場した\citep[cf.][]{snyder08:_what_is_thir_wave_femin}。
レベッカ・ウォーカーは、1992年に \emph{Ms. Magazine} で「私は第三波である」と宣言し\citep{walker92:_becom_thir_wave}、
この運動の幕開けを告げたとされる。
ウォーカーやその他の第三波フェミニストたちは、第二波フェミニズムの課題を完全に否定するわけではない。
しかし、彼女たちはフェミニズムをより若い世代にとって魅力的で実践的なものにすることを目指している。
第三波フェミニストたちは、まず、フェミニズムは女性の自律をより尊重すべきだと考える。
彼女たちの見解では、第二波フェミニストたちは、女性の選択が家父長制を支えたり、伝統的なジェンダー規範を強化するものであると見なした場合に、過度に批判的な態度をとる傾向がある\citep[p.xxii]{hernández02:_colon_this}。
ウォーカーは次のように述べる。

\begin{quote}
  第二波フェミニストたちは、フェミニストであるためには貧困生活を送り、常に批判的であり、結婚を避け、ポルノグラフィを検閲し、女神を崇拝しなければならないと言う。
フェミニストは妥協してはならず、お金や愛のために譲歩してはならず、常に女性の向上に専心しなければならず、高潔で自己犠牲的な生き方{\DDASH}願わくば女性団体で働くなど{\DDASH}をしなければならない。
彼女たちは、セックスの前にスパンキングを望むこと、BMWの自動車を所有すること、禅宗の僧侶となること、「レディ」として扱われることを望むこと、人種的抑圧をジェンダー抑圧よりも優先して問題化すること、\ruby{女性蔑視的}{ミソジニスティック}なヒップホップ音楽を愛すること、自分を虐待した父親といまだに会話すること、結婚し、モンタナの農場で3人の子供を育てたいと願うなどを選べば、フェミニストではなくなると恐れる。
\citep[p.xxxii]{walker95:_to_be_real}。
\end{quote}

第三波フェミニストたちは、女性が自らの選択を自由におこなう権利をもつべきだと主張する。
たとえその選択が第二波フェミニストから見て、自らの従属を助長しているように見えるとしても、それは批判されるべきではない。
そして、多くの第二波フェミニストたちが、メインストリーム文化を家父長制と不可分に結びついたものとして拒否すべきだと訴えたのに対し、第三波フェミニストたちは、メインストリーム文化を逆用し、解放の手段として活用できると考える。
ジェニファー・バウムガードナーとエイミー・リチャーズは、次のように述べる。

\begin{quote}
  私たちの欲望は、家父長制が仕掛けた罠であるとは限らない。
「ガーリー」とは、バービー人形、化粧品、ファッション雑誌、ハイヒールなど、女性らしさを形作るタブー視されたシンボルを包摂し、それらの使用が「私たちは騙されている」証拠などではないことを示すものだ。
化粧をすることは、市場や男性の視線に屈服することを意味するのではなく、それはセクシーであり、\ruby{洒落たこと}{キャンピー}であり、\ruby{皮肉的}{アイロニック}であり、あるいは単に自己装飾の行為だ。
\citep[pp.302--303]{baumgardner06:_manif}

\end{quote}

第三波フェミニストたちは、「セックス・ポジティブ」な立場をとるよう女性に呼びかける。
彼女たちは、ポルノグラフィやセックスワークですら、楽しめるものであり、解放的なものでありえると主張し、第二波フェミニストたちによるそれらの批判の背後には、セックスそのものに対するネガティブな態度が潜んでいると指摘する。
レスリー・ヘイウッドは、新しいフェミニズム運動について、次のように述べている。
「この運動は、ポルノグラフィ、セックスワーク、サドマゾヒズム、ブッチ/フェムの役割を擁護し、さらには異性愛、性交、結婚、セックストイを再評価する」\citep[vol.1, p.260]{heywood06:_women_movem_today}。

第三波フェミニストたちはまた、家父長制を理解する際にはインターセクショナリティ(交差性)の視点を取り入れる必要があると主張し、貧困層や非白人女性、クィア女性、さらにはトランスジェンダーの人々の視点を統合することを優先課題としている。
ヘイウッドは、第三波フェミニズムを「包括性の一形態」と定義する\citep[vol.1, p.xx]{heywood06:_women_movem_today}。
インターセクショナリティという概念は、キンバリー・ウィリアムズ・クレンショウによって提唱され、女性が経験する抑圧がジェンダーのみならず、人種や階級によるものであることを説明するために用いられた\citep{crenshaw89:_demar_inter_race_sex}。
ヘイウッドは、第三波フェミニズムの目的について「人種、民族、宗教、経済的地位に基づく女性間の違いを尊重するだけでなく、一人の個人のなかに存在する異なるアイデンティティにも配慮する」と述べている\citep[vol. 1, p.xx]{heywood06:_women_movem_today}。

第三波フェミニストたちは、第二波フェミニズムに対して常に好意的な態度をとってきたわけではない。
しかし、第二波フェミニズムも当初から多様な声を包含しており、有色人種の女性やレズビアン活動家もその一部であったし、またセックス・ポジティブな立場は、第二波フェミニストの多くにとって優先事項だった。
多くの第二波フェミニストたちは、女性の選択する権利を批判したいのではなく、それらの選択がなされる広範な文脈に注目すべきだと主張している。
女性たちは、自らの欲望が家父長制によってどのように形成され、選択肢がいかに制限されているのかを進んで検討すべきだというのが彼女たちの主張だ。
これは、必ずしも女性の選択そのものを批判することを意味しない。
また、近年では、第二波フェミニズムと第三波フェミニズムの対立を過度に強調することに疑問を呈する声も増えている。
たとえば、フェミニズムにおける「横の違い」、すなわちリベラルフェミニズムとラディカルフェミニズムの対立の方が、フェミニズムの「波」とされる区分よりも、より持続的であり、哲学的に重要だと考えられてきている。

\section{ポスト構造主義}

ポスト構造主義は、ジャック・デリダ、ジャック・ラカン、ミシェル・フーコー、ジル・ドゥルーズ、リチャード・ローティ、ジュディス・バトラーなどの思想家に関連する概念と思想の集合だ。
セックスとセクシュアリティに関する問いは、多くのポスト構造主義者にとって中心的な関心事であり、特にフーコーとバトラーの著作はこの分野において重要だ。
ポスト構造主義の主要な著作は、しばしば非常に複雑で難解であり、また統一された体系的思想を構成するものではない。
むしろ、ポスト構造主義者たちは体系的な思考そのものに対して懐疑的な立場をとる傾向がある。
それにもかかわらず、ポスト構造主義的アプローチが社会問題を分析する際に共通して持つ特徴をいくつか挙げることができる。

第一に、ポスト構造主義者たちは、客観性や真理そのものといった観念を問い直す。
彼らは、私たちが現実にアクセスする際には、常に言語や社会の概念的枠組みを介するため、純粋に客観的な世界に直接アクセスすることは不可能だと主張する。
したがって、ポスト構造主義者は極端な社会構築主義の立場をとり、「男性」「女性」「ゲイ」「ストレート」といったアイデンティティのカテゴリーには、普遍的な本質はなく、歴史の特定の時点で形成されたものであり、その意味は時代とともに変化しうると考える。
フーコーは、このような概念体系の複雑なネットワークを「ディスクール(言説)」と呼び、特定の社会において存在する概念体系が、個人の行動や自己理解をどのように構造化するかを分析した。
たとえば、19世紀の科学的言説は、誰を「精神異常者」として分類し、誰を施設に収容するかを決定した。
ポスト構造主義者は、私たちと現実の間には常にこうした言説のネットワークが介在し、私たちの認識を形成すると考え、これを不可避なものとみなす。

第二に、ポスト構造主義者は、社会の言説的枠組みが、それを支配する人々の利益と結びついていると主張する。
「知は権力である」とフーコーが述べたように、科学者、裁判官、哲学者などの権威ある人物が「客観的真理」へのアクセスを主張することは、既存の社会秩序を維持し、新たな形の社会的統制を確立する手段にすぎないとポスト構造主義者は考える。

第三に、ポスト構造主義者は、人々が合理的な主体として自由に行動やアイデンティティを選択するという考え方を批判する。
個人のアイデンティティは社会的な力によって形成され、その社会の概念的・言語的枠組みによって構造化されるため、ポスト構造主義者は個々の決定や行動をより大きな言説のネットワークと結びつけて分析しようとする。
彼らは「自律的な自己」という概念自体を解体しようとし、クロード・レヴィ=ストロースの言葉を借りれば、「私は存在する」から「私とは、何かが生じている場である」へと視点を転換することを目指す。

最後に、ポスト構造主義者は、社会のシンボル的・言説的枠組みが社会的統制の手段であると考えるが、これを一枚岩的なものや不変のものとは見なさない。
むしろ、それは非常に不安定なものであり、常に支配的な概念を転覆させたり抵抗したりする機会があると考える。
多くのポスト構造主義者は、政治に対して臆面もなく急進的なアプローチをとる。
そして、ポスト構造主義的批判そのものを政治的行為として捉える。
彼らの見解では、社会を変革するための第一歩は、現行の言説的枠組みがけっして不可避なものではないと認識し、それに代わる新たな枠組みを創造する可能性を検討することである。
ただし、主要なポスト構造主義者の著作に、社会を再構築するための積極的なビジョンが提示されているわけではない。
彼らは、政治改革の究極的な終着点を想定することに懐疑的であり、たとえばマルクス主義者のように歴史には最終的なゴールがあるとする考え方を退ける傾向がある。

セクシュアリティに関する議論においては、フーコーは、おそらくポスト構造主義者の中で最も影響力をもっている思想家である。
彼は『性の歴史』という三巻からなる著作を執筆し、「抑圧仮説」と呼ばれる考え方に異議を唱えた。
抑圧仮説とは、ヴィクトリア朝の人々がセックスやセクシュアリティに関する言説を抑圧しようとし、現代社会においてもその影響が続いているというものである。
フーコーは、実際にはヴィクトリア朝の人々はセクシュアリティに取り憑かれており、それについて議論し、人々の性的行動を科学的分類体系に組み込もうとしていたと指摘する。
そして、彼らのセクシュアリティに関する理解は、他の言説と同様に、その社会の権力構造と結びついていた。
彼は次のように述べる。
「セクシュアリティは、権力が適用される権力の外部にある領域ではなく……むしろ、権力の\ruby{意図}{デザイン}の産物であり、その道具である」\citep[p.152]{foucault81:_histor_sexual}。

ポスト構造主義的アプローチのセックス倫理学では、個人のアイデンティティや選択よりも、より大きな制度的構造や、特定の問題に関する思考を形作る概念に焦点を当てる。
また、それは、周縁化された人々の立場に注意を向けることを目指す急進的な政治的アプローチでもある。

ポスト構造主義は批判を受けるだけでなく、一部の批判者からは本能的な敵意さえ引き起こしている。
その理由の一つは、ポスト構造主義の著作が、不快なほど不明瞭であり、それがしばしば意図的であるように思われることだ。
その著者たちは、目が回るほどの新しい用語を導入し、明快な議論を避けることを誇りとしているようにも見える。
このため、批判者たちは、ポスト構造主義の議論は高度な教育を受けたエリート層に向けたものであり、より具体的な政治的闘争とはかけ離れたものだと非難する。
また、ポスト構造主義は社会の物質的条件よりも文化的言説に焦点を当てるため、左派の一部からは、その視野が狭く、エリート主義的だと批判されている。
さらに、ポスト構造主義者は社会変革のための具体的な処方箋を提示しないため、彼らが社会改革に対して静観する態度を促していると批判されることがある。
マーサ・ヌスバウムは、(ジュディス・バトラーに対する批判として)このような態度を「ヒップな敗北主義」と表現している\citep{nussbaum99:_profes_of_parody}。
もし私たちが、権力関係は不可避であり、いかなる抵抗運動もそれらの関係を再構成することしかできず、私たちをそこから解放することはできないと信じるならば、変革に向けて努力する意味はあるのだろうか?また、社会構築主義が、ジェンダーや人種といったアイデンティティ・カテゴリーを疑問視することによって、人々がそれらの集団に属することで連帯やコミュニティを見出す可能性を損なっているという批判もある。

しかし、ポスト構造主義が最も優れている点は、私たちが自分自身や社会について抱いている前提を問い直すことを促すことだ。
人々はしばしば、セクシュアリティに関する私たちの考え方、特に道徳的信念は、自然に根ざしているか、あるいは歴史を通じて不変であると考えがちだ。
ポスト構造主義者は、こうした見方をけっして当たり前のものとして受け入れず、倫理的問題をより広い社会的文脈の中で考察するよう私たちを促している。

\section{クィア理論}

現代のクィア理論は1990年代に登場した。
「クィア理論」(Queer Theory)という名称は、テレサ・デ・ラウレティスが編集した1991年の学術誌 \emph{differences} の特集号「クィア理論:レズビアン・ゲイセクシュアリティーズ」によって広まった\citep{lauretis91:_diff}。
クィア理論はフーコーをはじめとするポスト構造主義者の思想に大きく影響を受けている。
しかし、それだけでなく、1970年代のレズビアン・フェミニズム運動の著作{\DDASH}シャーロット・バンチ\footnote{要訳注}、ザ・フューリーズ\footnote{要訳注}、パープル・セプテンバー・スタッフ\footnote{要訳注}、モニーク・ウィティッグなど{\DDASH}にも依拠している。
また、セクシュアリティが人種や植民地主義の問題と交差することを強調するため、これらの分野の研究とも積極的に接続している。
クィア理論が1990年代に台頭したことは、当時のAIDSアクティビズムと無関係ではない。
理論の提唱者たちは、クィア理論がゲイ、レズビアン、トランスジェンダーの人々の平等な権利を求める闘争と密接に結びついていると考えている。

クィア理論はあからさまに政治的ではあるが、しかしそれは、多くのゲイ・レズビアン活動家たちが採用してきた公民権運動を手本とした戦略に対しては異議を唱えている。
こうした活動家たちは長らく、ゲイやレズビアンは、生得的な性的指向によって定義される社会的に不利な少数者として保護されるべきであり、その点において、人種を理由に保護されるべきブラックの人々と同様であると主張してきた。
クィア理論は、ゲイあるいはストレートといった固定的なアイデンティティ・カテゴリーの存在そのものが疑問視する。
むしろ、同性愛的な性的\ruby{嗜好}{プリファレンス}は数あるエロティックな可能性の一つにすぎず、ある特定の社会において、ある性的指向が当人の社会的アイデンティティの中核を占めることになったのは、単なる偶然の産物だと考える。
イヴ・セジウィックは、その代表的な著作『クローゼットの認識論』の中で、次のように述べている。

\begin{quote}
  ある個人の性器的活動を別の個人のそれと区別するためには無数の次元があるにもかかわらず(行為の種類の好み、部位や感覚、身体的なタイプ、その頻度、象徴的な投資、年齢関係や権力関係、種、参加者の数など)、たった一つの次元{\DDASH}すなわち、対象選択にあたってのジェンダー{\DDASH}だけが残され、それが「性的指向」と呼ばれる今やどこにでも見られるカテゴリーとして普及し、現在に至るまで維持されてきたのは、驚くべきことだ。
\citep{sedgwick90:_epist_closet}

\end{quote}

現代社会は、単に人々を性的指向によって分類するだけでなく、この分類に基づくヒエラルキーを導入してきた。
クィア理論家たちは、人工的な「ゲイ対ストレート」の二項対立が社会にどのような影響を与えてきたのかを分析し、この二項対立が特定の人々を周縁化し、別の人々を特権的な立場に押し上げる仕組みを批判する。
この構造を指す言葉として、「ヘテロノーマティヴィティ(異性愛規範)」(heteronormativity)という概念が導入された\citep{warner91:_introd}。

クィア理論家の多くは文学研究や文化研究の分野で活動しており、ヘテロノーマティヴィティの複雑な根源を掘り起こすために、さまざまなテクストを批判的に分析することを重要な課題としている。
その代表例が、セジウィックの著作だ。
彼女は、広範囲にわたる文学作品や哲学的テクストを精査し、近代のアイデンティティの中核として異性愛/同性愛の二分法がどのように形成されたのかを探求した。
彼女は、単に「ゲイ」「ストレート」といったカテゴリーが何を意味するのかを問うのではなく、こうしたカテゴリーが「どのように機能し、何を演じ、どのような関係を生み出しているのか」を問わねばならないと主張する\citep[p.27]{sedgwick90:_epist_closet}。

デ・ラウレティスは、クィア理論の目的は「もう一つの言説的地平、性的なものを考える別の方法」を生み出すことであり、「異性愛者、ゲイ、レズビアンといった単一的なアイデンティティ」を超越することにあると述べる\citep[p.iv]{lauretis91:_diff}。
クィア理論家たちはヘテロノーマティヴィティを乗り越えるために、ローズマリー・ヘネシーが「不安定で多様な立場の集合体」と呼ぶような、より流動的な個人のアイデンティティを採用することを提唱する\citep[p.965]{hennessy93:_queer_theor}。
社会学者のダイアン・リチャードソン\ig{Diane Richardson}は次のように述べている。

\begin{quote}
クィア理論はクィア・ポリティクスと交差し、自己をXやYやZといったラベルで分類することを戦略的に拒否する。
そして、性的なジェンダー・システムの解体を呼びかける。
私たちは、すでにそうしたアイデンティティを超えている{\DDASH}ポスト・ウーマン、ポスト・マンとして、私たちはトランスジェンダーであり、ポスト・レズビアン、ポスト・ゲイ、ポスト・ヘテロセクシュアル(おそらく?)、私たちはクィアなのだ。
\citep[p.38]{richardson00:_rethin_sexual}
\end{quote}

この視点は、ゲイとストレートという二項対立を否定するだけでなく、ジェンダーが固定的であり、自然に根ざしているという考え方も否定する。
おそらく、このアイデンティティ観を最も影響力をもって広めたのはジュディス・バトラー\ig{(Judith Butler)}であろう。
彼女はジェンダーを「一種のパフォーマンス」として捉えることを提案する。
バトラーは次のように述べる。

\begin{quote}
ジェンダーの現実はパフォーマティヴである。
これは単純に言えば、ジェンダーの現実は、それが実際に演じられる限りにおいてのみ現実となることを意味する……。
ジェンダーの現実が持続的な社会的パフォーマンスを通じて創り出されるということは、本質的な性別や、真実のあるいは持続的な男性性や女性性といった概念そのものが、ジェンダーのパフォーマティヴな側面を隠蔽する戦略の一部として構築されていることを意味する。
したがって、ジェンダーは、それが表現するものか、それとも内面の「自己」を隠すものなのかといった問いに還元することはできない。
むしろ、パフォーマンスとしてのジェンダーは、自己の心理的内面性という社会的フィクションを構築する「行為」なのだ。\citep[pp.527-528]{butler88:_perfor_acts_gender_const}

\end{quote}

クィア理論の政治的意図は、その名称自体に表れている。
「クィア」(Queer)という言葉は長らくゲイやレズビアンに対する蔑称であったが、クィア理論家たちはこの言葉を取り戻すこと自体を政治的行為とみなす。
ヘネシーはこう述べる。
「クィアネスを誇示することは、見えない存在であることや、あるいは人々に対して弁明せざるをえない異常者であることを強いるプレッシャーへの反抗のジェスチャーである。
それは、ヘテロノーマティヴィティへの「適切な」反応を拒否する、ある種の「\ruby{抗議活動}{アクティング・アップ}」の一形態なのだ」\citep[p.867]{hennessy93:_queer_theor}。
クィア理論は、人々を性的指向やジェンダー・アイデンティティによって分類するためのカテゴリーが持つ偶発性を明らかにすることで、これらの分類がもたらす周縁化を解体し、挑戦するための理論的ツールを提供することを目指している。
ジェンダー・アイデンティティと性的指向が流動的であり、パフォーマティヴであるという考え方は、LGBTQの活動家の間でも依然として議論を呼んでいる。
特にアメリカでは、この問題が憲法上の法理と深く関わっているために、より一層の論争を引き起こしている。
アメリカの憲法裁判においては、不変的な特性に基づく差別は、より厳格な司法審査の対象となる。
同性婚の推進者たちは、同性愛が生得的なものであるという主張に基づいて訴訟を展開し、成功を収めた。
州法による同性婚の禁止を無効とする判決において、アメリカ合衆国最高裁は次のように述べている。
「精神科医やその他の専門家は、性的指向が人間の性的な表現の一つとして正常であり、かつ不変的なものであると認識している」(\emph{Obergefell v. Hodges}, p.8)。
さらに、最高裁は、性的指向が不変的であるがゆえに、ゲイやレズビアンは性的関係の種類を選ぶ自由がないと結論づけた。
「彼らの不変的な本質が、同性婚こそがこの深遠なコミットメントに至る唯一の道であることを規定している」(\emph{Obergefell}, p.4.)。
同様に、トランスジェンダーの権利擁護者たちも、自らのジェンダー・アイデンティティが生得的であり、不変的なものであると主張することが多い。
しかし、クィア理論家たちは、この主張に全面的に反対しているわけではない。
むしろ、彼らの目的は、ジェンダー・アイデンティティや性的指向が個人の不変的な特性を反映しうることを否定することではなく、むしろ社会がこれらのカテゴリーをどのように定義しているか、また、それに基づいて特定の社会的役割を強制し、さらにはそれを社会的ヒエラルキーや不平等の基盤として利用していることを批判する点にある。

\section{本章のまとめ:セックス倫理学を研究するための基本原則}

本章の冒頭で、哲学者たちはセックス倫理学の問題をほとんど無視してきたと述べた。
しかし、ここまで見てきたように、例外も存在し、それらは極めて興味深く、重要なものだ。
哲学の歴史におけるセックス倫理学についてより深く学びたいと考える人々にとって、残念ながらその作業は容易ではない。
個々の哲学者や特定の問題に関する研究はあるものの、現時点ではこの分野を包括的に扱ったアンソロジーや教科書は存在しない。
本書の最終章では、セックス倫理学に関する議論で用いられる最も重要な哲学的原則について詳しく論じる。
これらの原則は、本章で検討した哲学者たちの思想に基づいている。
読者は、より詳細な議論を求める場合、そちらを参照されたい。
ここでは、それらの原則を簡潔に要約する。

\begin{enumerate}
    \item \textbf{自由と自律:}
    自由主義社会の基本的な前提は、十分な理由がない限り、人々は自分の人生を自らの判断で生きる権利をもつということだ。
これは道徳的な領域にも同様に適用される。
原則として、個人が何を最善とするかを決定できるようにし、彼らが自己決定をおこなえる環境を尊重すべきだ。

    \item \textbf{平等:}
    社会は、制約を課す場合や利益を与える場合に、すべての人を平等に扱うべきだ。
また、すべての人々に平等な機会を提供し、人種、性的指向、ジェンダー・アイデンティティ、障害の有無、その他の個人的特性に基づく差別から保護すべきだ。
道徳的観点からも、私たちは他者を平等に扱うよう努め、自らの偏見に気づく努力をしなければならない。

    \item \textbf{効用:}
    他の条件が等しい場合、社会全体に利益をもたらすような道徳的および法的原則を採用すべきだ。
社会的善は、主観的な幸福や経済的繁栄など、さまざまな方法で測定され得るため、特定の事例において何が最も重要であるかを決定する必要がある。
また、社会全体の利益と個人の自由との間でトレードオフを求められる場合もある。

    \item \textbf{美徳:}
    私たちは、一般に称賛される生き方を目指し、正しい行動をとるよう自らの人格を涵養すべきだ。
道徳的な美徳を育むことにより、個々の状況に応じて適切に判断し、行動できるようになる。

\end{enumerate}

これらの原則の相対的な重要性については意見が分かれることがあるし、相互に両立しない場合もある。
また、ある原則が、特定の問題について異なる方向性を示すこともある。
そのため、単にこれらの原則のうち一つを持ち出して議論を決着させることはできない。
しかし、これらの原則は、道徳的・政治的議論を進めるための枠組みを提供し、ある程度の共通理解を見出す助けとなる。
たとえ他者と意見が対立するとしても、その立場が理解可能であり、尊重できる一般原則に基づいていることを認識することができるだろう。

\phantomsection{}
\section{討論のための問い}
\begin{enumerate}
\item アリストテレスは美徳の統一的なリストを提示していない。
ある性格特性が美徳的または称賛に値するかどうかをどのように判断すべきだろうか?また、保守主義者がしばしば徳倫理学に惹かれるのはなぜだろうか?

    \item 功利主義者にとって、国家は人々の性的行動にどのように関与すべきだろうか?功利主義者は人々の性的自由に制限を課すことを正当化しうるだろうか?

    \item カントは「モノ化」(objectification)をどのように定義しているだろうか?彼の定義は十分に明確か?また、その定義は彼自身の極端なセックス倫理学観を正当化するだろうか?

    \item フェミニズムの「波」を区別することは有益か?第三波フェミニズムの支持者による第二波フェミニズムへの批判のうち、説得力があるものはあるだろうか?
\end{enumerate}

\chapter{出会い、デート、セックス}

もし18世紀や19世紀のイギリスやアメリカに暮らす若者だったなら、デートの形は今とは大きく異なっていただろう。
それは「バンドリング」と呼ばれる慣習を伴っていた。
若い女性が若い男性を自宅に招き、彼は彼女のベッドで一夜を過ごすことができた。
しかし、それは想像するほど楽しいものではなかった。
女性の両親は彼女を「バンドリング・バッグ」に入れた。
それは即席の貞操具のようなもので、腰のあたりや時には首のあたりで縛られた。
場合によっては、求婚者の男性も自身のバッグに詰め込まれた。
そして、こうしてしっかりと縛られた状態で、二人は隣り合ってベッドに入ることになった。
しかも、それはしばしば女性が両親と共に使うベッドであり、念のため二人の間には長い木の板が置かれることもあった。
そして{\DDASH}ようやくデートが始まるのだった(D'Emillo and Freedman, 1997, p.22; Zarrelli, 2017)。

\nocite{demilio97:_intim_matter}\nocite{zarrelli17:_awkwar_centur_datin}

人と出会い、デートをし、セックスをすることは、それ以来少なくともいくつかの点では容易になった。
今ではキャンバスの袋に縛り付けられる必要はない。
しかし、それでもなお課題は残る{\DDASH}実践的なもの、感情的なもの、そしてもちろん倫理的なものだ。
本章では、人と出会い、デートをし、(うまくいけば)セックスをすることに伴う倫理的問題を扱う。
まず、概念的な問いから始める。
セックスとは何か、そして私たちはどのようにして「自分がセックスをしている」と認識するのか? 哲学者たちはセックスや性的行為の定義についてさまざまな見解を提示してきた。
次に、セックスは常にロマンティックな愛と結びつくべきなのか、またカジュアルセックスの道徳性について考察する。

さらに、デートの相手を見つける際に生じる問題へと進む。
インターネット・デートの倫理について論じた後、デートの相手を選ぶ際の選択が差別とみなされることがありえるかについて検討する。
最後に、ある特定の種類のセックス、すなわちボンデージ、支配と服従、サディズムとマゾヒズムを含むBDSMに関する問題について論じる。

\section{「セックス」を定義する}

グレタ・クリスティーナはエッセイ ``Are We Having Sex Now or What?'' の中で、自身が初めて他者とセックスをするようになったとき、それまでのパートナーの人数を記録しようとしたと述べている。
しかし、年齢を重ねるにつれ、男性だけでなく女性ともセックスをするようになり、さらに、プレイ・パーティーで女性とイチャついたり、\ruby{覗き部屋}{ピープ・ショー}でヌード・ダンサーとして働いたりと、さまざまな種類の性的経験を持つようになった。
彼女はこれらの経験のうち、どれが実際に「セックス」と見なされるのかを考えようとしたが、最終的にはその試みが無意味であると結論づける。
そして彼女はこう認める。
「私はいまだに答えを持っていない」\nocite{christina92:_are_we_havin}。

セックスの定義に悩むのは、彼女だけではない。
〔元アメリカ大統領の〕ビル・クリントンは、〔在職中に〕モニカ・ルインスキーとのセックスを疑われた際、それを否定した。
しかし、実際にはオーラルセックスをしていたことが明らかになると、彼はそれが「本当の意味でのセックス」には当たらないと主張した。
彼は聖書を引き合いに出し、その研究によって自身の見解が正当化されるとまで述べた。
この主張は当時広く嘲笑されたが、実際には、現在においても多くの人々がオーラルセックスを正式な「セックス」とは見なしていないことを示す研究結果がある\citep{dotson-blake12:_explor_social_sexual}

このようなデータが示すように、私たちは誰もが認める「セックス」の定義はもっておらず、したがってそれによって性的行為と非性的行為の境界を決めることができていない。
そして、この境界がどこに設定されるかによって答が左右されてしまう倫理的問題がいくつもある。
たとえば、パートナー同士が、何が「セックス」とみなされるのかについて合意していなければ、何が「不貞」に当たるのかも判断できない。
哲学者たちは、セックスの定義に対して大きく二つのアプローチをとってきた。
それが、通常「(「セックス」の)外在説」(internalism)と「(「セックス」の)内在説」(externalism)として知られるものだ。

\subsection{セックスの内在説と外在説}

セックスを定義する最も単純な方法は、それを特定の種類の身体的接触を伴う行為とすることだ{\DDASH}より具体的には、ある人の特定の身体部位が他者によって刺激されることを指す。
この考え方は「セックスの外在説」と呼ばれることがあるが、それを身体的要素に還元するため「還元主義的視点」とも呼ばれる。
アラン・ゴールドマンは、この立場を擁護し、「セックスとは[他者の身体との接触が]生み出す行為である」と述べている\citep[p.268]{goldman77:_plain_sex}。\ig{Alan Goldman}
しかし、この定義にはさらに精緻化が必要だ。
まず、どの身体部位が刺激されると「性的」な経験と見なされるのかを特定しなければならない。
たとえば、誰かに腕を撫でられたとしても、それが心地よく感じられることはあるが、私たちは通常それを「セックス」とは考えない。
イゴール・プリモラッツは、性的快楽とは「性的部位」、すなわち性器に関連する快楽であると主張する\citep[p.46]{primoratz99:_ethic_and_sex}。

外在説の利点は、もしこの立場を受け入れるならば、セックスを一義的に定義できる点にある。
しかし、それは私たちの実際の経験とはかけ離れたものになる。
たとえば、医療検査では性器が刺激されることがあっても、それを「セックス」とみなす人はほとんどいない。
また、\ruby{性感}{センシュアル}マッサージのように、性器への直接的な接触がなくとも性的であると思われる行為もある。
さらに、BDSMでは性器の刺激や性的な快楽を伴わないことが多い。
加えて、人によっては、相手と同じ場所にいなくてもセックスは成立すると考える{\DDASH}たとえば、性的な会話やテキストのやりとりをしているときなどがそうである\citep[pp.19--20]{soble06:_activ_sexual}。

これに対する代替案として、セックスを「特有の身体的興奮を生じさせる行為」と定義する方法がある。
その興奮には性器が関与することが多いが、それだけに限られるわけではない\citep{janssen02:_sexual_inhib_sis1,janssen02:_sexual_inhib_sis2}。
しかし、この立場をとったとしても、二人が性交をしていながら、まったく身体的快楽を感じないというケースを想定することは可能だ。
プリモラッツは、そのような場合、二人は本当の意味ではセックスをしているとは言えないと述べる\citep[pp.47--49]{primoratz99:_ethic_and_sex}。
しかし、こうした主張は、私たちが一般的に理解しているセックスの概念からかけ離れているように思われる。
性交は、典型的な性的行為と見なされるからだ。

多くの哲学者は、還元主義的視点の問題点は、セックスの「精神的・想像的な側面」を見落としていることにあると考える。
彼らは、セックスとは通常、少なくとも関与する者の一人が、それを「性的行為」と認識することを要すると主張する。
セイリオル・モーガン\ig{セイリオル・モーガン}は、「セックスは頭のなかにある」と述べる。
彼は、「私たちの身体的経験の性質が精神生活と結びついているのでなければ、無数の性的現象を理解することはできない」と論じる\citep[p.5]{morgan03:_sex_in_head}。
この考え方は「セックスの内在説」と呼ばれる。
内在説においては、ある経験が「セックス」に数えられるかどうかは、その経験にどのような意味が与えられるかによる。
性的経験は通常、身体的快楽を伴い、その快楽は多くの場合、性器に関連する。
しかし、常にそうとは限らない。
重要なのは、関与する人々がその経験にどのような意味を見出すかなのだ。

外在説は失敗すると一部の人が考える理由の一つは、性的欲望がその本質において意図的であるという点にある。
つまり、性的欲望は単なる感覚に留まらず、自己の外部に向けられた思考も伴うという主張だ。
サシャ・セットガストは次のように説明する。
\begin{quote}
  性的興奮や性的欲望の経験は身体的な感覚を伴うが、通常は意図的対象にも焦点が当てられる{\DDASH}私たちはある対象に「対して」興奮を感じ、「~に向かって」欲望を抱く。
私たちは合理的存在であるので、そのような対象は単に知覚や想像に現れるだけでなく、私たちはそれらについての\ruby{概念}{コンセプション}を形成し、それが経験の質に影響を与える。
すなわち、私たちの興奮や欲望の対象との関係は、それらについての\kenten{思考}によって媒介され、私たちの性的価値観や信念と結びついている。  \citep[p.384, 強調は原文]{settegast18:_prost_good_sex}。
\end{quote}

内在説の利点は、各人が自らにとって何が性行為と見なされるかを自由に決定できる点にある。
性的経験とは、私たちが適切な意味を付与するものだ。
しかし、これは真空中で自動的に成立するものではなく、人々の性に関する信念はその人の文化や教育や経験によって形作られている。
スペクトルの一端には、ほとんどの人がほぼ常に性的行為と認める典型例{\DDASH}たとえばペニス・ヴァギナ性交{\DDASH}が存在し、他の端にはセクスティング〔メッセージアプリ等で性的なやりとり〕のような活動があり、これを「セックス」と捉えるか否かは人によって異なる。

私たちの言語には、このようにさまざまな程度で「開かれた」概念が多く含まれている。
こうした開かれた概念を用いる際には、私たちは一般にいくつかの典型的な事例については合意しており、それによって互いに意思疎通を図ることが可能になるが、境界的な事例については意見が分かれることもある。
たとえば「\ruby{家族}{ファミリー}」という言葉が人々によってさまざまに理解されることを考えてみればよい。
ある人々は、\ruby{二いとこ}{セカンドカズン}〔はとこ、自分の親のいとこの子供〕や\ruby{三
  いとこ}{サードカズン}〔自分の祖父母のいとこの孫〕までも家族に含めると考えるかもしれないし、別の人々は親しい友人も含めるかもしれない。
また、さらに他の人々はより狭い見方をとるだろう。

しかしながら、性をこのような内的で主観的な方法で定義することは問題を孕む。
すなわち、私は人々は何がセックスで何がセックスでないかについてしばしば異なる見解を持つと述べたが、内在説はこれらの違いを解消してくれない。
むしろ、それらの違いを必然的なものにしてしまう。
こうしたセックスの定義の相違は、人々が共有した経験について意見が分かれた時に重要な意味を持つ。
こうした意見の不一致は、倫理的・法的な問題を生じさせる。
たとえば、もしあなたが一人のパートナーと\ruby{一夫一婦}{モノガミー}の関係にありながら、他の誰かとセクスティングをおこなっている場合、あなたとパートナーがセクスティングがセックスに該当するかどうかについて合意しているか否かが問題となる。
ビル・クリントンは、自分がモニカ・ルインスキーとセックスはしていないと本気で信じていたのかもしれない。
しかし彼は、自分がその関係の性質を隠蔽しようとしていることについては、十分に自覚していた。

理想的には、こうした意見の相違の多くは、コミュニケーションによって解決されるべきだ。
もし、あなたとパートナーが「誰かとセクスティングする」という行為の意味について異なる見解を持つならば、その違いを率直に共有し、双方が合意する形で関係の境界を交渉すべきだ。
しかし、現実生活は必ずしもこんな理想に沿ってくれるものではない。
現実生活には、常に誤解や欺瞞の可能性が存在している。

さらに、内在説を支持する一部の哲学者は、セックスと見なされるものは、少なくとも二人が自分たちが性的活動に関与していると合意する必要があるという特定の制限を設ける。
私はこれを、「セックスの相互説」と呼ぶことにする。

\subsection{相互性、倒錯、パラフィリア}

セックスの相互説を擁護する者たちは、ある\ruby{相互行為}{インタラクション}が真に性的なものと見なされるためには、その行為には、自分たちがおこなっていることを理解して望んでいる二人の参加者が含まれている必要があると考える。
パディ・マクイーンは、セックスを本質的に相互的なものとして次のように定義する。
「性的な相互作用とは、二人以上の人間が、性的欲望(または何らかの性的欲望群)を満たすことを意図し、互いに反応的かつ助けあう形でおこなう活動である」\citep{mcqueen21:_sexual_inter_sexual_infid}。
マーク・ミゴッティとニコル・ワイアットは、セックスを会話にたとえて説明する。
彼らは「セックスをするということは、明らかにセックスに関することであるが、それはまた、一緒に何かをすることでもある」と述べる\citep[p.19]{migotti17:_very_idea_sex_robot}。
相互説は、性的欲望を現象学的に分析し、その構造において本質的に相互的であると考える立場から出発する。
トマス・ネーゲルは、性的欲望を「他者についての感覚」として記述する\citep[p.8]{nagel69:_sexual_perver}。
また、ジャン=ポール・サルトルは『存在と無』(1943)において、「二重の相互的受肉」という概念に基づき、性的欲望を詩的に分析している。
「私は、他者の肉体を私と他者自身に対して現実化するよう他者を促すために、私自身を肉体とする。
私の愛撫は、それが他者を他者自身に対して肉体として誕生させるかぎりにおいて、私に対して私の肉体を誕生させる」\citep[p.514]{sartre43:_being_nothin}。
言い換えれば、相互性への欲望は、性的欲望の不可欠な要素だ。

相互説には、いくつかの明らかな反例がある。
第一に、マスターベーションの問題がある。
相互説の擁護者のなかには、マスターベーションでさえ他者に向けられた欲望によって動機づけられると主張する者もいる。
なぜなら、そこには他者への幻想が伴うことが多いからだ。
しかし、すべてのマスターベーションが幻想を伴うわけではない。
第二に、無生物に対する性的欲望(オブジェクトフィリア)の問題がある。
これは、必ずしも他者に関する思考を含むものではない。
第三に、二人の人間が関わっていたとしても、それが非同意のセックスである場合、一方の欲望は他方に応答されていない。
この場合、相互性が欠如していることになる。

相互説は、次の三つの異なる主張として読むことができる。
「相互性がないセックスは、そもそもセックスとは言えない」、あるいは、「相互性がないセックスは、不完全な形態のセックスだ」、あるいは、「相互性がないセックスは、倫理的に問題がある」。
哲学者たちは、これら三つの立場をそれぞれ擁護してきた。

第一の立場では、幻想を伴わないマスターベーションやオブジェクトフィリアはセックスとは見なされない。
ミゴッティとワイアットは次のように述べる。
「もしセックスが共有された性的主体性を必要とするならば、セックスには最低でも二人が必要だ。
つまり、自分自身とセックスをすることはできない」\citep[p.20]{migotti17:_very_idea_sex_robot}。
また、二人が関わる性的行為であっても、それが非同意的であり、したがって相互的でない場合、それはセックスとは見なされるべきではない。
ティモシー・チェンバースは次のように述べる。
「レイプ犯は、ある人物を特定の身体的動作に強制する。
しかし、こうした強制された動作を「セックス」と呼ぶことは、最初の暴力にさらに侮辱を加えることになる。
それが成り立つのは、私たちのセックスのイメージが著しく歪んでいる場合だけである{\DDASH}すなわち、女性の精神状態や自律性の行使を無視するようなイメージである。
しかし、それは単に不快なだけではなく、忌まわしいことだ」\citep[p.4]{chambers09:_no_you_cant_steal_kiss}\ig{Timothy Chambers}。

第二の立場は、ネーゲルやサラ・ラディックによって提唱されている。
この立場では、セックスには「完全性の度合い」があり、相互性のレベルによって異なるとされる\citep[p.89]{ruddick75:_better_sex}。
ラディックの言葉を借りれば、「個人的で、本質的に自己愛的で、無反応で、非身体的で、受動的で、あるいは強制されたセックス」は、依然としてセックスではあるが、ある程度不完全なものである\citep[p.87]{ruddick75:_better_sex}。
ネーゲルはこれを精神分析的な観点から説明し、非相互的なセックスは「性的感情の最初の段階における原始的なバージョン」であると述べる。
この段階では、個人は自身の欲望が相手に応答される必要があることをまだ十分に理解していない\citep[p.14]{nagel69:_sexual_perver}
ネーゲルとラディックは、非相互的なセックスを「倒錯」(perversion)とみなし、それゆえ不完全なものと考えている。

最後に、相互説を明確に道徳的な観点から定義する哲学者たちがいる。
すなわち、彼らは非相互的なセックスや性的欲望をセックスとは認めるものの、それを本質的に不道徳なものとみなす。
ロジャー・スクルートンは、ネーゲルやラディックにならい、非相互的なセックスを「倒錯」と呼ぶ。
しかし、彼はネーゲルらよりもはるかに踏み込んでおり、それが単に不完全であるだけではなく、実際には道徳的にも問題のあるナルシシズムの一形態だと主張する。
彼によれば、倒錯的な欲望はまず第一に、アリストテレス的な意味で非倫理的だ。
なぜなら、それは「人間の繁栄を阻害する」ものだからだ。
それらは「性的衝動を人間関係の結びつきから逸脱させる」ものであり、このような結びつきは正常で充実した人生において不可欠な要素だ。
したがって、そのような欲望は「理性的な人間が熟慮の上で最も望むであろう人生の実現を妨げる」\citep[p.317]{scruton06:_sexual_desir}。
また、スクルートンは、倒錯的な欲望が第二の意味でも非倫理的であると述べる。
それは、道徳の基盤となる「他者への配慮」を弱めるからだ。
彼は次のように述べる。「私たちが倒錯した欲望に耽るとき、私たちは自己の最も深い部分{\DDASH}すなわち生命そのもの{\DDASH}を道徳的な交流から切り離し、それを道徳法の支配が及ばない領域へと追いやる。
その領域とは、身体が支配し、猥雑さが支配する奇妙な快楽の世界だ」\citep[p.289]{scruton06:_sexual_desir}。

相互説は、たしかに多くの人が持つ信念を的確に捉えている。
それは、セックスとは本来的に、二人(あるいはそれ以上)の人間が自由意思のもとに共におこなう相互的な活動であるべきだ、という考え方だ。
実際、私たちは、少なくとも大部分の場合において、サルトルが語るような(ネーゲルの言葉を借りれば)「重なり合う相互的知覚」を求める傾向がある。
また、相互説の大きな利点は、セックスにおいて私たちが相手の意志や楽しんでいるかどうかに注意を払う必要があることを強調する点にある。
非相互的なセックスは、多くの場合、深刻な害を及ぼし、さらには犯罪にすらなりうる。
性的暴行が極めて重大な道徳的悪であり、深刻な犯罪と見なされるのは、それが非相互的な行為だからである\citep[cf.][]{woollard19:_promis_paedop_rape_signif_sexual}。
しかし、相互説の擁護者たちは、これ以上のことを論じなければならない。
彼らは、非相互的なセックスがすべて、実際には性的ではないか、不完全であるか、あるいはなんらかの問題を抱えていることを示さなければならない。

もし非相互的なセックスは本当の意味でのセックスではないと主張したいのならば、二つの選択肢が考えられる。
第一に、直観に反する結論を受け入れることだ。
すなわち、多くの人が性的であるとみなす行為{\DDASH}たとえば、無生物とのセックス{\DDASH}は、実際には性的行為ではないと主張することだ。
しかし、この立場をとることで、内在説の持つ利点の一つ{\DDASH}セックスの定義を当事者自身に委ねること{\DDASH}を失うことになる。
あるいは第二に、「性的活動」(sexual activity)と、実際に「セックスをすること」(having sex)を区別することだ。
スティーブン・ローは、性的暴行の議論の中でこの区別を次のように説明する。
「レイプが「性的」であるということは、それが被害者の同意や快楽を伴うことを意味するものではない。
また、それが二人の人物が「セックスをした」(having sex)ことを意味するわけでもない(というのも、ある人物たちが「セックスをした」と言う場合、通常は双方の同意があることを\kenten{示唆する}からだ」\citep[p.69]{law09:_rape_is_sex_act}。
この区別は、確かに重要な意義を持つ。
すなわち、同意のあるセックスと、同意のないセックスの間には根本的で重要な違いがあることを強調し、この違いを言葉のレベルで明確にする点において有益だ。
しかし、この言語的区別を導入することによって、少なくとも記述的言語のレベルでは、明白な害は何も及ぼさない行為(マスターベーションや無生物とのセックスなど)と、性的暴行などの重大な不正行為を同列に扱うことになってしまう。

もし、ネーゲルやラディックの立場を採用し、非相互的なセックスは何らかの形で不完全なセックスであると考えるならば、この不完全性をどのように理解するかを決める必要がある。
ネーゲルもラディックも、こうしたセックスをすべて不道徳なものとして非難しているわけではない。
ラディックは「快楽をもたらすあらゆる性的行為は、一応のところ善である」と述べている\citep[p.100]{ruddick75:_better_sex}。
しかし、もしそうであるならば、完全なセックスと不完全なセックスを区別することによって、私たちは何を達成したことになるのだろうか。
彼らが非相互的な、したがって不完全なセックスを指すのに「倒錯」という(明らかに偏見を帯びた)用語を使用していることも問題を複雑にする。
これは、非相互的なセックスには何らかの病理的、あるいは問題のある要素が含まれていることを暗示しているように見える。
ラディックは、倒錯は本質的に有害なものではないと述べている。
しかし、彼女はそれが特定の種類の害を引き起こすリスクを伴うと考えている。
彼女は次のように述べる。「ある性的行為が不完全であればあるほど{\DDASH}つまり、それがより個人的で、本質的に自己愛的で、無反応で、非身体的で、受動的で、あるいは強制的であるほど{\DDASH}それは誰かにとって有害である可能性が高くなる」\citep[pp.100--101]{ruddick75:_better_sex}。
しかし、これが真であるとは必ずしも明らかではない。
すでに述べたように、個人的な自己愛的なセックスのほとんどは、実際には完全に無害だ。
性的暴行のような有害な非相互的セックスは、マスターベーションよりもむしろ「個人的でない」。
したがって、完全なセックスと不完全なセックスを区別することが、セックスの健全性、充実度、倫理性についての私たちの判断とどのように対応するのかは不明瞭だ。

スクルートンの見解{\DDASH}すなわち、非相互的なセックスはそれ自体が不道徳であるという立場{\DDASH}もまた問題を抱えている。
彼は同性愛を倒錯の一形態として分類するのだが、それは、同性愛者たちは自分とまったく同じようなパートナーを求めることで、本物の「他者」(Other)との対峙を避けているのだ、という馬鹿げた理由に基づいたものだ。
これによって彼は自分の議論を弱体化させてしまっている。
しかし、彼はまた、非相互的なセックスを避けようとすることは、人間として完全に発展した人生の重要な要素を見逃すことであり、ひいては自己に対して害を及ぼすと考えている。
この主張はアリストテレス的徳倫理学に基づくものであり、充実した人生には必然的に特定の普遍的要素が含まれるべきだという考えに由来する。
しかし、たとえこのアリストテレス的枠組みを受け入れるとしても、さらに次の三つの前提を認めなければならない。

第一に、相互的な性的関係が、充実した人生に不可欠な普遍的要素の一つであるという点で合意しなければならない。
しかし、少なくとも現存するアリストテレスの著作に表れている限りでは、これはアリストテレス自身の見解ではなかった。
また、これが真であるかどうかについても疑問の余地がある。
確かに、私たちの多くは相互的な性的関係に魅力を感じるが、さまざまな理由で性的関係を放棄する人も多い。
たとえば、多くの宗教ではそうした禁欲が価値あるものとされている。
また、多くの人は単に独身を選ぶ。
こうした人々が、したがって幸福や充実を得ることができないという証拠は存在しない。

第二に、たとえ相互的な関係が充実した人生に必要だと認めたとしても、非相互的な性的関心が、相互的な関係を持つことを妨げる、あるいは少なくともそれを阻害するという前提も受け入れなければならない。
しかし、これは明らかではない。
ある人は、自身の非相互的な性的関心を、満足のいく相互的な関係と完全に統合することができるかもしれない。

最後に、スクルートンは、非相互的なセックスをおこなう人々は、他者に対する道徳的規範を守る能力が低いと主張する。
しかし、一生独身を貫いた人や、無生物に対する性的関心を持つ人が、他の人々よりも道徳的に劣っているという証拠はなく、スクルートン自身もその証拠を提示していない。

相互説についてどのような意見を持つにせよ、最終的には、なぜネーゲル、ラディック、スクルートン、そして多くの哲学者が非相互的な性的欲望を指すのに「倒錯」という用語を選ぶのかを問わなければならない(倒錯に関する哲学文献は驚くほど多い)。
この用語は、歴史的に強い含意を持っている。
この言葉はもともと、異端的な宗教的信念を指すために用いられた。
やがて、それは性的な文脈で使われるようになり、支配的な宗教的道徳によって認められない欲望や行為を非難するための用語となった。
初期の心理学者たちはこれを引き継ぎ、彼らが異常で病理的であるとみなす欲望や実践を分類するために使用した。
したがって、この長い歴史的経緯を考えれば、この用語が真に哲学的な機能を持ちうるのか疑問視する人々に共感するのは容易だ。
たとえば、プリモラッツは「この用語は何の有益な目的も果たさない……したがって、単に廃止すべきだ」と述べている\citep[p.64]{primoratz99:_ethic_and_sex}。
グレアム・プリーストも、「性的倒錯とは……思想史の廃棄場に送られるべきもう一つの概念である」と述べている\citep[p.371]{priest97:_sexual_perver}。

心理学の分野では、「倒錯」という用語は「パラフィリア」(paraphilia)という別の用語に置き換えられていることに触れておくべきだろう。
この用語は現在、『診断と統計マニュアル(DSM)』\ig{診断と統計マニュアル}やその他の教科書で使用されている。
「パラフィリア」という言葉自体は、100年以上前から使用されており、倒錯という言葉が侮蔑的かつ非科学的だと考えた心理学者たちによって導入された。
DSMは1980年に「パラフィリア」という用語を採用したが、その定義は暗黙のうちに相互説を反映している。
パラフィリアは「表現型的に正常で、生理学的に成熟し、同意のある人間のパートナーとの性器刺激または準備的愛撫以外のものに対する、強く持続的な性的関心」と定義されている\citep[p.685]{APA13:DSM}。

DSMの第五版(2013)までは、パラフィリアは定義上、心理的障害と見なされていた。
しかし、第五版(DSM-5)では「パラフィリア」と「パラフィリア性障害」(paraphilic disorder)の区別が導入された。
後者は、「臨床的に有意な苦痛や機能障害を引き起こす」パラフィリア的な性的関心と定義された\citep[p.694]{APA13:DSM}\ig{診断と統計マニュアル}。
こうした苦痛や機能障害がない場合、パラフィリアはもはや自動的に病理と見なされることはなく、治療の対象とも考えられなくなった。

この用語変更の動機は評価に値する。
すなわち、主流から外れた性的関心が、自動的に病理と見なされるべきではないことを認めようとするものだ。
しかし、それによって「パラフィリア」という概念自体が不要になったとも考えられる。
モーザーとクラインプラッツは次のように指摘する。

\begin{quote}
  DSMの精神障害の定義に従えば、いかなる性的行動も、それが個人の機能に支障をきたす場合には病理と見なされうる。
しかし、精神障害の行動的表出と、その精神障害自体、あるいは根本的な問題とを混同してはならない。
\citep[p.103]{moser05:_dsm_iv_tr_parap}\ig{診断と統計マニュアル}

\end{quote}

DSM\ig{診断と統計マニュアル}の定義によるパラフィリア的な性的関心が、その他の苦痛や機能障害を引き起こしうる関心や性向{\DDASH}それが性的なものであれ非性的なものであれ{\DDASH}とどのように異なるのかは明らかではない。
たとえば、鉄道模型の収集、有名人への執着、高級シャンパンの嗜好{\DDASH}これらはいずれも、極端に走れば幸福な生活を送る能力に影響を与えうる。
この観点からすれば、たとえば足フェティシズムには特別なものは何もない。
逆に、ある人の足フェティシズムがその人の人生に完全に統合されている場合もある。
モーザーとクラインプラッツは次のように述べている。
「少なくとも一部の個人にとっては、特定の性的行動が健全なセクシュアリティの表現であり、彼らにとって有益である可能性がある。
特定の性的行動が社会的に受け入れられない、あるいは違法であるという事実は、診断プロセスにおいて無関係であり、無関係であるべきだ」\citep[p.95]{moser05:_dsm_iv_tr_parap}。

したがって、精神障害の診断に関して言えば、パラフィリアという概念はまったく意味がない。
ある関心が個人に苦痛を与えたり、彼らの生活を妨げたりするかどうかが問題なのであり、そうでないならば問題にはならない。
そのいずれの場合においても、それが特に性的なものであるかどうかは直接的には関係がない。
たしかに、性的関心は特に\ruby{しつこい}{パーシステント}ものであり、潜在的には破壊的になりうる。
しかし、主流から外れた性的関心が、より一般的な性的関心{\DDASH}たとえば、人間のパートナーへの報われない欲望{\DDASH}よりも特に破壊的であると考える理由はない。
エルヴィン・J・ヘーバーレは次のように結論している。

\begin{quote}
  結局のところ、「パラフィリア」という言葉は、それが置き換えようとした伝統的な宗教的用語と同様に、イデオロギー的であり、前科学的である。
これは否定的な価値判断を表すものであり、客観的な事実を記述するものではない。
精神科医がこのような道徳的な用語を用いるとき、彼らは自らの立場を損なっている。
\citep{haeberle16:_parap}

\end{quote}

臨床医とは異なり、私たちは倫理的探求をおこなっており、したがって道徳的判断を回避する義務はない。
しかし、「倒錯」という概念であれ、あるいは新しい形式である「パラフィリア」という概念であれ、特定の性的行為が道徳的に正しいか間違っているかを理解する上で有益であることは示されていない。
相互説を受け入れるかどうかにかかわらず、非相互的なセックスを倒錯またはパラフィリアとして定義することは、議論を前進させるものではない。

\subsection{セックスの目的}

哲学者たちは、何がセックスに当たるのかについてだけでなく、そもそもセックスには目的があるのか、あるとすればそれは何なのかについても意見が分かれている。
進化論的な観点から見れば、セックスには明白な機能がある。
それは私たちが生殖し、種として存続する手段である。
しかし、これがセックスの「目的」であるとは限らない。
つまり、セックスがその目的を達成するために「設計された」と考える必要はない。
進化はこの意味での目的を持たない。
自然選択の過程によって、特定の適応が生き延びる種の間で存続するようになっただけだ。
性的生殖は、多くの種、そして私たち人間にとって、生存を確保する上で有効であることが証明されてきた。

セックスの生殖機能が、その道徳的境界を決定すると主張する者もいる。
彼らは、セックスが道徳的に許容されるのは、子供をもうける意図がある場合に限られると考える。
この立場は「生殖主義」(procreationism)として知られている。
しかし、この立場を擁護する哲学者は非常に少ない。
見てきたように、アウグスティヌスやトマス・アクィナスのように、セックスについて比較的厳格な立場をとる哲学者でさえ、別の理由でセックスをすることを認めている。
彼らにとっては、夫婦間の感情的なつながりを維持するためのセックスは許容されるものであった。
この見解は、多くの現代の保守派も共有している。
たとえば、ジョン・フィニスはアクィナスの立場を説明しながら、セックスは「配偶者と婚姻関係における個人的な誓約を表現し、具体化するもの」だと述べている\citep[p.392]{finnis08:_marriag}。

私たちが生殖主義を否定するとしたら、セックスは結婚に限定されるべきだと考える理由があるだろうか。
一部の哲学者は、それはセックスが本質的に特別な意味を持つ行為だからだと主張する。
したがって、セックスは、相手を尊重し、またセックスという行為自体を尊重するためにも、愛と献身の関係の中でのみおこなわれるべきだという。
デヴィッド・ベネターはこの立場を「特殊意義説」(significance view)と呼ぶ。
彼は次のように述べる。
「セックスが道徳的に許容されるためには、それが(ロマンティックな)愛の表現でなければならない。
つまり、それは性的行為の親密さにふさわしい愛情の感情を示すものでなければならない。
この見解においては、セックスの結びつきは、当事者の相互的な愛と慈しみを反映する場合にのみ許容される」\citep[p.182]{benatar02:_two_views_sexual_ethic}。
特殊意義説を採用するすべての人が、セックスにおいて結婚を必要とするわけではない。
しかし、彼らは少なくともお互いと長期的な\ruby{関係を続ける意思}{コミットメント}が必要だと考える。

ここで重要なのは、特殊意義説は、当事者の実際の意見に依存しているわけではないという点である。
この立場では、カジュアルセックスは、それに関与する人々全員がそれを道徳的にまったく問題ないと考えていたとしても、道徳的に誤っているとされる。
しかし、この立場の擁護者は、実際のところ、大多数の人々がこの見解に同意すると考えている。
さらに、多くの擁護者は、この立場が人間に関する基本的かつ普遍的な真理であるとまで主張する。
哲学者G.E.M. アンスコムは次のように述べる。
「\ruby{気まま}{カジュアル}で無意味な性的行為などというものは存在しない。
そんなことは誰もが知っていることだ」\citep{anscombe72contraception}。
ウィリアム・ベネットは、文明社会は長い間、セックスを「最も特別で強力な」人間の相互作用であり、「本質的に道徳的な活動」と見なしてきたと述べる。
彼は次のように言う。
「セックスは価値中立的なものではない。
それどころか、セックスはおそらくあらゆる人間の行為の中で最も価値を帯びたものなのかもしれない」\citep[p.19]{bennett98:_death_outrag}。
ベネターは、私たちの態度が特殊意義説の正しさを示していると考える。
彼は次のように述べる。

\begin{quote}
  この見解は、多くの人々が性的行為そのものではないが、性的親密さと無関係ではない行為について下す判断とよく一致する。
たとえば、(1) 自分の性病の情報を、 (a)ただの知人に共有する場合、あるいは(b) 配偶者や親しい家族に共有する場合。
(2)裸になること、それが(a) 街中で、 (b) 寝室のプライバシーの中で配偶者の前で。
これらの例において、(a) と (b) のどちらにもまったく同じ感覚を抱く人はほとんどいないだろう。
\citep[p.197]{benatar02:_two_views_sexual_ethic}
\end{quote}

ここには二つの問題がある。
第一に、特殊意義説は道徳的な主張をおこなっている。
すなわち、カジュアルセックスは常に道徳的に誤っているとする。
しかし、それは心理学的な主張、すなわち「ほとんどすべての人間がセックスに特別な意義を与えている」という考えに基づいているように見える。
この普遍的事実が、なぜか非意義的なセックスを不道徳なものにするとされる。

しかし、心理学的事実から道徳的事実へと飛躍することは常に危険を伴う。
ベネター自身もこの点を認めているが、彼はこれを次のように擁護する。
「記述的な心理学的主張は規範的判断を導くものではない。
しかし、すべての(あるいはほとんどすべての)人間に共通する不変の心理的特性を否定しようとする道徳観は、いずれも欠陥のあるものであろう」(ibid.)。
しかし、たとえこれを受け入れたとしても、人間の心理に関するこの経験的主張が真であるとは到底明らかではない。
ベネター自身も次のように述べている。
「セックスを意義あるものとみなすことが、人類全体に共通するのか、それとも特定の文化に特有のものなのかは、明らかに経験的な問題であり、心理学者や人類学者、その他の専門家が判断するのが最も適している」\citep[p.198]{benatar02:_two_views_sexual_ethic}。

しかし、特殊意義説の妥当性はこの点に依存している以上、この問題を単に脇に置いておくことはできない。
さらに、この分野の研究から得られる知見を見る限り、疑う余地はほとんどない。
人間は性的に多様な種であり、社会によってセクシュアリティに与える意味はさまざまであり、セックスをする理由も多様だ(cf. Hatfield et al., 2012, pp.1--38, Pukall, 2020, p.280ff.)\nocite{hatfield12:_cultur_social_gender}\nocite{pukall20:_human_sexual}。

それどころか、特殊意義説の中心的な道徳的主張{\DDASH}すなわち、セックスは安定した関係のなかに限定されるべきだという主張{\DDASH}は、その経験的主張、すなわち「すべての人が普遍的にセックスを意義あるものと見なしている」という前提が真ではありえないことを前提としているように思われる。
もし、人々が自然に、かつ普遍的にこの制約を受け入れるのであれば、それを道徳的格律として要求する必要はないはずだ。
この立場の擁護者たちは、少なくとも一部の人々が何らかの理由で「セックスは本質的に意義あるものだ」とする一般的見解を共有していないことを認めている。
彼らは、この少数派が文化全体に影響を及ぼし、アンスコムが「避妊道徳」(contraceptive morality)と呼ぶものを生み出してしまうことを懸念している。
アンスコムによれば、避妊道徳とは、カジュアルセックスを問題のないものとみなす広範な社会的見解のことであり、これはピルの発明と1960年代の性革命によって広まったとされる。

普遍的な人間の本性に反する文化的規範が支配的なものになりうるという考えは、あまり説得力がないように思われる。
しかし、それが不可能であるとも言い切れない。
寛容な社会環境は、短期的なカジュアルセックスへの欲求を満たすことを優先し、長期的な利益を犠牲にする可能性もある。
この点については、本書2.2.2節でより詳細に検討する。

特殊意義説に対する最も有力な対抗理論は、リベラルな「単なるセックス」(plain sex)の視点である。
この立場によれば、セックスは自然な生物学的プロセスであり、必ずしも特定の目的を持つ必要はない。
この立場を最も有名に擁護したのはアラン・ゴールドマンである。
彼は、セックスには必然的な外的目標や目的があるとする「手段・目的分析」(means-end analysis)を否定する。
ここでいう目的とは、生殖、愛の表現、単なるコミュニケーション、対人的承認などのことである\citep{goldman77:_plain_sex}\ig{Alan Goldman}。

リベラルな立場では、セックスの本質は身体的快楽の追求にあり、それには何の恥ずべきことも、不道徳なこともない。
もちろん、当事者が特別な意味を与えることは可能であるが、それを道徳的に必須とする理由はない。
リベラルな立場は、セックスが他の社会的相互作用と同じ道徳的制約を受けると考える。
この点について、ピーター・シンガーは次のように述べる。
「セックスには特別な道徳的問題は何もない。
セックスに関する決定には、正直さ、他者への配慮、慎重さなどの考慮が関与するかもしれない。
しかし、これは自動車運転に関する決定についても同じことが言える」\citep[p.2]{singer79:_pract_ethic}。

リベラル派は、特殊意義説が、他者に害を及ぼさない行為に対して道徳的判断を下すことを批判する。
対照的に、リベラルな視点は、ピアーズ・ベンが「許容の推定」(permissive presumption)と呼ぶ原則に基づいている。
彼はこれを次のように説明する。
\begin{quote}
  私たちがおこなっているあらゆる行為について、それに異議を唱えようとする者は、その異議の正当性を立証する責任を負う。
たしかに、その責任が容易に果たされる場合もある。
たとえば未成年が関与している場合や、同意なしに人々が危害を受ける場合などだ。
しかし、この種の異議が存在しない限り、私たちは自ら選んだ性的生活を、良心の呵責なく送る道徳的権利をもっている。
\citep[p.237]{benn99:_is_sex_moral_special}。
\end{quote}
リベラル派は、特殊意義説がセックスに対する不健全な見方を助長すると考える。
もしセックスを他の行為とは別のものとして特別視するならば、それに対する不安やスティグマを生じさせる可能性がある。
人類学者アヴァ・ミラウスツィーヘンは、セクシュアリティ関連活動家の集会で次のように語った。
「セックスは何か奇妙で神秘的な概念ではない。
それは食事や睡眠と同じくらい普通で必要なものだ……セックスを日常とは異なる特別なものと見なしてしまうと、それが不安や恐怖、機能不全を引き起こしてしまう」\citep{fein13:_why_sex_is}。
また、特殊意義説は、\ruby{継続的関係}{リレーションシップ}の外でセックスをする人々を蔑視することにつながる{\DDASH}たとえば、性的に活発な若い女性に対する「スラットシェイミング」〔ふしだら女呼ばわり〕を助長する。
多くの人々にとって、リベラルな視点の大きな利点は、それが「セックスポジティブ」であること、つまり、性的快楽を罪悪感や恐れなしに追求できることを認める点にある。

特殊意義説の擁護者は、リベラルな視点が退けられるべきであることを示す別の論拠を提示する。
彼らは、セックスに特別な意義があるとする立場こそが、性的暴行や児童の性的搾取によって生じる害の特別な重さを説明できると考える。
リベラル派が本当にセックスを食事や睡眠と同じようなものと考えるならば、性的暴行や児童虐待がこれほどまで重大な犯罪と見なされる理由を説明できないのではないか、というのだ。
ベネターは次のように述べる。
「リベラルな視点の擁護者にとって問題なのは、性的暴行が、たとえば誰かに無理やり何かを食べさせることと比べて、それほど深刻な干渉である必要がないという点だ。
したがって、リベラルな視点はレイプがなぜ間違っているのかを説明することはできるが、それが特別な種類の間違いである理由を説明することはできない」\citep{benatar02:_two_views_sexual_ethic}。
この問題については、本書3.3節でさらに詳しく議論する。

リベラル説と特殊意義説は、それぞれカジュアルセックスの倫理に対して非常に異なる態度をとる。
私はカジュアルセックスについてより詳細に検討したい。
その議論は、セックスに関する特殊意義説とリベラル説の対立を超えたものを含むことになる。

\section{カジュアルセックス}

2015年、\emph{Vanity Fair}は読者に「デートの黙示録」が訪れたと信じ込ませようとした。
同誌は、「約100年間にわたって醸成されてきた\ruby{その場のセックスだけの関係}{フックアップ}文化が、デートアプリと衝突した。
デートアプリは、今や時代遅れとなった求愛の儀式に迷い込んだ隕石のようなものである」と主張した。
その結果、若者たちはもはや恋愛関係を求めることなく、ひたすら\ruby{一夜かぎりの関係}{ワンナイト・スタンド}を繰り返す社会が生まれたという。
同誌は、投資銀行勤務のアレックスという人物の言葉を引用している。
「Tinderのようなアプリがあれば、常に狩りをしているようなものだ……週に2、3回Tinderのデートを設定し、そのすべてとセックスできるかもしれない。
だから、1年で100人の女性と寝ることだってできるわけさ」\citep{sales15:_tind_dawn_datin_apoc}。

この\emph{Vanity Fair}の記事は、若者の間で「フックアップ文化」が支配的になったとするメディアの報道の一例にすぎない。
ミレニアル世代の性生活が、単なるカジュアルな出会いの連続にすぎないという考え方は、統計データと矛盾する。
というのも、現在の若者は、同じ年齢の過去の世代よりも性的に活動的ではないことを示すデータがあるからだ\citep[cf.][]{twenge17:_declin_sexual_frequen,twenge17:_sexual_inact_durin,monto14:_new_stand_sexual_behav}(\emph{Vanity Fair}の記者は、この広範なデータが「解釈の余地がある」と主張している。
ただしこの主張の説明や、それに反するデータの提示はなかった)。
しかし、大学キャンパスなど特定の環境では、社会的慣習が変化し、カジュアルセックスがより一般的になった可能性はある\citep[cf.][]{bogle08:_hookin}。
統計や傾向の正確な数値はともかく、カジュアルセックスは相当広くおこなわれていることは確かだ。
調査によれば、アメリカ人の約3分の2、あるいはそれ以上が人生のどこかでカジュアルな性体験をしたことがあるとされる。
ヨーロッパ諸国では、この割合は同等かそれ以上だ\citep{met17:_one_nigh_stan,how15:_this_is_how}。

カジュアルなセックスが道徳的に許されるかどうかについても、人々の間には対立がある。
2020年にPew Researchがアメリカ人を対象に行った調査によれば、62%がカジュアルなセックスは道徳的に容認できると考えており、つまり残りの三分の一以上はそう考えていないということだ\citep{brow20:_near_half_u}。
本節では、人々がカジュアルなセックスを不快に思う、あるいは少なくとも道徳的に問題があると感じる理由を検討する。
カジュアルなセックスに対する私たちの態度は、単なる個人的倫理の問題にとどまらず、公共政策の問題でもある。
性教育は学校で若者たちに教えられており、それがどのような価値を促進すべきかについては大きな議論がある。
「禁欲オンリー性教育」(abstinence-only education)の推進者たちは、学校は、カジュアルなセックスだけでなく、結婚外のあらゆるセックスを否定すべきだと考えている。
しかし、それほど過激でない性教育プログラムであっても、カジュアルなセックスについてはさまざまな態度がとられている。

私はカジュアルセックスを、継続的な関係にない二人の間でおこなわれ、かつその後関係が続くことを期待されていない性的な\ruby{接触}{エンカウンター}と定義する。
カジュアルセックスは、多くの場合、お互いをよく知らない二人の間でおこなわれるが、必ずしもそうとは限らない。
むしろ、「セックスフレンド」(friends with benefits)の関係は、若者の間でかなり一般的になっている\citep{dube17:_why_frien_benef}。

\subsection{カジュアルセックスは相手をモノ化するのか?}

カジュアルセックスはモノ化(objectification)の一形態であると主張する哲学者がいる。
つまり、それは他者を自らの目的のために利用し、彼らを身体、すなわち自身の快楽を満たすための物理的手段へと還元し、対等な人間として認識しないということである。
もしこれが真であるならば、一般にモノ化を問題視する人々にとって、カジュアルセックスは倫理的に懸念すべきものとなる。

モノ化を道徳的問題として最初に考察した主要な哲学者はカントである(本書1.3節を参照)。
マーサ・ヌスバウムは、モノ化とその道徳的問題性について影響力のある現代的説明を提示している。
彼女は次のように述べる。
「人間を道具的に扱うこと{\DDASH}すなわち、人間を他者の目的の手段として扱うこと{\DDASH}は常に道徳的に問題がある」\citep[p.289]{nussbaum95:_objec}。
この考え方は明らかにカントを想起させる。
カントは、人を単なる手段としてではなく、常に同時に目的としても扱うべきだと述べている。
しかし、ヌスバウムは、モノ化がなぜ間違っているのかをカント主義特有の前提に依存せずに説明する。
彼女は、モノ化には七つの問題点があると特定する。
それは、道具性、自律の否定、受動性、代替可能性、侵害可能性、所有可能性、主観性の否定である。
彼女の説明の核となるのは、モノ化は他者を自己の目的のための単なる手段として扱うことを含み、それがその相手の自律と主体性を否定し、またその個別性をも否定するという見解である。
それは、他者を自らの目的を達成するために利用可能な誰とでも交換可能な存在にしてしまう(ibid., pp.256-257)。

カジュアルセックスは本質的にモノ化なのだろうか? この議論は、比較的異論の少ない前提から始まる。
すなわち、セックスには通常、相手の身体への欲望が、少なくとも一要素として含まれる。
マイケル・ルースは次のように述べる。
「セックスの出発点は、単純に他者の身体への欲望である。
人は肌に触れたい、髪の匂いを嗅ぎたい、目を見つめたいと望む{\DDASH}そして、自らの性器を相手の性器に触れさせたいと望む」。
この議論は、次に「私たちが他者の身体を欲望するという事実が、本質的な道徳的危険を孕んでいる」と主張する。
ルースはこう述べる。
「この欲望は、他者を単に自らの性的欲求を満たすための手段、すなわちモノとして扱うことに非常に近づく{\DDASH}他者を目的ではなく手段として扱う危険がある」\citep[p.185]{ruse88:_homos}。
\ruby{継続的関係}{リレーションシップ}の中でのセックスは道徳的な懸念を引き起こさない。
それは、第一に、肉体的欲望に伴うモノ化は一時的なものであり、相手との長期的な関係の中では断片的なエピソードにすぎないからである。
第二に、欲望そのものが、相手へのより深い配慮や敬意と結びついている。
ティモ・ユッテンは、これが「親密さ、対等性、相互性の文脈のなかにある」と述べる\citep[p.31]{jutten16:_sexual_objec}。
しかし、カジュアルな接触では、相手の\ruby{人格性}{パーソンフッド}が矮小化されることを防ぐものが何もない。
ヌスバウムは次のように述べる。

\begin{quote}
  相手との間にいかなる物語的歴史も存在しない状況において、欲望が偶発的なもの以外の何かに向かうことは可能なのか?また、相手の身体を自己の状態を満たすための単なる道具として利用する以上のことができるのか?……匿名的な精神の下でセックスをする場合、果たして本当に相手を敬意と配慮をもって扱うことができるのだろうか?\citep[p.287]{nussbaum95:_objec}
\end{quote}

しかし、モノ化がなぜそれほど悪いことなのかということについては疑問の余地がある。
私たちは日常的に互いを手段として利用している{\DDASH}たとえば、膝を枕代わりにしたり、強い握力を利用してスパゲッティの瓶を開けてもらったりすることがある。
カントとヌスバウムはともに、モノ化が害を及ぼさない場合もあることを認めている。
しかし、カジュアルセックスは「\kenten{単なる}モノ化」の一形態であると主張される。
それは、相手を完全に自己の快楽の手段へと還元し、その人間性を完全に消し去るというものだ。

日常的なモノ化と、道徳的に問題となる「単なるモノ化」との境界線を引くことは困難だ(本書1.3節を参照)。
これを最も理解しやすい方法は、「単なるモノ化」が、相手を自己の目的のための単なる手段としか見えなくする点にあると考えることだ。
カジュアルセックスが欲望に駆られた一時的なものであるという性質こそが、この単なるモノ化という状況を生み出すと主張されているのだ。

\subsection{カジュアルセックスは搾取の一形態か?}

カジュアルセックスは本質的に不道徳であると主張する別の理由として、それが「搾取」(exploitation)の一形態であるという見解がある。
つまり、ある人々が他者を自分の利益のために利用する形でおこなわれるということだ。
これは、カジュアルな性的接触が常に両者にとってカジュアルなものとして認識されているとは限らないためだ{\DDASH}あるいは、少なくとも両者が常にそれを望んでいるとは限らない。
むしろ、カジュアルセックスが一般的な社会においては、それを本当に望み、楽しむ人々のグループが、理想的な世界であるならばカジュアルセックスを避けたいと考える別のグループを利用する形になってしまっている。
この章のはじめに、カジュアルセックスを経験している人の数は、それを道徳的に許容できると考える人の数よりもはるかに多いと指摘した。
これは単なる偽善かもしれない{\DDASH}つまり、人々は世論調査では、自分の実際の行動ではなく、「そう答えるべきだ」と考える意見を述べている可能性がある。
しかし、別の可能性として、多くの人々がカジュアルセックスをさせられるプレッシャーを感じており、本当はセックスはもっと真剣な関係のために控えておきたいと考えている。
若者は特にこのような圧力にさらされやすい。

単純化のために、ここでは社会の中に二つのグループが存在すると仮定しよう。
一方のグループを「ネスター」(巣づくり者)、もう一方を「ドリフター」(放浪者)と呼ぶことにする。
ドリフターは、総合的に判断した上でカジュアルセックスを好む人々であり、ネスターはセックスを\ruby{恋愛関係}{リレーションシップ}の中に限定したいと考える人々だ。
もちろん、現実には物事はそれほど単純ではない。
多くの人は時にはカジュアルセックスを求め、また別の時には恋愛関係を求める。
また、自分がどちらのグループに属するのか確信が持てない人もいるだろう。
それでも、この単純化されたモデルは、関連する問題を明確に提示するのに役立つ。

この問題の明白な解決策は、ドリフターはドリフター同士でセックスをし、ネスターはネスター同士でつきあうようにすることだ。
もしネスターがドリフターと関係を持ち、相手が陣営を変えることを考えてくれるかもしれないと期待するのであれば、それは自己責任ということになる。
他の領域では、私たちは通常こうした個人的好みの違いをこのようにして解決している。
たとえば、スポーツファンは通常、他のスポーツファンと試合を観戦し、オペラファンは同じ趣味を持つ人と一緒にオペラに行く。
オペラファンがスポーツファンの恋人につきあってスポーツ観戦に行くこともあるが、事前にどこへ行くのかを知っている限り、それが道徳的に問題だとは考えられない。

しかし、カジュアルセックスには、他の社会的相互作用とは異なる道徳的懸念があり、それが他の種の社会的交流と異なったものにしている。
それはカジュアルな性的接触にはある固有の特徴があるためだ。
第一に、人々はセックスや恋愛関係について明確なコミュニケーションをとることを困難に感じることが多い。
特に、よく知らない相手に対しては、他の社会的な場面にはない困難が生じる。
グレイシー・オルムステッドは次のように述べる。
「互いに好意をもっている男女であっても、カジュアルな性行為の最中にシグナルを誤解したり、重要なことを言いそびれたりすることがあるものだ。
見知らぬ相手に対して率直になることは難しく、それによって重要な真実を伝える能力を失ってしまう」\citep{olmstead18:_divor_sex_love}。
ネスターは、自分が期待していることを明確に表現できないまま、ドリフターのパートナーに誤解を与えることがあるかもしれない。
また、自分の期待を伝えても、それが曖昧で誤解される可能性がある。
あるいは、そもそも二人の間でこうした話がまったくなされないこともある。

この問題は、第二に、性的行動に関して人々が強い社会的圧力を感じるという事実によってさらに悪化する。
私たちの選択は、集団の期待や受け継がれた\ruby{行動パターン}{スクリプト}によって形作られる。
これにより、人々は自らの熟慮した選好に反する行動をとることがある。
ドナ・フレイタスは、大学の性的文化についての著書の中で次のように述べている。

\begin{quote}
  フックアップ文化に一人で反対するのを恐れて、多くの学生はそれに従う。
たとえ彼らが内心、別の選択肢を求めていたとしても。
彼らは、セックスをカジュアルなものと見なさないようにしたら、社会生活が崩壊してしまうと考えている。
同調圧力は絶大だ……個別インタビューでは、多くの学生が「フックアップが好きではないけれど、大学の社会生活の大部分を占めているため、好きなふりをしている」と語っていた。
彼らは仲間内で浮くことを恐れているのだ。
\citep{freitas13:_time_stop_hook_up}

\end{quote}

特に女性は「堅物」(prude)だと思われたくないと考える{\DDASH}しかし、同時に、カジュアルセックスに適応すると「尻軽」(slut)と見られる危険もある。
このように、女性は二重の圧力にさらされる。
一方で、大学生に関するリア・フェスラーの研究によれば、男性は別の種類の圧力を受けている。
彼女は次のように述べる。
「男性が本当に何を望んでいるかにかかわらず、フックアップ文化が支配的な環境では、彼らは異性愛者としての公的なアイデンティティを「何人の女性と寝たか」「その女性がどれほど魅力的であったか」に基づいて確立するよう求められる」\citep{fessler16:_lot_women_dont}。

このような同調圧力は、たとえ社会集団内の多数派、あるいは大多数が実際には異なる行動を望んでいたとしても存在しうる。
この現象は「多元的無知」(pluralistic ignorance)バイアスとして知られ、社会心理学者のフロイド・オールポートによって提唱された概念だ(Allport, 1924; O'Gorman, 1986, pp.333-347)。
\nocite{allport24:_social_psyc,ogorman86:_disc_plur_ignor}多元的無知の状況では、個々人は、その場で支配的な価値観と思われるものを他の人々の多くが信じていると想定してしまい、それに異を唱えることを恐れる{\DDASH}たとえ、多くの人々は実は同じような疑念を抱いているとしても。
ある研究者は次のように述べる。
「〔ふるまいの〕規範に関する信念が濃密な性行動を許容するものである場合には、多元的無知が人々の行動に影響を及ぼしている可能性がある。
フックアップの過程において、多元的無知は、性的パートナーのどちらか、あるいは両方が、自分自身の信念ではなく、周囲の規範だと思いこんでいるものに従って行動するように促してしまうかもしれない」\citep[p.130]{griggs03:_plur_ignor_hook_up}。
大学生を対象とした研究では、学生たちは自分の仲間たちが実際よりもはるかに性的に寛容だと信じていることが示されている\citep{chia06:_how_media_cont,reiber10:_hook_up}。
多元的無知は、セックスに関する場合のように、\ruby{あけっぴろげ}{オープン}なコミュニケーションにタブーがある場合に特に助長される。

最後に、カジュアルセックスは継続的関係の\ruby{事前要件}{プレリュード}となってしまっており、ネスターたちが自分の選好に厳格に従おうとすると、彼らが本当に望んでいる「コミットした関係」を手に入れることが非常に困難になってしまっている。
社会的規範の変化によって、ドリフターたちは意図せずして、ネスターを自分たちの選好に縛りつける形になっている。

私がネスターとドリフターと呼んだ二つのグループは、しばしばジェンダーに結びつけて語られる。
ある人々は、ドリフターの圧倒的多数が男性であり、ネスターの圧倒的多数が女性だと主張する\citep[cf.][]{regnerus12:_cont_matin_mark}。
デヴィッド・バスは次のように述べている。
「研究は一貫して、男性が平均してより多くの性的パートナーの多様性を求めることを示している。
男性は一日に女性より頻繁に性的な思考を抱き、複数のパートナーを含む性的空想をより多く持ち、カジュアルセックスのみを目的としてオンラインマッチングサイトに登録する傾向が高い」\citep{buss16:_what_do_you}。

バスは、この明らかな違いを説明するために進化論に訴えているが、この男女差が有意でないと考える人も多く、また、たとえ存在するとしても、それは完全に社会的に構築されたものだと考える人もいる。
いずれにせよ、この違いは普遍的ではない。
ネスターの中にも男性は多数存在し、ドリフターの中にも女性は多数存在する。
しかし、もし女性がネスターである可能性が男性よりも高いと認めるならば、カジュアルセックスを禁止することは、特に若い女性が男性によって搾取されることを防ぐ手段にもなりうる。

\subsection{カジュアルセックスは有害か?}

一部の論者は、カジュアルセックスがさまざまな形で危害をもたらすと主張する。
彼らは第一に、病気や望まない妊娠のリスクがあると指摘する。
第二に、彼らは、少なくとも多くの人にとって、心理的なダメージを与える可能性があると主張する。

20世紀以前は、カジュアルセックスに対する明白かつ説得力のある反論が存在した。
カジュアルセックスは病気の蔓延や望まない妊娠の非常に現実的なリスクだった。
現在では、この問題は解決されたと考えられるかもしれない。
コンドームは広く容易に入手可能であり、他の避妊法も少なくとも妊娠の防止には効果がある。
しかし、避妊手段へのアクセスが容易であるだけでは十分ではない。
人々がそれを確実に使用しなければならない。
アンスコムは、避妊の普及がカジュアルセックスに対する抑制を弱める一方で、人々が避妊手段の使用に対して注意を払わなくなる可能性があると主張する。
彼女は次のように述べる。
避妊が容易になればなるほど、「ますます多くの人々が、責任を感じず、抑制することもなく、\ruby{性交}{インタコース}をするようになる。
\kenten{しかし}、彼らは常に確実に避妊手段を使用するほど慎重ではない」\citep[p.146]{anscombe72contraception}。
ボストン大学の学生健康・福祉センターの教育者であるベス・グランペトロは、ボストン大学のニューズレターで、学生たちはフックアップに関して誤った安心感を抱いていると指摘している。
彼女は次のように述べる。
「多くの学生は、人の見た目や服装、つきあっている人々を見れば、その人がクリーンかどうか、つまり病気にかかっていないかどうかわかると考えている。
しかし、残念ながら、それは事実ではありません」\citep{noa07:_prob_hook_up}。

このような「リスク補償」バイアスは、社会科学者にはよく知られている現象だ。
たとえば、自動車の安全性が向上すると、人々はより不注意に運転するようになり、実際には事故が増加する可能性がある。
同様に、若者は望まない妊娠に対処する準備が年長者たちよりも整っていないのに、セックスに必要な予防策を怠る傾向がある。
また、望まない妊娠は後に妊娠中絶によって対処できるが、それには女性の健康へのリスクが伴い、多くの女性は中絶に対して道徳的な懸念を持っている。
さらに、コンドームなどのバリア法を使用しても感染する性感染症も存在する。

一部の論者は、カジュアルセックスは心理的な悪影響をもたらすと主張している。
ある研究者らは、「カジュアルセックスは不安や抑うつなどの心理的苦痛、さらには低い自尊心や人生の満足度の低下と関連している」と指摘している\citep{napper16:_asses_pers_negat}。
これは、カジュアルな性的な出会いにおいては、人は特別な\ruby{傷つきやすさ}{ヴァルネラビリティ}を感じるためかもしれない。
ジェームズ・ロチャは次のように述べている。

\begin{quote}
 通常のフックアップであっても、参加者の予期せぬ傷つきやすさを浮かび上がらせてしまい、道徳的な応答が求められることがある。
その問題は、簡潔に言えば、参加者がどれほど「フックアップは意味のないセックスだ」と振る舞おうとしても、私たちは人間である以上、セックス行為から感情を完全に切り離すことはおそらくできない、という点にある。
無感情なセックスを常におこなえる人がまったく存在しないとは言えないが、そうした人は稀である可能性が高く、典型的なフックアップの中でさまざまな感情が生じる人は多い。
ほとんどの人にとって、少なくともいくつかのフックアップのなりゆきではさまざまな感情に襲われるものだ。
\citep[Chapter 7]{rocha19}
\end{quote}

カジュアルセックスの心理的悪影響について考えられるまた別の理由は、カジュアルな出会いが、同意のないセックス、完全には同意されていないセックス、あるいは単に満足のいかないセックスなどにつながりやすいという点だ。
リア・フェスラーは次のように述べている。
「キャンパスにおけるレイプは、フックアップ文化の一部として生じるものではないでしょうか{\DDASH}つまり、より流動的になりつつある「性的文化のスペクトラム」の極端で不穏な側面なのではないでしょうか。
この問題を議論することは、レイプ、性的暴行、望まないセックスを減らす努力に役立つでしょう」(Friedersdorf 2016での引用)\nocite{friedersdorf16:_how_does_hook}。
カジュアルな出会いにおいては、禁止の境界線を設定することがより難しく、それが無視されることも多い。
フェスラーは次のように述べる。

\begin{quote}
 私は、性的暴行とカジュアルなセックスの区別に苦しんだ複数の女性たちを知っていて、実際にインタビューもおこなってきました。
そうした曖昧な境界線が生み出す深い痛み、混乱、そして損傷については、私自身が身をもって証言できます。
また、犯罪が通報されない無数の理由のなかには、「それは性的暴行だったのか、レイプ(非常に複雑な意味を帯びている語)だったのか、それともそうではなかったのか」という不確かさがあります。
ここで「同意」は非常に重要な役割を果たしています。
多くのサバイバーたちは、性的な活動の一部あるいは大部分には同意していたけれども、自分に対してなされたすべての身体的行為に同意していたわけではないような場合には、自分をレイプした相手をレイピストとして告発するためにさえ苦しまねばならないのです。
\citep{friedersdorf16:_how_does_hook}
\end{quote}

複数の調査によると、多くの人々がカジュアルな性的接触の後に「望んでいたよりも先に進まれた」と感じている。
ある研究では、女性の約5分の1がこの経験をしたと報告しており、別の研究では、大学生の16\%がフックアップの際に\ruby{圧力}{プレッシャー}を感じたと述べている(Lambert et al., 2002, p.130; Paul et al., 2000, p.81)。
\nocite{lambert02:_plur_ignor_hook_up}\nocite{paul00:_hook}
また、たとえセックスが完全に同意の上でおこなわれたとしても、\ruby{継続的交際関係}{リレーションシップ}の中でのセックスほど満足のいくものではないかもしれない。
カジュアルな相手は私たちのことをよく知らないか、まったく知らない場合が多く、また二度と会わない可能性が高いため、相手の満足を気にかける動機も弱くなる。
そのため、カジュアルセックスでは女性がオーガズムに達する確率が恋愛関係の中でのセックスよりもはるかに低いというデータがある\citep{armstrong10:_orgas_coll_hook_relat}。

哲学者の中には、カジュアルセックスが別の種類の害をもたらすと主張する者もいる。
すなわち、それは私たちの人格を損なうというのだ。
アンスコムは次のように述べる。
「セックスを単なる享楽の手段として認めようとする者は、その代償を払うことになる。
彼らは\ruby{浅薄}{シャロー}になるのだ」\citep{anscombe72contraception}。
この懸念には二つの側面がある。
第一の懸念は、私たちが浅薄になるということだ。
すなわち、カジュアルなセックスを追い求めることによって、私たち自身が\ruby{もっぱら性的な存在に}{セクシュアライズ}されてしまうという懸念だ。
刹那的な肉体的関係を絶えず求めることによって、セックスにセックスが本来値するもの以上の重要性を与えてしまうのだ。
その結果、私たちは人生におけるより繊細で奥行きのある喜びに費やす時間を失い、それらへの関心も薄れてしまう。

第二に、カジュアルにセックスをすることによって、愛情に基づく継続的関係を築く能力が損なわれてしまう可能性がある。
ロッド・ドレーアは「Tinder時代のモダン「ラブ」」(ここでの「ラブ」のカギ括弧は、単にカジュアルな関係を追い求めることを意味する)について、「この「モダン・ラブ」は、本当の愛をほとんど不可能にしてしまう。
こんなもので誰が幸せになれるだろうか?」と述べている\citep{dreher15:_moder_love_age_tind}。
もっとも、ドレーアはそれがいかにしてそうなるのかを正確には説明していない。
ただし、カジュアルなセックスに反対する立場の者たちは、しばしば食べ物の比喩に訴える。
クリスチャン・クリスチャンソンはこう述べる。
「愛は人間の生における基本的価値の一つであり{\DDASH}おそらく最も基本的な価値だ。
私たちは、それを、私たちの「愛の味蕾」(love-buds)を過剰に刺激することによって危うくしたいのだろうか?」\citep[p.105]{krisjansson98:_casual_sex_revis}。
この見解によれば、カジュアルなセックスは、刹那的な刺激に対する一種の中毒を引き起こすか、あるいは、長期的関係におけるより繊細で獲得に手間のかかる満足を育むために必要な忍耐を、人々から奪ってしまうというのだ。
クリスチャンソンはこの見解を「キノコ化論法」と呼んでいる。

\begin{quote}
  キノコ採りをする人は、同じ種類であって食べられさえすれば、大きさや見た目がだいたい同じである限り、それぞれのキノコを区別しない。
個々のキノコの独自性など、彼にとってはまったく重要ではない……。
さて、彼がベッドを共にする相手もキノコと同じように無差別に選ぶとすれば、彼にとって「特定の人物P」とのセックスも、他のどんな{\DDASH}失礼、「キノコ」ではなく{\DDASH}人物とのセックスとも区別のないものとなる。
そしてその結果として、他者との深く真剣な性的関係を築く可能性を、自ら放棄することになるのだ。
\citep[p.101]{krisjansson98:_casual_sex_revis}
\end{quote}

もっとも、これはあまり適切な比喩ではないかもしれない。
というのも、実際のところキノコ狩りをする人々はかなり選別に慎重だからだ。
間違った種類のキノコを選べば、命に関わる可能性すらあるからだ。
しかし、この点はひとまず脇に置いてよい。
肝心なのは、カジュアルなセックスが、少なくとも長期的には、真の親密さを育む能力を妨げる効果をもつかもしれないということだ。

\subsection{カジュアルセックスの擁護論}

リベラルな立場からのカジュアルセックスの擁護の核心には、自律(autonomy)への訴えがある。
この立場の支持者は、カジュアルセックスが成人によって自由意思のもとに合意された行為である以上、それを道徳的に非難すべきではないと考える。
ローガン・レヴコフは次のように述べる。
「私は、いかなる人(男性であれ女性であれ)も、セックスをするために恋愛関係を結ぶ必要があるとは思わない。
誰もが、感情的な負担を伴わないフックアップをする権利をもっている」\citep{levkoff13:_penn_grad_resp}。

また、カジュアルセックスは単に許容されるべき行為であるだけでなく、積極的に良いものであり、社会として寛容な態度をとる方が望ましいと考えることもできる。
第一に、多くの人はカジュアルセックスを楽しんでいる。
そうでなければ、彼らはそれに関与していないだろう。
また、それは一時的ではあるが、複数のさまざまな人々と親密なつながりを持つ機会を提供する。
これはそれ自体として価値がある経験になりうる。
ハリー・リベルトは、これを旅行による世界探訪の経験にたとえる。
「人々の間にある性的親密さや経験は、地理や文化が多様であるのと同じくらい多様だ。
恋人との経験や海外旅行は、ともに私たちの人生を変え、新しい視点で世界を見せ、私たち自身についての痛烈な教訓を教えてくれるものだ」\citep[p.414]{liberto17:_prob_sexual_prom}。
また、カジュアルセックスは、特別な絆を共有した友人のネットワークを形成する助けにもなる。

第二に、カジュアルな出会いは、長期的な関係を害するどころか、むしろそれを助ける可能性がある。
もし最終的にモノガミーの関係に入る選択をするとしても、カジュアルセックスの経験がその選択をより十分な情報に基づいたものにし、結果としてモノガミーに対する満足度を高めるかもしれない。
「若いときは遊べ」という古い格言のように、カジュアルな経験を経ておけば、パートナー以外の人々に惹かれる誘惑を感じにくくなる可能性がある。
また、パートナーが過去にさまざまな性的経験をしていることが、お互いをより良い\ruby{恋人}{ラヴァー}にし、関係を強化することにもつながることもありえる。

さらに、継続的関係にコミットする前に相手とセックスをすることは、適切なパートナーを選んだかどうかを判断する助けにもなる。
ジル・フィリポヴィックは、性的な相性が長期的な関係にとって重要であり、セックスに対する健全な態度が不可欠だと主張する。

\begin{quote}
  セックスが非常に多くの人々にとって極めて重要なものであること、そしてほとんどのカップルにとって、性的相性が素晴らしい結婚には不可欠であることを認識できれば、私たちはもっと幸福になれるだろう。
実際にセックスをしなければ、性的相性があるかどうかを知ることはできない。
婚前セックスが汚らわしいとか倒錯的だと主張することは、二人に必要な対話を妨げてしまう。
そして、結婚前にセックスを恥ずべきもの、悪いものと考える世界観は、結婚初夜に突然消え去るわけではない。
\citep{filipovic12:_moral_case_sex_marr}
\end{quote}

さらに、カジュアルセックスは、長期的な関係が破局した後の自尊心や感情的な回復にも役立つ。
セックスセラピストのパトリシア・リッチは、オンラインマガジン\emph{The Daily Beast}に次のように語っている。
「失恋後、人は\ruby{交際関係}{リレーションシップ}の中で味わっていたスキンシップに対して禁断症状のようなものを経験します。
再びスキンシップを味わうことは助けにもなり、癒しにもなります」\citep{shire14:_peop_who_have}。

\subsection{カジュアルセックスの害への反論}

カジュアルセックスに対してリベラルな立場をとる者の中には、「モノ化」の議論の前提そのものを否定する者もいる。
彼らは、セックスが単なる快楽の追求であること自体に、何ら本質的に問題があるとは考えない。
たとえそれが、相手との物語的歴史に基づくものではなく、また相手の個別性を特別に認識するものではなかったとしてもだ。

二人の人間が、単に身体的満足を目的として合意の下でセックスをすることは十分に可能だ。
カジュアルセックスには、たしかに、ヌスバウムが指摘するような問題のある特徴{\DDASH}すなわち、それが完全に利己的であり、相手の人格や自律を否定するという特徴{\DDASH}をもつものもあるかもしれない。
しかし、カジュアルセックスの擁護者は、それらがすべてのカジュアルセックスに必然的に当てはまるわけではないと主張する。
むしろ、カジュアルセックスは、相互的な快楽を目的としておこなわれることができるものであり、そのような場合には道徳的にまったく問題はないとされる。
アラン・ゴールドマンは次のように述べている。

\begin{quote}
〔カジュアルセックスが〕その本性からして相手を「モノ化する」行為であるとしても、自分自身も相手の欲望に応じ、一つの性的なモノとなり相手に快楽を与えることによって、あるいは自分たちの行為の快楽が相互的なものであることを確保することによって、人はパートナーを欲求と欲望をもった主体として認めている。
\citep[p.283]{goldman77:_plain_sex}\ig{Alan Goldman}
\end{quote}

しかし、カジュアルセックスの最も問題のある事例{\DDASH}つまり、一方または双方が自己の快楽のために相手の利益を完全に無視するようなケース{\DDASH}によって、すべてのカジュアルセックスを判断して非難すべきではない。
それゆえ、カジュアルセックスが本質的にモノ化的であるという主張は退けられるべきだ。
むしろ他の社会的相互作用と同様に、相手をモノ化したり、不適切に扱ったりしないように注意すべきだということになる。

また、すでに述べたように、多くの人が、よく知っている相手とカジュアルなセックスをするという点にも注目すべきだ。
たとえば「セックスフレンド」としての関係の一環としておこなわれる場合などがそうだ。
このような場合、そうしたセックスは、両者の間に存在する「物語的歴史」の一部をなしていることもありえる。
したがって、それはヌスバウムが提起するような懸念には必ずしも当てはまらない可能性がある。

カジュアルセックスの擁護者の多くは、同じように、「搾取」という議論の前提そのものも否定する。
少なくとも通常の状況において、相互に合意されたセックスが(少なくとも哲学者たちが定義する意味で)搾取に当たるとは認めない。
カジュアルセックスは、当事者が自由に合意するものだ。
参加者は拒否することもできるし、他の関係の選択肢も十分に開かれている。
相手の意図が明確になるまでセックスを待つことも可能だ。
どちらの当事者も、明白な形で不利益を被るわけではなく、また、より利益を得られたはずの別の選択肢があったわけでもない。
カジュアルなセックスを断ったとしても、通常、それによって誰かが当然得るべき何かを失うわけではない。

リベラルな立場は、無神経さを正当化するものではない。
この立場の支持者は、セックスが傷つきやすさを生じさせる可能性があることを認めるし、多くの人がセックスに感情的な意義を持たせていることも否定しない。
また私たちはセックスに対してはスティグマを押しつけているため、コミュニケーションが困難になることもある。
特にカジュアルな性的接触では、相手の意図はしばしば口に出されないままだ。
そのため、パートナーたちは意図を明確に伝え、誤解を生じさせないようにする責任を負う。
これらの点は、リベラルな立場が本質的に誤っていることを意味するものではない。
それは単に、カジュアルセックスにはリスクが存在し、それを適切に管理する必要があることを示しているにすぎない。
エリザベス・ブルーニグは次のように述べている。

\begin{quote}
  セックスは極めて親密で個人的な領域であり、他の社会的状況よりも大きな害が生じる可能性がある。
この高まった危険性を考えれば、人々がこの領域において、他の状況以上に慎重さや配慮、思いやりをもって行動することを期待するのは当然だ。
\citep{bruenig18:_aziz_ansar_debac}

\end{quote}

同じように、カジュアルセックスの擁護者は、健康と安全に関するリスクを認めた上で、それに対処するための最良の手段が道徳的な禁止であるとは考えない。
むしろ、これらの問題は教育と意識啓発によって最も効果的に対処されるべきだ。
包括的な性教育を提供し、性的健康にかかわる各種サービスをすべての人が利用できるようにすることが求められる。
リベラルな立場の支持者は、カジュアルセックスを非難する道徳的規範がむしろ事態を悪化させてきたと主張する。
たとえば、禁欲オンリー性教育は逆効果だ。
データによれば、禁欲オンリー性教育は若者の間で性感染症や望まない妊娠の割合をむしろ増加させる。

カジュアルセックスにスティグマを与えるのではなく、むしろすべての人がセックスについて自由に議論できる環境を整え、正確な情報にアクセスできるようにするべきだ。
意識を高め、オープンなコミュニケーションを促進することで、人々がコンドームやその他の避妊手段を適切に使用し、パートナーとそれについて話し合うことができるようになる。
また、「同意」に関する曖昧な境界線にも対処できる。
これは、カジュアルな性的接触において望まないセックスが深刻な問題ではないと言っているのではない。
カジュアルセックスの擁護者も、その問題を認めている。
しかし彼らは、解決策は「同意カルチャー」を促進することにあると言うだろう。
すべての人が、同意を得ることと境界を尊重することの重要性を理解する文化を築くことが必要なのだ。
カジュアルセックスの一律禁止は、この問題の解決には何の役にも立たない。
(同意については、第3章でさらに詳しく議論する。)

カジュアルセックスが完全に合意の下でおこなわれた場合でも、それが心理的な害をもたらす可能性があるという議論は、経験的・実証的な問題だ。
既存のデータを検証した研究の一つでは、カジュアルセックスが本質的に心理的なダメージを与えるという証拠は見つからなかった。
それが苦痛を伴うことがあるのは、そうした行為に対する禁忌意識が強い人々にとってであり、これは驚くべきことではない。
この研究者たちは次のように述べている。
「行動と欲望の間の不一致、とりわけ社会的・性的関係に関して生じる不一致は、身体的および精神的健康に劇的な影響を及ぼす」\citep{garcia12:_sexual_hookup_cultur}。
しかしとりあえずのところ、ロバート・ワイス\ig{Robert Weiss}の結論が妥当であるように思われる。
「カジュアルな性的関係の心理的影響に関する研究はまだ初期段階にあり、科学者たちはようやくその表面を引っ掻き始めたばかりだ。
カジュアルセックスが人の心理的健康にどのような影響を与えるのか、また与えないのかを真に理解するには、まだ長い時間が必要だ」\citep{weiss15:_what_are_psyc}。

\ruby{人の性格}{キャラクター}に関する議論は、批判的に評価するのが難しい。
それを論駁する証拠がどんなものになるかということ自体がわかりにくいからだ。
カジュアルセックスに対する寛容な態度は、歴史的に見ても比較的新しい現象であり、男女双方が現在のように自由にセックスをし、それを公然とおこなうことが許されている社会の事例はほとんどない。
したがって、その影響を評価できる前例は限られている。
カジュアルセックスが広くおこなわれているという事実は、一つの安心材料にはなりうる。
すでに述べたように、現在では多くの人が人生のどこかでカジュアルセックスを経験している。
リベラルな立場からすれば、これまでのところその結果は自信づけになるものだ。
セックス革命以降、カジュアルな出会いが増えたにもかかわらず、人々がより浅薄になったという証拠はなく、また、感情的に悪影響を受けたり、深く持続的な愛を築く能力が損なわれたりしているという証拠もほとんどない。
大規模な精神衛生の危機が生じていると示唆するものはない。
これらの事実は、カジュアルセックスがほとんどの人にとって、心理的健康や人格に深刻な害を及ぼすものではないことを示唆している。

カジュアルセックスに対する寛容さが、特に男性が女性を搾取することを助長するという意見に対して、ハンナ・ローシンはまったく逆の主張をする。
彼女によれば、カジュアルセックスはむしろ女性にとって有利に働いている。
それによって、女性は恋愛関係に縛られることなく、教育やキャリアに集中することができるからだ。
彼女は次のように述べる。

\begin{quote}
  率直に言えば……現時点でのフェミニズムの前進は、フックアップ文化の存在に大きく依存している。
しかも、驚くべきことに、この文化を推進しているのは、男性ではなく、むしろ女性だ。
特に学校では女性たちは巧みにこの文化を操っており、自分の成功のための空間を確保し、自分の目的を常に念頭に置いている。
今日の女子学生にとって、過度に真剣な求婚者は、19世紀における予期せぬ妊娠と同じ役割を果たしている{\DDASH}有望な未来を阻害する危険なものとして、あらゆる手を尽くして避けるべき対象なのだ。
\citep{rosin12:_boys_side}
\end{quote}

ローシンの見解、すなわちカジュアルセックスが女性の社会的進歩にとって不可欠だという主張を受け入れるかどうかは別として、進化心理学の一部の研究者が主張する「女性は本能的にカジュアルセックスを嫌うようプログラムされている」という従来の見解を退ける根拠は十分にある。
カジュアルセックスを受け入れ、それを楽しんでいる女性は多く存在するため、単純なジェンダー分析による説明は説得力を持ちにくい。

\subsection{本節のまとめ}

多くの人々がセックスの特殊意義説を受け入れているという社会学的事実は依然として存在する。
これは、カジュアルセックスを道徳的に誤っていると考える人の割合が依然として高いことからも明らかだ。
しかし、この見解が人間社会における普遍的な真理を表しているという主張を裏付ける証拠はほとんどない。
セックスの特殊意義説を受け入れる人々は、それを自分自身の関係に適用する権利をもっている。
しかし、この見解を共有しない他者に対して、自らの道徳的判断を押し付ける正当な根拠を見出すのは困難だ。

これは、カジュアルセックスが道徳的考慮から完全に自由であることを意味するものではない。
本論では、リベラルな立場の支持者も、カジュアルな出会いに伴う特有の脆弱性を認識すべきであることを示唆した。
これには、意思疎通での食い違いや、それぞれが期待しているものが異なっているリスク、さらには、\ruby{圧力}{プレッシャー}を感じて当人が望んでいる以上に踏み込まされるというリスクが含まれる。
したがって、カジュアルセックスに対しては、慎重かつ配慮をもってアプローチされるべきだ。

\section{インターネットデートの倫理}

インターネット・デート(ネットデート)は、現在の独身者の生活の一部となっている。
アメリカの成人の30\%が、これまでにマッチングアプリを使用したことがあると答えており、LGBTQ+の人々に限るとその割合は55\%に達する。
そして、多くのユーザーは全体的にポジティブな経験をしたと報告している。
アメリカ人の12\%は、オンラインで出会った相手と結婚するか、真剣な関係を築いたと述べている\citep{anderson20:_virt_down_onlin_datin}。
\ig{Monica Anderson}

しかし同時に、人々はネットデートに完全に満足しているわけではない。
Pew Researchの報告によれば、「過去1年間にマッチングサイトやアプリを使用したアメリカ人のうち、経験を通じて希望を持った(28\%)と答えた人よりも、フラストレーションを感じた(45\%)と答えた人の方が大幅に多かった」とされる\citep{anderson20:_virt_down_onlin_datin}。
このデータは一見すると矛盾しているように思われる。
人々はネットデートに満足しながらも、不満を感じている。
しかし、実際には、これらの統計は私たちがこのシステムに対して抱く複雑な感情を反映している。
私たちは、ネットデートが提供する出会いの機会を評価しながらも、それが必然的にもたらす困難に苦しんでいる。

ネットデートは明らかに消えることはない。
しかし、私たちはそれに関して倫理的な選択をすることができる。
第一に、私たちはまずそもそもマッチングアプリを使用するかどうかを決めなければならない。
第二に、アプリをどのように使用し、マッチした相手をどのように扱うかについての判断が求められる。
第三に、マッチングアプリの設計自体が改善されるべきかどうかを問うこともできる。
この点を考える際、ネットデートは肉食にたとえることができる。
大多数の人々は肉を食べるが、どれくらいの量を食べるのか、どこで買うのかといった選択をしている。
そして、一部の人々は菜食主義者として、少数派であるにもかかわらず、肉を食べないという明確な選択をしている。
一方で、多くの人々が畜産をより倫理的なものにするための改革を求めている。
同様に、ネットデートに関しても、使用するかどうかの選択、使用方法の選択、そしてシステム全体の倫理的改善に関する議論が可能だ。

\subsection{ネットデートの利点}

マッチングアプリには数々の利点がある。
アプリは、多くの人にとってデートすることをずっと簡単にした{\DDASH}そうでなければこれほど多くの人々が利用するはずがない。
パートナーを見つけることは容易ではない。
ネットデートが登場する以前、出会いの場は、バーでの長く憂鬱な夜、パーティーでの気まずい会話の連続、あるいは高額で効果の疑わしい結婚仲介サービスなどに限られていた。
おそらく著者たちの世代にとって最も過酷だったのは、出会いを求めてアルティメット(フリスビー競技)のリーグ戦に参加しなければならなかったことだろう。
特に、出会いの機会が限られている人々{\DDASH}たとえば、内向的な人々、社会的に孤立している人々、または田舎や小さな町に住んでいる人々{\DDASH}にとって、オフラインでの出会いはたいへん困難だ。

ネットデートは、経済学者が「厚い市場」(thick markets)と呼ぶものを生み出す。
これは、消費者が多数の選択肢を持つ市場のことだ。
一般的に厚い市場はより良い結果をもたらすとされる。
この傾向は、特にデートにおいて顕著だ。
なぜなら、人々はパートナーを選ぶ際に非常に選り好みする傾向があるからだ。
私たちは、自分と相性が良く、共通点のある相手を求めるが、そのような相手は全体のごく一部にすぎない。
選択肢が増えれば増えるほど、相性の良い相手を見つけられる可能性が高くなる。
マリナ・アドシェイドは次のように述べている。

\begin{quote}
  オンラインマッチングサイトは、まさに厚い市場だ。
私はこの市場における「取引価格」を、売り手としても買い手としても、自分にとって最適な相手と出会う可能性として解釈する。
この市場では、私自身が相手にとって最適なパートナーであると同時に、相手も私にとって最適であるという関係を見つけることが可能なのだ。
オンラインでの恋愛探しが簡単だからではなく(実際多くの点で簡単ではない)、厚い市場ではより質の高い関係を築ける可能性が高いからこそ、私はオンラインデートを利用するのだ。
\citep[p.8]{adshade13:_dollar_sex}

\end{quote}

マッチングアプリは、出会いの機会を増やすだけでなく、その多様性を広げることにも寄与する。
ある調査では、回答者の3分の2が、ネットデートサービスによって自分とは異なる人種や民族の相手と出会うことが容易になったと報告している\citep{hergovich17:_stren_absen_ties}。
これは個人的な利点にとどまらず、社会全体にとっても有益であり、異なる社会集団間の障壁を取り払う役割を果たしている。

また、マッチングアプリは、特定の性的嗜好を持つ人々や性的マイノリティの人々が、同じ関心を持つ相手をより簡単に見つける手助けをしている。
これにより、彼らの孤立感や周縁化が軽減されることになる。
ゲイやレズビアンの人々が、オンラインデートをいち早く積極的に取り入れたのは偶然ではない。
Pew Researchの調査では、アメリカのゲイとレズビアンの55\%がマッチングアプリを使用したことがあり、21\%がそれを通じて長期的なパートナーを見つけたと報告されている。
この割合は、全体の人口と比較してはるかに高い。
さらに、マッチングアプリは、特定のタイプのパートナーを求める人々にも新たな機会を提供している。
たとえば、Feeld、Fetlife、\#openは、ノンモノガミー(非モノガミー)を実践する人々を対象としている。
また、FarmersOnly、LDS Singles、Star Trek Datingといった、非常に専門的なデートサービスも登場している。

\subsection{ネットデートに関する懸念}

ネットデートには、いくつかの懸念すべき点がある。
オンラインマッチングアプリがパートナーを見つける主要な手段となるにつれ、外見のような表面的な基準にこれまで以上に重きを置く傾向が生じている。
ミア・レヴィタンは、「スワイプ機能によって、私たちは皆、外見だけで瞬時に判断する10代の若者のようになってしまった」と述べる。
彼女は、男性がデートプロフィールに特定の言葉を含めることでマッチング率がどの程度向上するかを調査した研究を引用し、「ひきしまった体型」(physically fit)と記載すると返信率が96\%向上し、最も有利な単語は「6ft (183cm)」であったと報告している\citep{levitan20:_futur_seduc_lond,dodgson18:_usin_thes_word}。

また、オンラインデートでは、出会う相手が自分の直接的な社会的ネットワークの外にいるため、相手に対して悪い振る舞いをすることが容易になる。
多くのユーザーがプロフィール上で虚偽の情報を記載し、また「ゴースティング」(ghosting){\DDASH}突然、何の説明もなく連絡を絶つこと{\DDASH}が蔓延しているという不満を抱いている。
若者の約5分の1が、恋愛関係にあった相手からゴースティングされた経験があると報告している\citep{yougov14:_poll_resul,navarro20:_psyc_corr_ghos_bread_exper}。
さらに、オンラインでマッチングしたものの実際にはまだ会っていない相手からゴースティングされた人を含めれば、その割合はさらに高くなるだろう。
これは、相手が自分の社会的ネットワークと無関係であるため、人々が無神経な行動をとりやすくなるためだと推測される。

こうした振る舞いは、単に無神経であるだけでなく、虐待的または暴力的になることもある。
Pew Researchの調査によると、かなりの数のユーザーが、マッチングアプリで相手に興味がないと伝えたにもかかわらず連絡を取り続けられた(37\%)、求めていない性的なメッセージや画像を送られた(35\%)、侮辱的な呼び方をされた(28\%)、身体的危害を加えると脅迫された(9\%)と回答している\citep{anderson20:_virt_down_onlin_datin}。
\ig{Monica Anderson}
オンラインデートがオフラインでの出会いよりも暴力のリスクを高めるかどうかは不明だが、イギリスの国家犯罪対策庁(NCA)は、オンラインデートの急増が「新しいタイプの性犯罪者」の出現を促していると警告している\citep{doria20:_are_datin_apps_safe_unsaf}。

ユーザーの振る舞いに加えて、アプリ自体の設計にも懸念すべき理由がある。
第一に、これらのアプリは少数の企業によって所有されており、その運営方法の多くは秘匿されている。
その結果、私たちは、十分な透明性がなく、ユーザーに対して直接的な説明責任も負っていない企業に、自らの恋愛生活のコントロールを事実上委ねてしまっている。
プライバシーやセキュリティに関する懸念はテック業界全体に広く見られるが、インターネット上の出会いに関しては、こうした懸念はとりわけ深刻だ。
というのも、これらのアプリは、きわめて個人的な情報の提供を私たちに求めるからだ。
そこには、性的指向のように、差別の対象となったり、法域によっては刑事訴追のリスクさえ生じかねない側面も含まれている。

さらに、私たちが人々と出会うためにマッチングアプリに依存してしまっていることからして、誰がマッチングアプリにアクセスできるかどうかという問題も重要だ。
これらのアプリは、どんな理由からであっても恣意的にユーザーを排除する権限を持っている。
つまり、どんなカテゴリーの人でも排除されうるということだ。
たとえば、Eharmonyは、同性愛者が同性パートナーを探すことを認めていない。
さらに、マッチングアプリはとても透明性があるとはいえない基準で、ユーザーを\ruby{排除}{バン}している。
たとえばTinderは、多くのトランスジェンダーのユーザーを排除しているが、これはおそらくトランスフォビックなユーザーがそうした人々のプロフィールを通報したためと考えられる\citep{tierney17:_why_are_tran}。
同様に、セックスワーカーたちも、個人的なデート目的でアプリを利用していたにもかかわらず、アカウントを削除された事例がある\citep{al-othman18:_sex_work_say}。

また、人々を直接に排除するという点を離れても、アプリの設計自体が、特定の人々にとってパートナーを見つけにくくする要因になっている。
多くのアプリは、異性愛者/同性愛者という二分法の区別に基づいており、\ruby{両性愛}{バイセクシュアル}や\ruby{全性愛}{パンセクシュアル}のユーザーにとっては使いにくいものとなっている。
TechHub社の創業者エリザベス・ヴァーリーは、「一部のアプリは、バイセクシュアルやパンセクシュアルという概念が存在すること自体を忘れているようだ。
最も大きな問題は、選択肢が二分法であることだ」と指摘している。
女性向けテック系メディア \emph{Gadgette} の編集者ホリー・ブロックマンは同意する。

\begin{quote}
  パンセクシュアルな人々は、進歩的なアプリでさえもしばしば排除されるか、同一ではないバイセクシュアルとして登録を強制される場合があります。
さらに、トランスセクシュアル、アセクシュアル、インターセックスの人々や、彼らが誰に表示されるかという点にも問題があります。
理想的には、すべてのアプリは、ユーザー自身の性的指向および性自認と、出会いたい相手の性的指向および性自認の\kenten{両方を}尋ねるべきです。
\citep{knowles16:_bisex_prob}
\end{quote}

マッチングアプリは、既存の恋愛市場の不平等をさらに助長する仕組みになっている。
最も人気のあるユーザー(スワイプ数が最も多いユーザー)は、より多くの相手に表示されるようになっており、結果として、最も魅力的なユーザーが圧倒的に多くのマッチングとメッセージを受けとることになる\citep{carr16:_i_foun_out}。
この不平等は、経済的不平等を測定するジニ係数を用いて分析されたこともある\footnote{このように測定すると、Hinge の女性向け「エコノミー」はジニ係数0.376であり、おおよそロシアとカンボジアの間に位置する。
男性向けのジニ係数は0.542であり、これは世界で最も不平等な国の上位10ヵ国に匹敵し、ブラジルやモザンビークと同等だ。cf. \citet{kopf17:_these_statis_show_why_its}.}。
アプリは年齢に関する偏見も強化しており、特に年齢の高い女性にとってマッチングの機会を減少させている\citep{schrobsdorff21:_paulin_poriz}。
クレメント・ノックスは、現在の状況を次のように要約している。
「経済的不平等が拡大するなか、何十万人もの人々が、日常生活の中で直面している以上に不平等な恋愛市場を提供するオンラインプラットフォームで愛を求めている」\citep[p.408]{clement20:_seduc}。

アプリのこの機能は、見た目が魅力的なユーザーに有利であるだけでなく、魅力度という点では最大公約数的な状態を助長する。
ユーザーに与えられている選択肢は、見込みのある相手に対する「いいね」か「よくないね」かという二者択一のみであり(しかも、その強度を表現する手段は、高額でやや使いにくい「スーパーいいね」を除けば存在しない)、その結果、最も多くのマッチを得るのは、強い印象を与える一方で評価が大きく分かれてしまう人よりも、大多数の人に「そこそこ魅力的」と思われる人々になる。
これは、機会が与えられれば、人々はより極端な反応を引き起こすユーザーにメッセージを送る傾向があるという証拠があるにもかかわらずのことだ\citep{rudder11:_mathem_beaut}。
実際に誰かと対面での関係を築くためには、通常かなり多くのマッチが必要となるため、アプリは、私たちに対して、大多数の人々に無難に魅力的だと思われるような自己演出を促すことになる。
その結果、自分のよりマイナーな関心に共鳴してくれるような相手を惹きつけうる側面よりも、万人受けする側面を優先して示すよう動機づけられてしまう。

マッチングアプリは、異なる人種的・文化的背景を持つパートナーとの出会いを助ける一方、差別のフォーラムにもなりえる。
出会いにおける差別という一般的な問題については、本書2.4節で論じる。
そこでさらに詳しく述べるように、人々が受ける注目度は民族的・人種的背景によって大きく異なる。
こうした人種差別的な態度はオフラインで蔓延しているものを反映しているが、オンライン環境の匿名性は多くの場合それを悪化させる。
たとえば、Grindrなど一部のアプリは、ユーザーの公開プロフィールにおける人種差別的および侮辱的な言語を容認しており、「アジア人不可」「\ruby{太め}{ファッティー}不可」「\ruby{トランス}{トラニー}不可」といった宣言がよく見受けられる。
これは明らかに問題だ。
クリス・ストケル・ウォーカーは、「現実の世界では「黒人不可、アイルランド人不可」というサインはもはや社会的に許されない。
それなら、私たちの出会いの大部分を占めるプラットフォーム上でなぜそのような表現を許容するのだろうか……」と述べる\citep{stokel-walker18:_why_is_it_ok_onlin}。

テクノロジーはオンラインコミュニティの創出を可能にする一方で、現実のコミュニティに有害な影響を与えることもある。
マイケル・ホッブズは、技術利用の結果としてゲイバーやその他の社交空間が衰退したことに一部起因するとする、いわゆる「ゲイの孤独の蔓延」を記録している。
彼はアダムという名の男性の言葉を記録している。
\begin{quote}
  Grindrでフックアップの相手を見つける方が、一人でバーに行くよりもはるかに簡単です。
特に新しい街に引っ越してきたばかりの場合、マッチングアプリは簡単にあなたの社交生活の場となってしまいます。
社交の場を探すのはもっとたいへんです。
そこではもっと苦労しなければなりません。
\citep{hobbes17:_toget_alone}
\end{quote}

統計によれば、長らくLGBTQ+コミュニティの要として機能してきたゲイバーの数は著しく減少している。
ある研究では、2007年以降、米国におけるゲイバーの数は37%減少したとされる\citep{mattson19:_are_gay_bars_closin}。
この研究の著者は\emph{The New York Times}に、レズビアンバーの数は2007年以降半数以上、1986年以降では90%以上減少したと述べている\citep{wilson20:_where_did_all_lesbian_bars_go}。
この減少はテクノロジーの影響だけによるものではないが、それが大きな役割を果たしていると推測できる。
ゲイやレズビアンはマッチングテクノロジーの\ruby{早期導入者}{アーリーアダプター}だが、異性愛社会も遅れてはいない。
たとえば、若者がパーティーなどの社交の場に参加する時間の減少を見ると、対面の社会的ネットワークに対するテクノロジーの影響が明らかだ\citep{wayne15:_death_party}。

また、現代のテクノロジー環境には一つの謎がある。
さまざまな方法でつながりを促進するにもかかわらず、全体として人々のセックスの回数は減少しているように見える。
いわゆる「性の不況」については多くの説明が提案されている\citep{julian18:_why_are_young_peopl_havin}。
一部の説では、私たちがテクノロジーに依存する一方で既存のテクノロジーに不満を抱いているとされる。
また、テクノロジー依存が単に選択肢を過剰に提供し、その結果、私たちが下す選択に対する満足度を低下させると主張する者もいる。
バリー・シュワルツは「選択のパラドックス」と呼ばれる現象を指摘する。
すなわち、選択肢が多ければ多いほど、私たちは自らの選択に対して満足しにくくなるというものだ。
これは現代の出会いの世界にも当てはめられ、性行為の減少を説明する一因として考えられている\citep{svoboda16:_probl_moder_roman_is_too_much_choic}。
明らかに複数の要因が作用しているが、こうした要因も影響を及ぼしている可能性があるのは確実だ。

\subsection{本節のまとめ:ネットデートをより倫理的にすることは可能か?}

ネットデートは今後も定着し続けるだろう。
それは広く普及しており、多くの人々が、総じてデートをより容易で充実したものにしたと考えている。
しかし、それでも私たちは、それに関して倫理的な選択をすることができる。
極端な選択肢として、マッチングアプリを一切使用しないことも可能だ。
しかし、これは大きな代償を伴う。
最も顕著なのは、出会いの機会が大幅に減少することだ。
そこまでしないにしても、アプリの使用方法に注意を払い、それが自分自身や社会にどのような影響を与えるかを意識することは可能だ。

また、どのアプリを使用するかを選択することで、消費者としての判断を通じて、企業に社会的責任のあるアプリ設計を求める圧力をかけることができる。
たとえば、ユーザーの偏見を助長するのではなく、それを軽減するような設計変更が提案されている。
マッチングの基準を人種や民族といった人口統計的要素ではなく、個人的な関心や性格に基づいて設定することが可能だ。
興味深い例として、日本のゲイ向けマッチングアプリ{9monsters}がある。
このアプリでは、ユーザーを9種類の架空の「モンスター」のカテゴリーに分類するが、その分類はユーザー自身のタイプの好みと、コミュニティによる評価の両方を考慮したプロセスに基づいておこなわれる\citep{miksche17:_meet_gay_app,hutson18:_debias_desir}。

また、マッチングアプリは、ユーザーの安全を確保するためにさらに多くの対策を講じることができる。
たとえば、Tinderは、「パニックボタン」を導入しており、デート中に危険を感じた場合に警察に通知することが可能だ。
また、 「不快な思いをしましたか?」(``Does This Bother You?'') という機能を備えており、不適切なメッセージを自動検出し、ユーザーが報告するかどうかを尋ねる。
しかし、これらに加えて、他にも施策が考えられる。
アプリは、アカウントが実際に存在する本人によって使用されていることを確認するための対策を強化できる。
企業は、ユーザーからの苦情を追跡し、対応する標準プロトコルを整備することが可能であり、また、データを共有することもできる(特にMatch Groupが主要なデートプラットフォームの大部分を所有していることを考えれば、これは比較的容易だ)。

とはいえ、ネットデートに関する多くの課題は、結局のところ、デートそのものに関する課題でもある。
私たちは、それを理想的な世界の基準で評価するのではなく、他の選択肢と比較して判断する必要がある。
デートには、必然的にストレスやリスクが伴う。
マッチングアプリは、それを悪化させることもあれば、逆に軽減することもある。
ユーザーにとって、そのバランスをとることは容易ではない。
しかし、これらのアプリが存在しない世界に戻りたいと考える人は、けっして多くはないだろう。

マッチングアプリは非常に多様であり、新しいものが頻繁に登場している。
このことは、ユーザーがアプリの使用\ruby{体験}{エクスペリエンス}を向上させる力を持っていることを意味する。
もし私たちが、どのアプリを利用するかについて慎重な選択をおこなうことをいとわないのであれば{\DDASH}たとえ一時的に出会いの機会が減るとしても{\DDASH}それによって、アプリの設計者はユーザーのニーズにより応えるようになり、最終的にはすべてのユーザーにとってより良い体験が提供されることになる。

\section{デートと差別}

2017年8月、ゲイ向けマッチングアプリ{Grindr}は、 ``What the Flip?'' というウェブシリーズの第1話を公開した。
このシリーズでは、2人のユーザーがプロフィールを交換し、お互いの視点からアプリを体験することができる。
第1話では、白人男性とアジア人男性がプロフィールを交換した。
アジア人男性が白人男性のプロフィールを使うと、彼は大量のメッセージを受け取った。
一方、白人男性がアジア人男性のプロフィールを使うと、彼のメッセージはほとんど無視された。
さらに、返信があった場合も、それらは人種的ステレオタイプや露骨な\ruby{けなし言葉}{スラー}を含むものばかりであった\citep{wong17:_watch_what_happen_when_these}。

経験的研究によっても、人種がオンラインデートの行動に影響を与えることが確認されている。
2009年、マッチングサイトOkCupidの共同創設者クリスチャン・ラダーは、「人種がマッチングに及ぼす影響」(``How Race Affects the Messages You Get'')と題するブログ記事を発表した。
彼はOKCupidのアメリカにおける詳細な利用データにもとづき、白人男性が他のどのグループよりも多くのメッセージを受け取り、送信したメッセージへの返信率も高いことを明らかにした。
一方、黒人女性は最も少ない関心しか得られなかった\citep[pp.101-109]{rudder14:_datac}。
ヨーロッパの研究でも、オンラインデートにおいてアフリカ系および中東系の人々は、白人と比べて受けとるメッセージや返信の数が少ないことが示されている\citep[p.332]{potarca15:_racial_prefer_onlin_datin_europ_count}。

オンラインデートは、親密な関係における差別の問題を悪化させる可能性があり、少なくとも、それを数値化しやすくしたことは確かだ。
しかし、この問題を生んだのはオンラインデートではない。
人種は、オンラインであろうとオフラインであろうと、人々のデート相手の決定に影響を与えている。
私たちは職場や教育の場における人種差別を非難し、それを防ぐための法律まで制定している。
しかし、誰とセックスをするか、誰とつきあうかという選択に関しては、一般に個人の好みの問題とされ、道徳的評価の対象外と見なされる傾向がある。

しかし、一部の人々は、デートの相手に関しても差別を容認すべきではないと主張する。
彼らは、デート相手の選好も道徳的評価の対象となりえるし、そうすべきだと考える。
もし私たちが人種を理由に相手を選ばないのであれば、それは道徳的に間違っており、私たちは自らの選好を変える努力をするべきだ、というのだ。
また、人種を基準に相手を選ぶことも、同様に問題があるとされる。

親密な関係における差別に対処すべきだという基本的な論拠は単純だ。
すなわち、人種差別は間違っており、それに対処する必要があるということだ。
もし他の領域で人種に関して中立的であるべきだと私たちが考えるのであれば、デート相手の選択においても同じことが言えるはずだ。
デートにおける差別の影響を受ける人々は、自尊心、社会的地位、そして人生の展望において不利益を被る。
一方で、デート相手の決定は極めて私的なものであり、誰に魅力を感じるかはしばしば自分の意志ではどうしようもないことだ。
こうした理由から、多くの人々は、デート相手に関する決定を道徳的評価の対象から除外すべきだと主張する。
本節では、まずこの立場の議論を検討する。

\subsection{デート相手選択は道徳的評価の対象外であるべきか?}

デートの選好が他者による道徳的評価の対象とならないべきだとする主張には、大きく五つの論拠がある。
第一に、誰とデートするかという選択は、私たちが自己の人生に対して持つべき自律的な決定権の範囲内にあると考えられる。
親密なパートナーの選択は、私たちにとって最も個人的な決断の一つだ。
米国最高裁は\emph{Lawrence v. Texas}の判決において、性的行動を「最も私的な人間の行為」と位置づけ、それゆえ外部の干渉から保護されるべきだと主張した(\emph{Lawrence et al. v. Texas})。
エレン・ウィリスは、「同意に基づく関係において、パートナーが自らの性的嗜好を持つ権利をもっていることはほぼ自明の前提だ」と述べている\footnote{ただしウィリスはさらに次のようにつけ加えている。
  「しかし、ロザリンド・ペッチェスキーの言葉を借りれば、本当に\ruby{根本的}{ラディカル}な運動は、単に選ぶ権利を主張するだけでなく、根本的な問いに注目しなければならない。すなわち、なぜ私たちは私たちが選ぼうとしているものを選ぼうとしているのか? もし本当の意味で選択肢があるならば、私たちは何を選ぶだろうか、と」。
}\citep[p.14]{willis92:_no_more_nice_girls}。

また、パートナーの選択は自己のアイデンティティや自己認識と密接に結びついている。
誰かにデートやセックスを強制されることを不快に感じるのと同様に、私たちは誰とデートしないかを選択する自由ももっていると考えるべきだろう。
フェミニストたちは、特にこの点に関して強い理由を持っている。
彼女たちは、女性が男性に対して性的に応じる義務がないことを確立するために闘ってきた。
また、女性の欲望が正当なものであることを認めさせるために努力してきた。
アミア・スリニヴァサンは次のように述べている。

\begin{quote}
私たちの性的な\ruby{選択}{プレファレンス}を政治的な査問の対象とすることには、たしかに本物のリスクがある。
私たちは、フェミニズムは、私たちの欲望の根拠を問い直すことができるものであってほしいと思っている。
しかし、それがスラットシェイミングやお上品な潔癖主義、あるいは過剰な自己抑制につながるものであってほしくないとも思っている。
つまり、それが、個々の女性に対して、彼女が同意しておこなっていることについて、「あなたは本当は自分が何を望んでいるのかわかっていない」とか、「あなたが実際それを望んでいるとしても、あなたはそれを楽しめているはずがない」と押しつけるようであってはならない。
フェミニストのなかには、そんなふうに欲望の根拠を問うことは不可能だと考える人もいる。
欲望の批判を許せば、それは不可避に権威主義的モラリズムへとつながってしまう、と。
\citep{srinivasan18:_does_anyon_have_right_sex}
\end{quote}

第二に、性的な\ruby{惹かれ}{アトラクション}は理性的なコントロールの及ばないものだと主張できる。
レイチェル・モランは、人種間結婚禁止法の歴史を論じる中で、私たちの欲望は「媒介されず、測定不能であり、言葉では説明できない」と述べている\citep[p.14]{moran01:_inter_intim}。
彼女は、性的欲望には不可解な「Xファクター」があり、「親密さとは、個人的な独自性を肯定する行為だ」と指摘する。
そして、「愛とアイデンティティは理性的判断の及ぶ範囲を超えている」と結論づける(ibid.)。
道徳的評価は、合理的に考えて、別の行動をおこなうこともできるような状況について適用されるべきだ。
しかし、誰に魅力を感じるかは、単なる事実であり、選択の問題ではないとも考えられる。
この点では、個人の人種的背景と同様に、選択の余地がない特性だ。
もし選択の余地のない特性を理由に人を批判することが不当であるならば、性的魅力の対象に基づいて人を批判することもまた不当だ。

第三に、私たちの恋愛対象の選択は複雑な要因によって決まる。
もし誰かが特定の人種の人々を事前に排除しているのであれば、それは問題視しやすいかもしれない。
しかし、実際にはほとんどの人が複数の要素を考慮し、それらが総合的に自分にとっての相手の魅力を決定する。
オンラインデートに関する統計が示すように、人種が大きな影響を与えている可能性はあるものの、個別の判断の中でそれを決定的な要素として特定することは困難だ。

第四に、誰かをデートの対象として選ばないことは、誰にも害を与えていないという主張もありえる。
雇用や教育の差別とは異なり、これは社会的資源の配分には影響を及ぼさない。
マッチングアプリで「左にスワイプする」ことで、その人から何かを奪うわけではなく、単に自分との親密な関係の機会を提供しないだけだ。
これは誰も当然に享受すべき権利ではない。
レベッカ・ソルニットは次のように述べている。
「あなたは、相手が望まない限り、その人とセックスする権利をもたない。
それは、誰かが自分のサンドイッチを共有してくれない限り、あなたにそのサンドイッチを食べる権利がないのと同じだ」\citep{solnit15:_men_explain_lolit_me}。
セックスを拒まれることも、サンドイッチを一口分けてもらえないことも、抑圧の一形態とは見なされない。
それは、いずれの場合も、個人の自由な選択に基づくものであり、誰もそれを要求する権利をもたないからだ。

最後に、一部の人々は、私たちは自らの偏見を克服するという意識的な選択を通じて、誰かに自分とデートするという特権を与えるべきだとする考え方には、\ruby{おせっかいな}{パトロナイジング}態度があるのではないかと懸念を示している。トム・オシェアは次のように述べている。「そこには、次のようなメッセージが潜んでいる危険がある。本来なら、私は、あなたのことを気持ち悪いか、地味か、あるいはぜんぜんセクシーでないと感じていて、とても関心を持つ気にはなれないでしょう。でも、私は、この寛大な義務感から、あなたへの嫌悪感や無関心を克服するよう自分を訓練してきたです、と」。オシェアは、人々は個人として欲望されることを望んでいると指摘しており、したがって、「欲望を公正な不偏不党の原理に基づいて再構築しようとする試みは、自己破壊的になりうる」と述べている。なぜなら、そうした試みによって生まれるものは、多くの人々が他者から性的に欲望されることに見出している価値を十分には満たさない可能性があるからだ\citep{oshea20:_sexual_desir_struc_injus}。。

\subsection{デートバイアスに対する批判}

デートにおける偏見を是正すべきだと考える人々は、人種差別が間違っているという単純な前提から出発する。
彼らは、すべての人が民族的背景や肌の色に関係なく平等に認められるべきだと主張する。
彼らは、パートナーの選択が個人的なものであり、リベラルな社会において自律的な決定として保護されるべきであることを否定しない。
そして、誰かが自分にとって魅力を感じない相手とデートすることを強制すべきだとも考えていない。
彼らが求めているのは、他の個人的な選択に適用されるのと同じ道徳的基準を、デート相手の決定にも適用することだ。

親密な関係における差別に反対する人々は、一つのアナロジーを提示する。
それは、私たちには誰と友人になるかを選ぶ権利があるが、その選択に対して道徳的評価がなされることがあるという点だ。
もし誰かが「私は白人以外の人とはつきあわない」と公言したり、実際にそうした行動を取っているとわかった場合、多くの人はそれを人種差別とみなすだろう。
では、デートの選択については、なぜ異なる基準が適用されるべきなのか?

これに対する反論として、友情の選択とデート相手の選択には重要な違いがあると主張することができる。
すなわち、デート相手の選択は、私たちが理性的に決定できるものではなく、誰に魅力を感じるかは制御不能だからだ。
しかし、デートにおける偏見に反対する人々は、欲望が完全に固定的なものではなく、むしろ可変的であることを指摘する。
二つの独立した研究によって、特に長期的な関係を考慮する際に、人々のパートナーの選択が意外に柔軟であることが示されている(Bleske-Rechek and Ryan, 2015; Gerlach et al., 2019; cf. Cahill, 2016, p.293)。
\nocite{bleske-rechek15:_contin_chang_emerg,gerlach19:_predic_valid_adjus}
\nocite{cahill16:_sexual_desir_inequal_possib_trans}
したがって、私たちは自分の\ruby{惹かれ}{アトラクション}の範囲を広げるよう意識的に努力すべきであり、またそれは可能だ。
ハディヤ・ロデリックは、「黒人としてのデート経験」について論じる中で次のように述べている。
「「好みは説明のつかないものだ」とよく言われる。
身体的な魅力にしてもその他の魅力にしても。
しかし、そもそもデートとは本来、探求的で予測不可能なものであるはずだ」\citep{roderique17:_datin_black}。

私たちの欲望もまた、社会的な文脈や、より一般的な信念によって形づくられている。
親密な関係における差別(intimate discrimination)に反対する人々は、性的な人種差別が、より広範な人種差別的態度と密接に相関していることを示す統計を指摘する。
つまり、驚くべきことではないが、デートにおいて強い人種的な選好を持つ人々は、マイノリティ人種に対してステレオタイプ的で否定的な態度を抱いている傾向がある\citep{callander15:_is_sexual_racis_reall_racis}。

しかし、人は自己反省や教育や経験を通じて、より人種的に開かれた考え方を持つようになることができる。
そして、それに伴ってデート相手の選好も変化しうることがわかるだろう。
私たちは、自分の欲望の基盤を吟味してみることができる。
特定の人種に対して無意識の嫌悪を抱いているかもしれないが、それは自己反省を通じて薄れていく可能性がある。
また、特定のステレオタイプを持っている場合、それが合理的ではないことを認識すれば、それを克服することもできる。

親密な差別に反対する者は、私たちの惹かれが複雑なものであることを認める。
しかし、まさにこの理由から、私たちは外見に特化して注目する必要は少なくなるはずだ。
ほとんどの人は、誰かと出会い、初めはその人に魅かれなかったが、知り合ううちに実際にはその人に興味をもつという経験をしている。

さらに、デートにおける偏見を批判する人々は、デート相手の選択が誰にも害を与えないという主張にも異議を唱える。
確かに、特定の個人が誰かに選ばれなかったとしても、それによって彼らの権利が侵害されたとは言えない。
しかし、デート市場全体において人種的偏見が蓄積されると、かなりの程度の害が生じる\citep{lopez19:_sexual_racis_is}。

デート相手の選択によって不利益を被る人々は、自己評価(self-esteem)に悪影響を受ける。
また、社会的な移動の機会が減少することによって、経済的不平等が拡大する。
社会学者が「同類配偶」(assortative mating)と呼ぶ現象は、同じ人種・階級の人々が互いに交際する傾向を指すが、これは社会全体の経済的不平等を悪化させる要因となる\citep{milanovic19:_rich_like_me}。

これらの主張から、私たちに特定の人種の人々とデートする義務があるというわけではない。
誰かが異なる人種に属しているからといって、他者がその人とデートすべきだと主張することはできない。
そのような主張をする人々はもちろんいない。
デートにおける差別を批判する人々が求めているのは、私たちが自らのデート相手の選択について省察し、偏見に気をつけ、できる限り開かれた態度をとることだ。

\subsection{肯定的なバイアスも批判されるべき差別か}

場合によっては、特定の人種のパートナーを避けるのではなく、むしろ積極的に求めることもある。
このような肯定的なバイアスは、否定的なものと同様の道徳的懸念は引き起こさないと主張したくなるかもしれない。
もし、親密な関係における差別が問題となるのはそれによって特定の集団の人々が過小評価され、恋愛の機会を奪われるからであるならば、人種的な好意的選好にはそのような問題が当てはまらないように思われる。

デートにおける最も顕著な肯定的バイアスの一例は、白人男性がアジア系女性を好む傾向だ。
ロビン・ゼンは、「イエロー・フィーバー」と呼ばれるこの現象が、否定的バイアスと同様に問題だと論じている。
彼女は、これがアジア系女性に対して「非個人的」または「画一的」な扱いをもたらすと主張する。
これはマーサ・ヌスバウムが「交換可能性」と呼ぶモノ化の一形態であり、すなわち、彼女たちは同じ人種の他のメンバーと交換可能な存在として扱われる。
そのため、彼女たちは自分が個人として愛されるのかどうかを疑うようになる。
ゼンは、\emph{OC Weekly} に寄せられたあるアジア系女性の証言を引用する。
「私はいつも、自分が交換可能な存在なのではないかと考えてしまいます」(Zheng, 2016, p.407; Chang, 2016を引用している)\nocite{chang06:_yellow_fever}。

また、ゼンは、アジア系女性が白人男性からの関心によって「異質な存在」として扱われることを指摘する。
彼女はある女性の言葉を引用している。
「私は自分自身や自分の外見を褒められていると感じたことがありません。
ただ単に、エキゾチックに見えるアジア系女性であるがゆえに褒められているのだと感じます。
私は、外見に関係のない褒め言葉であっても、信頼することができなくなっています」(ibid.)。
ゼンは、白人男性のアジア系女性に対する選好が、より広範な構造的な人種差別に寄与していると指摘する。
すなわち、アジア系女性を一つの集団として誤って表象することに加担しているのだ(ibid., p.408; Chan, 1988を引用している)。
\nocite{chan88:_asian_americ_women}
ゼンは、「イエロー・フィーバー」はアジア系女性を受動的で性的な存在とみなす認識を強化し、それ自体がそのような偏見を助長する原因となっていると主張する。
ゼンは、アジア系女性を好んでデートする男性を対象とした研究を引用し、その研究の著者が述べた次の観察を紹介している。
「ほぼすべてのインタビュー対象者は、最初に「アジア系女性が従順だというわけではない」という否定の文言を述べる。
しかし、それにもかかわらず、彼らは皆、何らかの形でアジア系女性が従順だと発言している」\citep[p.410]{zheng16:_why_yellow_fever_isnt_flatt}。
ゼンは、「イエロー・フィーバー」が助長する態度が、アジア系女性の生活に具体的な影響を与えると指摘している。
「アジア系女性に関するステレオタイプは、彼女たちを性的嫌がらせや暴力の標的にしやすくする」(Zheng, 2016, p.410; Kim, 2011, p.237を引用している)。
\nocite{kim11:_asian_femal_caucas_male_coupl}

\subsection{他の形態のデート差別も同じように批判すべきか}

場合によっては、人種以外の要素、たとえば身長や体重、さらには性別についても、上記の議論が当てはまるように思われるかもしれない。
つまり、人種と同様に、これらの要因についても差別しないようにすべきではないかと考えられる。
しかし、デートにおける差別に反対する人々は、レース(人種)は社会的文脈において特別な意味を持つと主張する。
ハディヤ・ロデリックは次のように述べている。
「人種は特別な問題だ。
私たちの人権法に制度的な保護が組み込まれ、数十年間にわたり反差別の原則が説かれてきた理由がある。
私たちのいわゆるポスト・レイシャルな社会は、この問題を克服し、人種が社会的構築物であることを認識し、私たちは皆、人間だという理解に至るはずだったのだ」\citep{roderique17:_datin_black}。
ソヌ・バディは、西洋社会における長い人種差別の歴史を考えると、人種差別は特に有害だと指摘する。

\begin{quote}
  既存の人種的ヒエラルキーやステレオタイプを強化する形での人種差別は、身長や体重に基づく差別とは異なる。
確かに、背が高く、スリムな人々は社会的に優位な立場にあるが、これらの特徴は人種ほど政治的に意味を持たない。
なぜなら、法律や社会制度が明示的に人種を理由に差別をおこなってきたからだ。
奴隷制、ジム・クロウ法、学校、レストラン、その他の公共施設における人種隔離は、この明白な事実を示している。
身長や体重は、人種と同じように深い社会的不平等を構造化し、またはそれを際立たせるものではない。
\citep[p.1004]{bedi15:_sexual_racis}

\end{quote}

人種に焦点を当てることは、他のバイアスを無視することを意味しない。
むしろ、一部の人々は、社会における既存の美の基準そのものに挑戦することが、人種差別の問題に取り組む一環となると主張している。
彼らは、欲望を可変的なものと捉え、純粋な外見に基づかないものとすることが、政治的な行為になりえると考えている。
アミア・スリニヴァサンは次のように述べる。

\begin{quote}
黒人女性、太った女性、障害を持つ女性の間で広まっている急進的な\ruby{自己肯定}{セルフラブ}運動は、私たちに性的嗜好を完全に固定されたものなどではないとみなすよう求めている。
「ブラック・イズ・ビューティフル」や「ビッグ・イズ・ビューティフル」は、単なるエンパワーメントのスローガンではなく、自分たちの価値観を再評価しようという提案だ。
リンディ・ウェスト\ig{Lindy West}は、太った女性たちの写真を見て、こうした身体{\DDASH}それまで恥や自己嫌悪感を抱いていた身体{\DDASH}を客観的に美しいと見ることはどういうことかを考えたという。
彼女にとってこれは理論の問題ではなく、知覚の問題であった。
つまり、ある身体{\DDASH}自分自身の身体であれ、他人の身体であれ{\DDASH}を別の視点から見て、嫌悪から賞賛へと認識を\ruby{一気に変更}{ゲシュタルト・シフト}する試みだったのだ。
\citep{srinivasan18:_does_anyon_have_right_sex}
\end{quote}

また、特定のジェンダーを好む傾向についても疑問を投げかけることができる。
これもまた、一種のバイアスではないだろうか。
イアン・エイヤーズとジェニファー・ジェラーダ・ブラウンは次のように主張する。
「異性愛者も同性愛者も「差別」をしている{\DDASH}つまり、最も親密な関係を築く相手の候補から、あらかじめ人口の半分を除外しているのだ。
このような差別がなぜ正当化されるのか、厳しく説明を求められるべきだ」\citep[p.31]{ayres05:_straig}。
すでに述べたように、性的指向は流動的なものであることが多い。
したがって、この流動性を促進するような措置を講じるべきではないだろうか。
エイヤーズとブラウンは、たとえば親が子供にバイセクシュアリティを奨励することを提案している。

しかし、人種と性別は異なるものだと考える理由もある。
性的な流動性が過小評価されがちであるとはいえ、エイヤーズとブラウンが「\ruby{性別}{セックス}にこだわらないバイセクシュアル」(sex-blind bisexual)と呼ぶ人々は少数派にとどまる。
ほとんどの人は、ある時点で異性愛者または同性愛者としての自己認識を持ち、それが生涯にわたって完全に固定されるわけではないとしても、特定の性別のパートナーを求める傾向がある。
この問題は、トランスジェンダーの人々をデートの対象としない人が多いことを考えると、さらに複雑になる。
ゲイとストレートの人々の両方が、強い「性器選好」(genital preference)をもっており、ある研究によれば、約90\%の人がトランスジェンダーの人とデートすることに抵抗を感じていると報告されている\citep{blair19:_trans_exclus_world_datin}。
この場合、人種や性別と同様に、誰かのトランスジェンダー・アイデンティティがその身体的外見の一側面だと常に主張できるわけではない。
ブリン・タネヒルは次のように述べている。
「魅力的で、知的で愛嬌のあるトランスジェンダーの人々は確かに存在する。
そして彼らは、シスジェンダーの人々と身体的に見分けがつかない場合もある」\citep{tannehill19:_is_refus_date}。
アビゲイル・カーリューは、トランスジェンダーの人々とデートすることへの抵抗感について、他のデート相手の嗜好と同じように批判的に検討すべきだと主張する。

\begin{quote}
私自身を含め、多くのトランス・フェミニストは、\ruby{嫌悪}{ディスガスト}の感覚〔トランス女性との交際を考えた際に一部のレズビアンが抱く嫌悪感〕は、あらかじめ定められた性的アイデンティティにおける必然的なものではなく、批判的な自己反省を通じて変化しうる柔軟な心構えだと主張する。
さらに、異性愛の男性やレズビアン女性を含む多くの人々が、トランス女性に惹かれて驚くことになるかもしれない。
\citep{curlew18:_whats_wrong_no}
\end{quote}

トランスジェンダーに対する差別の不正さは、多くの人々、特に多くのリベラルが受け入れる前提から導かれるものだ。
すなわち、\ruby{性}{セックス}とジェンダーが区別されるものであるという前提からの論理的な帰結だ。
広範なトランス権利運動の一環として、性器に対する好みが根絶されうる可能性がある。
リディア・グリーン\ig{Lydia Green}は次のように述べる。

\begin{quote}
  社会が変われば、そもそも人々がトランスフォビックな偏見を持つように育てられることがなくなるだろう。
公共政策、教育、メディアの三つの分野でトランスを受容する変化が起これば、人々のジェンダーに対する見方が変わり、トランスフォビックなデート相手の嗜好も過去のものとなるはずだ。
\citep{green17:_to_be_effec}
\end{quote}

しかし、こうした嗜好が将来消滅するか、また消滅すべきかについては意見が分かれる。
ゲイやレズビアンの中には、性的指向こそが自己のアイデンティティの中核を成すと考える者もいる。
ある匿名ブロガーはこう述べる。
「もし同性愛が「同じ\ruby{性}{セック
  ス}への惹かれ」と定義され、それが性器を含むものだとすれば、しかし一方で性器が明らかにその人の性と関係がないのであれば、何がその人を同性愛者たらしめるのか?」\citep{soldier17:_no_havin_genit}。
また、セックスとは相手の身体との親密な接触を伴うものであり、私たちがその人に抱く魅力は、その人の身体の特定の性質と密接に結びついている。
ブロガーのジョイラインは、自らの快楽体験が特に女性の性器を持つ相手とのセックスと結びついていると語る。
彼女はこう述べる。
「私がヴァギナとのセックスを好むのは、セックスで快楽を感じる能力を重視しているからにすぎない。
私は自分はセックスを楽しむに値すると学んだからだ。
私は自らの身体を探求する努力を重ね、ついに自分に必要なもの、そしてどのようにすれば自分の身体が最も快楽を得られるのかを理解するに至ったのだ」\citep{maenzanise19:_i_dont_find_piv_sex}。

デートにおける差別に対処しようとするいくつかの提案を見てみると、それらはこうした懸念をある程度緩和する助けとなるかもしれない。
誰かに対して望まない相手とデートするよう圧力をかけたり強制したりすることを主張する人はいない。
親密な関係における差別に反対する人々の多くが提案しているのは、むしろ内省のプロセスだ。
もしゲイやレズビアン、あるいはその他の人々が、この内省を誠実におこなった上で、それでも自分の選好は変わらないと判断するのであれば、自分が望む相手とつきあう権利は当然認められるべきだ。

\subsection{本節のまとめ:前進への道}

たとえ私たちのデート相手の選択が道徳的に問題がある可能性を認めたとしても、デートのバイアスに対処するための倫理的義務がどのようなものかをより明確に定義する必要がある。
それは、正義の抽象的な要求に従って、自らの嗜好を無理に変えなければならないという意味ではない。
むしろ、このアプローチの支持者は、デート相手やパートナー探しの過程に、継続的な省察と探求の余地を確保すべきだと主張する。
結果として、ある嗜好は固定的であり、自己概念にとって重要な要素であることが判明するかもしれない。
多くのゲイやレズビアンにとって、同性のパートナーを選ぶことはその一例だ。
すでに述べたように、デート差別に反対する人々は、誰もが誰かとデートする権利を主張できるとは考えていない。
彼らはむしろ、自らのバイアスを減らすための戦略を検討することを求めている。

デートのバイアスを軽減するために、二つの戦略が提案されている。
エリザベス・エメンズは、意識的な倫理的自己省察を通じてバイアスを軽減できると提案する。
彼女は「私のパートナーであることの本質的な役割とは何か?」と問いかけるべきだと考えている \citep[p.1360]{emens09:_intim_discr}。
この問いを立てることで、「これらの特性を文脈化し、親密な差別の規範に基づく直感的な反応以上のものを促すことができる」(ibid., p.1362)。
第二の戦略は「再習慣化」だ。
新しい人々や新しい経験に触れる機会を増やすことで、バイアスを軽減できるかもしれない。
異なるタイプの人々と交流することで、魅力の感じ方が変わる可能性がある。
ミッチェル\ig{Megan Mitchell}とウェルズはこれを、私たちが新しい食べ物に触れることで食の嗜好を変えることができるのと同様のプロセスだと比較している\citep{mitchell18:_race_roman_attrac_datin}。
ジャスティン・レームラーは、「オペラント条件付け」という概念を用いて、経験が欲望をどのように形成するかを説明している。
彼は次のように述べている。
「基本的な考え方は、私たちは報酬や肯定的な強化を受けた経験を繰り返したいと望む一方で、否定的な強化や罰を受けた経験は避けようとする、というものだ」\citep{lehmiller19:_where_do_our_sexual_attrac_come_from}。
したがって、新しい人々と出会い、彼らと肯定的な交流を持つ機会を自らに与えることで、結果として私たちの魅力の感じ方が変わるかもしれない。
こうした手法を試みても、デート相手の嗜好がほとんど変わらない人もいれば、大きく変化する人もいるかもしれない。
いずれにせよ、この努力をする価値はある。
それによって、自分自身にとっても、出会う相手にとっても、デートがより充実したものになる可能性があるからだ。

\section{BDSM}

今では誰もが映画『フィフティ・シェイズ・オブ・グレイ』を知っている。
若い女性が奇矯な億万長者と繰り広げる官能的な関係を描いたこの物語は、BDSMを一般に広めるきっかけとなった。
しかし、BDSMの実践は連続体に沿っており、『フィフティ・シェイズ』はその比較的穏やかな端に位置している。
その対極にあるものを見てみるために、エミリー・ウィットが書いた記事を参照しよう。
彼女は、職業的ドミナトリックスでありポルノ監督でもあるプリンセス・ドナ・ドローレがサンフランシスコで組織した過激なポルノ撮影について記述している。
その撮影は\emph{Public Disgrace}というオンラインポルノシリーズのためのものであり、このシリーズでは「女性が公衆の面前で拘束され、裸にされ、罰を受ける」映像が制作されている。
その撮影はバーでおこなわれ、男性パフォーマーや観客の一部が、同意の上で若い女性を平手打ちし、鞭打ち、挿入した\citep{witt13:_what_do_you_desir}。

BDSMは、拘束(Bondage)、支配(Dominance)、サディズム(Sadism)、マゾヒズム(Masochism)の略だ。
この用語は、性的快楽の名のもとに、誰かに対して意図的に苦痛や苦しみを与えたり、誰かを支配下に置く行為全般を指す。
そして、これらの行為はすべて当事者の合意の下でおこなわれる。
実践者の間では、BDSMをさらに三つのサブカテゴリーに分けることがある。
それは、拘束・調教(Bondage/Discipline, BD)、支配・服従(Dominance/Submission, DS)、サディズム・マゾヒズム(Sadism/Masochism, SM)だ。
それぞれ独自のダイナミクスを持っている。

多くの人々が、なんらかの形のBDSMというアイディアに魅力を感じている。ある研究によると、最大で70\%の人々がBDSMに少なくとも何らかの関心を示し、またはBDSM関連のファンタジーを持ち、ほぼ半数が何らかの形でBDSM的な行動を経験したことがあるという\citep{holvoet17:_fifty_shades_belgian_gray}。
ただし、一部の人々にとってBDSMは単なるファンタジーや実験的な行為ではなく、性的アイデンティティの不可欠な要素となっている。
こうした人々はしばしば自らを「キンキー」(kinky)と称する。
彼らは少数派ではあるが、その数はけっして少なくない。
同じ研究によれば、全体の約7\%が自らをキンキーであると認識している。

BDSMは当事者の同意に基づく実践であるにもかかわらず、それを何らかの形で道徳的に問題視する人々が多く存在する。
まずは彼らの主張を検討し、その後、BDSMの擁護者がどのように反論するのかを見ていく。
多くの法域において、BDSMの実践は法的に曖昧な位置にある。
合意に基づくBDSMが法的に訴追されることは稀だが、それを明確に保護する法律が存在する法域は少ない。
最後に、BDSMの合法化の是非について考察する。

\subsection{BDSMは有害か?}

BDSMは定義上、ある程度の危害を伴う。
それがBDSMの本質だ。
しかし、実践者たちは、明らかに、より大きな快楽や満足を得るためにそれをおこなっているのだろう。
とはいえ、BDSMに反対する人々は、その擁護者がいくつかの危険を見過ごしている可能性を指摘している。
まず、参加者が想定する以上の深刻な身体的危害のリスクがある。
DSM-IVは次のように警告している。
「サディスティックまたはマゾヒスティックな行動は、軽度から生命を脅かすものまで、さまざまな程度の傷害を引き起こす可能性がある」\citep[p.567]{DSM4}。
英国の1993年の \emph{R. v. Brown}判決では、合意の上でBDSMに参加した5人の男性が暴行罪で有罪判決を受けた。
この判決でテンプルマン卿は「一部のサドマゾヒズムの参加者は、彼らの性的接触によってどの程度の身体的危害が生じるのかを予測する手段を持たない」と述べた。

BDSMの実践において、意図しない負傷がどれほどの頻度で発生するかについての正確な統計は存在しないが、それが発生する可能性は確かにある。
多くのBDSM実践者は、BDSM行為が予定された範囲を超えてエスカレートし、深刻な危害を引き起こすことがあると認めている\citep[cf.][]{moser87:_explor_descr_study_sado_masoc_orien_sampl}。
これは事故によるものか、あるいは実践者が十分な訓練を受けていないことによる場合がある。
そして、実践者が掲げるスローガン「安全で、理性的で、合意のもとに」(Safe, Sane, and Consensual)とは裏腹に、エロティック・アスフィクシエーション(性的窒息プレイ)のような活動をおこなう人々もおり、これは完全に同意の上でおこなわれたとしても、本質的に重大なリスクを伴う\citep[pp.122-123]{busby12:_every_breat_you_take,downing07:_beyon_safet}。

危害が意図的に加えられることもある。
BDSM実践者の一人であるキティ・ストライカーは次のように述べている。
「BDSMコミュニティに入ったとき、何度望まないセックスに誘導されたり、圧力をかけられたり、あるいは強制されたのか、実際に数えることすらできない」\citep{striker11:_i_never_called_it_rape}。
4,000人以上のBDSM実践者を対象とした調査によると、29\%が何らかの同意侵害を経験しているという\citep{wright15:_consen_violat_survey}。
BDSMコミュニティ内の虐待についてのインターネット討論スレッドの投稿者が、ある友人の経験を次のように述べている。

\begin{quote}
彼女は当初、痛みのないプレイをおこなうと交渉していた。
しかし、彼女がプレイの最中に意識が朦朧とし、言葉を発することができなくなった後、彼はプレイの内容を再交渉し、彼女にボディパンチングへの同意を得た。
彼女は強めのマッサージのようなものを想像していた。
しかし、彼女は肋骨を三本骨折させられた。
彼は腎臓を殴り彼女は倒れたが、彼は彼女を床に押さえつけ、そのまま殴り続けた。
\citep{nick11:_bdsm_rape}
\end{quote}

BDSMは、意図的に他者に危害を加えたいと考える人々を引き寄せる可能性がある。
チャーリー・グリックマンは次のように述べている。
「BDSMコミュニティの人々があまり認めたがらない事実の一つは、そう、実際に他者を傷つける口実としてBDSMに惹かれる人々が存在するということだ」\citep{glickman11:_bdsm_rape}。
BDSMコミュニティの一員であるブロガーは次のように指摘する。
「BDSMコミュニティがターゲットへの接近手段を提供しており、無意識的にあるいは無責任に虐待行為を隠蔽する環境を提供しているのであれば、なぜ\ruby{捕食者}{プレデター}がBDSMコミュニティに引き寄せられないはずがあるだろうか?」(ibid.)。

また、BDSMは家庭内暴力の隠れ蓑としても利用される可能性がある。
本来BDSMに興味のなかった被害者が、意図せずそのような行為に巻き込まれるケースがあるのだ。
BDSMが主流文化に浸透するにつれ、家庭内暴力の加害者やその他の犯罪者が、「\ruby{乱暴}{ラフ}なセックス」(rough sex)の一環であったと主張することで、自らの行為を正当化するケースが増えている。
\emph{The Guardian} は、イギリスにおいて1996年には「ラフなセックス」を理由とする死亡・傷害事件が2件あったのに対し、2016年にはその数が20件に増加したと報告している\citep{grierson20:_gover_consid_law_curb_use}。
この弁護戦術は有効であることが証明されている。
We Can't Consent To Thisという団体によると、イギリスで男性が「ラフなセックス中に女性を殺害した」と主張したケースのほぼ半数で、罪状が殺人から過失致死に軽減されたり、完全に無罪となったりしているという\citep{harman20:_rough_sex_gone_wrong_defen,woodyatt20:_grace_millan_rise_shades_defen_murder_trial}。
BDSMに関する問題を論じるブロガー、アンドレアス・ザニンは次のように述べている。
「「俺を嫌わないでくれ、俺はただキンキーなだけなんだ」という弁明が、非合意の暴力をおこなう者たちによって利用される危険がある。
そして、私たちBDSMコミュニティが無批判にそれを容認し{\DDASH}あるいは、さらに悪いことに、彼らを支持するような立場をとることで{\DDASH}暴力の被害者たちが再び沈黙させられる危険がある」\citep{zanin14:_poor_persec_perver}。

一部の論者は、BDSM嗜好が心理的問題と関連していると主張する。
それは単なる症状であるだけでなく、原因ともなりうるというのだ。
BDSMは長らく病理として分類されてきたが、BDSM嗜好と抑うつや自殺傾向といった心理的病理との関連を示す研究も存在する\citep{brown17:_suicid_risk_bdsm_pract}。

\subsection{BDSMは本質的に貶めか?}

BDSMに反対する人々の中には、特に暴力や屈辱を伴う性行為が極めて屈辱的であり、そのような行為に対する道徳的正当性は、当事者の自由な同意だけでは保証されないと主張する者もいる。
アラン・ジェイコブズは、ウィットが記述した過激なポルノ撮影に関与した人々について、次のように述べている。
「彼らは、意識的であれ無意識的であれ、完全な屈辱を追求しており、それによって公然とセクシュアリティを貶めている。
彼らは自分自身にとっても他者にとっても極めて破壊的な存在だ……」\citep{jacobs03:_in_which_noah_millm_i}。

ジェイコブズは、この堕落をあからさまに宗教的な言葉で定義している。
彼は、この撮影に関わった者たちについて、 「彼らは、自分たちが神の姿において創られたという事実を曇らせている」\citep{jacobs03:_in_which_noah_millm_i}と述べている。
しかし、本書5.1.3節で論じるように、同様の議論は、世俗的な観点から、特にカント主義的な観点からも展開されてきた。
すなわち、すべての理性的存在は本来的な尊厳を持ち、それを自ら放棄することはできず、また、彼らは不可侵の権利として尊重されるべきだという主張だ。
R.~A.ダフは、合意の上でおこなわれる暴力的な性行為が、関与者の合意にもかかわらず犯罪として扱われる可能性があるかどうかを検討している。
彼は、その可能性を否定しつつも、次のような世俗的な定式化を提示している。
「これらの行為が公的な不正であると見なされる可能性があるのは、それらが私たちが互いに払うべき尊重を深刻に侵害し、その結果として(少なくとも暗黙的に)その対象となる者の道徳的地位を否定するからだ」\citep[p.232]{duff14:_towar_modes_legal_moral}。

\subsection{BDSMは悪しき性格を助長するか?}

BDSMは他者に苦痛を与える。
しかし、これ自体は特別なことではない。
たとえば、医療行為、コンタクトスポーツ、身体改造など、多くの活動が痛みを伴うが、これらが道徳的に問題視されることはない。
しかし、BDSMがこれらと異なる点がある。
それは、これらの活動では、参加者が「苦痛を与えること自体を楽しむ」ことは期待されていないという点だ。
もし医師、アスリート、あるいはタトゥーアーティストらが「自分の仕事でいちばん楽しいのは他者に痛みを与えることだ」と言ったならば、私たちは彼らの性格について深刻な疑問を抱くだろう。
しかし、BDSMにおいては、ドミナント(支配する側)が相手に与える痛みを楽しむことが期待され、さらにはそれを性的に捉えることが求められる。
批判者たちは、このような性質の肯定が、けっして賞賛されるべきでない性格特性を支持し、それを養成すると主張する。
また、サブミッシブ(従属する側)も、一般には容認されない性格特性に慣らされてしまう。
すなわち、彼らは自身を従属的な存在と見なし、苦しみを受けるに値すると考えるようになる。

BDSMは参加者の欲望を反映している。
しかし、徳倫理学の観点からすれば、私たちは単に現在の欲望を満たすのではなく、高貴な性格特性を養い、それに応じて欲望をより有徳な方向へと向かわせるべきだ。
ロッド・ドレアは次のように述べる。

\begin{quote}
  私たちは欲望を陶冶することができるし、無秩序な欲望を抑制することもできる(私は、苦痛を与えることや受けることに快楽を見出すことが極めて無秩序なものであると考える)。
また、生命を育むような欲望を促進することを学ぶこともできる。
『フィフティ・シェイズ・オブ・グレイ』に安っぽいスリルを感じる女性はいるし、BDSMポルノを楽しむ男性もいる。
だが文明の目的は、内なる野蛮さを抑圧することにある。
\citep{dreher12:_million_shades_of_gross}
\end{quote}

また、BDSMが市民の間に残虐性を育み、最終的には社会そのものに害を及ぼす可能性があるとも考えられる。
イギリスの1993年の \emph{R v. Brown}判決 において、テンプルマン卿は判決文の中で次のように述べた。

\begin{quote}
証拠から明らかになるのは、ドミナントの行為が予測不可能な危険性を持ち、身体と精神に対して屈辱的であるという点だ……私は、サディスティック・マゾヒスティックな行為に対して、同意という名目で正当化される弁護を発明するつもりはない。
このような行為は、残虐性を育み、それを称賛するものだ……社会には、暴力カルトから自身を守る権利と義務がある。
苦痛を与えることから得られる快楽は邪悪なものだ。
残虐性は文明とは相容れない。
\ig{(\emph{R v. Brown})}
\end{quote}

\subsection{BDSMは家父長制を再強化するか?}

フェミニズムの中では、BDSMが家父長制的な態度や規範を再強化するのではないかという議論が長く続いている。
シーラ・ジェフリーズは、BDSMは「異性愛的欲望を駆動する性差の粗野な力関係をエロティックにし、それを終わらせるのではなく強化する」と主張する\citep[p.86]{jeffreys96:_heter_desir_gender}。
キャスリーン・バリー\ig{Barry}は、『性の植民地』(\emph{Female Sexual Slavery})で、BDSMを「女性をその意志に反して性的に従属させる行為を偽装したものだ」と表現している\citep[p.209]{barry79:_femal_sexual_slaver}。
アンチBDSMフェミニストたちは、女性が服従的であるように社会化されており、BDSMの実践者はこれを利用し、さらに悪化させていると主張する。
彼女たちは、BDSMに関わる欲望は「社会において女性に降り注ぐ性的イメージへの条件反射的な応答」であり、BDSMを実践する女性は「私たちの生涯を通じて叩き込まれた性的相互作用のモデルに応じているのだ」と述べる\citep[p.139]{nichols82:_is_sadom_femin}。
C.~K. エグバートは次のように述べる。

\begin{quote}
 BDSMは定義上、不平等、支配、痛み、虐待をエロティックにするものであり、したがって、いかなる家父長制的規範にも挑戦することはない。
家父長制は、男性が女性を傷つけ、支配することを楽しむからこそ存在する…… BDSMは家父長制と同じことを性について主張している。
すなわち、女性を傷つけることはセクシーだということだ。
それを豪華なレザーの衣装で装飾し、許容される性的暴力のレベルを「\ruby{通常}{ノルム}」(たとえば、痛みを伴うまたは望まれない性交、強制的な性交)から、より極端なもの(性的拷問、切断)へと引き上げているだけだ。
\citep{egbert15:_bdsm_faq_frequen_asser_quibb}
\end{quote}

フェミニストの批判者たちは、BDSMの実践が日常生活にも波及する可能性を懸念している。
反BDSMフェミニストは、女性がBDSMに参加することで、結果的にフェミニズムを弱体化させ、「寝室の外での権力の不均衡の正当性を強化する」ことになると主張する(Nichols et al., 1982, p.140; cf. Scott, 2012)\nocite{nichols82:_is_sadom_femin}\nocite{scott12:_think_kink}。
この影響は二つの形で現れる。
第一に、BDSMはそれを実践する人々(男性も女性も)の権力に対する家父長制的な態度を強化する。
第二に、人々が女性のBDSMへの関心を「彼女たちは支配され、屈服することを望んでいる」という証拠として受けとる可能性がある。
すべてのドミナントが異性愛男性であり、すべてのサブミッシブが異性愛女性であるわけではない。
男女ともに両方の役割を引き受けることがあり、多くの実践者は伝統的な性別役割にとらわれない。
しかし、BDSMの実践が伝統的な性別に沿って構造化される傾向があることを示す証拠もある。
マーゴ・ワイス\ig{Margo Weiss}がBDSM実践者を調査した研究によれば、彼女のインタビュー対象者のうち、異性愛女性の71\%がボトム/サブミッシブだと自己認識しており、異性愛男性の75\%がトップ/ドミナントだと答えている\citep[p.262 fn.10]{weiss11:_techn_pleas}。
さらに、異性愛女性でトップ/ドミナントだと答えたのはわずか14\%であった。
彼女の研究対象者はまた、サブミッシブであることが「弱さ」と同一視される傾向にあることを報告している\citep[p.176]{weiss11:_techn_pleas}。
フェミニズムの立場からBDSMに反対する者は、関与する者の性別にかかわらず、BDSMは支配と権力を伴う関係をエロティックなものとして描き、平等や相互性を重視しないと主張する。
エグバートは次のように述べる。

\begin{quote}
  レズビアン、ゲイ、女性のドミナントもまた、他の誰とも変わらず、異性愛中心主義的および女性蔑視的な規範を内面化することがある。
女性が男性を虐待する事件が発生したとしても、それは暴力や性的暴行におけるジェンダーの力学を変えるわけではない。
未成年者が親を虐待する事件が起きたとしても、それが子供虐待の存在を否定するものではないのと同様だ。
\citep{egbert15:_bdsm_faq_frequen_asser_quibb}
\end{quote}

BDSMの実践者は、自分たちが私的におこなうことは他人には関係ないと主張する。
しかし、マージョリー・ジョレスは、フェミニズムにおいて「個人的なことは政治的なことである」という考えが自明のものとされていると指摘する。
つまり、私たちは自らの私的な行動が家父長制的な権力構造をどのように維持・強化するのかを検討しなければならないのだ。
彼女は次のように述べる。

\begin{quote}
  性的自由を理由にBDSMを擁護する問題は、それがセクシュアリティを批判から免除し、プライバシーのヴェールの後ろに隠して倫理的な調査から遮断することにある……私たちが私生活でどのように行動しているかをよく考察することは、自由の侵害ではない。
したがって、BDSMをプライバシーの問題として擁護することは正当化されない。
なぜなら、セクシュアリティは「現実世界」から切り離されているのではなく、それに関連して形成されるものだからだ。
もし不道徳な社会イデオロギーが世の中に蔓延しているなら、それは私たちの性的生活の中にも巡り巡っていると考えるべきだ。
\citep[p.268]{jolles15:_pleas_pain_femin_polit_rough_sex}

\end{quote}

キャスリーン・バリー\ig{Barry}は、BDSMが不平等な社会構造をエロティックにし、それを促進すると主張する。
「BDSM、ボンデージとディシプリン(拘束と服従)、これらは奴隷制に起源を持つ……もし私たちが、犯罪的であり、人間を侵害する社会制度(たとえば奴隷制)を美化し、崇拝するならば、私たちはそれを継続させることに加担しているのだ」\citep{murphy12:_part_two_two_part_series_bdsm_femin}。
\ig{Meghan Murphy}

フェミニストのBDSM実践者たちは、自分のBDSMへの関心と政治的信念をどのように調和させるかに苦悩している。
ダフネ・マーキンは、\emph{The New Yorker}に寄稿した自伝的エッセイの中で、自身を「\ruby{手強い}{フォーミダブル}」女性と認識しているにもかかわらず、スパンキングされることに惹かれることについて考察している。
彼女はこう書いている。
「男女の平等、あるいはその見せかけを維持することには多くの努力を要する。
そしてそれは、必ずしも性的興奮にとって最も確実なルートではないかもしれない」\citep{merkin96:_unlik_obses}。
多くのフェミニスト作家たちは、女性たちが自らの欲望を与えられたものとして受け入れるのではなく、それを吟味することを求めている。
BDSMの批判は、家父長制が女性の欲望をどのように形成するかを理解する努力の一環となりうる。
ノーマ・ラモスは次のように述べている。
「私はこの(服従に対する)性的快楽を得ている。
しかし、では、それをどうすればよいのか? それを変えるために努力するのだ。
それに挑戦しなければならない」\citep[p.62]{gillespie95:_where_do_we_stand_pornog}。

\subsection{BDSMの害に対する反論}

すでに述べたように、社会はある種の害を伴う活動を許容している。
たとえば医療行為、スポーツ、身体改造などだ。
BDSMの擁護者は、これらを許容しながらBDSMのみを問題視するのは一貫性に欠けると主張する。
彼らによれば、身体改造のもたらす害の方がBDSMよりも永続的であることが多い。
BDSMを区別するのは、その害ではなく、多くの人が非伝統的な性的実践に対して抱く嫌悪感や、それに関する道徳的見解だ。
アイオワ州控訴裁判所は、1985年のある判決の中で、サディスティック・マゾヒスティックなセックスは「スポーツ、社交、その他の活動」には当たらないと判断し、その理由として、次のように述べている。
「この活動が他の法域で繰り返し非難され、私たちの社会の一般的な道徳原則と対立するものと見なされてきたことは明らかである」(\emph{State v. Collier})。

BDSMの擁護者は、適切におこなわれる場合、BDSMはむしろ通常のセックスに伴うリスクを軽減すると主張する。
BDSMは明確で積極的な同意を事前に必要とするため、非BDSMのセックスよりもむしろ安全であると言われる。
参加者は、どのような行為が許可され、何が禁止されるのかを慎重に交渉し、安全語を設定する。
これにより、サブミッシブがセーフワードを発した場合、即座に状況が停止される。
多くの場合、事前に契約が交わされ、許容される行為とそうでない行為が明文化される。
シャナ・カッタリは次のように言う。

\begin{quote}
コミュニケーションと交渉は、ほとんどのBDSMのやり取りにおいて不可欠な要素である……参加者は事前に時間を設け、性感染症の状況や安全なセックスの方法、どのようなプレイがおこなわれるか、活動の時間、健康上の懸念(トラウマの引き金、アレルギー、障害、薬の必要性など)、使用する道具、および潜在的な安全問題について話し合う。
\citep[p.887]{kattari15:_gettin_it}

\end{quote}

BDSMのやり取りは、詳細なコミュニケーションと交渉によって始まるだけでなく、広範な「アフターケア」を伴う。
ブロガーのミシー・Bは、次のように述べている。

\begin{quote}
参加者はお互いにケア、注意、慰めを提供します。
ドミナントがサブミッシブに一定のプライベートな時間を与えることもあれば、毛布に包んで抱きしめることもあります。
アフターケアの時間には、シーンの振り返りがおこなわれ、何を改善すべきか、どの部分が最も楽しめたか、次回は何を省くべきかが話し合われます。
\citep{michea18:_can_bdsm_teach_us_what}
\end{quote}

BDSMの擁護者たちは、こうした点を挙げ、BDSMは単なる許容可能な実践にとどまらず、むしろより健康的なセクシュアリティのモデルを提供すると主張する。
ミシー・Bは次のように言う。
「もし社会全体が、BDSMやキンクのコミュニティで扱われるように同意とコミュニケーションを重視したならば、性的暴行に関する混乱は大幅に減少し、性的暴行そのものも減少するでしょう(Michea B, 2018. cf. Rocha, 2016)。
\nocite{rocha16:_aggres_hook_ups}

BDSM実践者たちは、BDSMの場において暴行や虐待が発生することがあることを否定しない。
しかし、彼らは非BDSMの文脈での暴行の頻度を考慮するよう求める。
BDSMの擁護者は、知識のある人々であれば、合意のある状況と虐待的な状況を明確に区別できると主張する。
心理学者のコリンヌ・ハイムは次のように述べる。
「合意のあるBDSMと虐待の違いを見分けるのは非常に簡単です。
前者では、事前に交渉がおこなわれ、シーンが計画されており、セーフワードなどの出口戦略があります」\citep{mcarthur16:_its_traves_that_bdsm_isnt_techn_legal}。
経験豊富な実践者は、これらすべての点について証拠を確保するよう努めている。

BDSMが特定の心理的問題と関連しているという主張に対して、擁護者たちは、BDSM実践者が一般人口と比べて有意な心理的差異を示さないことを示した研究を指摘する\citep{hebert14:_examin_person_charac_assoc_bdsm_orien,powls12:_descr_review_resear_relat_sadom}。
さらに、多くの実践者は、BDSMは心理的な問題を引き起こすどころか、むしろ治療的な効果を持つ可能性があると主張する。
ある研究は、「BDSMの実践には、人間関係や内面的成長を促す可能性がある」と結論している\citep[pp.157-158]{weille02:_psych_consen_sadom_domin_submis_sexual_games}。
職業的\ruby{女王様}{ドミナトリックス}を研究したダニエル・リンデマンは、彼女の被験者がBDSMを「セクシュアルな抑圧への健康的な代替手段」「贖罪の儀式」「過去のトラウマを克服するための手段」として語っていたことを報告している\citep[p.157]{lindemann11:_bdsm_therap}。
彼女によれば、ドミナトリックスたちは自身が提供する体験について、「性的抑圧に対する健全な代替手段、贖罪の儀式、過去のトラウマを克服するための手段、そして(「屈辱セッション」の場合には)クライアントが羞恥を通じて心理的な活性化を経験するプロセス」として説明している(ibid.)。
カップケーキ・シンクレアという名を与えられたサブミッシブは、\emph{VICE}に次のように語っている。

\begin{quote}
多くの人は痛みを避けようとします。
しかし、それを受け入れることで、私自身は必要なカタルシスを感じることができると同時に、自分が直面しているどんな問題よりも強い存在であることを思い出すのです。
私はこのライフスタイルを約六年間続けており、私にとってそれはセラピーのようなものです……信頼できる相手に身を委ねるとき、私は不安を手放すことができるのです。
\citep{barrett-ibarria17:_bsdm_can_provid_profoun_healin_exper}
\end{quote}

BDSMは、ジェレミー・トーマスが「トラウマ・プレイ」と呼ぶものとして機能することもある。
つまりBDSMは「過去のトラウマや虐待を「演じる」ことによって意図的に再現し、制御された環境で記憶を再構成する手段」になりうる\citep{thomas20:_bdsm_traum_play}。
あるキンク実践者は\emph{VICE}に対し、BDSMは「トラウマの記憶を再訪し、再現し、あるいは書き換えることが安全にできる、管理された環境を作り出す」と語っている\citep{barrett-ibarria17:_bsdm_can_provid_profoun_healin_exper}。
ロイ・バウマイスターは、BDSMは「自己からの逃避」を許してくれると捉えている\citep[p.29]{baumeister88:_masoc_escap_self}。
BDSM実践者たちは、「トップスペース」および「ボトムスペース」について語る。
これは、BDSM活動中に生じる「独特で、主観的に快適な、変容した意識状態」だ\citep[p.77]{ambler17:_consen_bdsm_facil_role_specif}。
スタシ・ニューマールはこれを心理学でいう「フロー」の概念で説明している。
「SMの参加者は、自らのプレイについて、恍惚的な体験、あるいはフローとして理解できるものとして語る。
彼らは無重力感について語り、リズムに乗る感覚、飛翔する感覚、認知プロセスの停止、そして周囲の世界の消失を経験する」\citep[p.328]{newmahr10:_rethin_kink}。
また、アンドレア・ベックマンが「BDSMの持つ変容の可能性」と呼ぶものを指摘する者もいる\citep{beckman01:_decon_myths}。
彼らは、それが人々の意識を変容させる能力において、魔術的あるいは宗教的儀式と類似していると論じている\citep{comfort78:_sexual_idios,norman04:_i_am_leath_shaman}。

BDSMの実践者が称賛に値しない行動に慣れ親しんでいるという主張に対して、擁護者たちは、BDSMが「\ruby{遊び}{プレイ}」の一形態であり、したがって参加者はその行為を日常生活で価値を置く行動とは切り離して考えることができると反論する。
レスリー・ロジャーズと名乗るドミナントは\emph{The Atlantic}に対し、「僕たちが本当にやっているのは、大人にもう一度遊び方を教えることなのです。
子供の頃にずっとやりたかったけど、結局できなかったことがあるでしょう? 今ならそれができるんです」と語っている\citep{morin15:_that_time_i_tried_bdsm_therap}。
パトリック・ホプキンズは、BDSMは不道徳な行為そのものではなく、それを模倣することをエロティックなものとしていると主張する。
BDSMは遊びの領域に属するため、それに関与する欲望は、残酷さや服従への欲望といった特性を反映するものでも、それらを助長するものでもないと論じている。
彼は次のように書いている。

\begin{quote}
  SMの参加者が実際に奴隷の拷問やレイプ被害者の叫び、女性の屈辱、あるいは加害者による執拗な暴力に快楽を見出していると想定すべきではない……サドマゾヒストは、その模倣自体を欲望するのであり、それを現実の劣ったコピーとしてではなく、何かのコピーとしてでもなく、純粋に模倣として欲望するのだ。
\citep[p.125]{hopkins94:_rethin_sadom}

\end{quote}

BDSMは悪徳を育むどころか、むしろ他者への敬意、相手のニーズへの敏感さ、そしてコミュニケーションへの開放性といった称賛に値する性格特性を表現し、育むものとして捉えることもできる。
ある研究によれば、BDSMの実践者の性格に対する純粋な影響は肯定的なものであるとされている。
調査対象となったキンキーの人々は、レイプに関する誤った信念を抱く傾向が少なく、性的暴行の被害者を非難する傾向も低く、女性に対する家父長制的な態度を持つ割合も低かった\citep[p.1]{klement16:_partic_cultur_consen_may_be}。

また、BDSMは人々にさまざまなアイデンティティを試す機会を与えることで、開かれた思考を育むことができる。
フェム・ダイクのミストレス・ミーン・マミーは、ロビン・バウアーに対し、それによって自分自身が異なるキャラクターを演じることを想像できるようになり、したがって自身の個人的経験の範囲を超えた人々への共感を育むことができると語る。
「私たちは探求することができる。
私にとって、それは本を読むのと何ら変わらない……もしかすると、それによって私の視点が変わるかもしれない」\citep[p.141]{bauer18:_bois_grrrl_meet_their_daddies}。

\subsection{フェミニストBDSM}

BDSMを擁護するフェミニストは、それが参加する女性の自由な選択を反映しており、したがって彼女たちの性的自律の表現として成り立つと主張する。
シュガーカントというブロガーは次のように述べている。

\begin{quote}
合意に基づくラフセックスを推奨することで、私たちの言葉を歪め、非合意の性的暴力を正当化しようとする人々に誤ったメッセージを送ることになる、という考え方があるかもしれない。
しかし、私のラフセックスは、私のフェミニズムと共存している。
それは私自身のエージェンシーによる行為であり、私が自ら進んで参加を決めたものだ。
\citep{tallon-hicks16:_can_you_be_femin_like_rough_sex}

\end{quote}

マーゴ・ワイス\ig{Margo Weiss}がおこなったインタビューの中で、インタビュイーの一人であるテラミスは次のように語っている。
「奴隷であることとフェミニストであることの間に、私はまったく矛盾を感じません。
奴隷であることは選択の問題であり、フェミニズムもまた選択の問題だからです」\citep{weiss11:_techn_pleas}。

シャナ・カッツは、BDSMがフェミニズムのより広い目的に貢献する特定の側面を指摘する。
それは、BDSMが促進する「オープンなコミュニケーションの理想」だ。
彼女は次のように述べる。
「ラフセックスを「フェミニスト的」なものにするのは、すべての参加者が自分の欲望を共有し、それが認識され、尊重される権利と空間を持つことだ」\citep{tallon-hicks16:_can_you_be_femin_like_rough_sex}。
BDSMは、フェミニストたちが長年すべての関係において求めてきた、明確で開かれた同意を前提とする。
そして、たとえ女性が男性の支配に服従する場合でも、実際には最終的な制御を保持しているのはサブミッシブ側だ。
サブミッシブは、どの行為が許容されるかを決定し、いつでもシーンを停止できる権利をもっている。

また、一部のフェミニストは、BDSMが既存の性規範に挑戦しうると主張する。
これらの性規範は、もともと家父長制的な秩序と結びついている。
ジョルズは次のように述べている。

\begin{quote}
  セクシュアル・ラディカルの立場からすると、性規範はしばしば社会統制のメカニズムとして機能する。
そのため、BDSMはその実験的かつ解放的な精神ゆえに本質的に価値を持つ。
性的自由へのコミットメントに基づき、それを楽しむ人々は、スティグマを受けることなく実践できるべきだ。
\citep[p.268]{jolles15:_pleas_pain_femin_polit_rough_sex}

\end{quote}

パットリック・カリフィアは、「BDSMにおける制服、役割、対話などが主流社会から取り入れられると、それらは権威へのパロディとなり、挑戦となる」と述べる\citep{califia79:_unrav_sexual_fring}。
ライラ・シセロは、BDSMが女性にとって安らぎの場となる可能性を指摘する。

\begin{quote}
  支配されることは、リラックスにつながることがある。
そう、リラックス{\DDASH}とりわけ、強くて自立していて、常に物事をコントロールしている女性にとっては。
大人でいることはたいへんだ。
自己主張することが、まるで仕事のように感じられることもある。
だからこそ、遊びの時間には、それから解放される「バケーション」を求めたくなることもある。
パートナーに指示され、主導権を握られ、支配されることは、多くの女性が日常的に抱えている「支配されまいとする緊張状態」からの休息のようなものだ。
女性にとって{\DDASH}あるいは男性にとっても{\DDASH}支配されることは、むしろ自由を感じさせることがある。
\citep{cicero12:_six_myths_kink_bdsm}
\end{quote}

この避難の感覚は、特に人種的マイノリティの女性にとって重要となる場合がある。
アリアーヌ・クルーズは次のように述べている。
「黒人女性のBDSM実践者たちが示すように、同意は、黒人が日常的に経験する身体的・精神的苦痛を、キンクという「安全な」遊び場において、恍惚と能動性の領域へと変換する機能を果たすことができる」\citep{wachter-grene16:_conver_arian_cruz}。

BDSM実践者たちは、BDSMが主流社会では不可能な流動性を可能にする点を指摘する。
カリフィアのインタビューを受けたある人物は次のように語った。
「もしトップやボトムの役割が合わなければ、キーを交換(役割変更)すればよい。
それを生物学的な性別、人種、社会経済的地位でやるのは無理だろう」\citep[pp.173-174]{califia00:_public_sex}。

\subsection{BDSMとクィア・アイデンティティ}

BDSMは長らくゲイおよびレズビアン・コミュニティにおいて特別な地位を占めてきた。
キャメロン・グローバーは、「キンクとBDSMをラディカルなクィア史の一部として再評価する時が来た」と述べる\citep{glover18:_its_time_recen_kink_bdsm}。

J.P. ラロックは、「スパルタからワイマール共和国に至るまで、キンクは常に私たちのセクシュアリティの一部であった」と主張し、第二次世界大戦後にゲイ・コミュニティ内で形成された独特のレザー・カルチャーについて述べている。
これは1960年代および70年代のゲイ・カルチャーにおいて重要な役割を果たした\citep{larocque14:_brief_histor_bdsm}。

ロバート・ホプケは、BDSMがゲイ男性にとって政治的に力を与える実践であり、次のような役割を果たすと述べる。
「BDSMは、家父長制および異性愛中心主義社会に対する痛烈な一撃となる。
ゲイ男性が互いに持つ男性的な力を活用し、彼らを未熟で女性的だと決めつける社会の枠を打ち破ることができるのだ」\citep[p.71]{hopcke91:_jung_jungian_homos}。

BDSMの支持者は、BDSMコミュニティがゲイやレズビアンだけでなく、その他の社会的に周縁化された人々をも包摂する空間を作り出してきたと主張する。
アレクサンダー・チェヴスは次のように述べる。
「キンクは包括的なコミュニティだ……私たちは皆、ここに属している。
私たちは無数の異なる文化、背景、肌の色、ジェンダー、政治的視点を持つ人々が集まるコミュニティなのだ」\citep{cheves18:_kink_is_part_my_ident}。
BDSMは、主流社会でスティグマや偏見にさらされる人々にとっての安全な空間を提供する。
キャメロン・グローバーはこう述べる。
「キンクは排除するのではなく、包括する。
それは、主流社会ではしばしば忌避され、誤解されるものを受け入れることを基盤としているからだ」。

多くのBDSMスペースは、トランスジェンダーの人々を歓迎する姿勢を強調している\citep{bauer07:_playg_new_territ}。
あるBDSMクラブのメンバーを対象とした研究では、次のように報告されている。
「トランスジェンダーのメンバーとシスジェンダーのメンバーの双方が認めていることだが、このクラブの性的空間において、トランスジェンダーの人々は受け入れられている。
クラブは、アイデンティティが流動的であり、身体が多様な意味を持ちうる場として機能しているのだ」\citep[p.1652]{stone13:_flexib_queer_serious_bodies}。
あるトランスジェンダーのメンバーは、クラブのイベントに参加した当初から歓迎されていると感じたことについて語る。

\begin{quote}
 The Clubは、はっきりとすべてのジェンダーやセクシュアリティの人々を受け入れています……そして、人々を歓迎しようという努力がなされている。
私は、派閥意識や「あなたは入れない、なぜならジェンダーが間違っている、あるいは正しい性器を持っていない」といったことを言われるのを聞いたことがありません。
そうしたことは、ここでは許されないのです。
明文化されたルールはありませんが、そういうことを警察のように取り締まるような行為は、この空間では暗黙のうちに禁じられています。
そして、人々が異なるジェンダーの人々をオープンに受け入れていることが、その雰囲気を作り出しているのです。
\citep[p.1653]{stone13:_flexib_queer_serious_bodies}
\end{quote}

こうした包括性は障害を持つ人々にも広がっている。
ドーン・レイノルズは、BDSMを「障害者にとっての重要な自己エンパワーメントの手段」と呼ぶ。
彼女は次のように述べる。
「BDSMは、異なる身体やオルタナティブなライフスタイルを受け入れる性的コミュニティを提供する。
さらに、BDSMは痛みと快楽の間の不安定な境界を遊ぶものであり、これは特に慢性痛を抱える人々にとって重要な意味を持つ」\citep{reynolds07:_disab_bdsm}。

\subsection{BDSMの非犯罪化は可能か?}

BDSMは法的に曖昧な立場に置かれている。
それは自発的な合意に基づく行為だ。
西洋の法体系においては、古代ローマにまで遡る基本原則として「\emph{volenti non fit injuria}(自発的に同意した者には不法行為は成立しない)」がある。
しかし、同じく長い歴史を持つもう一つの原則として、「人は自らに対する危害に対しては有効に同意することはできない」という考え方も存在する。
米国法曹協会の模範刑法典の言葉を借りれば、危害が「死亡の重大なリスクを生じさせる、または深刻かつ永久的な外貌の損傷、もしくは身体の各器官の機能の長期的な喪失や障害を引き起こす」場合、それは重大な傷害と見なされ、違法とされる(American Law Institute, Model Penal Code, 1980, § 210.0(3))。
この法律の背景には、公的秩序を維持する国家の責務がある。
社会全体の利益として、人々が紛争を解決したり娯楽の一環として決闘や暴力行為に及ぶことを防ぐ必要がある。
また、暴力行為の被害者が、加害者の圧力によって同意を主張せざるを得なくなることを防ぐ目的もある。

法は、一定の社会的価値を持つと見なされる行為に関して例外を設けている。
モデル刑法では、合法的な運動競技や競争スポーツ、または法的に禁じられていないその他の活動において発生する「合理的に予測可能な危険」を伴う傷害は例外とされる(Model Penal Code, § 2.11(2)(b))。
また、ピアスやタトゥーなどの身体改造についても、本人の同意の上で許容されている。
しかし、多くの法域ではBDSMはこの例外に明確に含まれておらず、そのためBDSMセックスは潜在的に犯罪と見なされる。
たとえば、1999年の\emph{People v. Jovanovic}判決では、ニューヨーク州最高裁は「公的秩序の観点から、たとえ被害者が行為を望み、同意していたとしても、傷害や重大な危険を伴う暴行の刑事責任を免れることはできない」との判断を下している。

このような起訴は稀だが、けっして存在しないわけではない。
実際の傷害がフットボールの試合中に負う怪我よりも軽度であっても、刑事訴追の対象となる場合がある。
事件では、BDSMプレイ中に「深刻な身体的傷害」を与えたとして被告が有罪判決を受けた。
被害者が医療処置を必要とした証拠はなく、目に見える外傷もなかった。
しかし、裁判所は、溶けたロウソクの蝋が「熱くて痛みを伴った」と推定し、乳首クランプが「きつく締め付けて切れる可能性がある」と認定した(\emph{State v. Guinn}。
cf.  \emph{State v. Collier}, \emph{Commonwealth v. Appleby})。

シェリル・ハンナは、BDSMの実践において真に合意のある行為に対して人々が起訴されるケースは極めて稀だと主張する。
刑事訴追がおこなわれる大半のケースでは、被告は前述の「ラフセックス・ディフェンス」を利用し、非合意の暴力行為について無罪を主張しようとしている。
ハンナはこう述べる。
「刑事司法制度に持ち込まれる事件の大多数において、同意はせいぜい疑わしいものだ」\citep[p.248]{hanna01:_sex_is_not_sport}。
彼女は、BDSMの非犯罪化はこうした事件の訴追をより困難にするだろうと考えている。
また、スポーツとの類比は誤ったものだとも主張する。
彼女によれば、コンタクトスポーツのルールは、主に男性同士の間で対等な立場における暴力を制御するために発展してきたのに対し、ラフセックスは通常、男性が女性に対して暴力的に振る舞うことを伴うからだ(ibid., p.254)。

しかし、BDSMの非犯罪化が必要であり、その実践者を保護する法律が制定されるべきだと主張する者もいる。
彼らは、いくつかの論拠を提示している。
まず第一に、刑事訴追が稀な地域であっても、BDSMを実践する人々は頻繁に差別の標的にされると指摘する。
彼らの行為は法的に認められた権利ではなく、むしろ厳密には違法とされているため、自らのライフスタイルを理由に標的にされた際に法的救済を受けることができない。
雇用主は、従業員がキンキーであることを理由に自由に解雇することができる\citep{keenan14:_can_you_reall}。
BDSM活動は、親権争いにおいて問題視されることがあり、それによって養育能力がないと判断される場合もある\citep{zavadski15:_paren_can_lose}。
2008年に全米セクシュアル・フリーダム連合(National Coalition for Sexual Freedom, NCSF)が実施した調査によると、キンキーの人々のうち4分の1が何らかの差別を経験したと報告されている\citep{sexual08:_secon_nation_survey_violen_discr}。
スーザン・ライトによれば、その後この数は著しく減少したものの、問題は依然として解決されていない。
マスター・ガーディアンと名乗るドムの一人は、筆者のインタビューに対し、「BDSMが理由で迫害を受ける人がいるって? いるに決まってんだろう!」と語っている。

第二に、BDSMが非犯罪化されれば、非合意のラフセックスの被害者に対する保護が実際に強化されると主張する。
現在、BDSMの状況下で非合意の行為が発生した場合、被害者が検察に起訴を求めても受理されにくいことがある。
また、たとえ起訴されたとしても、通常の暴行罪として扱われ、性的暴行として訴追されることはない。
これにより、被害者は匿名性の保護を受けることができない。
ライトは、この法律の問題を改革することが現在彼女の団体の最優先課題だと述べている。
彼らは、BDSMの状況下で暴行を受けた人々を保護するためのモデル法案を起草するべく、アメリカ法律協会(American Law Institute)と協力して取り組んでいる\citep{mcarthur16:_its_traves_that_bdsm_isnt_techn_legal}。

第三に、BDSMを保護するための法律が制定されれば、家庭内虐待の加害者を訴追する際にも有効に機能する可能性がある。
BDSM行為の特徴を明確に定義することで、警察や検察官が実際の虐待を見極める手助けとなり、加害者が被害者の「同意」を主張して責任を逃れようとすることをより困難にできる。
法律は、これらの特徴を明確に理解させるための指針として機能しうる。

第四に、BDSMを法的に認めることは、キンキーなセックスに対する社会的スティグマを払拭し、それを正当なアイデンティティとして主張することを可能にする。
ゲルフ大学でカップル・家族療法を教え、北米のBDSMシーンで活動しているルース・ノイシュティフターは、社会の態度が変化しつつあるとはいえ、「多くの人々の「普通」という概念からはまだ非常にかけ離れており、多くの人を怖がらせるものだ」と指摘する\citep{mcarthur16:_its_traves_that_bdsm_isnt_techn_legal}。

法律は単に何が許され、何が禁止されるのかを示すだけではない。
それは社会の価値観を伝える役割も果たす。
たとえ同性間の性行為を禁じる法律がほとんど執行されなかったとしても、それらの法律は同性愛嫌悪を正当化し、ゲイの人々に対し、自分たちが主流社会の一部ではないという明確なメッセージを送っていた。
同様に、BDSMを権利として認めないことは、キンキーな人々に対し、自分たちは「普通ではない」というメッセージを発していることになる。
BDSMの権利を明確に認めることは、単に「キンキーなセックスが許される」というだけでなく、それを正当な性的アイデンティティとして承認するという強い効果を持つだろう。

\section{本章のまとめ}

人々がキンクに興味を持つのは今に始まったことではない。
サディズムの語源となったマルキ・ド・サドは18世紀に、マゾヒズムの語源となったレオポルト・フォン・ザッヘル=マゾッホは19世紀に生きていた。
そして彼らが初めてこうした性的実践をおこなったわけではないことは確かだ。
しかし、今日ほどBDSMに関する認識が広がり、キンクな嗜好を持つ人々が自らのアイデンティティを隠さずに済む時代はなかった。

興味深いことに、BDSMが社会的に受け入れられるにつれ、「バニラ」の人々{\DDASH}つまり、BDSMや他の「\ruby{代替的}{オルタナ}セックス実践に興味をもたない人々{\DDASH}へのスティグマを懸念する声も聞かれるようになった。
ローレン・マッキオン\ig{(Lauren McKeon)}は次のように述べている。
「私は長年にわたりセックスポジティブなコミュニティを研究してきたが、その中で「バニラは悪い」という主張にしばしば出会った」。
彼女は、「性的自己実現は冒険的であることと必ずしも相関しない」という認識の重要性を強調する\citep{mckeon17:_when_it_comes}。
これは正しい。
BDSMが万人向けのものではないのは確かだ。
そして人々がBDSMに見出すスリルは複雑なもので、快楽と苦痛の関係、さらにはセックスと権力の関係といった複雑な要素を含んでいる。
こうした特性ゆえに、BDSMを実践するには、パートナー同士が率直に話し合い、深い信頼関係を築くことが必要不可欠だ。
しかし、このリスクを適切に管理するために、BDSMコミュニティが発展させてきた同意とコミュニケーション、そしてアフターケアというモデルは革新的で刺激的なものだ。

\phantomsection
\section{討論のための問い}

\begin{enumerate}
    \item すべての当事者が同意し、楽しんでいる場合でも、「倒錯的」または「\ruby{異常}{アブノーマル}」と形容できるセクシュアル・アクティビティは存在するだろうか?
    \item よく知らない相手とのカジュアルセックスと、継続的な「セックスフレンド」関係の間には倫理的に重要な違いがあるだろうか?
    \item デートにおける人種差別は、倫理的観点から見て、背の高い相手や痩せた相手を好むといった外見に基づく他の好みとは異なるだろうか?
    \item BDSMへの嗜好は正当な性的アイデンティティと見なされるだろうか?もしそうであるなら、宗教的少数派など他の集団に与えられている法的保護を受けるべきだろうか?
\end{enumerate}

\chapter{同意}

数年前に話題となった動画があり、それは人々に性的同意について教えることを目的としていた。
その動画では、同意を紅茶に喩え、自分が誰かに紅茶を差し出す場面を想像するように言う。
「もしその人が「いらない」「ノー」「紅茶はあまり好きじゃないんだ」と言った場合、それでも紅茶を渡しますか?おそらく渡さないでしょう」とナレーターは語る。
この比喩の基本的なメッセージはシンプルで強力だ。
それは、両者がセックスをするという明確な意思を持ち、それぞれその意志をはっきりと表明しなければ、私たちは「他者との境界線」(personal boundaries)を侵害するようなことを始めてはならないのだと思い出させることを意図したものだ。
これが意味するのは、よく知られたスローガンである「ノーはノーを意味する」(No Means No)ということ、そして私たちは常にこれを受け入れなければならないということだ。\citep{may15:_consen}。

「セックスはすべての関与者が自由に同意した場合にのみ成立する」という基本メッセージはシンプルだ。しかし、\#MeToo運動は、人々(ほとんど例外なく男性)が、特に他者に対して権力を及ぼす地位にいる際に、いかにこの基本的なルールを無視し続けているかを浮き彫りにした。
女性たちはみずから名乗り出て、一連の衝撃的な証言の中で、名高い権力者たちからハラスメントや暴行を受けた事例を次々と暴露した。
こうした人目を引く事件の背後には、同じくらい陰鬱な日常の現実がある。
セクシュアルハラスメントや性的暴行は日常茶飯事だ。
米国のある統計によると、女性の約5人に1人、そして男性の1.4%に性的暴行の被害経験があるとされている\citep{black11:_nation_intim_partn_sexual_violen_survey}。
さらに多くの人々が、セクシュアルハラスメントやその他の望まない性的な関心の押しつけの被害を受けている。
一方で、性的暴行の有罪判決は稀である。また、有罪判決を受けたとしても加害者には軽い刑罰しか科されないことが多い。
悪名高い例を挙げると、2016年にスタンフォード大学のキャンパスで意識を失っている女性を暴行したブロック・ターナーは、わずか6か月の刑を宣告されたにすぎない\citep{stack16:_light_senten_brock}。

私たちは社会として同意をもっと真剣に受け止める必要がある。
これには、性的暴力の被害者をより適切に支援することを含まれる。
また、同意をもっとよく理解しようと努めることも必要であり、そのためには同意に関連する複雑な問題について考えることが求められる。
まずは、性的同意が紅茶を飲むことに同意することとはどう異なっているかについて考えることが有益だろう。

まず第一に、セックスは紅茶よりもはるかに大きな利害を伴う。
人が他人に無理やり紅茶を飲ませることはめったにない。
そして、もし私たちがそんなことをされたら、それは奇妙で不快な経験になるだろう。
だが、それは性的暴行を受けた場合に感じる侵害とはほとんど比較にならないのはほぼ間違いがない。
セックスにおいては、同意を\ruby{蹂躙}{じゅうりん}されることは、他の多くの文脈では経験しないような深いトラウマを引き起こす。
身体的な暴行ですら、一般的には、性的暴行が引き起こすような種類の苦痛や長期的な被害を引き起こすことはない。

第二に、性的な場面では、他の文脈に比べて同意の問題が論争の対象になりやすい。
これは部分的には性的行為の性質によるものだ。性的行為は、なにより、自由な同意があるかどうかによって、親密で喜ばしいものにもなれば、トラウマ的で犯罪的なものにもなりうる。
誰かが殴られたりナイフで刺されたりした場合、それが同意に基づくものかどうかを議論する必要はほとんどない。
なぜなら、人は通常そうした行為に同意しないからだ{\DDASH}仮に同意するとしても、それは非常に特殊な文脈に限られる。
しかし、同意のあるセックスも同意のないセックスも、どちらも日常的に発生している。
また、セックスはほとんど常にプライベートな場面でおこなわれるため、実際に争われるケースでは一方の証言が他方の証言に対立する形になりがちだ。

第三に、性的同意は社会的慣習やタブーの複雑なネットワークに埋めこまれているため、紅茶について話すほどオープンに話し合うのが難しい。
多くの場合、人々は同意を非言語的に、あるいは間接的に示すものだ。
これは、多くの文化がセックスを恥ずかしいことだとしてきたことに一因がある。
また女性は特にセックスについて複雑なタブーや期待に直面させられている。
異性愛の\ruby{性的接触}{エンカウンター}は、女性が男性よりも権力を持たない社会という広い枠組みの中でおこなわれる。
また、同性間の接触も、同性愛嫌悪が根深く、ゲイやレズビアンがスティグマや暴力に直面する社会でおこなわれる。

こうした理由やその他の理由から、同意はセックス倫理において常に中心的なトピックにならざるをえないが、それはまた複雑な問題でもある。
一部の論者は、セックス倫理の議論では、同意が過度に重視されていると主張している。
さすがに同意を完全に無視できると主張する人は存在しない。だが、私たちは同意の適切な役割を誤解していると考える哲学者や作家は存在する。
私はまずこうした人々の議論を概観する。
その後、同意のさまざまな定義と、同意のないセックスが引き起こす被害の性質について議論し、
また特定の状況下で誰かが同意する能力を持つのかどうかを巡る議論を調査する。さらに、同意の基準を再定義する必要があるかどうかを検討する。つまり、我々は同意の構成要素に関する異なるモデル、たとえば「肯定的同意モデル」や「熱烈同意モデル」を採用すべきかどうかを検討する。
最後に、性的コミュニケーションの代替モデルについて考察する。

\section{同意の限界}

かつては、どのようなセックスが道徳的に認められ、法的に許されるかを判断する際に、同意はそれほど中心的な役割を果たしていなかった。
ロビン・ウェスト\ig{Robin West}は、同意が西洋社会においてセックスや他の私たちの生活において中心的役割を果たすようになった事情を次のように説明する。
何世紀にもわたり、セックスに関しても他のあらゆることと同様に、人々の権利や責任は地位によって決定されていた。
セックスで重要だったのは、関係者が結婚しているかどうかだった。
婚外セックスは不道徳と見なされ、しばしば刑罰の対象とされた。
結婚できないゲイやレズビアンは、当時の道徳的および法的基準に照らしてセックスをおこなうことが許されなかった\citep[p.7]{west20:_consen_legit_dysph}。
結婚している場合、セックスは義務とされ、妻には夫の求めを拒否する権利はなかった。
フェミニスト、ゲイアクティビスト、その他の改革者たちは、セックスに関する道徳と法律の両方において、同意を否定的・肯定的な基準として確立するために何年も闘ってきた。
つまり、彼らは、セックスをしたい時にそれに合意できる権利と、したくない時に拒否できる権利{\DDASH}結婚してても{\DDASH}を認めるように求めたのだ。

性道徳と法の枠組みが同意に基づくものへと移行したことは、大きな成果と正当に評価されている(この変化については本書5.1節で論じる)。
しかし、同意のみに焦点を当て、イゴール・プリモラッツが言うところの「道徳的に許容されるセックスの試金石」\citep{primoratz01:_sexual_moral}にすると、私たちの性的行動に対する道徳的判断において、同意と同様に重要な他の考慮すべき事柄を見落とすことがあると主張する人々もいる。
これらの批判は、保守派と進歩派のスペクトラムの両端からも寄せられている。

保守派は、同意が人々の性的選択を導くために十分な条件なのかどうかについてさまざまな理由を挙げて疑問を呈している。
彼らは同意が性的活動のための重要かつ必要不可欠な前提条件であることを否定してはいない。
しかし、グレイシー・オルムステッドは\emph{American Conservative}の2016年の記事のタイトルで「同意だけの問題ではない」と述べている\citep{olmstead16:_its_not_just_consen}。
保守派の人々は、ロス・ダウザットが「より厚みのあるセックス倫理」(thicker sexual ethics)と呼ぶものを求めるべきだと主張しており、これは「同意だけ」にとどまらない倫理とされている\citep{douthat17:_age_consen_its_discon}。
ダウザットは、1960年代から1970年代にかけての「セックス革命」で登場した同意ベースの道徳体系に起因する社会問題のカタログを挙げている。
それには結婚の減少、離婚や一人親家庭の増加、出生率の低下による人口問題などが含まれる。
また、デヴィッド・フレンチ\ig{David French}は、同意に焦点を当てることで社会がよりセクシュアル化(sexualize)されてしまったと主張している。

\begin{quote}
同意に焦点を当てた道徳の実際的な結果は、すべてがセクシュアル化(sexualization)されることだ。
欲望だけが一線を引く基準とされることで、もはやセックスが関係ない空間は存在しなくなった。
仕事の会議やレストランさえも、情熱的な密会の場所になりえる。
すでに決まった相手と交際していても、あるいは結婚していてすらも、潜在的な\ruby{その場のセックスだけの関係}{フックアップ}に対する防火壁とはならない。
\citep{french17:_its_past_time}
\end{quote}

保守派は、同意を重視するアプローチが性道徳に関する他の制約を考慮する必要性を無視していると考えている。
彼らは、こうした制約を考慮に入れた方が、個人および社会全体の利益により適うと主張する。
彼らはたとえば、クリスティン・エンバが言うように、「慎重さ、節度、敬意といった美徳、さらには愛を再び取り入れる必要がある」と考えている\citep{emba17:_lets_rethin_sex}。
また、保守派は、特定の性的行為はあまりにも人を貶めるものであるので、誰も同意すべきではないと主張する。
ロッド・ドレハー\ig{Rod Dreher}は、ある女性が過激なポルノ作品の撮影に参加した体験を語った記事について、「すべての参加者が同意していたとしても、そのポルノ撮影が道徳的に正当化されるだろうか? 私たちの判断では、彼女が同意したからといってその行為が道徳的に正当化されるわけではない」と述べている\citep{dreher13:_porn_cultur_consen}。(本書2.5.2節および5.2.3節を参照。)

一方、フェミニスト哲学者たちは、まったく異なる観点から同意を焦点にする考え方を批判している。
彼女たちは、それがセックスのおこなわれ方を偏狭でジェンダー化された捉え方に限定してしまうと主張する。
クィル・レベッカ・ククラは「セックスをめぐる交渉について語る際に、同意と拒否にほぼ排他的に焦点を当ててしまうことは、セックス倫理と性的コミュニケーションの理解に深刻な\ruby{歪}{ゆが}みと悪影響をもたらした」と述べている\citep[p.75]{kukla18:_thats_what_she_said}。
また、メラニー・ベレスは、同意に焦点を当てることが性的な接触を本質的に問題含みのものにしてしまうと指摘している。

\begin{quote}
この視点は、すべての性的活動が道徳的に問題含みであり、また強制的なものとして始まると仮定してしまっている。
まずはじめにセックスは「悪い」ものであり、それが「良い」ものに変更されなければならないとする想定は、すべての人を潜在的に性的犯罪を犯す加害者、つまり潜在的な犯罪者に仕立て上げてしまうことになる。
暴力的なセックスは性的活動全体のごく少数を占めるにすぎないという現実にもかかわらずだ。
\citep[p.102]{beres07:_spont_sexual_consen}
\end{quote}

同意に焦点を当てることは、「行為者/被行為者」モデルのセックスを前提としていると批判されている。
このモデルでは、一方の人物(通常は男性であると想定されている)がセックスを提案し、もう一方の人物(通常は女性と想定されている)がその提案を受け入れるか拒否するという構図が描かれる。
このセックスのモデルは、古くからある不正確なジェンダー・ステレオタイプを助長している。
つまり、男性が女性にセックスを求め、女性がそれを男性に与えるか拒否する門番の役割を担うという考え方だ。
キャロル・ペイトマンは次のように述べている。

\begin{quote}
「同意」という言葉の伝統的な使い方は、性別の「\ruby{自然的}{ナチュラル}な」特徴や性的なダブルスタンダードに関する信念を強化する助けとなっている。
……同意は常に何かに対して与えられるものとされる。そして、男女関係においては常に、女性は男性に対して同意するものと考えられている。
「\ruby{自然的}{ナチュラル}に」優れ、積極的で、性的に攻撃的な男性が主導権を握り、契約を提示する。
一方で「\ruby{自然的}{ナチュラル}に」従属的で受動的な女性がそれに「同意する」のだ。
\citep[p.164]{pateman80:_woman_consen}
\end{quote}

ペイトマンはこのモデルには多くの問題があると指摘している。
まず第一に、このモデルは明らかに異性愛者間のセックスをその模範としている。
第二に、このモデルは常に男性がセックスを主導し、女性はしばしば愛情や物質的支援といった他のものと引き換えにいやいやながらセックスをするものだと仮定している。
第三に、このモデルは女性に男性の行動に対する最終的な責任を負わせるものだ。

また、フェミニスト哲学者たちは、特定の状況で個人の同意に焦点を当てることが、より広い社会的文脈を無視することになると主張している。
つまり個人の決定は、常に家父長制的な力関係に埋めこまれている、という点を無視することになるというのだ。
彼女たちは、より広い文化的文脈が私たちの嗜好や選択を形作っていることを指摘する。
メーガン・マーフィー\ig{Meghan Murphy}は、「同意は魔法である」と考える\ruby{心的態度}{エートス}が、これらのもっと大きな問題を無視していると主張する\citep{murphy13:_tyran_consen}。
マーサ・チャマラスは次のように説明している。

\begin{quote}
  「同意」という言葉の社会的意味は、男性が能動的に性的関係を主導し、女性がその主導に応答する受動的な役割を割り当てられるという不平等な性的関係のシステムに本質的に結びついている。
抽象的に見れば、同意はジェンダー中立的なものかもしれない。
しかし、実際には、女性が男性と平等な条件で性的関係を主導する機会を持たない限り、同意という概念は、女性が性的関係において受動的な立場であることが適切なのだということを示唆し続ける。
\citep[pp.814-815]{chamallas88:_consen_equal_legalb}
\end{quote}

こうした広い文脈を無視してしまうことで、私たちは、女性がセックスに同意するかどうかを決定する際にしばしば感じている\ruby{圧力}{プレッシャー}を見落としてしまう。
この圧力には、経済的な脆弱性、拒否した場合の暴力の暗黙の脅威、女性は性的にオープンであると同時に性的に抑制的であるべきだという社会的期待などが含まれる。

一部のラディカルフェミニスト思想家は、さらに踏み込むべきだと主張している。
彼女たちは、家父長制の支配があまりにも強力かつ広範囲に及んでいるため、最も日常的な性的なやり取りであっても女性の同意の有効性を疑うべきだと考える。
強制されたセックスはすべての異性愛的セックスが含まれる連続体の一部にすぎず、どれも完全に同意に基づくものとは見なせないというのだ。
キャサリン・マッキノン\ig{Catharine MacKinnon}はこう述べている。
「レイプの不正さを定義するのがこれほど難しいのは、レイプが\ruby{性交}{インタコース}とは別物であるという疑いのない前提から始まっているからだろう。
しかし、男性支配という条件下では、女性にとってこの二つを区別するのは困難だ」(MacKinnon, 1989, p.175; cf. Dworkin, 1987, pp.142-143)。
\nocite{mackinnon89:_towar_femin_theor_of_state}\nocite{dworkin87:_inter}
ロビン・モーガン\ig{Robin Morgan}は「アメリカ中の上品な人々の結婚生活の寝室のほとんどが、毎晩のレイプの場となっている」と述べている\citep[pp.136-137]{morgan80:_theor_and_pract}。

このような急進的な見解は、女性の性的主体性を軽視するものだと批判されてきた。
リン・ヘンダーソン\ig{Lynne Henderson}は、同意に対する急進的批判者たちは「女性の意思や経験を否定し、皮肉なことに、異性愛関係における女性のコントロールの欠如をさらに強化している」と指摘する\citep[p.56]{henderson93:_gettin_know}。
さらに、ロビン・ウェスト\ig{Robin West}は、急進派の批判が暴力的な攻撃事例への焦点をそらす効果を持つと述べている。

\begin{quote}
仮にレイプが本当に至るところでおこなわれているものであるならば、犯罪とされるべき同意のない性的行為がいまだ犯罪化されていないという主張はごく当たり前に真になる。
しかし、それがその政治的主張の本質部分であって真剣な刑法改革の基盤とはなりうるものではないと見なされるならば、結果的にそれにはたいした意味がない。
もしすべてのセックスが文字通りにレイプであるならば、刑法が標的とするべき不正行為を定義する基準が存在しなくなるからだ。\citep{west10:_sex_law_consen}
\end{quote}

マッキノン自身は、すべての異性愛的セックスをレイプと同一視したことはないと強調しているが、その批判者たちは彼女の前提に基づけばその結論に至るのは避けられないと主張している。

この問題に対してどの立場をとるにせよ、重要なフェミニスト的研究は、性暴力と倫理的なセックスの間に広がるグレーゾーンの存在を指摘している。
人々、特に女性は、同意はあるものの、望んでいないセックス、不満なセックス、あるいはトラウマ的なセックスを経験することがしばしばある。
このようなセックスは法的な暴行の基準を満たさないかもしれないが、それでもなお深刻な問題を引き起こしうる。
ニコラ・ゲイヴィー\ig{Nicola Gavey}は、このようなセックスが起こりえる状況を記録しており、次のように説明している。

\begin{quote}
こうした状況には、男性が実際の身体的強制や暴力を伴わない形でプレッシャーをかけ、それに対して女性が抵抗できないと感じる状況や、男性が荒々しく乱暴な態度をとったために女性がセックスから逃がれられないと感じ、受け入れることになった状況が含まれる。
また、男性パートナーが直接的な強制をおこなわなかったにもかかわらず、女性がセックスを避ける権利がないと感じたり、拒否する方法がわからなかったために望まないセックスに応じることになったケースも含まれる。
\citep[p.136]{gavey04:_just_sex}
\end{quote}

シャルリーン・ミューレンハードは複数の共同研究者とともに、男女の間でどれほど多くの人が同意はあるが望んでいないセックスを経験しているかを示す研究をおこなっている\citep[cf.][]{muehlenhard05:_wantin_not_wantin_sex,peterson07:_concep_wanted_women_consen_noncon_sexual_exper}。
ロビン・ウェスト\ig{Robin West}は、望んでいないが同意のあるセックスが引き起こす被害を説明している。
彼女は次のように述べている。

\begin{quote}
同意の上ではあるが、快楽も欲望も伴わないセックスに応じる女性たちは、少なくとも四つのかたちで自己の感覚に現実の損傷を被る可能性がある。第一に、彼女たちは自己主張の能力に損傷を受けるかもしれない。すなわち、快楽・欲望・動機・行為とのあいだの「心的なつながり」とでも言うべきものが、弱まるか、あるいは断ち切られてしまうのである……。第二に、望まぬセックスに同意した女性たちは、自己所有の感覚に傷を負うおそれがある。
第三に、女性が、パートナーの愛情や経済的地位に対する依存(それが現実のものであれ、そう感じられるものであれ)ゆえに、望んでいない快楽のないセックスに同意する場合、彼女たちは自律の感覚を損なう。というのも、それによって、独立に必要な自己維持力を確保するために本来取るべき手段を怠ったことになるからだ。
そして第四に、こうした快楽も欲望も伴わない性行為のあとに、自分は全体を楽しんだなどという事実に反する発言{\DDASH}いわば「快楽の嘘」{\DDASH}がなされるかぎりにおいて、そうした行為に応じた女性たちは、自らの誠実性の感覚に重大な損傷を受けることになる。\citep[p.53]{west95:_harms_of_consen_sex}
\end{quote}

これらの論者の誰も、同意を完全に無視すべきだと結論づけているわけではない\footnote{しかし、たとえば\citet{ichikawa20:_presup_consen}を見よ。
イチカワは、同意をセックスの必要条件とも十分条件ともせずに、同意のないセックスを重大な不正行為だとする枠組みを描こうとしている。
\nocite{ichikawa20:_presup_consen}}。
むしろ彼らの研究は、同意に過度に焦点を当てた倫理的枠組みの限界を認識しておく必要があることを示すことを目的としたものだ。
ここからは、同意をどのように定義するか、そしてどのような条件下でその同意が有効と見なされるのかを議論する。

\section{同意を定義する}

ロビン・ウェスト\ig{Robin West}が指摘するように、同意は私たちの法律や道徳の基礎的な概念になっているため、同意が何であるか、またその有効な同意の基準について単一の厳密な定義が存在すると思われるかもしれない。
しかし、実際にはそうではない。
法律には単一かつ明確な定義といったものは存在しない。
また、セックスに関する一般の意識を調査すると、ある状況で同意するということが何を意味するのかについて多くの人が不確かであることがわかっている\citep[pp.462-463]{muehlenhard16:_compl_sexual_consen_colleg_studen}。

性的活動の文脈において、哲学者たちは時に同意を、干渉されない権利を一時的放棄することとして定義する。
この干渉されない権利はすべての人がもっている基本的な権利だ。
この権利は、人を他者からの望まない接触や、望まない、あるいは不当な行動制限から保護する。
干渉されない権利は、私たちが自分自身の身体と、身体的な\ruby{境界}{バウンダリー}をコントロールする権利をもっているという考えに基づいている。
誰かが私たちの許可なく密接な形で接触してきた場合、この権利が侵害されたことになる。
しかし、私たちがその行為に同意している場合には、権利の侵害は起こらない。
ハイジ・ハード\ig{Hurd}は、同意を「道徳的な変換力」(moral transformative)または「道徳的な魔法」(moral magic)と表現している。
彼女は、まったく同じ行為がある人物に対しておこなわれたとしても、それが無害なものか犯罪かは、その人物がその行為に同意しているかどうかで決まるとして、
次のように印象的な言葉で説明している。
「同意は、レイプを愛の営みに、誘拐を日曜日のドライブに、暴行をフットボールのタックルに、窃盗をプレゼントに、不法侵入をディナーパーティーに変える」\citep[pp.503-504]{hurd05:_blamin_victim}。

同意の性質をめぐる議論では、どのような状況下でならば、ある人がこの干渉されない権利を一時的に放棄(waive)していると言えるのかという問題に焦点が当てられている。
これには二つの競合する理論が存在する。
同意の内部説(internal theory)と外部説(external theory)だ。
この競合する理論によれば、同意を個人の「心の状態」として定義するか(内部説)、または同意を行為、具体的には個人がセックスに対する意思を表現する行為として捉えるか(外部説)によって異なる(内部説の例として\citet[pp. 124--125]{hurd96:_moral_magic_consen}、\citet[pp.166--167]{alexander96:_moral_magic_consen_ii}、外部説の例として\citet[p.69]{brett98:_sexual_offen_consen}、\citet[p.422]{schulhofer05:_rape_twilig_zone}を見よ)。
多くの場合、個人の心の状態は行動と一致しているものであり、その行動は心の状態を明らかにしている。
つまり、何かをしたい、または何かが起こることを望むという心の状態は、その意思を行動として表現することにつながっている。
しかし、このつながりが断たれる場合もある。
裁判所が性的暴行事件で内部説または外部説のどちらを適用するかについては一貫していない(内部説の例として\emph{People v. Bink}、外部説の例として\emph{People v. Burnham}を参照)。

一部の哲学者は同意の内部説を擁護している。
彼らは、非同意のセックスによる被害は個人の自律的意志{\DDASH}つまりセックスをしたくないという欲求{\DDASH}が侵害されたことから生じるため、もし自律が侵害されていないのならば不正はおこなわれていないと主張する(Bryden 2000, p.255, Hurd, 1996, pp.124-125, Alexander, 1996, p.165)。
\nocite{bryden00:_redef_rape}\nocite{hurd96:_moral_magic_consen}\nocite{alexander96:_moral_magic_consen_ii}
内部説を支持する立場からは、刑法は一般的に私たちが実際に経験する被害から守ることを目的としているのであって、私たちが被害として経験するはずだと他の人が判断するものから守るものではないと主張できるだろう。
たとえば、ある人物Aがある人物Bの同意を得ずにBに対してある行為をおこなったが、それにもかかわらずBがAの行為を歓迎していた場合には、Aに対してその行為を重大な不正行為として責任を問うことは不当だと考えられる。
また、おそらくもっと重要なケースとして、一方が同意を表明したが実際にはセックスを望んでいなかった場合がある。
テキサス州で起きたある事件では、一人の男がナイフを持って女性のアパートに押し入り、セックスを要求した。
女性はコンドームを使うという条件でセックスに同意した。
このケースでは、女性が同意を表明したとして大陪審(起訴陪審)は男を起訴しなかった。
しかし、このケースでは女性は明らかにレイプされており、内部説はその理由を説明してくれる。
たしかに彼女は同意を表明したが、彼女にはセックスをしたいという欲求が欠けていたからだ\citep[cf.][p.137]{hurd96:_moral_magic_consen}。

外部説の擁護者たちは、このようなケースは自分たちの理論でも説明できると主張する。
また、上にあげたもう一方のケース、つまりAがBに対してセックスを開始して、Bが実際にはセックスを望んでいたが結局その意思を一切表明しなかったというようなケースは、現実世界では非常に稀だと主張する。
法律は一般的なルールを確立し比較的よくあるケースをカバーして、全体として市民を被害から守ることを目指さなければならない。
私たちは人々を望まない性的接触から保護したいと考えており、外部説の擁護者たちは、その最善の方法はすべての性的行為の前に同意の表明を必要とすることだと主張する。
ジョン・ダナハーは次のように述べている。

\begin{quote}
  パフォーマンス説〔外部説〕の良い点の一つは、同意があったと主張する側にその証明責任があることだ。
リベラルな社会においては、人々のデフォルトの地位は「身体的な干渉を受けない権利をもっている」というものだ。
同意の存在は私たちをこのデフォルトの地位から移行させるが、その移行には正当な理由が必要だ。
そうした理由を提供できるのは、客観的に評価可能な証拠だけだ。
\citep{danaher13:_some_notes_consen_sexual_offen_part_one}
\end{quote}

外部説の擁護者たちは、他人の心の状態を判断するのは、その人の実際の行動を見る以外には非常に難しいと主張する。
アメリカ法律協会(American Law Institute)は、改訂モデル刑法(revised Model Penal Code)の中で同意を外部的に定義する決定を正当化し、次のように述べている。

\begin{quote}
「同意」の主観的な定義には魅力があるが、実際に機能するのは行動に基づく同意の概念だけである。
同意の有無を評価する際、事実認定者は、申し立て者の主観的な感情と、それが被告人に対すしてはっきり知らされたかの両方を理解するために、必然的に当事者たちの行動と言葉に注目しなければならない。
観察可能な行為がなければ、同意の存在を推定したり、被告人が法的な境界を超えたかどうかを判断する公平かつ信頼できる方法は存在しない(American Law  Institute, ``Model Penal Code: Sexual Assault and Related Offences,'' Discussion Draft No. 2, Apr. 28, 2015, § 213.9 cmt., p.145)。
\end{quote}

捜査や裁判の文脈において、被害者の心の状態を判断することは難しいだけでなく、侵襲的でトラウマを引き起こす可能性もある。
また捜査官や被告側弁護士が、被害者の服装や被告との過去の関係といった文脈的要素を指摘し、そうしたものが被害者の心の状態を示す証拠だと主張する余地を与えてしまう。

だが、たとえ外部説を受け入れたとしても、具体的な状況で全当事者が同意していたかどうかを判断する困難さが常に解決されるというわけではない。
同意はセックスを望んでいるという欲求を明示的に表現する形でおこなわれることもあるが、必ずしもそうとは限らない。
ワートハイマーが述べているように、人々が同意を表明するのは、「明示的な場合もあれば暗黙的な場合もあり、言語的な場合もあれば非言語的な場合もある」\citep[p.346]{wertheimer03:_consen_sexual_relat}。
同意が表現される多様な方法が困難な問題を生みだすのだ。
あとで見るように、一部の人々はセックスを始める前に明確な肯定的同意を要求すべきだと主張する{\DDASH}つまり、どんなものであれ親密な接触を開始する前に、明確かつ現在進行形の承認を得ることが必要だというのだ。
のちに、この基準を規範的基準として確立することについての議論を評価する。
だが、事実として、日常生活においては同意はしばしば微妙な方法で与えられていることは確かだ。

ここからは、難しいが重要な問題について議論しておきたい。
それは、非同意のセックスが引き起こす被害だ。
多くの哲学者たちが、非同意のセックスによる被害の性質を分析し、その影響を理解する助けとなる研究をおこなっている。

\section{同意のないセックスの害悪}

哲学者カリン・フリードマン\ig{Karyn L. Freedman}による著書『パリでの一時間』(\emph{One Hour in Paris} \citep{freedman14:_one_hour_paris})は、若い頃にヨーロッパ旅行中に暴力的なレイプ被害を受けた自身の体験を語る一人称の記録だ。
この書籍は衝撃的でありながらも美しいもので、フリードマン自身の体験を振り返ると同時に、トラウマと心的外傷後ストレス障害(PTSD)の心理を探求している。
また彼女はアフリカを旅し、現地の女性たちに彼女たち自身の性的暴行の体験とその後の影響について話を聞いた。
性的暴行がもたらす被害は、他の形態の身体的暴行被害者が経験するものを超えている。
これは他の暴力が引き起こす被害を軽視するものではない。
しかし、哲学者たちは非同意のセックスがこれらと区別されるいくつかの特徴を指摘している。

まず第一に、性的暴行被害者が他の暴力犯罪の被害者よりも大きなトラウマを経験するという証拠がある。
データによれば、レイプ被害者は心的外傷後ストレス障害(PTSD)のリスクが高く、ほぼすべての他の犯罪被害者よりも自殺願望や自殺未遂の報告が多い\citep[p.104]{wertheimer03:_consen_sexual_relat}。
フリードマン\ig{Karyn L. Freedman}はレイプ被害が彼女にもたらした永続的な影響を衝撃的に語っている。

\begin{quote}
毎日が苦しい闘いだった。
私は感情的な断絶状態で人生を送っていた。
毎朝、〔被害を受けた〕8月1日の映像が押し寄せ、沈んだ気持ちで目を覚ました。
日常のルーチンをこなしながらも、一種の霧の中にいるようで、レイプの記憶から思考を持続的にそらすことができなかった。
\citep{freedman14:_one_hour_paris}
\end{quote}

デヴィッド・ベネター\ig{David Benatar}は、性的暴行が特別な被害をもたらす理由は、人々が自然に、セックスに特別な意義を与えていると考える場合にのみ説明できると主張している(本書2.1.3節参照)。
ジーン・ハンプトン\ig{Jean Hampton}が指摘するように、「私たちのセクシュアリティはそれぞれにとって非常に重要であり、自己認識において中心的な役割を果たす」ため、性的暴行がトラウマを引き起こす一因であることは確かだ\citep[p.151]{hampton99:_defin_wrong_and_defin_rape}。しかし、これを説明するためにセックスの「特殊意義説」(significance view)を採用すべきだとするベネター\ig{Benatar}の意見に同意する哲学者はほとんどいない。

確かに、多くの人々はセックスに特別な意義を認めており、これは性的暴行被害者が感じる侵害の一部を説明している。
だが、セックスという行為に特別な意義を認めるかどうかにかかわらず、ほぼすべての人はまた別のあるものに価値を認めている。すなわち、「性的自律」(sexual autonomy)だ。
つまり、私たちは親密な接触に関して自分の身体とその境界をコントロールする能力を重視しているのだ。
フリードマン\ig{Karyn L. Freedman}が述べるように、性的暴力は「被害者の個人としての自律を侵害し、その身体的一体性(bodily integrity)を破壊する行為」である。
セックスそのものにどんな意義を認めるかにかかわらず、セックスにかかわる暴行の身体的侵害の程度は、ほとんどの他の身体的暴行が引き起こすものを超えている。
性的暴行は被害者を特に内密なしかたで標的にするものであり、被害者を自分の目的のためにコントロールするという形で自律を侵害する。
ローリー・カルホーン\ig{Laurie Calhoun}は次のように述べている。
「レイプは侵略の犯罪である。加害者は他者を自分の個人的な所有物として扱い、彼女が自己決定権と平和と安寧に生きる権利をもった知性と感性を持つ人間であるという事実を完全に無視して行動する」\citep[p.109]{calhoun97:_rape}。

哲学者たちは、性的暴行が被害者に与えるもう一つの危害の形態として、その「表出的価値」(expressive value)に注目している。
性的暴行は被害者に対する侮蔑の態度と、彼女の人間としての価値を無視する行為を表現するものだ。
ハンプトンは次のように述べている。
「この行為の表出的内容{\DDASH}犯行の遂行とその結果の両方において{\DDASH}が、加害者を主人として、被害者を劣位の\ruby{対象}{オブジェクト}として表現する点に、その不正行為の理由がある」\citep[p.135]{hampton99:_defin_wrong_and_defin_rape}。
レイプは単に性的欲望を満たすための手段ではない。
その欲望の目的自体が相手の主体性を覆し、無視することなのだ。
マーサ・ヌスバウムはこの点を力強く次のように表現している。

\begin{quote}
その欲望は、死体との性交によって満たされるものではないし、ましてや動物とでも満たされない。
そこで性的に興奮を引き起こしているのは、まさに自分の心の暗い片隅ではすでに人間だと認識している相手を物体に変える行為、
相手を「誰か」ではなく、「何か」に変化させるその行為だ。
\citep[p.281]{nussbaum95:_objec}
\end{quote}

レイプは、人間がなしえる最も完全な形のモノ化(objectification)の一つだ。
フリードマン\ig{Karyn L. Freedman}はレイプを「モノ化され、身体が強制的に、性的に、暴力的に利用されるという経験だ」と説明している。

この「表出的」な被害は、レイプ被害者の大多数が女性だという事実と密接に関連している。
家父長制社会においては、レイプは女性の従属的な地位を肯定するという特別な表出的機能を果たしている。
ハンプトンは、性的暴行は単に被害者個人に害を与えるだけではないと言う。性的暴行はすべての女性に対しても害を及ぼしているのだ。
彼女は次のように述べている。
「私たちの社会で発生するレイプの最も重要な側面の一つは、それが単なる被害者個人への道徳的損害にとどまらず、多くの男性が多くの女性に対して男性としての主導権を確立しようとするパターンの一部である点だ」\citep[p.135]{hampton99:_defin_wrong_and_defin_rape}。
これは、男性がレイプされないという意味ではないし、レイプされた男性が被害を受けないというわけでもない。
男性被害者も確かに深刻な被害を受けており、これは真剣に受け止めるべきだ。
しかし、女性にとってレイプは繰り返され続けている脅威である。つまり、それは女性が常に警戒しておかなければならない常在的な危険として存在している。そのため、これを憎悪犯罪(hate crime)の一形態とみなすべきだという意見もある\citep{campo-engelstein16:_rape_hate_crime}。

性的暴行の影響は、レイプおよびレイプ被害者に対する社会的態度によってさらに悪化している。
フリードマンは「社会的、文化的、宗教的、政治的状況すべてがレイプ被害の主観的な体験の質に影響を与えている」と指摘している。
被害者はしばしば被暴行後に恥の感覚をもつことがあり、またしばしばスティグマに直面する。
このスティグマへの恐れから多くの被害者が犯罪を通報するのを妨げられており、また周囲の人々の目に対して、自分は損なわれた存在だという持続的な感覚をもちつづけてしまうことがある。
このスティグマは男性被害者にも影響を与えている。
実際のところ、スティグマの恐れが男性被害者が女性被害者よりも性的暴行を報告する可能性をさらに低くしているかもしれない\citep{mezey87:_male_victim_sexual_assaul}。

ここからは能力の問題に移りたい。
つまり、誰かが自由に同意したと言えるために満たすべき条件についての議論だ。

\section{同意の能力}

ある人がセックスに同意するためには、その人は同意する能力を有していなければならない。
セックスへの同意は、契約書への署名や医療処置への同意など、他の生活領域で用いられる条件と同様のものを満たす必要がある。
同意は、自発的(voluntary)であり、判断能力があり(competent)、情報に基づいた(informed)ものでなければならない。
銃を突きつけられた状況で本当の同意をおこなうことはできない。
また、ある人がはっきりとした思考ができないほど泥酔している場合や、幼い子供、判断能力が損なわれるような精神病エピソードの最中にある場合も同様だ。
そして、同意者は自分が何に同意しているのかを理解していなければならない。

私たちは性的同意に関する道徳的・法的なルールを必要としている。それは、
自由で合理的な意思決定ができない状況の人々を守るようなものであるべきだ。
しかし同時に、私たちは人々の積極的自律(positive autonomy)も保護するべきだ。
つまり、同意の基準を過度に厳しすぎるものにしないようにし、特に個人のアイデンティティや幸福にとって重要なセックスに関しては、彼らが自分の意思で自分の人生を生きる権利を妨げないようにしなければならない。
誰かに「あなたはセックスに同意する資格がない」と告げることは、\ruby{おせっかい}{パトロナイジング}で\ruby{保護主義的}{パターナル}な行為になりかねない。
それには擁護可能でもっともな理由が必要だ。
したがって、性的同意に関する道徳的・法的ルールは、望んでいないセックスから人々を守ることと、人々が自分の身体を自由に扱う権利を保護すること、この二つの競合する目的のバランスをとる必要がある。

同意とその能力に関する哲学的アプローチにはおおむね二つの一般的な方向性がある。
第一のアプローチは、同意の有効性に対して画一的制限を支持するものだ。
画一的制限(categorical restriction、「類型的制限」とも訳される)とは、すべての関連ケースに適用される厳格で明細的なルールだ。
たとえば、私たちは飲酒運転の判断については画一的制限を採用している。
私たちは、人によって同じ量のアルコールに対して異なる反応を示すことを知りながらも、法律は血中アルコール濃度の具体的な法定上限を設定している。
この上限を超えた場合、実際の精神状態にかかわらず運転は許されない。

第二のアプローチは文脈的なアプローチだ。
文脈主義者は、同意の能力は状況や関わっている人々に応じて変化する可能性があると考え、上のような厳格なルールに反対する。
道徳的・法的なルールを提案する際には、哲学者は消極的自律と積極的自律のバランスをとる必要がある。
つまり、人々を不当な干渉から守る一方で、人々が本当にそうしたいと思っているときにセックスする判断能力を過度に制限しないようにすることが重要だ。
文脈主義者は、自分たちのアプローチの方が人々の積極的自律をより適切に保護できると主張している。

以下で議論するすべての問題に対して単一のアプローチを採用する必要はない。
たとえば、同意可能な年齢については類型的アプローチを採用し、酩酊下での同意については文脈的アプローチを採用することも可能だ。
また、これをさらに複雑にする要因として、法律が何を義務として要求するべきかということと、私たちが道徳的・倫理的に何を考えるべきかということについては、別の意見をもつ可能性があることを指摘しておくべきだ。
たしかに、非同意のセックスは重大な犯罪であり、、同意が損なわれていると思われる状況はある。しかし、その行為が不道徳であるとしても、必ずしも「暴行」というレベルのものではないと考えるべき場合があるかもしれない。

\subsection{強制、影響、権力}

同意は自発的(voluntary)でなければならない。
前述の通り、銃を突きつけられて契約書に署名してもそれは無効であるのと同様に、\ruby{有形力}{フォース}〔物理的な力〕やその脅しによってセックスへの同意を得た場合、その同意は自発的になされたものだとはみなされない。
有形力や脅迫を用いて誰かを強制することは明らかに不正だ。
強制的なセックスはレイプだ。
伝統的には、有形力の行使は性的暴行の十分条件であるだけでなく、必要条件でもあると考えられてきた。
法律は、過去には、レイプの有罪判決を下すために暴力が使用されたことを検察が立証することを要求していた。

現在では、性的暴行法において有形力という要件はほぼ排除されている。
この法的な変化は、すべての強制が暴力やその脅迫を通じておこなわれるわけではないという事実を反映している。
強制はより微妙な形で現れることもある。
第一に、ある人物が有形力を伴わない脅迫を用いて相手にセックスを強要することがある。
たとえば、相手が拒否した場合には、昇進を認めない、関係を終わらせる、自分はとても腹が立つだろう、などと告げるといった脅迫がある。
第二に、ある人物が相手を脅迫する代わりに、何か相手が欲しいものをセックスと引き換えに提供することがある。
たとえば、相手に昇給や高い成績を与えるという申し出がそうだ。
第三に、脅迫や誘因がなくても、当事者間に権力の不均衡が存在していて、弱い立場にある人が拒否できないと感じてしまう場合がある。

アメリカでは、裁判所が強制と「\ruby{苦渋の選択}{ハードチョイス}」とを区別することがある。
ある人がしぶしぶ同意したとしても、その人に他に選択肢がある場合には、その同意は有効だと裁判所が判断することがあった\citep[p.130]{buchhandler-raphael11:_failur_consen}。
この\ruby{判断手法}{ルーリング}に基づくと、ある人が合理的に別の行動をとることができた場合、その人は強制されていなかったことになり、同意は有効とされる。
しかし、これは強制・同意のアプローチとしては、多くの人が受け入れようとするものよりもかなり狭い、最小主義的なものだ。
こうした判断手法は、同意が完全に自由なものとは私たちが考えないような多くの状況において、人々を保護しないものになってしまう。
たとえば、雇用者が被雇用者に対して、仕事を続けるためには自分とセックスをしなければならないと強要した場合、たしかにその被雇用者には他の選択肢がある。
彼女は仕事を失うことを受けいれることができる、と言えないこともない。
しかし、多くの人は、この状況での同意は自由になされたものではないと考えるだろう。

強制についての別の定義は、人が要求に従うことを期待されている状況におかれており、その要求を拒否した場合には「事情が悪化する」(worse off)場合を強制とみなすものだ。
しかし、この場合は「事情が悪化する」とは何かを定義する必要がある。
たとえば、AがBに「ディナーをご馳走するからセックスをしよう」と提案した場合、Bが拒否するとBの事情は悪化する、と言うことは可能だ{\DDASH}Bは無料のディナーを食べられない。
しかし、多くの人はこれをBは強制されているとは考えないだろう。
一つの答えとして、セックスの要求がなされる前の状態より悪化しない場合、強制とは見なされない、とする考えがある。
この定義に基づくと、ディナーを拒否してもBの状況は元の状態より悪化していない。
しかし、この基準も厳しすぎると考えられる。
たとえば、ある人が本来は昇給するべき状況にいる場合に、上司から昇給をセックスと引き換えにすると言われた場合、それを拒否したとしてもその人の状況は以前よりも悪化していない。しかしこれは強制であるように思われる。

ワートハイマーは別の基準を提案している。
彼は、人が拒否した場合に「自分が持つべき権利を下回る状態」になる場合を強制とみなすべきだと主張する。
この「道徳的基準線」(moralized baseline)という概念は、ロバート・ノージック\ig{Robert Nozick}から採用されたものだ。
彼によれば、もし人が拒否した結果として道徳的基準線を下回るならば、その同意は無効だ\citep[pp.167--169]{wertheimer03:_consen_sexual_relat}。
たとえば、当人には、それに見合った昇給を受ける権利があるのに、上司がセックスをその引き換え条件とした場合、それを拒否すると道徳的基準線を下回ることになるため、その同意は強制された同意だと見なされる。
しかし、上司が当人に見合った昇給ではなく「セックスすれば当人に見合わない高い昇給を与える」と申し出た場合、拒否しても道徳的基準線を下回ることはなく、この同意は有効とされる。
同様に、教授が「セックスしなければ成績を下げる」と脅した場合、拒否すれば学生は道徳的基準線を下回る状況に置かれるため、その提案は強制と見なされる。
しかし、「成績を本来のものよりも上げる」という申し出の場合、その提案は強制とは見なされず、同意は有効とされる。

ワートハイマーの見解は直観的には理解しやすい。
しかし、それでもなお不確定のところが残っている。
まず第一に、脅迫が本当に強制的なものとなるためには、どの程度深刻な\ruby{福利}{ウェルビーイング}の損失である必要があるのかを判断しなければならない。
哲学者が好む奇妙な喩えだが、金魚を食べるぞと脅された場合、ほとんどの人はこれを強制とは見なさないだろう(ただし、映画『ワンダとダイヤと優しい奴ら』に登場する動物好きのマイケル・ペイリン演じるキャラクターは別だ)。
また、ある人が相手に同意させるために、自分を傷つける(たとえば手首を切る)と脅すケースも考えられる\citep[p.280]{husak06:_compl_guide_consen_sex}。

さらに、サラ・コンリーが述べるように、誰かが「なだめすかし、丸め込み、おだてあげ、懇願し、熱弁し、威圧して」同意を得ようとする場合をどう扱うかも検討する必要がある\citep[p.115]{conly04:seductionrapecoercion}。
このような場合、その人は相手の事情を悪化させるような明確な脅迫をしているわけではないが、相手の意思の弱さや善意につけこんで搾取しようとしている可能性がある。
また、AがBに対して脅迫ではなく「\ruby{提案}{オファー}」をするケースもある。

ワートハイマーの見解は、さらに、Aが道徳的に問題のある行動を取っている場合でもBの同意が有効である可能性があることを示唆してしまう。
たとえば、上司が従業員に不当な昇給を申し出たり、教授が学生に本来受けるべきでない成績向上を持ちかけたりする場合、これらの行為が道徳的に非難されるべきものであることは確かであり、違法と見なされる場合もあるかもしれない。
しかし、ワートハイマーは、それらの行為の不道徳性そのものが、従業員や学生の同意を無効にすべきではないと考える。
私たちがワートハイマーの見解を受け入れるかどうかは、有効な同意の基準が、同意を求める側の道徳的な問題とどれだけ密接に関連していると考えるかにかかっている。
誰かが道徳的に非難される方法で同意を得たとしても、それがなお有効だと合意できないかぎりは、ワートハイマーの見解は納得がいくものにはならないだろう。

一部の哲学者は、強制を、性的アプローチを受けた人の自律に与える影響という観点から定義すべきだと主張している。
彼らは、AがBを本当の選択をおこなう能力が制約される状況に置いた場合、それが強制だと考える。
モニカ・カワート\ig{Monica Cowart}は、自由な同意のためには、複数の選択肢が存在し、その選択肢の中には本人の利益や欲求に合致するものが含まれている必要があると主張している\citep{cowart04:_under_acts_consen}。
ジェームズ・ロウチャ\ig{James Rocha}も同様に次のように言う。

\begin{quote}
申し出が強制的であり不正であるのは、その申し出が相手の自律を抑制しようとする意図をもつか、あるいは、その状況で申し出をすることが相手の自律を侵害する可能性が高いと合理的に予見できる場合である。
ある申し出が強制的であるというのは、相手の自律を尊重しておらず、自律を保ったままでは抜け出すことが難しい困難な状況に追い込んでいるからである。\nocite{rocha11:_sexual_haras_coerc_offer} \nocite{cahill01:_rethin_rape}。
(Rocha, 2011, p.204; cf. Cahill, 2001, p.204)

\end{quote}

これらの自律に基づいた説明を法的な基準として確立することは、ワートハイマーが提案しているような基準よりも困難かもしれない。
そのため、私たちは道徳的領域と法的領域で異なる基準を採用せざるをえないことになるかもしれない。
この章の冒頭で、同意がないセックスと倫理的かつ快楽的なセックスとの間には重要なグレーゾーンが存在すると述べた。
ワートハイマーの「道徳的基準線」の基準を満たしているが、それでもなお相手の自律が尊重されていないようなケースが、このグレーゾーンに該当するケースの一例となるのかもしれない。

\subsection{年齢}

一定の年齢以下の子供はセックスに同意することができない。
法律上、同意年齢〔age of consent、同意可能になる年齢〕がどこに設定されるかは大きく異なるが、12歳未満に設定されている地域はほとんど見られない。
社会が子供とのセックスを禁止する理由は数多くある。
かつては、子供は無垢な存在であり、性的な意識や性的な感情を一切持たず、性的な接触によって堕落させられてしまうものだと信じられていた。
だが、現在ではこの見方が正しくないことがわかっている。
子供を対象に研究する心理学者たちは、「性的発達は乳幼児期に始まる」という見解をかなり以前から間示してきた\citep{louie19:_sexual_behav_child}。
幼い年齢であっても、子供がセックスに興味をもちある程度理解することは正常であり、むしろ避けられない現象だ。
デヴィッド・アーチャードは、子供が無垢であるという信念は誤りであるばかりか危険でさえあると主張する。
この信念は、結果的に、子供をエロティックな対象として見てしまうことにつながるだけでなく、現実的には適切な性教育を提供する必要性を否定してしまうことになる\citep[pp.118--119]{archard98:_sexual_consen}。
また、幼少期に性的虐待を受けた被害者をなんらかの意味で堕落してしまったと見るべきではない。

成人が子供とセックスをすることによって生じる害は別のところにある。
まず第一に、子供は身体的に傷つきやすく、成人との性的接触によって身体的リスクが発生する点が挙げられる。
また、子供は依存状態にあり、自分の欲望を自由に表明したり、自分の意思で選択したりすることが難しい。
子供たちは自分自身の行動や生活状況をコントロールすることができない。
子供は、身体的に優位にあり、保護者であり提供者である大人の言うがままにならざるをえない。
さらに、子供は自然に大人を尊敬し、権威ある存在とみなす傾向がある。
性的接触をおこなう大人は、多くの場合、子供に対して特別に権威的な立場にある。
こうした要素のすべてが、子供にとって成人とのセックスを拒否する自由な選択を不可能にしている。

また子供は、セックスに同意するために必要な判断をおこなうことができない。
彼らは他の大人の性格や意図を正確に判断することができず、その相手が自分にとって脅威であるかどうか、あるいは自分の利益を考慮してくれるかどうかを見極めることが難しい立場にある。
同時に、子供が他の大人とのセックスから何らかの利益を得るとは考えにくい。
一部には、これらの要因がすべての子供と成人の性的関係において必ずしも成立しているわけではないと主張する人々がいる。
仮にそれが事実であったとしても、そのような例外は非常に稀であり、事前に判断することは不可能だ。
したがって、子供と成人の性的行為を強く、一般的に禁止する十分な理由がある\footnote{ \citet{malon15:_adult_child_sex_limit_liber_sexual_moral}を見よ。
ただし成人と子供の間のセックスの有害性について疑問を呈している論者もいる。
たとえば\citet{levine02:_harmf_minor}、\citet{brongersma90:_boy_lover_their_influen_boys}を見よ。}。
\nocite{levine02:_harmf_minor} \nocite{brongersma90:_boy_lover_their_influen_boys}

子供がティーンエイジャーになると、状況はさらに複雑になる。
彼らは性的関係を本当に欲求するようになり、それを求めはじめる{\DDASH}通常は同世代の相手との関係だが、時には年上の人と関係を持ちたがることもある。
性的な発達は極めて個人的なプロセスであり、また社会的文脈にも大きく依存する。
フィッシェルは次のように提案している。

\begin{quote}
私たちは若者が性や性的選択、性行為の影響を完全に理解する神経学的な瞬間などといったものを発見することに焦点を当てるべきではない。
そのような瞬間が幻想にすぎないことを示す社会学的証拠は十分に存在する。
教育、\ruby{より安全な}{セーファー}セックス性教育、各種のリソース、ジェンダー、生育地域、家族関係、政治などがすべて、若者の性的主体性や同意能力を仲介し、抑制し、強化し、あるいは他の方法で形作る要因となっている。
\citep[p.305]{fischel10:_per_se_power}
\end{quote}

すべてのティーンエイジャーが必要な能力を発達させる特定の一時点を特定することは不可能かもしれないが、それでも私たちはティーンの性的活動を管理する一般的原則を策定しなければならない。
道徳的判断や法的判断を無数の個別ケースに合わせて調整することはできないからだ。
これらの一般的原則は、若者を搾取から保護する必要性と、彼らが自らの人生を決定する能力や権利が芽生えつつあることを尊重する必要性を、できる限りバランスよく調和させなければならない。
その最善の方法については意見が分かれている。

ティーンエイジャーの性的活動にはリスクが伴うため、パートナーの年齢にかかわらず、すべてのティーンがセックスを控えるべきだと考える人もいる。
ティーンの\ruby{性的禁欲}{アブスティナンス}の推奨は、多くの地域で社会政策の一環となっている。
これは以下の一つまたは両方の方法でおこなわれる。
アメリカ政府が資金提供する「禁欲オンリー教育」(abstinence-only education)は、「性的活動を控えることによって得られる社会的、心理的、健康上の利点を教えることを唯一の目的とする」としている。
また、未成年の性行為を犯罪化する法律を通じてもおこなわれる\footnote{Title V, Section 510 (b)(2)(A-H) of the Social Security Act (P.L. 104--193). \url{https://aspe.hhs.gov/reports/impacts-four-title-v-section-510-abstinence-education-programs-1}.}。
これらの政策手段はさまざまな社会的圧力によって補強されることがある。
たとえば、民間団体は「純潔の誓い」(``virginity pledges'')を通じて若者に禁欲を促すキャンペーンを展開する。
アメリカでは、これらの団体の一部は福音派教会と深いつながりを持つが、連邦政府から資金提供を受けているものもある。

ティーンの性的禁欲を推奨する政策は、婚前交渉の不道徳さに関するもっと一般的な信念に基づいている場合が多い。
しかし、このような政策の支持者たちは若者特有の要因にも訴えている。
まず第一に、ティーンエイジャーはリスクの高い性的行動をとる可能性が高いと指摘されている。
成人よりも避妊具を使用しない傾向が強く、その結果、性感染症にかかったり妊娠したりするリスクが高い\footnote{未成年者が性的意思決定に関して未発達な認知プロセスしか持たないことを示す証拠としては、\citet{drobac14:_neurob_decis_makin_high_risk}を見よ。}。
また、ティーンエイジャーは望まない妊娠に対処するための経済的および心理的能力が成人よりも低い。
子供の親となったティーンエイジャーたちは州の支援を必要とする可能性が高く、また教育からドロップアウトしてしまう可能性が高いという懸念もある。
さらに、思春期の性的活動それ自体が有害であるかもしれないという懸念がある。
ティーンエイジャーの性的選択は衝動に駆られやすいもので、その結果として彼らはしばしば後に自分の選択を後悔しており、さらに、心理的なダメージを受けることがあると主張されている。
禁欲推進派は、性的に活発なティーンがうつ病にかかりやすく、他の危険行動にも走りやすいというデータを指摘している\citep[pp.163--170]{hallfors05:_which_comes_first_adoles}。

ミシェル・オーバーマンは、特にティーンエイジャーの少女たちの脆弱性を強調している。
彼女は、思春期の少女たちがしばしばセックスを強要され、セックスをしたのちにそれを後悔し、望まない性的接触によって感情的および身体的な被害を受けることが多いという証拠を提示している。
オーバーマンは次のように述べている。

\begin{quote}
思春期に内在する脆弱性、たとえば極端に低い自尊心、変化しつつある自分の身体についての\ruby{両義的な感覚}{アンヴィヴァレンス}、自己主張することへの強いためらいなどが、ティーンエイジャーを自分が完全には望んでいない、あるいは一部すら望んでいない性的接触に同意させる要因となっている。
\citep[p.709]{oberman01:_girls_master_house}
\end{quote}

しかしながら、禁欲のみを推奨する方針は、いくつかの理由で批判されている。
まず、批判者たちはこの方針は実際にはティーンエイジャーの性的活動率を低下させていないと主張している。
また、禁欲の推奨はティーンが避妊具を使用する可能性を低下させることで、結果的にティーンのセックスをよりリスクの高いものにしてしまっていると指摘している。
禁欲オンリー教育に関する証拠を調査した研究者たちは、性感染症と望まない妊娠を防ぐ手段として「たしかに禁欲は理論上は100%効果的だが、現実にはその効果はほぼゼロに近い」と結論づけている\citep{santelli06:_abstin_abstin_only_educat}。

第二に、批判者たちは、ティーンがセックスをしないようにと抑制あるいは阻止しようとする際、意図せずして望ましくないメッセージを彼らに送ってしまうと主張している。
ティーンにはセックスに同意する能力がないと宣言することで、彼らに「自分たちの選択は重要ではなく、また自分たちの欲求は正当なものではない」と伝えてしまうのだ。
またそれは、セックスが悪いもの、あるいは不正なものだと教えることになる。
さらに、禁欲が性教育の中心となる場合、本来教育者が焦点を当てるべきもの、すなわち同意とオープンなコミュニケーションの重要性から注意を逸らしてしまう。

ティーンの禁欲を推進するために最も強力な法的手段は、ある年齢未満の者を対象とする性的行動をすべて犯罪化する法律だ。
批判者たちは、禁欲オンリー教育と同様に、この種の法律も逆効果を招く可能性があると主張している。
これらの法律は、若者が大人に助言や支援を求めることをためらわせてしまう。
また、ティーンエイジャーも実際には性的に活動しているという現実を、大人たちが受け入れることを妨げてしまう。
そして、権威ある人物が、多くのティーン自身が無害で望ましいと見なしている行動を禁止しようとすることで、結果的に権威への尊敬を損ない、さらには法律そのものへの全般的な不信感を生むことになる。

ティーンエイジャーの性行為を刑事罰の対象とすることは、明らかに不公正な起訴につながる。
たとえば、たった13歳で、同意のあるセックスをおこなったという理由で起訴されたケースが存在する。
このような起訴は、その対象者にとって破滅的な結果をもたらす可能性がある。
被告たちに軽い判決や執行猶予が与えられたとしても、そうした人々は学校を中退したり、仕事を解雇されたり、社会的支援プログラムの利用資格を失ったりすることがあり、また彼らは一生の間、性犯罪者リストに登録されてしまう可能性がある{\DDASH}多くの場合、犯罪がおこなわれた時点で彼ら自身が\ruby{同意年齢以下}{アンダーエイジ}であったにもかかわらずだ。
起訴された者が18歳以上であっても、被告と被害者の年齢がごく近いケースが見受けられる。
カリフォルニア州のデータによると、同州では少なくとも法定レイプ罪で起訴された人の30%が20歳未満であり、42%が被害者と5歳未満の年齢差であったことが示されている\citep[pp.51--52]{ccasa2008}。

法律は検察官に対して非常に広範な裁量権を与えており、検察官はそれを常に、集団によって異なる影響を及ぼすような形で行使している。
起訴の傾向は、関与者の人種や性別によって大きく異なり、男子は女子よりも頻繁に起訴される。
また、同性愛関係をもった男子や、人種的マイノリティに属する者は、その他の者よりも高い頻度で起訴される傾向にある(James, 2009, p.246; see Waites, 2005)。
\nocite{james09:_romeo_juliet_were_sex_offen}\nocite{waites04:_age_consen_sexual_consen}

若者の性的同意が有効かどうかを判断する際には、当事者間の年齢の差に焦点を当てるべきだと主張する者もいる。
多くの法域では、いわゆる「年齢差法」(age-gap laws)を導入しており、これは大きな年齢差がある者同士の性行為を犯罪化するものだ。
アメリカのほぼすべての州では、法定レイプ法に何らかの年齢差規程を組み込んでいる。
これらの規定において、同意年齢をどこまで低く設定すべきか、また許容される年齢差をどれくらいにすべきかについてはさまざまな見解が存在する。
たとえば、2歳以上の年齢差を禁じるべきだという提案もあれば\citep{waites04:_age_consen_sexual_consen}、当事者の年齢に応じて許容される年齢差を変えるべきだという提案もある。
具体的には、14歳の場合は3~4歳の差を許容し、17歳の場合は1~2歳に制限する、といった設定が考えられる。
また、年齢差の程度に応じて処罰のレベルを変えることも提案されている。

年齢差規定は、十代の若者(かなり幼い場合も含む)がセックスをおこなっているという明白な現実を認めつつ、若者を搾取から保護することを目的としている。
これにより、マシュー・ウェイツが「ティーンエイジャーの性的市民権」(sexual citizenship)と呼ぶものを承認することができる。
若者のセックスを全面的に禁止しようとしないことで、若者がセックスについて考え、話し合い、自らの選択について深く考えることを奨励する。
また、セックスそのものが本質的に悪いものだという誤ったメッセージを送ることを防ぐことができる。
年齢差に焦点を当てることで、法的には成人とされるが、年齢が近い未成年者とセックスをおこなった人々を保護することもできる。
先述したように、このようなケースでの起訴は実際に珍しくなく、起訴された者にとって悲惨な結果をもたらすことがある。

ただし、年齢差規定にも批判がある。
特に重要な点として、同年代のティーンエイジャー間の性的関係に潜んでいるかもしれない強制的な要素を見逃してしまう結果につながるという主張がある。
画一的な同意年齢を擁護するアラン・ワートハイマーは次のように述べている。

\begin{quote}
年齢差の大きなカップルのセックスが強制的なものでありやすいということはありえることだ。
そして、こうした強制的という理由で理由で問題になるようなセックスの割合は、同年代間の関係では年齢差のある関係よりも低いかもしれない。
これはそうかもしれないし、そうでないかもしれない。
しかし、年齢差が実際に強制の適切な代理的指標になるという証拠がない限り、この議論は年齢差アプローチを支持する根拠にはならない。
\citep[p.218]{wertheimer03:_consen_sexual_relat}
\end{quote}

オーバーマンは、年齢差規定が、当事者たちの年齢が近い場合に強制的なセックスのケースを検察官が単に無視してしまう結果を招き、特にティーンの女子が同年代の男子たちによる搾取に対して脆弱になると懸念している\citep[p.751]{oberman01:_girls_master_house}。

別の提案として、大人が若者に対して権力を行使する状況に特化して若者を保護する方法がある。
フィッシェルは次のように述べている。

\begin{quote}
未成年間のセックスを規制する際の中心となるべきは、単なる年齢や年齢差ではなく、信頼や権威や依存という関係であるべきだ。
依存関係を年齢差よりも上位に置くことで、私たちがもつ法的な想像力はかなり違ったものになる。
そこでは権力差とその乱用が標的となり、若者たちは判断能力と選択の意思をもつ存在としてみなされることになる。
つまり、法はもはや単に若者はセックスに「イエス」と言えないだろう想定するのではなく、特定の依存関係においては合理的に「ノー」と言えないことがあると考えるようになる。
\citep[pp.315--316]{fischel10:_per_se_power}
\end{quote}

この提案は、ティーンエイジャーの自律を承認するという点で前に挙げた提案よりさらに先に進んでいる。
しかし、これにも年齢差規定と同じ制約がある。
つまり、これもまた若者間の強制の問題に対応していないのだ。
若い人々、特に女子たちが、同年代の仲間から搾取される可能性があると信じるならば、この提案はその問題を無視しているだけでなく、権力関係がない関係においては問題はないはずだと暗黙のうちに示唆することで、問題を悪化させる可能性がある。

また、この提案はどのような種類の権力関係がそのターゲットに該当するかという難しい問題にも直面する。
義理の親、教師、スポーツのコーチのような場合は明確であろうが、もっと曖昧なケースもある。
たとえば、18歳の女性水泳インストラクターが、16歳の生徒と交際関係を持つことは禁じられるべきだろうか?

スチュアート・グリーン\ig{Stuart Green}は、権力関係が存在する場合には、その関係を「厳格責任」(strict liability)制度における決定的証拠としてではなく、強制があることの「反証可能な推定」(rebuttable presumption)とみなすべきだと提案している。
彼は次のように述べている。

\begin{quote}
このような制度では、被告が法律の文言に違反しているとしても、自分たちの関係が実際には真の同意に基づいていたことを示す積極的な証拠を提示することが許される。
こうした解決策が完全なものとは言えない。
それは無実の被告のプライバシーに侵入し、また被告にこの種の未立証の容疑に伴う\ruby{汚名}{スティグマ}を与えることになる。
しかし、少なくとも厳格責任制度の潜在的な過剰包摂効果を軽減することができるだろう。
\citep{green17:_how_crimin_inces}
\end{quote}

ティーンエイジャーに対して、高めの年齢を単一の同意年齢として設定するか、あるいは、低めの年齢に設定して年齢差規定を導入するか、あるいは、大人が権威的な立場にある関係を規制しようとするか、といったことはどうあれ、双方がそれを望んでいるケースの起訴を防ぎつつ、若者を搾取から守るためのさまざまな改革が考えられる。
ハイジ・キトロッサーは、18歳未満のセックスについて、肯定的同意(affirmative consent)の要件を導入することを提案している\citep{kitrosser97:_meanin_consen}
(「肯定的同意」要件の長所と短所については、後述する)。
他の研究者たちは、ティーンエイジャーを含むセックスのすべての起訴に対して、被害者協力要件を課すことを提案している。
この要件は、起訴を進行させるためには、被害者の協力がなければならないとするものだ(Olszewski, 2006, p.707; James, p.257)。
\nocite{olszewski06:_commen_statut_rape_wiscon}\nocite{james09:_romeo_juliet_were_sex_offen}
別のアプローチとして、刑事訴追の代わりに、親が子供のパートナーに対して接近禁止命令を裁判所に請求できるようにする方法がある。
この方法についてダリル・オルシュウェスキー\ig{Olszewski}は次のように述べている。
「息子や娘が疑わしい性的行動にかかわっていることに気づいている場合、親はそれに不賛成ならば息子や娘の交際関係を禁止する裁判所命令を求めることができるようにする」\citep[p.718]{olszewski06:_commen_statut_rape_wiscon}。
このような接近禁止命令の取得プロセスは比較的簡略なものにして、その命令の条件は状況に応じて調整されるべきだ。
また、こうした命令は対象者について刑事記録を残さず、性犯罪者登録に掲載されることもないようにするべきだ。
この提案は、親がティーンエイジャーの子供の私生活に対して過剰な力を持つことになるという批判を受ける可能性があるが、少なくともこの提案は紛争を刑事裁判から遠ざけるものだし、刑事起訴よりもはるかに寛大な措置だといえる。

\subsection{一時的能力低下}

同意能力は、合法・違法を問わずさまざまな\ruby{物質}{サブスタンス}によって損なわれる可能性がある。
これにはアルコールや各種の\ruby{薬物}{ドラッグ}が含まれる。
酔った(intoxicated)状態での同意を論じる際には、話すことができない、歩けない、完全に意識を失っているといった極端な状態を扱っているのではないことを留意しなければならない。
昏睡のような状況にある人が同意できないのは明らかだ。
しかし残念ながら、現実には、裁判所が深刻な酩酊状態にあった者が同意可能だったと判断したケースが存在する。
アルコールによって判断能力が著しく損なわれた人々をもっとうまく保護する必要がある。
また、明確かつ一貫した形で同意を表明できないほど酔っている人が性的暴行から保護されるようにしなければならない。
しかし、哲学的な同意論の議論は、基本的な機能を果たすことは可能だが、薬物やアルコールによって判断が鈍っているケースに焦点を当てている。

仮に基本的な機能を果たす能力が同意のための必要条件だとしても、それは十分条件ではない。
シャロン・コーワン\ig{Cowan}は「性的な親密関係についての同意能力には、家にふらつきながら帰りつき、ドアの鍵を開けてそのままトイレで嘔吐するのに必要なレベルよりも高い認知的・合理的能力が必要だ」と述べている\citep[p.919]{cowan08:_troub_drink}。
問題は、その適切な閾値をどこに設定するかだ。
法律は酔った状態が同意能力に及ぼす影響について明確な基準を設けていない。
また、人々の道徳的見解も同様に不明確だ。

この曖昧さは飲酒運転に関する具体的な法的制限とは対照的だ。
特定の血中アルコール濃度が人々の運転能力を正確に反映しているわけではないが、飲酒運転には制限値がある。
飲酒への反応は個人によって異なり、機能低下は段階的に進むものであり、突然すべての能力が失われるわけではない。
しかし、実際に機能する法執行システムを作るためには、ある程度恣意的であっても何らかの閾値を設定する必要があるということを私たちが認めているため、国家がその基準を設けている。

一部の人は、アルコールと酔いに関して厳格な見解をとるのが最善だと主張している。
すなわち、酔った状態にある者はセックスに同意できないとする立場だ。
ジョアン・マクグレゴール\ig{Joan MacGregor}は、「女性が酔っている、もしくは薬物を使用している場合、そのこと自体が彼女が自発的に同意する能力がないものにすると結論づけるべきだ」と主張している\citep[pp.244--245]{macgregor94:_force_consen_reason_woman}。
これを「\ruby{厳格}{ストリクト}説」と呼ぶことにする。
この厳密説の立場では、相手が飲酒や薬物の摂取により判断力に影響を受けている可能性がある場合、安全な選択肢として、たとえその相手が乗り気であってもセックスを控えるべきだとされる。

厳格説を支持する主な理由は、この立場を採用すれば望まれないセックスが減る可能性があるからだ。
アルコールや薬物は、人々の反応に影響を与え、意思を明確に伝える能力を制限する\citep{koss89:_discr_analy_risk_factor_sexual}。
アルコールは抑制効果を持つため、飲酒した人は望まない行為の進展に対して反応が鈍くなる可能性がある。
また、男女では状況の認識が大きく異なってしまう場合がある。
クリスティン・チェンバース・グッドマン\ig{Christine Chambers Goodman}は、酔った女性は「準備活動」(preliminary activity)と呼ぶものをセックスの代替物として受け入れることがあるが、パートナーがそれをセックスの前段階として捉えていることを理解していない可能性があると指摘している\citep[p.79]{goodman09:_protec_party_girl}。
また、カレン・クレイマーは、女性が抵抗しないことが、伝統的な男性の積極性と女性の受動性モデルを信じる男性にとって「イエス」と解釈される恐れがあると懸念している\citep[p.121]{kramer94:_rule_myth}。
有名な刑事事件の\emph{R v. Bree}裁判では、酔った状態で望まないセックスを経験した女性が「私は自分の身体のなかにいないような感覚でした……望んでいないことはわかっていましたが、それをどうやって\ruby{止}{と}めればよいのかわかりませんでした」と証言している(\emph{R v. Bree}, p.8)。

さらに、人々は飲酒や薬物の影響下にあるとき、リスクを取りやすくなる。
避妊具を使用しない可能性が高くなり、使用した場合でも正しく使わない可能性がある。
その結果、セックス自体を後悔しない場合でも、望ましくないセックスの結果として性感染症\ig{(STI)}や望まない妊娠といった問題が発生する可能性がある\citep[p.409]{george19:_alcoh_sexual_healt_behav}。
もう一つの提案は、セックスへの同意について、飲酒運転と同様の基準を設けるというものだ。
すなわち、血中アルコール濃度\ig{(BAC)}によって具体的な法的制限を設定するという考えだ。
運転は細かい運動能力を必要とし、ちょっとしたミスが命に関わる事故を招くため、セックスへの同意の基準よりもやや高く設定すべきかもしれない。
しかし、いずれにせよ画一的で客観的な基準が設けられ、これが誰かが性的行為に有効に同意できるかどうかを決定することになる。
しかし、この提案には実際的な困難が伴い、道徳的な基準として導入することは難しい。
ほとんどの人はデートの際にアルコール検知器を持ち歩いていない。
また、法的基準としても、どのようにして実施するかが問題となる。
検査は通常、事件から数日後、あるいは数時間後におこなわれるため、検査の時点での結果が意味を持たない可能性がある。
さらに、血中アルコール濃度の基準は飲酒量しか測定できず、他の薬物使用については何も示さない。

これらの画一的アプローチ(厳格説および血中アルコール濃度基準)は、普遍的で客観的である点で利点があるものの、より柔軟なアプローチが望ましい場合もある。
たとえば、セックスの参加者のそれ以前からの関係が重要だと考えることができる。
多くの人は、すでに確立された関係にあるカップルでは、両者ともかなり酔っている状態でも同意が成立すると認める傾向がある{\DDASH}多くのカップルは一緒に飲酒し、アルコールを媚薬のように利用することがあるからだ。
一方で、それと同じレベルの信頼関係がない二人の場合、その人々の行動に対するアルコールの影響をより懸念することもある。
また、その人がどのような状況で酔ったのかも重要だ。
誰かが自由にどれくらい飲むかを選択している場合、その同意を有効とみなす方が良いと主張する人もいる。
しかし、たとえば相手がオレンジジュースにこっそりウォッカを混ぜたり、飲酒を強要したりした場合は別だ。
ただし、この主張には異議もある。
すでに酔った状態であることを理由に被害者を非難することになりがちだからだ(cf. Buchandler-Raphael, 2017, pp.1045--1049; Shaw, 2016, pp.1414--1421)\nocite{buchhandler-raphael17:_conun_volun_intox_sex}\nocite{shaw16:_title_ix_sexual_assaul_issue_effec_consen}。

どんなものであれ、画一的基準は性的自律、特に女性の性的自律に対する脅威になると懸念する人もいる。
\emph{R v. Bree}裁判では、「女性を性的暴行から保護することを目的とした規定が、結果的に自律的な成人が個人的な決断を自ら下す権利に過干渉するシステムに変わる可能性がある」と述べられている。
人々の自律を維持するため、ワートハイマーは、酔ってはいても完全に機能し、自分の意思を伝える能力がある場合は、一般的にその同意を有効と認めるべきであり、酔っている人とセックスした人を非難すべきではないと論じている。

ワートハイマーは「少なくとも中程度の酩酊状態に達するまで飲むことは、一部の人々にとって望ましい性的・社会的経験の重要な要素だ」と指摘する\citep[p.251]{wertheimer03:_consen_sexual_relat}。
多くの人が酔った状態でセックスすることを選択することは、こうしたよくある行動を刑事罰の対象にすることをためらう理由になる。
彼はより緩やかな方針を採用することのリスクを認めつつも、「法的ルールは最悪のケースによって左右されるべきではない」と主張する(ibid.)。
彼は、酩酊時のセックスに関する方針が、事前に、つまり人々がまだ酔っていない段階で選ばれるとしたら、どのような方針が選ばれるかを想像してみるよう求める。
とりわけ女性の場合を想定し、もし当該集団の人々が、シラフの状態で同意に関する方針を選ぶ際、酔っていても自分の同意が有効であることを望むのが一般的であれば、その同意は有効とみなされるべきだと彼は言う。
多少のリスクがあったとしても、酔った状態でセックスする自由を認められることによって、自分の自律が尊重される方を、彼女らは最終的には選ぶだろうというのだ。
彼の立場を「寛容説」(lax view)と呼んでよいだろう。

ワートハイマーは、画一的な立場には不整合があると指摘している。
性的接触にアルコールが関与している多くのケースでは、当事者の両方が飲酒している場合が多く、しばしば行為を主導している側の方がもう一方より酔っていることも多い。
それにもかかわらず、裁判所は主導者に責任を問う一方で、もう一方の当事者は同意をするほどの判断能力がなかったと見なしている場合がある。
ワートハイマーは「もし酩酊中におこなった不正な行為について人が責任を問われるのであれば、同様に酩酊中の人の同意も有効と見なさなければならない」と主張する\citep[p.233]{wertheimer03:_consen_sexual_relat}。

イレーネ・セイドマンとスーザン・ヴィッカーズは、厳格説と寛容説の妥協案を提案している。
それは、酔った状態での同意を認める一方で、関係者が飲酒している場合は必ず肯定的同意を求めるべしというものだ。
彼女たちは「アルコールや薬物がどの程度まで判断能力を阻害するかを判定する明確な基準がない以上、最低限の基準が必要だ」と述べている。
そして「もしアルコールが関与している場合、女性が明確な口頭による意思表示をおこなって性的に\ruby{親密な行為}{インティマシー}(挿入を含む)を望んでいると示さない限り、同意はなかったと見なされるべきだ」と主張する\citep[p.486]{seidman05:_secon_wave}。

酔った状態における同意の問題は、道徳的判断と法的判断とで異なる基準を採用せざるを得ない事例の一つだとも言える。
たとえば、ある人物Aが、酔ってはいるが同意していた人物Bとセックスしたとする。
Aを起訴しようとすれば、前に述べた困難が生じる。
すなわち、「同意が不可能なほど酔っている」とはどの程度なのかという普遍的な基準を定めることは難しいこと、さらには、しばしばA自身もBと同程度に酩酊しているという事実にも直面せざるを得ない。
しかし、私たちが自らの生活において道徳的判断をおこなう場合には、より繊細で慎重な対応が適切だ。
相手が酔っている可能性がある場合には、明確で熱烈な同意を確認すべきであり、特に相手のことをよく知らない場合には、よりいっそう慎重であるべきだ。
もちろん、恋人関係にあるからといって、相手の自由で判断力ある同意を確保する必要がなくなるわけではない。
しかし、そうした関係は、相手が薬物やアルコールの影響下でどのように反応するか、どのようにコミュニケーションをとるか、そして特定の状況で本当にセックスを望んでいる可能性がどれほどあるかといったことについて、よりよく理解する手がかりを与えてくれる。

\subsection{心的能力}

2015年4月、アイオワ州の男性ヘンリー・レイホンズは性的暴行の罪で無罪判決を受けた。
レイホンズは、アルツハイマー病を患っていた妻ドナ・ルー・レイホンズとセックスをしたとして起訴されていた。
検察は、レイホンズ夫人がもはや同意する能力を失っていたと主張したのだ。
弁護側は、検察が主張する夜にレイホンズ氏が実際に妻とセックスをしたかどうかを争うことで無罪を勝ち取った。
しかし、アイオワ州の法律に対する検察の解釈、すなわち「同意能力を失った人とのセックスは必然的に性的暴行を構成する」という点については争わなかった\citep{belluck15:_sex_demen_husban_trial_age}。

まったく異なる事例として、アンナ・スタブルフィールドのケースがある。
スタブルフィールドは哲学教授であり、2018年に重度の知的障害と脳性麻痺を持ち言葉を発せない男性に対する繰り返しの性的暴行で有罪判決を受け、12年の懲役刑を言い渡された。
スタブルフィールドはセックスは同意の上でおこなわれたと主張している。
彼女はファシリテイテッド・コミュニケーション(支援付きコミュニケーション)という技法を通じて、キーボードやアルファベットボードを用い、しばしば補助者の助けを借りて被害者の同意を得たと主張した\citep{engber15:_stran_case_anna_stubb}。

認知機能に永続的な障害を持つ人々は、セックスへの同意能力に影響を受けることがある。
生まれつき存在するダウン症候群のような知的障害もあれば、認知症や脳損傷のように後天的に獲得されるものもある。
これらの障害は、判断力や意思決定能力に多様な影響を及ぼす。
知的障害を持つ多くの人はセックスや親密な関係への関心を持ち続ける一方で、搾取や虐待を受けるリスクが非常に高い\citep{justice16:_crime_person_disab}。

知的障害を持つ人が直面する法的な状況は非常に曖昧な場合がある。
裁判所はしばしば次の二つの質問をする。
第一に、その人が性的行為の性質やそのありえる結果を理解しているか。
第二に、その行為がおこなわれる道徳的・社会的文脈を理解しているか。
しかし、これらの基準は解釈の幅が広く、障害を持つ人々は裁判所で\ruby{偏見}{バイアス}にさらされることがある。
裁判所は彼らを虐待から守る必要性を強調する一方で、彼らの積極的な自律を守ることにはあまり関心を払わない。
裁判所と同様に、親や介護者、行政当局も予防的なアプローチをとることが多いが、それはしばしば障害を持つ人々からセックスの機会を完全に奪う結果を招いてしまう\nocite{appel10:_sex_right_disab} \footnote{\citet{appel10:_sex_right_disab}を見よ。
\citet{denno97:_sexual_rape_mental_retar}は1997年までの関連する事件について包括的な研究をおこなっている。
\citet[p.86]{kulick15:_lonel_its_oppos}も見よ。
}。

これはより大きな範囲の問題を反映している。
障害を持つ人々のセクシュアリティは、しばしば社会全体から否定されるか、あるいは管理と監視の対象とされる。
アン・フィンガーはこう述べている。
「セクシュアリティは私たちの最も深い抑圧の源だ。
そして最も深い痛みの源でもある。
雇用、教育、住宅における差別について話し、それを変えるための戦略を立てることは難しくないが、セクシュアリティと生殖からの排除について話すことははるかに難しい」\citep[p.9]{finger92:_forbid_fruit}。
研究者たちは、知的障害を持つ子供(成人を含む)の親たちが「自立、自己指導、責任と自律の想定」ではなく、「依存と服従と子供っぽい行動」を促進する傾向があることを発見している\citep[p.196]{mill10:_negot_auton_famil}。
より一般的な障害者に関して、ジェニファー・スコットは次のように述べている。

\begin{quote}
身体的な\ruby{主体性}{エージェンシー}は単にセクシュアリティの中核的要素であるだけでなく、人間性の中核的要素であり、障害を持つ人々が日々取り組まざるをえない問題だ。
彼らは自分自身の身体を所有していないと見なされることが多い。
介護者、医師、さらには見知らぬ他人までもが「助ける」という名目で、あるいはその人自身よりも自分たちの方が何がその人にとって最善かを知っていると信じて、介入してくることがある。
\citep[p.218]{scott15:_can_disab_peopl_have_sex}
\end{quote}

多くの知的障害者、特に認知症を持つ多くの人々は施設で生活しており、これがさらに監視と管理の\ruby{層}{レイヤー}を加えている。
トビン・シーベルスが指摘するように、グループホーム、長期介護施設、その他の施設にいる人々は、性的パートナーとなりうる人と二人きりになることを許されないことが多い。
多くの施設では、居住者を性別ごとに分けている。
また、ポルノへのアクセスが制限され、居住者がマスターベーションするための十分なプライバシーさえ確保できないこともある\citep[p.45]{siebers12:_sexual_cultur_disab_peopl}。

ここでも、強い画一的な見解をとることが可能だ。
スティーブン・シュルホーファーは「性的犯罪に関するモデル刑法」を作成し、次のような基準を提案している。「……被害者が身体的に無力であり、心理的に欠陥がある、または心理的に無能力である場合、同意は自由に与えられたものとはみなされない」\citep[p.283]{schulhofer98:_unwan_sex}。
しかし、シュルホーファーが使っている「心理的に欠陥がある」(mentally defective)という記述は、現在の学者たちがもはや使わない表現であり、また使うべきものではない。
だが、彼の基本的な主張は、一定のレベルの知的障害のある人とのセックスを禁止すべきだというものだ。

しかし、障害者の権利活動家は一般にこのような包括的なアプローチを拒否する。
彼らは、知的障害者も他のすべての障害者と同様に、充実した性的生活を送る権利を奪われるべきではないと主張する。
また、適切な政策があれば、この目標と搾取や虐待からの保護という目標は両立可能だと考えている。
本書のさまざまな箇所で強調してきたように、セックスは多くの人にとって\ruby{幸福}{ウェルビーイング}の源泉であり、しばしば人々のアイデンティティに不可欠な要素だと認めるならば、障害者が性的な自律を行使できるように努めるべきだ\citep{vehmas19:_person_profoun_intel_disab_their_right_sex,hollomotz10:_vulner_adult,evans09:_sexual_person_relat_peopl_intel_disab,kittay10:_person_is_philos_is_polit}。
問題は、これをどう実現するかだ。
まず、セックスと障害に関するステレオタイプと闘うことから始めるべきだ。
知的障害のある女性はしばしば受動的で脆弱な存在と見なされる一方で、男性は\ruby{捕食}{プレデター}的で過剰に性欲が強いと見なされることが多い\citep{feely16:_sexual_surveil_contr_commun_based,gill15:_alread_doing_it,barrett14:_disab_mascul}。
私たちは、知的障害者を正当な欲望と目標を持つ主体として見る準備をしなければならない。
これには、完全に独立し自由な主体による選択だけを自律とみなすという考え方を再考することが含まれる。
むしろ私たちは、意思決定というものはさまざまな方法で支援されうることを認めることができる。
ジョセフ・フィッシェルは次のように述べている。
「知的障害を持つ人々はある程度の自律を達成することができるかもしれないが、その達成は個人的な変革だけでなく、社会的な変革を通じて保証される」\citep[p.146]{fischel18:screwconsent}。

エイリオノア・フリンとアンナ・アースティン=カースレイクは、このような支援付き意思決定がどのように機能するかを次のように説明している。
「支援者の役割は、その人の意思と好みが何であるかを想像し、その意思と好みに基づいて意思決定をおこなうことだ」\citep[pp.81--104]{flynn14:_legis_person}。
アレクサンダー・ボニ=サエンズは、このような支援付き意思決定モデルを「\ruby{認知}{コグニション}プラス」と呼んでいる\citep[p.1234]{boni-saenz15:_sexual_incap}。
ルーシー・シリーズは次のように説明している。
「意思決定に関連する情報を取得し理解するのを助けたり、利用可能な選択肢の利点と欠点を話し合ったり、他者とのコミュニケーションを支援したりすることができる」\citep{series15:_relat_auton_legal_capac}。
シモ・ヴェフマスは、特別な訓練を受けたソーシャルワーカーの支援を受けて作成する書面契約を提案している\citep[p.527]{vehmas19:_person_profoun_intel_disab_their_right_sex}。

ドン・クーリックとイェンス・リュードストロムは、デンマークの事例から学ぶべき教訓を紹介している。
デンマークでは、ソーシャルワーカーや介護者が「性的アドバイザー」として訓練を受け、「障害者がマスターベーションをおこなったり、パートナーとの性行為をおこなったり、セックスワーカーから性的サービスを購入したりする際に支援する」\citep[p.18]{kulick15:_lonel_its_oppos}。
これらのアドバイザーは、さまざまな種類の障害を持つ人々だけでなく、介護者、管理者、障害者と共に働く他の関係者も支援するために訓練を受ける(ibid., pp.101--105)。
クーリックらはデンマークのグループホームで、スタッフが障害者向けにロールプレイ活動やディスカッショングループを組織している様子を紹介している。
これらの活動により、参加者は性的なやりとりでのコミュニケーション方法や交渉方法、許容される行動の範囲、拒絶や別れなどの感情的なリスクへの対処方法を身につけることができる(ibid., pp.108--109)。
また、身体的および知的な障害を持つカップルの例も紹介されている。
このカップルは異なるグループホームに住んでいたが、ソーシャルワーカーが協力し、互いに訪問できるようにし、部屋を準備してカップルが性行為をおこなえる環境を整えた。
アシスタントは定期的にカップルの様子を確認し、すべてが順調であることを確かめた(ibid., pp.980--101, p.111)。

ボニ=サエンズは、裁判所がこのシステムが障害者にとって有益に機能することを確保する役割を果たせると考えている。

\begin{quote}
システムの評価は文脈依存的におこなわれ、受託者責任法の「忠実性とケアの原則」に基づいて導かれるものになるだろう。
つまり、裁判所は、そのシステムが利益相反を排除し、個人およびその性的意思決定について十分な知識を持ち、認知障害のある個人を性感染症や妊娠の脅威から保護するために合理的な措置を講じているかといったことを評価することになる。
\citep[p.1234]{boni-saenz15:_sexual_incap}
\end{quote}

ボニ=サエンズは、彼の裁判所支援型「認知プラス」モデルの利点は、進行性認知症を持つ人々が直面する特有の課題に対応できる点だと主張している。
認知症患者の多くは、認知症が進行する前から続く長期的な関係をもっている。
ボニ=サエンズは、彼のモデルからは、認知症が進行する前に、どのような条件下で性的接触が許されるかを指定しておく事前指示(advance directive)を作成できる法的枠組みが支持されるとしている。
こうした事前指示のモデルとして使うことができる法的装置がすでに存在する。
すなわち、医療ケアにおける事前指示だ。

生命維持技術の発展が進む中、医療ケアにおける事前指示はますます一般的になりつつある。
その名の通り、これらの指示は、人が情報に基づく決定をおこなえない状況や意思を伝えられない状況において、どのような治療を受けるか、または受けないかをあらかじめ指定することを可能にする。
医療における事前指示を扱う連邦法は、実際に「患者の自己決定法」(The Patient Self-Determination Act)と呼ばれている。
多くの法域で、このような事前指示に法的効力を与えるために必要な条件と規定を具体的に定める法律が制定されている\citep[cf.][]{srebnik99:_advan_direc_mental_healt_treat}。

性的同意の事前指示(advance sexual directives)の提案は、すでに確立されたこの法体系に基づいている。
ボニ=サエンズは、医療分野で使用される事前指示に非常に似た形式で、人々がどのような条件下で自分との性的行為が許されるか、または許されないかを指定する事前指示を作成できるようにすることを提案している\citep{boni-saenz15:_sexual_incap}。
この指示は、他に個人が含めたいと思うあらゆる詳細を含むことができる。
この指示を作成する際には、作成者が精神的に健全であることを確認する必要がある。
作成者は、精神的に健全である限り、いつでも指示の内容を修正または撤回する自由がある。
契約は、患者が望む限り具体的なものにすることができる。
たとえば、どのような行為が許されるかを正確に指定したり、その条件が守られているかを確認するために定期的な医療検査を受けることを条件としたりすることが可能だ。
私たちは、このような検査を義務づける法律を制定することも検討できる。
事前指示には、医療分野における事前指示の代理意思決定者のように、契約で指定されていない行為に対して許可または拒否をおこなう第三者を指定したり、その他の問題に対応する権限を付与したりすることができる。

同意の事前指示に対しては、いくつかの理由で異議を唱えることができる。
第一に、パートナー間の相互性の欠如が懸念されるかもしれない。
セックスは本質的に相互的な体験であるべきだと主張する人もいる。
そのため、この相互性が性的な場面で欠けている場合、その場面は完全な同意が得られたものではないと考えられる。
カナダ最高裁に持ち込まれた事例で、男性が意識を失っているパートナーと性交渉をおこなったケースを議論する中で、ジャニーヌ・ベネデットとイザベル・グラントは、「性的行為とは、本来両者が身体としてその場に存在し、ある種の相互的な快感や欲望を感覚する場面であるはずだ」と主張している\citep[p.80]{benedet10:confusing}。
意識を失っている人とのセックスの場合、このような相互性が明らかに欠けている。
一方、進行性疾患を患っている人とのセックスの場合、状況はより複雑だ。
その人がどの程度「\ruby{しっかりしている}{プレゼント}」と見なされるかはケースごとに異なり、判断が難しい場合もある。
その状態が進行して完全に状況を認識できなくなった時点で行為がおこなわれるケースもたしかにあるだろう。
事前同意指示は、その人がパートナーとセックスを続けることを許可することになるかもしれない。

第二に、事前に同意が与えられている場合、どのようにして同意を撤回するかという問題がある。
同意を撤回する権利は性的自己決定権の重要な一部であり、法律で保護されなければならないと法改正者たちは主張している。
しかし、同意の事前指示はこの権利を奪うリスクがある。

これらの懸念を考慮する際には、性的同意の事前指示は通常、長期にわたる信頼関係に基づいてパートナーとの間で作成されることになるものであることを忘れてはならない。
指示の内容が必ずしも順守されない可能性があることは確かだ。
しかし、このリスクを完全には排除できなくても軽減する方法はある。
定期的な医療検査などの安全対策を契約に組み込むことでリスクを軽減することができるだろう。
事前医療指示と同様に、事前性的同意指示を作成しようとする人々をガイドするためのモデル契約やベストプラクティスが確立されるだろう。
ボニ=サエンズは、認知症患者の場合、性的活動はしばしばなんらかの監視が可能な長期介護施設内でおこなわれることが多いことを指摘している{\DDASH}そして事前性的同意指示はそのような文脈以外では無効とすべきだと考えている\citep[p.43]{boni-saenz15:_sexual_incap}。

事前性的同意指示において搾取のリスクを完全に排除することはできない。
しかし、もしこのリスクが事前に十分理解され、本人が自発的にそのリスクを引き受けるのであれば、事前同意の支持者たちは、その権利を妨げるべきではないと主張している。
リスクのある性的活動に対しても、他の危険な活動と同じ一貫した基準を適用すべきだ。
人々がリスクを引き受ける前に、そのリスクを十分に理解できるようにすることは重要だ。
しかし同時に、彼らの自律に対して十分な敬意を払うべきであり、もし私たちが事前の同意を禁じるようなことをすれば、それはパターナリスティックな介入に他ならない。

\section{同意と欺瞞}

ジョイス・ショートは若くして独身であり、ウォール街で成功したキャリアを築いていた。
ある日、仕事終わりに友人たちとバーへ出かけ、そこで「非常にハンサムで洗練された若い男性」と出会った。
彼はユダヤ人で独身であり、ニューヨーク大学の会計学の学位を持っており、彼女にとって理想的な相手に思えた。
しかし、二人が交際を始めた後になって、これらの話がすべて嘘であることを知ることになる。
その後、彼女は、彼女が「欺瞞によるレイプ」と呼ぶ行為に対して人々の認識を高めるために長年にわたり活動してきた。
「私はこのことをあらゆる場所で叫び続けます」と彼女は筆者に語った。
「性器に関する自己決定権を損なうあらゆる嘘は、一種の暴行です」\citep{mcarthur16:_is_lying_get_laid_form_sexual_assaul}。

ほとんどの人は、関心を持った相手に良い印象を与えるために、真実を少し誇張したり、あるいは情報を隠したりした経験があるだろう。
しかし、ジョイスが経験したような事例{\DDASH}ベッドに引っぱりこむために相手を騙す行為{\DDASH}が実際に違法とされる地域はごくわずかだ。
ただし、いくつかの例外がある。
多くの法域では、「誘因に関する詐欺」ではなく「行為に関する詐欺」として特定の性的欺瞞を犯罪として扱っている。
たとえば、誰かが夜中にこっそり寝室に忍び込み、相手のパートナーになりすますケースがこれに該当する(こうした事例は意外に多い)。
また、医師や医師を装った者が、患者にセックスを医療行為だと信じ込ませるケースも含まれる。
たとえば、ある男性は自分が医師であると偽り、ある女性に対し、「自分とセックスすることが致命的な血液疾患を治す唯一の方法だ」と欺いた事例がある\citep{chavez87:_woman_says_ruse_trick_her_sex}。

また、ごく少数ではあるが、「ジェンダー詐欺」、あるいは出生時のジェンダーについて嘘をついたことで起訴されたケースも存在する。
1996年にコロラド州で一人が有罪判決を受け、米国ではこの事例は孤立したものだったが、英国では2012年以降、この種の起訴が急増し、5人が有罪判決を受けている\citep{wilkinson17:_troub_case_uk_woman_convic_gender_fraud}。
さらに、多くの地域では、性感染症(特にHIV陽性)であることを開示しなかった場合、あるいは嘘をついた場合に起訴される可能性がある\footnote{Centers for Disease Control and Prevention, ``HIV and STD Criminal Laws.'' \url{https://www.cdc.gov/hiv/policies/law/states/exposure.html}を見よ。〔2025年現在アクセス不能〕}。

このような特定のケースを除けば、法律は一般的に性的な欺瞞には関与しない。
カナダの裁判所は次のように述べている。
「小さなものから時には大きなものまで、古来より欺瞞はロマンスや性的関係の副産物であった。
それらはしばしば欺かれた側に害をもたらすリスクを伴う。
しかし、これまでの文明の歴史において、こうした欺瞞は悲しいものであったとしても、歌や詩、社会的な非難の領域に委ねられてきた」(\emph{R. v. Cuerrier})。
また、1975年にニューヨーク州の裁判官はこれをより簡潔に表現している。
「法律の下では、\ruby{すべての男}{エブリマン}は自由である。
紳士であることも、\ruby{軽薄者}{キャッド}であることも」(\emph{People v. Evans})。
この点で法律は一般の通念と一致しているように見える。
性的欺瞞について重要な論文を執筆したトム・ドハティは、人々が一般にほとんどの性的欺瞞を容認していると考えている。
彼によれば、大多数の哲学者は、一般の人々と同様に「寛容説」(Lenient Thesis)を支持している。
寛容説によれば、「生まれつきの髪の色、職業、恋愛感情などの個人的特徴について誤解をさせて欺いてセックスすることは、軽い不正にすぎない」\citep[p.718]{dougherty13:_sex_lies_consen}。

だがドハティ自身は寛容説を否定すべきだと考えている。
彼は、誰かを欺いてセックスすることは重大な不正行為であり、実際には一種の暴行だと主張する。
この立場を「厳格説」(Strict Thesis)と呼ぶことができるだろう。
彼の見解は彼一人のものではない。
イギリスの全国慈善団体レイプ・クライシス(Rape Crisis)のフィオナ・エルヴァインズは、これを端的に表現している。
「もし誰かを騙してセックスにもちこむ必要があるなら、あなたは加害者です」\citep{sanghani14:_lied_your_way_sex}。
この見解は、性行為における同意の役割に関する広く受け入れられている見解から導かれるように思われる。
しかし、それは多くの人々が受け入れがたいと思う結論をもたらすことにもなる。

\subsection{厳格説}

厳格説を擁護する中で、ドハティは議論の出発点として「正当な同意がない状態で誰かとセックスをすることは重大な不正である」という誰もが受け入れられるであろう前提を提示している\citep[p.722]{dougherty13:_sex_lies_consen}。
ドハティはすべての欺瞞が同意を無効にするとは考えていない。
彼は、相手が「\ruby{交渉決裂要因}{ディールブレイカー}」とみなす情報{\DDASH}すなわち、もしその情報を事前に知っていればセックスを拒否したであろう事実{\DDASH}について嘘をついたり、情報を隠した場合には、同意が無効になると主張する。
その論理は単純だ。
もしその特定の事実について真実を知っていれば、相手は同意しなかった、つまりそのセックスはおこなわれなかっただろう。
したがって、そこには同意がなかったのだ、というものだ。

厳格説を受け入れるべき理由は主に二つある。
第一に、それは人は自分の身体に対して自律的コントロール権を有するという広く受け入れられている事実から導かれるように見えることだ。
第二に、厳格説を採用することで、二つの不整合を解決できるというものだ。
一つは私たちの性的同意に関する見解と、他の領域での同意の扱いとの間の不整合であり、もう一つは、私たちが許容している性的な欺瞞と、私たちがすでに非難している性的な欺瞞の間の不整合だ。

ドハティにとって、同意の重要性は自律という根本的な価値に根ざしている。
これまで見てきたように、自律は自分の身体に対するコントロール権を意味し、セクシュアリティを含め、他者からの干渉を受けない権利を伴う。
性的同意とは、この権利を特定の方法で一時放棄(waive、権利の行使保留)し、特定の相手に対してセックスを許可する行為だ。
しかし、ドハティは、何に同意しているかを完全に理解していなければ、そもそも自律を一時放棄(保留)することなどはできはしないと主張する。
言い換えれば、性的同意が道徳的効力を持つためには、他の同意と同様に、十分な情報に基づいたものでなければならない。
ある意味で、これは当然のようにも思える。
ビクトリア大学の暴力防止プロジェクトは「情報に基づいた同意とは、同意を求められた人が何に同意を求められているのかについて十分な情報(full information)をもっていることを意味する」と説明している\citep{project25:_consen}。

論争が始まるのは、「十分な情報とは何か」を具体的に定義しようとするときだ。
ドハティにとって、「十分な情報」とは、自分にとってディールブレイカーとなる情報について真実を知ることを指す。
彼は次のように述べている。

\begin{quote}
適切に評価された場合、性的自律は、「個人が自分の身体と性的能力が何のためにあるかを、なんら制約のない自分の考え方に基づいて自由に行動すること」\citep[p.70]{schulhofer92:_takin_sexual_auton_serious}を許容する。
したがって、性的な接触において、何が自分にとって特に重要な特徴であるかを決定することは、各個人に任されている。
\citep[p.730]{dougherty13:_sex_lies_consen}

\end{quote}

ドハティは、相手にとってのディールブレイカーについて欺くことは、実際には意識を失っている相手とセックスをするのとなんら変わらないと主張している。
これは一見すると非常に強い主張のように思えるが、彼はそれは彼自身の立場から論理的に導出されると主張している。
同意は二値的〔イエスかノーか〕であり、また有効であるかそうでないかのどちらかだ。
そして、もし同意が有効に与えられていない場合、他者がその身体的自律を侵害するならば、それは性的暴行と見なされるべきなのだ。

厳格な同意観を擁護する立場からすれば、こうした立場は他の文脈でも用いられているように思われる。
たとえば医療の場面では、医療提供者が患者の同意を得る際には、十分な情報を提供することが求められている。
医師は、患者が意思決定をするために必要だと考えられる情報を隠してはならない。
また、商取引においても、人々は詐欺から保護されている。
ここで守られるのは身体的自律ではなく、金銭的な利益というはるかに弱い権益であるにもかかわらずのことだ。
商法では、消費者が商品の購入やサービスの利用を自由に選択する際に影響を与えるような「重要な虚偽の説明」から保護される(\emph{FTC v. Colgate-Palmolive Co.}, p.387を見よ)。
私たちは誰かが欺瞞的な口実で財物を取得したならば、それは窃盗として有罪だとみなす。
このように、性的欺瞞からの保護を拒否している法律の姿勢は\ruby{場当たり的}{アド・ホック}に見える\citep[][pp.69--71を見よ]{estrich87:_real_rape}。
寛容説の擁護者は、このようなさまざまな形の欺瞞について、それらに違った扱いを適用している不整合を説明しなければならない。

さらにもう一つの不整合を指摘することができる。
先に見たように、性的欺瞞の中にはすでに深刻な悪と見なされ、法的に処罰されるものも存在している。
法的には「行為そのものに対する詐欺」(fraud in the factum)と「誘因に対する詐欺」(fraud in the inducement)の区別があり、後者は刑事罰の対象となる。
アラン・ワートハイマーが指摘するように、この二つの欺瞞の区別は恣意的だと主張できる。
ワートハイマーは「すべてはそのケースがどのように記述されるかにかかっている」とする\citep[p.206]{wertheimer03:_consen_sexual_relat}。
厳格説の擁護者は、すでに非難されている性的欺瞞のケースと、一般に許容されているもっと一般的な欺瞞との間に原理的な区別はないと主張する。
もし前者が問題視されるのであれば、後者も同様に問題視されるべきだ、と。
なぜなら、両者の悪は、いずれも私たちの自律的な同意能力を損なうという点に由来しているからだ。

\subsection{厳格説への反論}

厳格説は、その論理的な単純さゆえに一見魅力的に思える。
しかし、これを拒否すべき理由はいくつか存在する。
まず第一に、性的欺瞞が極めて広範に存在しているという事実がある。
オンラインデート時代以前でさえ、46%の男性と36%の女性が、デートの約束を取り付けるために少なくとも一度は嘘をついたと認めている\citep{rowatt99:_lying_get_date}。
オンラインではさらに頻繁に嘘がつかれていることは驚くべきことではない。
オンラインデートに関する調査では、プロフィール上で何らかの虚偽を記載している人が全体の90%にのぼることが判明している\citep{hancock07:_truth_lying_onlin_datin_profil}。
これらの嘘がすべてが相手にとってディールブレイカーになるようなものだとはかぎらないが、多くは実際にそうだ。
仮に性的欺瞞を強姦と道徳的に等価とみなし、またはそれを犯罪化した場合、非常に多くの人々が強姦者や犯罪者になってしまうだろう。
もちろん、多くの人が何かをおこなっているからといって、それが道徳的に正しいということにはならない。
たとえば、多くの人が脱税をしたり、飲酒運転をしたりするが、どれだけそれが頻繁でもこれらの行為は非倫理的だ。
しかし、人々の日常的な行動を非難したり犯罪化しすぎると、道徳や法の権威が弱体化する危険がある。
ごく一般的な行動が非難されてしまう社会では、人々は本当に有害な行為に対する非難を無視するようになってしまうだろう。

厳格説の擁護者は、道徳や法律の重要な役割の一つは、まさに人々の行動を変えることだと指摘できるだろう。
現在、人々が自由に嘘をついているのは、寛容説が広く受け入れられ、性的欺瞞に対して何の罰則もないためだ。
もし人々がこの欺瞞をもっと深刻に受け止めれば、その頻度が減るかもしれない。
しかし寛容説の擁護者は、このような変化はあまりにも現実とかけ離れているため、人々の行動に実質的な影響を与えることは難しいと主張するだろう。

第二に、厳格説は誤った等価性を生み出してしまうという懸念がある。
ドハティは、彼の立場が示唆する論理は、性的欺瞞が性的暴行と同程度に深刻な悪だと主張している。
しかし、実際に暴力的な性的暴行を受けた被害者は、この主張には同意しないだろう。
むしろ反対に、こうした主張は被害者たちの経験を矮小化するものとして受け取られるかもしれない。
そこで厳格説の擁護者は、ある特定の行為が悪しきものであるためには、それが最も重大なケースと等しいほど悪い必要はないと返答するだろう。
暴力的な暴行は、それをさして有害でないと感じる被害者がいたとしても、依然として悪であるとされるべきだ。
しかし寛容説の擁護者にとっては、被害者の経験の点でこれら二つのケースの距離があまりにも大きいため、このアナロジーはまったく意味を成さないように思われるかもしれない。

ドハティの議論に対するもう一つの批判は、同意は「十分な情報が与えられ」(fully informed)ていなければならないという考え方に異議を唱えるものだ。
代わりに、必要なのは、同意に「適切な程度に情報が与えられている」(adequately informed)ことだと主張できるだろう。
つまり、自分がセックスをしようとしていること、そして相手が基本的に誰であるかを知っていれば十分だろうという立場だ。
それは相手について知りたい情報すべてにアクセスする必要があることを意味するわけではない。
ハリー・リベルトはセックスをギャンブルにたとえている。
ギャンブルをする際、誰もお金を失うことに同意しているわけではない。
もし事前に賭けで負けると知っていたならば、その賭けはしなかっただろう。
しかし、それが賭けの同意を無効にするわけではない。
ギャンブルでは金銭的損失のリスクがあることに対して同意しているのだ。
同様に、誰かとセックスをする際にも欺瞞のリスクがあることは誰もが知っていることだ。
\citep[p.132]{liberto17:_inten_sexual_consen}
デートの世界に欺瞞が蔓延しているという事実を再び思い出す価値がある。
潜在的なパートナーに嘘をつかれるリスクがあることは、誰もが知っていることだ。
もし何かが本当に重要であるならば、相手に関する自分の信念が正しいかどうかを確認するための手段をとるべきだ。
サイコロを投げようとしているのなら、自分が騙されるかもしれないことを受け入れる準備をしておくべきだ。

性的同意に関しては、「十分な情報」という概念がさらに捉えがたくなる。
というのも、そもそも自分自身がどのような条件をディールブレイカーとみなすのかを把握していないこともあるからだ。
私たちは「特定の属性の集合体」とセックスをするのではなく、個人としての相手とセックスをするのであり、その動機は非常に複雑で、時に自分でもわからないことがある。
この点で、性的欺瞞は他の種類の欺瞞と重要な非類似性を持っている。
デヴィッド・ブライデンは次のように述べている。
\begin{quote}
商取引においては、取引は通常、土地の取得や金銭の獲得といった目的を達成するための手段にすぎない。
そこでの詐欺は通常、被害者がその取引から得ようとした利益の多く、あるいはすべてを奪ってしまう。
また、その財物の市場価値が、欺瞞が果たした因果関係を証明する強力な証拠となることも多い。
一方、セックスの動機ははるかに複雑だ。
セックスは時に結婚のような「大きな目的」の一部として望まれることもあるが、商取引とは異なり、セックスはそれ自体が楽しいものだ。
その結果、欺瞞がなければセックスがおこなわれなかったかどうか、そしてその欺瞞が相手にとってその体験の価値を損なったかどうかは、通常ははるかに不明確だ。
\citep[p.463]{bryden00:_redef_rape}

\end{quote}

さらに、ある人がディールブレイカーとみなす内容が、必ずしもその人が知る権利のある情報とは限らない。
厳格説は、相手がディールブレイカーと認識する情報を開示する義務を生じさせるが、私たちにはプライバシーの権利もある。
したがって、プライバシーの権利と相手の「十分な情報に基づいた意思決定をおこなう権利」との間でバランスを取らなければならない。
たとえば、ある調査によれば、半数以上の人がバイセクシュアルの人とはデートしないと答えている\citep{thorpe16:_why_won_some_peopl_date_bisex}。
では、バイセクシュアルの人は、自分が異性・同性の両方とつきあった経験がある、またはつきあう意思があることを必ず開示しなければならないのだろうか? 状況によっては、この開示が実際に危険を伴う場合もある。
スティグマのある少数派に属することを明かすことで、暴力のリスクにさらされる可能性があるからだ。

\subsection{結論:法的な含意}

厳格説は、欺瞞を性的暴行と同一視することで、単なる道徳的非難を超え、性的欺瞞を違法とすべきだという結論を導いているように見える。
本節冒頭で言及したジョイス・ショートはまさにそのように考えている。
「セックスに誘導するために嘘をつくのは誘惑ではなく犯罪です」と彼女は言う\citep{mcarthur16:_is_lying_get_laid_form_sexual_assaul}。
自身の経験を経て、ショートは性的パートナーに嘘をつく行為を刑法上の性的暴行として再分類する必要性を訴えるようになった。
そして、彼女に賛同する人々もいる。
アメリカの二つの州では、法律を拡大し、性的関係を得る目的で相手を欺く行為を違法とする提案がおこなわれた。
2008年にマサチューセッツ州の下院議員がこのような法案を提案し、2014年にはニュージャージー州の議員も同様の提案をおこなった。
しかし、両提案とも否決された。
とはいえ、性的暴行に関する全国的な対話が続く中、将来的に同様の試みがおこなわれる可能性は十分にある。

性的な欺瞞を犯罪化することは、実務的な課題をもたらすだろう。
捜査や裁判の過程では、その人物が何を言い、その発言の意図が何であり、それが本当に相手にとってディールブレイカーであったかどうかを立証する必要があるだろう。
多くの場合、欺瞞には感情に関する主張が含まれることになる{\DDASH}たとえば、本心から「愛している」と言ったかどうかということだ。
このような発言の誠実さを判断するのはほぼ不可能に近い。
また、性的欺瞞を犯罪とする法律は、国家が私的な生活に介入する範囲を拡大することにもなる。
デヴィッド・ブライデンは次のように述べている。
「私たちは、陪審員が事実をどう判断するかを考えるだけでなく、刑事司法制度がその事実を確認しようと関与することを私たちが本当に望むかどうかも考慮しなければならない{\DDASH}たとえば、恋愛関係が終わった後に警察が捜査を始めるような事態を本当に望むのかどうか、といった点だ」\citep[p.469]{bryden00:_redef_rape}。
ジェーン・ラーソンは1993年に影響力のある論文を発表し、刑事法ではなく民事法(\ruby{不法行為}{トート}法)を利用することで、欺瞞によってセックスに誘導された人々が救済を求めることができるようになると主張した\citep{larson93:_women_under_so_littl_they}。
しかし、民事訴訟であっても同様の懸念は生じるだろう。

ドハティは合法性の問題には踏み込まず、性的な欺瞞が重大な道徳的不正であることを立証することのみに関心があると述べている。
本章の他の箇所でも述べたように、セックスに実際には同意があったとしても、道徳的には問題がある場合がある。
私たちは、欺瞞が同意を無効にするというドハティの結論には同意できないかもしれないが、それでも、それが深く非倫理的であるという判断には同意しうる。
そして、ほとんどの人が親密な関係において率直さと誠実さが美徳であるという点に同意できるはずだ。

\section{肯定的同意}

過去20年間にわたって、アメリカ合衆国の大学キャンパスでは、性的行為を開始する前にパートナーから明確な同意を得ることを求めるルールが導入されつつあり、同意の概念が再定義されようようとしている。
いわゆる「肯定的同意」(affirmative consent)基準を確立することで、これらの教育機関は性的暴行に対する認識を変え、ひいてはセックスそのものに対する考え方を変えることを目指している。

アメリカにおいて、肯定的同意運動が注目を集め始めたのは1990年代のことだ。
1991年、オハイオ州にある小規模ではあるが歴史のある大学であるアンティオーク・カレッジ(Antioch College)の関係者は、学生たちが性的な接触をおこなう際に、パートナーから明確かつ継続的な同意を得ることを求める性的暴行防止ポリシーを発表した。
このポリシーは、学生グループ Womyn of Antioch によって最初に提案され、その後大学の管理部門によって採用された。
このポリシーは全国的な注目を集めたが、その多くは非常に否定的な反応であった。
この運動を主導した学生は、2014年に「私たちは笑いものになった」と認めている\citep{saltman14:_we_start_crusad_affir_consen}。

アンティオーク大学の試みに対する否定的な反応にもかかわらず、同大学の政策はその後広く模倣され、現在では同意の再定義を目指す広範な運動の先駆けとして認識されている。
高等教育リスク管理全国センター(National Center for Higher Education Risk Management)によれば、現在では800を超える大学が、性的暴行ポリシーに何らかの形で肯定的同意の定義を取り入れているという。
2014年、カリフォルニア州はSB967号法案を可決し、州内すべての大学キャンパスで肯定的同意を標準とすることを義務付けた。
他の多くの州でも同様の法律を検討している\citep{new14:_yes_means_yes_world}。

大学は独自の法制度を運営しているわけではなく、最も厳しい罰は退学処分だ。
しかし、カナダのように肯定的同意基準を性的暴行の定義に採用している法域も存在する。
2019年8月、アメリカ法曹協会(American Bar Association)は各州に肯定的同意を法律上の基準として採用するよう促す決議を採択した\citep[p.1]{domestic19:_repor_house_deleg}。
一部のアメリカの州はすでに刑法に肯定的同意を取り入れている(刑法において肯定的同意基準を用いることを提案している研究者としては、\citet{schulhofer98:_unwan_sex}、\citet{anderson05:_negot_sex}、\citet{decker11:_no_still_means_yes}などがいる)。
国内で最も影響力のある法的機関の一つであるアメリカ法律協会(American Law Institute)は最近、全国で肯定的同意を法的基準とすることを推奨する提案を検討したが、最終的にはメンバーによって却下された。

肯定的同意基準への移行は、法律および社会全体がセックスに取り組む方法において、確実に変化をもたらすだろう。
それは多くの人々、いやおそらく大多数の人々にとって行動の変化を必要とするだろう。
このような変化が望ましいのか、そしてそれが実際に実行可能なのかを問う必要がある。

\subsection{肯定的同意の定義}

肯定的同意(affirmative consent)ポリシーは、同意の基準を「ノーはノー」(no means no)から「イエスのみがイエス」(yes means yes)に移行させることを目的としている。
ジョン・ダナハー\ig{John Danaher}はこの基準を次のように要約している。
「ある性的行為が道徳的(または法的)に許容されると見なされるためには、その行為に関与するすべての参加者が自由で、積極的かつ明確にその行為を進める意思を示さなければならない。
ただ反対がないというだけでは不十分だ」\citep{danaher14:_yes_means_yes}。
肯定的同意ポリシーやそれに類する法律・法案にはいくつか鍵になる特徴がある。
まず第一に、両方のパートナーが同意を明確に表現することを求めている。
第二に、同意は継続的であり、いつでも撤回可能でなければならない。
第三に、パートナー同士の関係のあり方にかかわらず同意が必要だ。

肯定的同意を最も明確に表現する方法は言葉による承認だ。
アンティオーク・カレッジのポリシーはこれを要求しているようであり、同意を「特定の性的接触または行為に対して進んで言葉で(willingly and verbally)合意する行為」と定義している\citep{college25:_sexual_offen_preven_polic}。
しかし、肯定的同意ポリシーによっては、しばしば非言語的な同意形式も認めている。
カリフォルニア大学リバーサイド校の肯定的同意ガイドでは「同意は非言語的でも表現しうる。
言葉を使わずに性的接触に対する明確な意思を示す方法がいくつかある」と述べており、「うなずき」「親指を立てる」「相手を引き寄せる」「直接目を合わせる」といった例を挙げている\citep{university25:_what_is_consen}。
研究者たちは肯定的同意ポリシーを「\ruby{強硬}{ハード}バージョン」と「\ruby{柔軟}{ソフト}バージョン」に分類することがある。
厳格なバージョンは明確な言語による同意を要求する一方、柔軟バージョンは非言語的な同意も認める。
柔軟なバージョンは性的な場面で非言語的なシグナルが最も一般的かつ自然なコミュニケーション形態である現実により対応しているように見える。
しかし、それは同意の曖昧さを防ぐというポリシーの目的を損なうリスクも伴う。
カリフォルニア州の肯定的同意を義務付ける法律では「非言語的コミュニケーションのみに依存することは誤解を招く可能性がある」と警告している\footnote{California Senate, Bill Text: CA SB967, 2013--2014. \url{https://legiscan.com/CA/text/SB967/id/954680}.}。

肯定的同意基準は同意が継続的であることも要求する。
アンティオーク・カレッジのポリシーによれば、学生は「いかなる相互作用においても、身体的あるいは性的な行為のそれぞれの段階で明確な同意を得る必要がある」とされている\ig{\footnote{Antioch College.}}。
同意は性的活動の間中ずっと存在していなければならず、「同意について混乱が生じた場合は、その混乱が明確に解決されるまで活動を停止することが不可欠」だ。
同意はパートナーのどちらからでもいつでも撤回可能だ\ig{\footnote{Antioch College.}}。

最後に、肯定的同意ポリシーは、パートナーが継続的な関係にある場合、さらには結婚している場合でも明確な同意を要求することがある。
アンティオーク・カレッジのポリシーでは「過去に誰かと特定の性的親密さを持ったことがあっても、そのたびに必ず確認しなければならない」とされている。
ただし、いくつかのポリシーでは交際中の人々に異なるルールを認めている。

\subsection{肯定的同意の擁護}

肯定的同意(affirmative consent)基準を確立する最も基本的な理由は、性的暴行の発生を減らし、暴行をおこなった人物を罰しやすくすることだ。
現在のところ、キャンパス内外での性的暴力の発生率は高く、一方で告発率や加害者が罰せられる率は驚くほど低い。
肯定的同意基準を確立するだけでこれらの問題を解決できると考える人はいないが、提唱者たちはそれが一定の影響を与えると信じている。

まず第一に、肯定的同意基準は、パートナーが同意していると誤解する可能性を排除することで、発生する暴行の数を減らす可能性がある。

こうした件での本気の誤解は実際にはごく稀かもしれない。
検察官のリンダ・フェアステイン\ig{Linda Fairstein}は「被害者が示したシグナルは、言語的であれ身体的であれ、ほとんどの場合は非常に明確だ。
実際にコミュニケーションの失敗が原因で起こるレイプはほとんどない」と述べている\citep{fairstein94:_panel_discus_men_women_rape}。
しかし、それが起こりえることは確かだ。
本気の誤解が稀である一方で、性的暴行裁判において被告がそのような誤解を主張することは非常に一般的だ。
1975年の裁判\emph{People v. Mayberry}では、被告が被害者の同意について誤解したと主張して無罪となり、その後何百回も引用されている
肯定的同意基準は不当な「合理的な誤解」弁護を完全に防ぐことはできないが、それを困難にすることはできる。
また、もしそのような弁護が難しくなると知れば、潜在的な加害者は行動を再考するかもしれない。

肯定的同意基準はまた、被害者の沈黙を同意として解釈する可能性、あるいはそう主張する可能性を排除するという重要な役割を果たす。
性的暴力の可能性を恐れる状況では、人々が凍り付いたり沈黙したりすることが知られている。
この現象を説明するために、臨床医は「緊張性不動」(tonic immobility)や「トラウマ中解離」(peritraumatic dissociation)といった用語を使うことがある。
研究によれば、これは多くの暴行で見られる現象だという\citep{moller17:_tonic_immob_durin_sexual_assaul}。
ある女性は、自分の美容師が性的な接近を始めた際の反応を次のように説明している。
「私の前頭前皮質はオフライン状態であり、感情を表現する選択肢がありませんでした。
脳がそのシステムからエネルギーを切り替えたからです。
副交感神経系が筋肉を弱め、血流を極端に減少させたため、物理的に動いたり反応したりすることができませんでした」\citep{corvo18:_why_i_froze_smiled_durin}。
カレン・クレイマー\ig{Karen Kramer}は、一部の男性、特に性的状況での男性と女性の行動についてのある種の考え方を抱いている男性にとっては、女性が沈黙し抵抗しないことが、攻撃的にセックスを求める免罪符となってしまう可能性があると指摘している。
「伝統的な男性の攻撃性と女性の服従モデルを信奉する男性にとって、彼女の抵抗の欠如は明確な「イエス」と聞こえるかもしれない」\citep[p.121]{kramer94:_rule_myth}。
肯定的同意基準は、沈黙を同意ではなく拒否として解釈することを要求するため、このような状況にある人々を保護する助けとなるだろう。

被告や弁護人たちが同意を得たと虚偽の主張をすることも日常的に生じている。
肯定的同意基準によっても、このような場合を特定するという困難な問題を完全には解決できない。
しかし、肯定的同意基準は、こうした主張を立証するために用いられる証拠の一部を制限することができる{\DDASH}特に、被害者の加害者に対する過去の行動や、事件の文脈に訴えるような証拠の種類を制限することができる。
こうした証拠を用いる戦略によって、裁判官や陪審員が加害者とされる側に共感しやくなってしまうことがすでに知られている。
2011年のカナダの裁判では、裁判官は、事件は「誘いかけるような状況下で起きた」として、有罪のレイプ犯に執行猶予付きの刑を言い渡した。
裁判官は、被害者とその友人が加害者とその友人と一緒に飲酒し、チューブトップとノーブラ、ハイヒール、濃いメイクをしていたことに注目し、その夜は「空気がセックスで満ちていた」と述べている\citep{cbc11:_manit_judge_rebuk_sex_assaul_remar}。
模擬陪審員を対象とした研究によると、被告の友好的な態度や、女性がベッドルームに戻るように招いたといった要因を理由に無罪判決を出しやすいことがわかっている\citep{finch06:_break_bound}。
肯定的同意基準は、理想的には、このような要素を無視するよう陪審員を導くようになる。

ルチンダ・ヴァンダーヴォート\ig{Lucinda Vandervort}は、肯定的同意基準が女性を日常生活でハラスメントや暴行の被害に遭いやすい状況からも保護すると主張している。

\begin{quote}
  理論的な観点から言えば、肯定的な性的同意は「ノーはノー」パラダイムでは提供できない保護手段を提供する。
これを理解するのは、日常的な行動{\DDASH}睡眠、歯磨き、入浴、シャワー、あるいは洗濯物をたたむ、床を掃除する、CEOの机の上の書類を整理するなど{\DDASH}の行為中に知人や親戚から性的暴行を受けた経験がある人なら簡単だ。
日常生活のありふれた行動{\DDASH}ただ生きて呼吸していることさえ{\DDASH}が「誘惑的」であり、性的接触を招く行為だと主張されることが多いという被害者たちの経験を考えると、「イエスでなければノー」という肯定的同意基準が必要であることは明白だ。
\citep[p.405]{vandervort12:_affir_sexual_consen_canad_law}
\end{quote}

性的暴行の減少を超えて、肯定的同意ルールは、すべての人にとってセックスを\ruby{より良い}{ベター}ものにする可能性があると提唱者たちは期待している。
これらのルールは、パートナー双方が同意している状況でも、セックス中のコミュニケーション方法を変えることで、ベターなセックスを実現する可能性がある。
タラ・カルプレスラーは、肯定的同意の推進は「そもそものセックスへのアプローチ方法を大きく方向転換させることに関わっている」と述べている。
提唱者たちは、セックスについてもっと話し合うことでパートナー同士が互いに率直になり、反応しやすくなるため、すべての性的関係が改善される可能性があると主張している。
アン・フリードマン\ig{Ann Friedman}は「同意を確認することで、より情熱的な熱いセックスにつながる」と述べている(Friedman, 2014; cf. Friedman and Valenti, 2008, p.7)。
\nocite{friedman14:_oh_yes_means_yes}\nocite{friedman08:_yes_means_yes}
オレゴン州立大学は肯定的同意を説明するウェブページで次のように記している。
「これがセックスの楽しい部分です。
自分が何をしたいのか、どうやりたいのかを話すことができます。
\ruby{創造的}{クリエイティブ}になれるのです!」\citep{moyer14:_how_calif_yes_means_yes}。

肯定的同意の提唱者たちは、法律や大学のポリシーの変更が「\ruby{同意文化}{コンセントカルチャー}」を創造する一助となると主張している。
この文化においては、同意が話題にされ、価値あるものとして認識されるようになる。
活動家たちは、大学キャンパスや社会全体に広く蔓延ている「レイプ\ruby{文化}{カルチャー}」をしばしば非難している。
このような文化の例を見つけるのはけっして難しいことではない。
たとえば、\ruby{学生社交団体}{フラタニティ}が同意の重要性を嘲笑するような下品なスローガンを\ruby{唱え}{チャント}あげる例が数多く報告されている\citep{jackson18:_frat_barred_yale_years_is_back}。
肯定的同意ポリシーは、同意が社会全体で真剣に受け止められているという明確なメッセージを誰にでも発信する。
そして提唱者たちは、肯定的同意法がより開かれた、セクシュアリティを肯定的に捉える文化への変化を促進すると期待している。
そのような文化は特に女性のセクシュアリティに対してより親しみやすいものになるだろう。
アン・フリードマン\ig{Ann Friedman}は次のように述べている。
「私たちはまだ、「\ruby{よい子}{ナイスガール}はセックスが好きだとは認めないものだ、ましてやどう好きなのかを話すことなどはありえない」という考え方を解体している途中だ」\citep{friedman14:_oh_yes_means_yes}。

\subsection{肯定的同意への反対論}

同意基準に反対する人々は、パートナーたちに対して、いつもいつもお互いに許可を求め与えあうことを要求するならば、セックスからその本質的な自発性と神秘性が奪われることになり、ロマンスも失われると主張している。
サラ・クリクトンは\emph{Newsweek}でアンティオーク・カレッジの規定について「セックスの予測できない甘美さを犯罪化するものだ」と述べている\citep{crichton93:_sexual_correc}。
また、クリスティナ・ネーリングはカリフォルニア州のYes Means Yes法案について次のように語っている。

\begin{quote}

すべての偉大なラブストーリーには、主人公たちが言語の厳格なルールを捨て、官能の流れに身を任せる瞬間がある……安全がチェックされ責任が問われる私たちの文化、際限のないおしゃべり、メッセージを送り、ブログを書き、他人のご機嫌を取り、謝罪し、分析し、言語化しつづける私たちの文化の中で、エロティシズムは直観で探索できる最後のフロンティアかもしれない。
ダンスのように、セクシュアリティは言葉以前に存在し、言葉を超えている。
それを質問や定型文で固定するのは、蝶を壁にピンで留めるようなものだ。
蝶をピンで留めることで、それはよく見えるようになるかもしれないが、もはや舞い上がることはない。
そしてたぶん、あなたの心も舞い上がることはなくなるだろう。
\citep{nehring15:_are_today_legal_defin_rape}
\end{quote}

誰かに対して自分が何をしたいかを公然と話すことは、攻撃的または不適切だと感じられることさえある。
また、一部には、途中で気持ちが変わってしまった場合にそれをうまく操るのをひどく苦手に感じる人もいる。
サマンサ・ハリス\ig{Samantha Harris}とマイケル・サッド・アレン\ig{Michael Thad Allen}は、肯定的同意基準が「社交的に臆病であったり、性的経験が少なかったり、微妙な社会的シグナルを理解する能力が弱い人々」に対して不釣り合いに大きな影響を与える可能性があると懸念を示している\citep{harris20:_bad_vibrat}。

さらに、女性の欲望がスティグマ化されている社会的文脈を考えると、同意を明確かつ公然と表現することが特に女性にとって困難になりえると指摘する人々もいる。
ダグラス・フーサク\ig{Douglas Husak}とジョージ・トーマス\ig{George C. Thomas}は、私たちの社会が女性に対して「積極的すぎる」または「軽い」と見られないようにするため、暗黙的な方法で同意を表現するよう教えていると言う。
彼らは、こうした目的のために「求愛儀礼」(courtship rituals)と呼ばれる特別な社会的慣習が発展してきたのだと述べている。
肯定的同意は、自分の欲望について、これまでは避けるよう社会化されてきたレベルの明確さで公然と話すことを突然要求されてしまうため、特に女性にとってセックスをさらにストレスフルなものにするリスクがある\citep{husak92:_date}。

デヴィッド・アーチャードは、親密な関係の文脈において肯定的同意基準が特に問題であると懸念している。
彼は、この基準がそのような関係の基礎であるお互いに対する信頼という前提を侵害すると述べている。

\begin{quote}
親密で愛情に満ちたカップルが、性的接触のたびにお互いの明確な同意を求めなければならないとするのは不合理に見える。
明確な同意がないことは非同意と見なされるため、確認を怠った場合は有罪だということになる。
この不合理さは、愛、長い交際、相互理解、相互信頼などによって特徴づけられる関係に、見知らぬ者同士の関係ならば適切であるような推定基準を拡張しているところにある。
\citep[p.146]{archard98:_sexual_consen}
\end{quote}

肯定的同意に反対する人々は、肯定的同意を法的基準にしようとすることは、実際の世界での人々の行動を無視しており、ほとんどすべての人を犯罪者にする可能性があると主張している。
「YOU are a Rapist; Yes YOU!(あなたはレイプ犯です。
そう、あなたです!)」という衝撃的なタイトルの記事で、デヴィッド・バーンスタイン\ig{David Bernstein}は「肯定的同意はアメリカのほぼすべての成人を性的暴行の加害者にする」と述べている\citep{bernstein14:_you_are_rapis}。
\ig{David Bernstein}
事実として、ほとんどの人は、少なくとも毎回毎回肯定的同意を求めているわけではない。
仮に肯定的同意基準が導入されてもこの事実が変化するとは考えにくい。
結果として、少なくとも形式的には、私たちはこれまで通りのセックスをするだけで犯罪者となり、定期的に暴行をおこなっていることになる。
そして、問題となる行動の範囲は非常に広い。
カナダの裁判官は、肯定的同意法の影響を考慮しながら、次のような懸念を表明している。
「共同生活者たち、さらには厳密に言えば配偶者たちが、片方が眠っている間にキスやタッチをした場合、性的暴行を犯していることになる。
\kenten{明確な事前同意があったとしても}それは変わらない」(\emph{R. v. J.A.}, \ig{2011 SCC 28,} para. 74\ig{\footnote{\url{https://scc-csc.lexum.com/scc-csc/scc-csc/en/item/7942/index.do}.}}) 。

肯定的同意法の反対者たちは、私たちがセックスの前に肯定的同意を得ることを怠ることがしばしばであることを考えると、「ノーはノー」基準の下よりも虚偽の告発がこれまでよりも容易で頻繁になると主張している。
また彼らは特に、肯定的同意法が「後悔を犯罪化」することになるのではないかと懸念している。
ある性的接触自体は同意の上でおこなわれたものであっても、そこに明確なディスカッションがおこなわれたわけではなかった場合、どちらのパートナーも後になって相手を暴行で告発できる可能性がある。
キャシー・ヤングは次のように述べている。

\begin{quote}
肯定的同意は、当人が後悔している性的接触のほぼすべてを暴行として再構成できる世界を作り出す(もし当人が完全に\ruby{素面}{しらふ}の状態でその接触を始めたというわけでなければ)。
そして、過去を回想しての強制の認識は、その時点での同意の認識よりも常に信頼のおけるものとされるにちがいない{\DDASH}私たちが、現在のバイアスによって記憶が過去を「編集」してしまうことがよくあることがよく知られていてもだ。
\citep{young15:_femin_want_us_defin_these}
\end{quote}

もちろんこれまでも虚偽の告発(false accusation)は起こりえる。
虚偽告発がどの程度一般的かについては議論があるが、データを研究した研究者たちは、それが非常に稀だと結論づけることが多い\citep{ferguson16:_asses_polic_class_sexual_assaul_repor}。
しかし、反対者たちは、肯定的同意基準がそれをもっと一般的にしてしまうのではないかと懸念している。
たとえば、二人の関係が破局してしまった後や、カジュアルな関係ののちに一方が相手は継続的な関係を望んでいないことを知った場合に、意図的な告発がおこなわてしまう可能性を心配する人もいる。
また、ジャネット・ハーレイのように、さまざまな「誠実な(good faith)告発者」を懸念する声もある。
彼女は次のように述べている。

\begin{quote}
これには、現在はその時点で同意していなかったと主張しているが、その記憶が当時の自発的な薬物摂取によって破壊されてしまっている告発者、あるいはその後に記憶が歪んでしまった告発者が含まれる。
また、友人やボーイフレンドや両親らによって「あなたが経験したのは暴行なのだ」と説得され、当時はそう感じていなかったのに今ではそう信じている告発者や、当時は曖昧な気持ちだったが、時間が経つにつれてその出来事をより否定的に捉えるようになり、現在では当時は同意しなかったと本当に確信している告発者がいる。
また、現在怒りや恥を感じており、その感情によって、その時自分は同意できたはずがない、またはすべきでなかったと信じ込んでしまった告発者も含まれる。
\citep[p.272]{halley16:_move_affir_consen}
\end{quote}

もし肯定的同意が法律の一部となれば、結果として私たちの親密な生活に対する刑法の影響が広がることになる。
これはリベラルな改革派の多くが各種の行為の非犯罪化や刑務所収監の削減を推進している時代に逆行するものだ。
また、肯定的同意法はすでに広範な権限を持つ検察官の裁量をさらに拡大させることになる。
性的暴行に関する厳格な法律を支持する提唱者たちは、性的暴行がすでに法制度上の例外であり、有罪判決率が低く、一般的に罰則が他の暴力犯罪に比べて軽いことを指摘している。
厳格な法律はこの不均衡を是正するだけだと彼らは主張する。
しかし、同意をめぐる法規制を強化しようとする動きは、現在、広範な司法改革派の連合によって推進されている改革運動に逆行するものだ。
改革派は、現在警察や検察官に過剰な権限が与えられており、あまりにも多くの人々が刑務所に収監されていると考えている。

裁判と刑の執行は、さまざまな集団に対して不平等に実施されているのが常であり、性犯罪についても例外ではない。
ジャネット・ハレー\ig{Janet Halley}は、肯定的同意法による起訴は「性的に危険だと見なされている集団に特別に不釣り合いな仕方で影響を及ぼすだろう」と予測している。
それには黒人や他の有色人男性、告発者に比較して社会的・経済的地位が低い男性、そして社会的規範や告発者が抱いている性別による期待に従わない男性および女性が含まれるだろうという(Halley, 2016; cf. Ackerman and Sacks, 2018)\nocite{ackerman18:_dispr_minor_presen_u}。
ハレーは、ハーバード大学で目にしたケースの多くが黒人男性が関与していると報告している\citep[cf.][]{halley15:_tradin_megap_gavel_title_ix_enfor}。
残念ながら、現在大学キャンパスでは性的暴行の懲戒に関連する人種的データを報告することは少ない。
しかし、少なくとも一つのデータとして、エミリー・ヨッフェ\ig{Emily Yoffe}は2014年にコルゲート大学が前年度のデータを公表したことを報告している。
それによれば、黒人男性の学生は学生全体の4.2%を占めているが、大学に報告された性的不法行為の50%は彼らに対する告発であったという\citep{yoffe17:_quest_race_campus_sexual_assaul_cases}。

反対者たちはまた、肯定的同意基準が性的暴行裁判の訴訟方法を根本的に変え、被告に不利な影響を与えると主張している。
特に、法制度の基本原則である推定無罪の原則を取り除いてしまうことでそうなるという。
肯定的同意基準は、非同意をデフォルトとするため、証明責任は必然的に被告に課される。
被告は告発者が肯定的に同意しているかどうかを確認するために合理的な措置をとったことを証明しなければならないことになる。
アメリカ弁護士協会が各州に肯定的同意法を採用するよう求める決議を発表した際、批判者たちは「この変更は違憲であり、憲法修正第5条および第14条の\ruby{適正手続き}{デュー・プロセス}条項に違反する」と述べた。
これらの条項はそれぞれ、刑事訴追において、被告が自分に不利益な供述をしない権利を保護し、法の下での平等な保護を保証するものだ\citep{bauer-wolf19:_lawyer_group_disag_colleg_model_affir_consen}。

こうしたリスクは大学キャンパスでさらに高まる。
大学は、警察や裁判所ならば従わなければならない確固とした手続き規則に従うことは求められていないからだ。
大学キャンパスでは、通常は法律家ではない管理者たちに裁量権が与えられており、彼らはこうした複雑な紛争に対処する訓練を受けていないことが多い。
アメリカでは、処分対象者が、性的暴行処分手続きで大学側にバイアスがあったとして多数の訴訟を起こし、その多くが勝訴している\citep{anderson19:_more_title_ix_lawsuit_accus_accus}。
手続き上の問題は、その大学が肯定的同意基準を採用しているかどうかにかかわらず発生する可能性があるわけだが、反対者たちは、肯定的同意基準を導入することで、処分対象者の権利が侵害される可能性がさらに高まると主張している。
たとえばテネシー州の裁判官は、テネシー大学チャタヌーガ校(University of Tennessee at Chattanooga)の肯定的同意ポリシーが推定無罪の原則を排除していると判断した。
このポリシーでは「極めてプライベートな身体の一部を露出する行為に、証人がほとんど存在しない環境で、信頼できる肯定的な言葉での応答を証明すること」を被告に要求しているとされた(\emph{Mock v. University of Tennessee at Chattanooga}\ig{, No. 14-1687-II, Tenn. Ch. Ct. 10 August 2015\footnote{\url{https://kcjohnson.files.wordpress.com/2013/08/memorandum-mock.pdf}.}})。
2017年、トランプ政権は大学が調査をおこなう際の連邦政府の指針を変更し、オバマ政権が2011年に大学に送付したキャンパスでのセクシャルハラスメントに関する指針を撤回した。

\subsection{熱烈同意}

肯定的同意(affirmative consent)を超えて、さらに異なる基準{\DDASH}熱烈同意(enthusiastic consent){\DDASH}を基準として採用すべきだと主張する人々もいる。
熱烈同意という概念は、ジャクリーン・フリードマン\ig{Jaclyn Friedman}とジェシカ・ヴァレンティ\ig{Jessica Valenti}が2008年に発表したアンソロジーエッセイ集 \emph{Yes Means Yes! Visions of Female Sexual Power and a World Without Rape}で広められた概念だ\citep[pp.308--309]{friedman08:_yes_means_yes}。
このフレーズは、マスメディアのコメンテーターや性教育関係者の間で広く使われているが、哲学者の間でまだ十分な議論がおこなわれたとは言えない。
哲学研究の標準的なデータベースである \emph{Philosophers Index}には、この概念を論じた学術論文は2022年現在一本も掲載されていない。
しかし、この概念は検討に値する強力なアイデアだ。

熱烈同意の正確な定義を見つけることは難しい。
ギャビー・ヒンスリフ\ig{Gaby Hinsliff}はこれを「お互いに触れあわずにはいられないというはっきりした感覚」と表現している\citep{hinsliff15:_consen_is_not_enoug}。
多くの人は肯定的同意と熱烈同意を混同するが、両者は関連はあるものの異なる概念だ。
「熱烈同意」は肯定的同意よりも高い基準を設定している。
熱烈同意は単なる「イエス」ではなく、「イエス、ぜひ!」(``hell yes!'')と表現されることがある。

すべてのパートナーが熱心に参加するセックスは、特にパートナー同士があまり知らない状況において、同意に関するあらゆる曖昧さを取り除く。
熱烈同意を求めることで、相手が外見上は同意を示しているとしても発しているかもしれない微妙なサインに注意を向けることができる。

両方のパートナーが本当に同意している場合でも、熱烈同意を求めることはセックスをより楽しいものにする。
熱意を求めることで、パートナーの気分や欲望に敏感になりやすくなるからだ。
全米性的暴力終結連合(National Alliance to End Sexual Violence)のアドボカシーディレクターであるエボニー・タッカー\ig{Ebony Tucker}は、熱烈同意と肯定的同意の違いを次のように説明している。
熱烈同意は、肯定的同意のように、単にパートナーが現在の状況に「問題ない」と感じていることを確認するだけでなく、パートナーがそれを積極的に楽しんでいるかどうかに焦点を当てている。
タッカーは次のように言う。
「「熱烈同意」は、セックスをしている相手にきちんと注意を払い、相手がその状況を楽しんでいることを確認することに、より重きが置かれています」\citep{cooney18:_aziz_ansar_alleg_has_peopl}。

ジャミーラ・ジャミル\ig{Jameela Jamil}は熱烈同意を「性的同意のゴールドスタンダード」と呼んでいる\citep{jamil18:_what_we_need_learn_aziz_ansar_clust}。
しかし「ゴールドスタンダード」とは、定義として、すでに最低基準は超えているものだ。
現実の日常が常に理想通りだとは限らず、熱烈同意は普遍的に有効な同意の基準として機能しえないと主張されてきた。
まず第一に、これは法的あるいは道徳的基準として運用するための明確さに欠ける。
リリー・ゼン\ig{Lily Zheng}は次のように述べている。

\begin{quote}
熱烈同意は紙の上で理論的には理想的だが、現実の親密な場面では悪夢のようなものだ……熱烈同意モデルはあまりに曖昧なため、現実のやり取りが「熱烈な」ものであったかどうかを判断することはほぼ不可能だ。
\citep{zheng14:_how_ace_sex}

\end{quote}

法的な文脈においては、裁判官や陪審員が「イエス」と「イエス、ぜひ!」とを合理的な疑いを超えて区別することは極めて難しいだろう。
親密なコミュニケーションの多様性と複雑さを考えれば、なおさらのことだ。
また、この基準は、性的暴行の基準を非常に高く設定してしまうため、すべての暴行の告発の有効性に疑問を投げかける結果になりかねない。

熱烈同意は、現在私たちが同意のあるものとして受け入れているさまざまな状況で欠如している可能性が高い。
たとえば、ある人はその時点ではセックスをしたい気分ではないが、パートナーが落ち込んでいて身体的な触れ合いが助けになるとわかっている場合がある。
セラピストはしばしば、たとえカップルが完全にその\ruby{気分}{ムード}だとは言えなくても、特定の日にセックスを予定しておくことを勧める。
また、子供を作ろうとしているカップルは、妊娠の可能性を高めるために非常に頻繁にセックスをすることを自分たちに強いることがある。
これらすべての状況において、一方または両方のパートナーがセックスに対して真の熱意を欠いているかもしれない。
それでも私たちはそれらを道徳的に問題があるとは考えず、ましてや暴行と呼ぶことはけっしてない。
アセクシュアルの人々もまた、関係を維持するために、熱意はなくとも同意の上でセックスをすることがある。

熱烈同意基準は、参加者に対してコミュニケーションの負担も課す。
性的コミュニケーションの困難さゆえに、多くの人は本当にセックスを望んでいる場合でも熱意を表現するのに苦労しているものだ。
特に女性は、熱意を示すことで\ruby{尻軽女}{スラット}とレッテルを貼られたり、性的に奔放だという評判を得ることを恐れる場合がある。
そのため、同意を熱意を持って表現することが期待されると、それが女性に不釣り合いな負担をかける可能性がある。

\subsection{本節のまとめ}

メディアが肯定的同意とその基準の導入推進について議論する際、主にアメリカの大学キャンパスでの状況に焦点を当てることが多い。
2011年、オバマ大統領の下で、教育省市民権局(Office of Civil Rights)は「親愛なる同僚への書簡」(dear colleague letter, DCL)を発行し、「あらゆるハラスメントを終わらせ、敵対的な環境が作り出された場合はそれを排除し、再発を防止する」ために、高等教育機関が講じるべき措置を提示した。
この書簡は多くの学校が肯定的同意基準を導入するきっかけとなったが、基準そのものを明確に義務付けたものではなかった。
しかし、2017年にトランプ政権はこの手紙を正式に撤回した
\citep{melnick20:_analy_depar_educat_final_title}。
この撤回の主な目的は、ハラスメントまたは暴行が疑われるケースにおけるキャンパス内手続きの方法を変更することだった。
ベッツィ・デヴォス\ig{Betsy DeVos}教育長官の下、連邦教育省はこれらの手続きが処分対象者に対して不利益なバイアスを含んでいると見なした。
この政策転換がキャンパスでの同意ポリシーに与える影響は今後明らかになるだろう。
そして、トランプ大統領がすでに政権を去った今、アメリカ政府の政策が再び変化する可能性がある。

しかし、この問題はアメリカの大学に通う人々だけに影響を与えるわけではなく、実際には私たち全員に関わるものだ。
本節で示したように、性的関係において肯定的同意を法的基準として確立することには賛否両論の妥当な議論がある。
この問題は、政治ではなく原則に基づいて決定されるべきだ。

\section{同意についてのまとめ:性的コミュニケーションの別モデル}

本章では、私たちの性的な生活において同意が果たす重要な役割と、健全な性的コミュニケーションを概念化する方法としての限界の両方を説明しようと試みた。
同意に対する批判に共感する人々は、私たちがどのような代替案を持ちえるのかと疑問に思うかもしれない。
ジョセフ・フィッシェル\ig{Joseph Fischel}は次のように述べている。
たしかに同意に関する会話をすべて放棄することはできないが、私たちはそれでも

\begin{quote}
同意に対して「ひねったり、締めつけたり、圧しつけたり」して、同意に働きかけることができる。
同意という概念が私たちの想像力を捉えている状態を解き放つことで、より有望な価値、規範、概念を導入し、より安全で、より民主的な\ruby{快楽文化}{ヘドニック・カルチャー}を築こうとする試みに道を開くことができるのだ。
\citep[p.3]{fischel18:screwconsent}
\end{quote}

本章の冒頭で議論したフェミニストによる同意批判を出発点として、一部の哲学者は、性的コミュニケーションの概念を捉え直し、行為のイニシアティブをとる行為者(主体)に対して受動的なパートナーが同意するかしないかの問題だという従来の同意モデルを超えようと試みている。
ジョン・ガードナーは、セックスをチームワークとして想像することを提案する。
彼は次のように説明している。

\begin{quote}
性的パートナーは、お互いがもたらすであろう快楽と満足に関して共通の意図を持ち、その快楽と満足を実現する一般的な方法についても共通のメニューをもつ。
だがそれだけでなく、私が先に記述したような意図、すなわち、お互いがそれを一緒におこなうこと、つまり単なる共通の追求(persuit in common)であるだけでなく、協働的な追求(joint pursuit)であることを意図している。
\citep[p.54]{gardner18:_oppos_rape}
\end{quote}

ガードナー\ig{Gardner}はこれを「持続的な対人間フィードバックループ」と呼んでいる\citep[p.55]{gardner18:_oppos_rape}。
このようなフィードバックループにおいては、パートナーのどちらが同意を与えたかと考えることは無意味だ。
なぜなら、そこには明確な行為者(主体)や受動者(客体)は存在せず、互いに望まれた関係に参加する二人だけがいるからだ。

同様の観点から、クィル・レベッカ・ククラはセックスを対話として捉えることができると主張している。
彼女は次のように述べている。

\begin{quote}
将来することになるかもしれない\ruby{性的接触}{エンカウンター}ついての対話を私が始めようとするとき、私は必ずしもセックスをねだっているわけではない。
たとえば、私は単に自分の\ruby{空想}{ファンタジー}を言語化しようとしていたり、相手を喜ばせるだろうと思う可能性をほのめそうとしていたり、ある活動や役割について相手がどう感じるかを探ろうとしていたり、自分がそれについてどう感じているかを探る助けを求めていることがある。
セックスのよい交渉には、何をするのが楽しいだろうかということについての積極的な共同のディスカッションがしばしば含まれるものだ。
また、その限界や制約や退出条件についての会話も含まれることが多い。
\citep[p.76]{kukla18:_thats_what_she_said}
\end{quote}

こうした哲学者たちは、発話行為理論を用いて、セックスにつながる会話の種類を想像している。
たとえば、招待を伴う会話や「贈り物の申し出」を含むものがある\citep[p.74]{kukla18:_thats_what_she_said}。
いずれの場合も、セックスには参加者両者の積極的な関与が必要だ。
彼らは、性的暴行を単に「同意のないセックス」としてではなく、むしろ「参加のないセックス」として捉えるべきだと主張している。

BDSMを実践する人々もまた、自分たちには同意をめぐる議論に提供できる\ruby{教訓}{レッスン}があると考えている。
その中でも特に興味深いものの一つが「アフターケア」だ。
この概念は、性的コミュニケーションは単にセックスの前だけおこなわれるべきものではないことを強調している。
セックスの後にもパートナーと話をし、それが相手にとっても楽しい経験だったかどうかを確認すべきだ。
これにより、性的コミュニケーションは単に相手の許可を得ること以上のものとして概念化される。
性的コミュニケーションは、関わるすべての人の欲望や感情に調和する継続的なプロセスであるべきなのだ。

同意の役割を再考し、それを超える新しい性的コミュニケーションのモデルを提案するこれらの試みは比較的新しいものだ。
しかし、現在多くの哲学者が新しい同意モデルの構築に取り組んでおり、今後数年の間にこの分野の思考がさらに発展していくことが期待される。

\section{討論のための問い}

\begin{enumerate}

\item 同意教育プログラムの焦点は何であるべきか? 強調すべき主要な点は何か? そのメッセージは、対象となる聞き手の年齢によって変えるべきだろうか?

\item  私たちが、ある人々はセックスに同意する十分な能力を持たないかもしれない{\DDASH}たとえば未成年であるとか、何らかの理由で判断力が損なわれているなど{\DDASH}と同意するのであれば、同じレベルの能力の欠如が、性的暴行の加害者として訴えられた人物の責任を軽減する理由となりうるだろうか?

\item  たとえ私たちが、性的な欺きのすべてが犯罪とされるべきだとは考えないとしても、例外的にそれが犯罪とされるべきだと考えるケースはあるだろうか? もしあるなら、それはなぜか?

\item  大学が、刑法には規定されていない肯定的同意(affirmative consent)の基準をキャンパス内で施行する権限を持つべきだろうか? 継続的な関係にある人々と、一時的な関係にある人々とで、同意に関する基準を変えるべきだろうか?
\end{enumerate}

\chapter{コミットメントと結婚}

2011年のクリスマス直前、イタリア人のアントニオは古いタンスを整理していたところ、妻のローザが恋人と交わした一連の手紙を見つけた。
彼は妻を問い詰め、彼女は事実を認め、アントニオは離婚を申請した。
このような出来事は残念ながら日常的に起こっている。
しかし、このケースが特異なのは、アントニオが99歳であり、ローザが97歳であったこと、そして彼らが結婚して77年も経っていたことだ。
不倫は1940年代のことだった\citep{squires11:_divor_wife_he_discov_affair}。

前の章ですでに、セックスはコミットメント関係のために取っておくべきかどうかという問題について論じた。
どの立場をとるにせよ、多くの人々にとってセックスとコミットメント関係が密接に結びついていることは否定できない。
私たちの多くは、人生のある時点でコミットメント関係の一部となる。
しかし、そのような関係の形態はさまざまであり、私たちはパートナーと共に関係のあり方を自由に定義する権利をもっているとはいえ、そうした決定には倫理的な側面があると私は主張する。
本章では、モノガミーを倫理的観点から考察する。
モノガミーの人生は最良の生き方なのか。
そして、私たちはパートナーに対してモノガミーを求めることが公平なのか。

西洋諸国では長年、結婚が合法的にセックスを許される唯一の手段であった。
現在でも、結婚は特定の種類の性的関係に特別な地位、つまり社会や国家における正統性を与えている。
それだけではない。
結婚は法的な便益を提供し、法的な義務を課す。
これにより、性的関係は社会的な地位と結びつくことになる。
私たちは市民として、どのような関係がこのような承認に値するのかを決定しなければならない。

この章では、現在25ヵ国以上で合法化されている同性婚について論じる。
現在の傾向として合法化が進んでいることを考えれば、同性婚の問題はすでに決着がついたものと考えられるかもしれない。
実際、同性婚を認めた法域において、ゲイやレズビアンが結婚する権利を失うことは想像しがたく、また、これらの国々では同性婚を支持する世論が多数を占めている。
しかし、世界の多くの国々ではいまだにゲイやレズビアンに結婚の権利が認められておらず、また、合法化された国々においても、依然として多くの人々が同性婚に反対している。
この問題は、いまだに議論の対象であり続けている。

同性婚をめぐる論争を超えて、ゲイやレズビアンに対する結婚の平等が達成されたことは、依然として未解決の多くの問題を提起している。
たとえば、同性婚を支持する論拠は、複数の人と結婚したいと望む人々にも適用されるのだろうか。
個人の生き方に対して中立であることが求められるリベラルな民主国家において、そもそも国家が結婚を認めるべきなのだろうか。
国家による結婚の承認は、独身者に対する差別を生み出しているのだろうか。
本章では、結婚の概念を根本的に再定義し、非恋愛的な関係の法的な承認を認めるべきだとする議論や、結婚を国家が関与しない単なる私的契約へと変えるべきだとする議論についても考察する。

\section{モノガミー}

ほとんどの人々は、自分のパートナーが\ruby{一夫一婦}{モノガミー}的であることを期待している。
そして、アントニオとローザの例が示すように、この期待が裏切られた場合、人々はそれを非常に深刻なこととして捉える。
しかし、社会的慣習としてのモノガミーはけっして普遍的なものではない。
多くの人々は、長期的でお互いにコミットした関係はモノガミー的なものになるだろうと当然のように考えているが、それは私たちが社会として、この期待を作り出すことを選択してきたからにすぎない。
動物の中でモノガミーであるものはほとんどなく、多くの人間社会では、少なくとも一部の男性に対しては一夫多妻制が許容されてきた。
さらに、ごく少数の社会では一妻多夫が認められており、また結婚制度そのものが存在しない社会もある。

本節では、次の問いを提起する。
すなわち、私たちはモノガミー的であるべきなのか?あるいは、言い換えれば、モノガミーには何らかの価値があるのか?その価値は、私たち自身、お互い、そして社会全体に対する代償を上回るのか?私たちはモノガミーという社会的規範を支持すべきなのか、あるいは積極的にこれに挑戦すべきなのか?ここでは、モノガミーを支持する理由と、また他の選択肢を採用する可能性のある理由を検討する。

\subsection{社会的規範としてのモノガミー}

「{モノガミー}(一夫一婦)」という用語は、次の二つの意味のいずれかを指す。
一つは、ただ一人との恋愛関係(romantic relationship)に専念するという\ruby{誓約}{コミットメント}であり、もう一つは、一人とのセックスにのみ限定するという誓約だ。
社会科学者は、前者を社会的モノガミー、後者を性的モノガミーと呼ぶのだ。
これら二つは通常は連れそって現れるものだが、必ずしもそうだとは限らない。
両者を受け入れる人もいれば、両者を拒絶する人もいる。
しかし、一方のみを実践し、他方を実践しない人々も存在する。

いずれの意味においても、モノガミーは私たちの社会における規範だ。
しかし、それが強制されているということではない。
私たちのなかには独身を貫く者や、単なる\ruby{気軽}{カジュアル}な関係を選ぶ者もある一方、複数との\ruby{親密関係}{リレーションシップ}を形成する人々や、親密関係の枠外での恋愛または性行為を許容する人々も存在する。
彼らがそのために法的制裁を受けるわけではない。
しかし、大多数の人々は、ある時点で\ruby{恋愛的}{ロマンティック}にも性的にも排他的な関係に落ち着くことを期待している。
そして、たとえ短期的な関係であっても、その継続期間中はたいていは排他的だ。
多くのメディア評論家は、モノガミーが何らかの形で崩壊しつつあるのではないかと懸念している(またはその崩壊を喜んでいる)。
2015年9月11日号の\emph{The Time}は、表紙に「モノガミーは終わったのか?」という大見出しで特集を掲載した\citep{magazine15:_is_monog_over}。
\emph{The Guardian}や\emph{NBC News}を含む他の主要な報道機関も、「モノガミーは死んだのか?」という見出しの記事を掲載している\citep{alexander05:_is_monog_dead,jeffries12:_sex_issue}。
しかし、現状のデータは、今現在において恋愛やセックスにおける排他性が広く拒否されていることを示唆しているわけではない。
ほとんどの人がこれらを実践し続けているという純粋に統計的な意味では、社会的モノガミーと性的モノガミーはどちらも\ruby{標準的}{ノーマル}だ。

だが、それにもかかわらず、何らかの形態の\ruby{非}{ノン}モノガミーに関与している人々の数はけっして\ruby{僅か}{トリヴィアル}ではない。
入手可能なデータは乏しいものの、ある研究では約4\%の人々が同意に基づく非モノガミーの関係にあると報告している。
また、ある時点で20\%の人々がそのような関係にあったとされる\citep{haupert17:_preval_exper_consen_nonmon_relat}。
2020年におこなわれた米国成人に対する調査によれば、彼らの三分の一が、自分の理想とする関係を何らかの形で非モノガミー的であると述べ、ミレニアム世代の43\%がこれに含まれている\citep{ballard20:_millen_are_less_likel_want_monog_relat}。

いずれにせよ、こうしたことが倫理的検討の対象になるとは思われないかもしれない。
\ruby{交際関係}{リレーションシップ}は、他の自由に参加する合意と同様であり、当事者は、提示された条件を受け入れるか、他の選択肢を探すことができるのだ、と。
しかし、もし私たちが自律的な生活を送ろうとするのであれば、自己の選択や信念が検証に耐えうるかどうかを吟味してみる必要がある。
モノガミーは、慣習または習慣によって受け入れられているにすぎない可能性がある。
もし私たちがそれを真剣に考えたならば、そんな選択はしないかもしれない。
また、モノガミーは単に自分自身について選択するにとどまらない。
私たちはパートナーにもそれを求める。
実際、私たちはしばしば交際する条件として相手にモノガミーを課す。
この種の制限を他者に課すことが倫理的であるかどうかを私たちは問うことができる。

一部の人々は、モノガミーはより強い意味で社会的規範として機能しており、人々はそれに従うよう圧力を受けていると主張する。
彼らは、社会が人々にモノガミーを強いる圧力を説明するために、「\ruby{モノガミー主義}{モノガミズム}」、「モノノーマティビティ」または「強制的モノガミー」といったレッテルを用いる\citep[p.277]{emens04:_monog_law}。
エリック・アンダーソン\ig{Eric Anderson}は著書 \emph{The Monogamy Gap} で次のように述べている。
モノガミーは「健全で適切で道徳的かつ自然であると信じこむように私たちは育てられている。
そして、この規範から逸脱し、そうした\ruby{行動パターン}{スクリプト}に挑戦する者は\ruby{烙印}{スティグマ}を押されることになる」\citep{anderson11:_monog_gap}。
もし、社会が本当にこの実践から逸脱する者に対して烙印を押そうとしているとすれば、それはモノガミーの倫理性を検証し、この規範が十分に根拠づけられているかどうか、またその烙印に抗すべきかどうかを判断するもう一つの理由となる。

\subsection{モノガミーの代替手段}

モノガミーについての賛否両論の議論を評価する前に、少なくとも最も一般的なその代替手段が何であるかを考察する必要がある。
まず第一に、社会的モノガミーを維持しながら性的モノガミーを拒否する人々が存在する。
この中には、「オープン\ruby{関係}{リレーションシップ}」を維持しようとする人々が含まれる。
この関係、つまり、各個人が他の人々とセックスをすることは許されるが、通常は\ruby{主要な}{プライマリー}関係以外の者との間に感情的な絆を形成しないという条件が課せられる。
ダン・サヴェッジはこのような関係を「モノガミッシュ」と呼んでいる\citep{savage12:_savag_monogamish}。
オープン関係のパートナーは、他の者に関して許される行為について、さらに特定のルールを設けている場合が多い。
たとえば、同一人物とのセックスは一度のみにかぎるとか、一方の旅行中のみ他とのセックスが許されるといった具合だ。
このような関係は「スウィンギング」と混同してはならない。
スウィンガーは、主要なパートナーの眼前で、またはそのパートナーが参加するイベントにおいて、他の者とセックスする人々だ。

「ポリアモリー」という用語は、しばしば非モノガミーの同義語として用いられるが、厳密には、もっと特定的な意味を持つ。
ポリアモリーを実践する者は、社会的モノガミーおよび性的モノガミーの両方を拒否する。
彼らは複数のパートナーと継続的かつ感情的な関係を維持する。
ポリアモリーを実践する者の中には主要なパートナーを維持する者もいるが、他方でパートナー間に階層を設けないよう努める者も存在する。
時として、ポリアモリーを実践する者のパートナー同士も互いに関係を持つ場合があるが、必ずしもそうであるとは限らない。

以下に続く議論の大部分は、非モノガミーのあらゆる形態に当てはまる。
しかし、そうでない場合には、その旨を示すことにする。

\subsection{モノガミーは自然なものか?}

一部の者は、モノガミーがその代替手段よりも好ましいか否かという問題は、自然への訴えによって解答されえると主張する。
この立場によれば、私たちは本質的に生物学によって特定の種類の関係を求めるようにプログラムされており、この自然な傾向を否定すれば、結果として不満を抱くことになる傾向がある。
しかしながら、私たちの本性が何を要求するかについては、人々の意見が一致しているわけではない。
心理学者のヘレン・フィッシャーは、困難な環境下で\ruby{生存}{サバイブ}する手段として、私たちは永続的なペアボンドを作るように、またそれを求めるように進化してきたと考える。
彼女は次のように述べている。
「私たちの祖先が危険な土地での生活を選択するにつれて、\ruby{つがい形成}{ペアボンディング}は女性にとって不可欠のものとなり、男性にとっても実用的なものとなった。
そして、モノガミー{\DDASH}一度に一人相手との\ruby{つがい}{ペアボンド}を形成するという人間の慣習{\DDASH}が進化してきた」\citep[p.131]{fisher04:_why_we_love}。
フィッシャーによれば、私たちは進化の歴史の結果として、セックスと親密さを結びつけ、一人との永続的なコミットメントを形成したいと欲求するように作りあげられている
(Zeifman and Hazan, 1997およびFisher, 1992を見よ)。
\nocite{zeifman97:_proces_model_adult_attac_format}\nocite{fisher92:_anatom_love}
一部の理論家は、私たちの自然的な傾向は生涯にわたる絆を形成することだと考えるが、フィッシャー自身は、実際には私たちは数年ごとに新たな絆を形成する\ruby{経時的}{シリアル}モノガミー主義者として進化したと考える。

論争を呼んだベストセラー『性の進化論』\citep{ryan10:_sex_dawn}の共著者であるクリストファー・ライアンとカシルダ・ジェタは、私たちが排他的な関係を\ruby{固定的に組み込まれている}{ハードワイアド}という考えに異議を唱える。
ライアンは「モノガミーが人間にとって自然に生じるものだと信じる理由は何もない」と述べ、「実際、何百万年にもわたって進化の力は人間の\ruby{性欲}{リビドー}を育んできた。
そしてその結果、私たちはおそらく地球上で最も\ruby{性欲的}{セクシュアル}な種であると言えるまでになっている」と主張する\citep{ryan10:_monog_unnat_our_sexy_species}。
ライアンとジェタは、進化が示す物語はフィッシャーが提示するものとはまったく異なると論じる。
彼らは次のように述べる。

\begin{quote}

いくつかのタイプの証拠は、農業以前(先史時代)の私たちの祖先が、成熟した個体のほとんどが同時に複数の継続的な性関係を持っていた集団で生活していたことを示唆している。
たとえこれらの関係がしばしば\ruby{一時的}{カジュアル}なものであったとしても、それは無作為または無意味なものではなかった。
正反対に、これらは高度に相互依存している共同体を一体に保つ重要な社会的絆を強化していた。
\citep[pp.9--10]{ryan10:_sex_dawn}
\end{quote}

ライアンとジェタは、モノガミーは数千年前の農業革命の際に発明され、男性が自分の財産を保護する手段として女性に課せられたものだと主張する。
すなわち、いったん土地の一区画を自分の財産として所有すると、その土地を次世代に引き継ぐべき子供が自分の子供であることを確実にする必要があるのだ。

一般に、\ruby{自然}{ネイチャー}への訴えには非常に慎重になされる必要がある。
まず第一に、私たちの行動のうち、どれが自然な傾向に根ざしているかを特定することは、ほぼ不可能に近い。
人間は特定の社会環境の中で育ち、際限なく環境によって形成されている。
これは、私たちが白紙として生まれ、私たちに関する何事も生物学の結果ではないということを意味するわけではない。
むしろ、「私たちはこういうものだ」という主張は証明が常に非常に困難だということだ。
第二に、進化によって私たちがおこなうようにプログラムされていると主張できる多くの行動が存在するが、それらをおこなわない方がむしろ望ましい場合もある。
たとえば、高脂肪・高カロリーの食物を摂取することがその例として挙げられる。
かつては、飢饉の時代を生き延びるためにそのような食物に依存していた。
しかし、現在では安定した食糧供給があるために、そのような食物を過剰に摂取する食生活は長期的には健康問題を引き起こす。
これは、進化が現代社会の文脈の中で私たちを誤った方向へ導いている一例だ。

しかし、これらの事実から、私たちの\ruby{本性}{ネイチャー}への訴えにまったく価値がないと結論づけるべきではない。
生物学が私たちの行動を完全に決定することはできなくとも、それに制約を課すことは可能だ。
仮に私たちが食生活を自由に選択できたとしても、すべてのものを食べられるわけではない。
たとえば、反芻動物のような草食中心の食生活は私たちが適応していないものであり、そのような食生活を追求することは、せいぜい無意味であり、最悪の場合有害だ。
恋愛やセックスに関しても、私たちの進化の側面の中には克服することが非常に困難なものがあるかもしれない。
もし、私たちがセックスと親密さを結びつけるようにあらかじめ\ruby{組み込まれ}{ワイアド}ているのであれば、複数の相手と性関係を持ちながら、なお一方でパートナーとの排他的で親密な絆を維持することは非常に難しいかもしれない。
しかし、もしライアンとジェタの主張が正しく、私たちが単に多くのセックスをおこなうために生まれてきたのであれば、モノガミーへの試みは、自己のけっして排除できない一部を抑制しようとすることになり、それは私たちの\ruby{幸福}{ウェルビーイング}と\ruby{親密関係}{リレーションシップ}の安定性の双方に影響を及ぼすだろう。
ライアンは「菜食主義者になると決めたからといって、ベーコンの香りがおいしそうでなくなるとは限りません」と述べる\citep{bishop10:_ask_chris_ryan_ph}。

また、男性と女性は異なる進化を遂げたため、モノガミーに向かう傾向に性差が存在する可能性があると論じる者もいる。
これは、男性にとってセックスは「\ruby{安価}{チープ}」なものであって、妊娠のリスクを伴わない上に、複数のパートナーを持つことで遺伝子を次世代に伝える可能性が高まるためだ。
一方、自分で子供を産む必要がある女性にとっては、セックスは負担が大きいものだ{\DDASH}妊娠は困難かつ危険であり、子供の生存と生育のためには手厚いケアを必要とする。
初期の社会において、女性は保護者および提供者としての男性に依存していた。
ダニエル・バーグナーは「親の投資理論」を説明しながら、男性は「安価な種子を広めることによって遺伝的遺産を拡大するようプログラムされているのに対し、女性は本質的に慎重さを求め、長期的な資源提供者となりえる男性を確保することで子供に対する投資を最大化するよう組み込まれている」と述べる\citep{bergner13:_unexc}。
その結果、男性は女性よりもモノガミーに対して関心が弱い。

科学者たちは、人間がモノガミーに進化したかどうかを判断するため、男性と女性の相対的な体格、私たちの遺伝的構成、最も近い霊長類の祖先の行動、さらには世界各地のさまざまな社会に関する人類学的研究など、さまざまなデータを検討している。
これらのデータは必ずしも同じ方向を示しているわけではないが、総合的には暫定的に、さほど劇的ではない結論を示唆している。
すなわち、人間は状況に応じてモノガミーにも非モノガミーにも傾く性質がある。
生物学者のロバート・サポルスキーは「私たちは古典的なペアボンドを形成する種ではない。
私たちはまた、一夫多妻やトーナメント型の種でもない……公式には、私たちは悲劇的に混乱した種である」と述べる\citep{sapolsky25:_biolog_human_behav}。
これはけっして私たちを驚かせるものではない。
キャリー・ジェンキンスは、人間が多様な環境に適応してきたことを指摘し、私たちのパートナーシップの形態もそれに応じて変化するのは理にかなっていると論じる。
彼女は「私たちは多様で適応力のある種です。
したがって、子供を育てるという課題に対して多様なアプローチを可能にする一連の生物学的メカニズムが存在することを予測すべきです。
柔軟性こそが、人間としての私たちの特徴です」と述べている\citep{dominus17:_is_open_marriag_happier_marriag}。

哲学的観点からすれば、この結論はまったく歓迎されないものだというわけではない。
\ruby{証拠}{エビデンス}が示しているのは、哲学者たちが一般に想定する通り、私たちは感情的および性的生活において道徳的選択を行い、自分自身の良い生のビジョンを決定する能力を持つ存在だということだ。
自然は、私たちが特定の種類の関係にのみ幸福を感じるようあらかじめ決定しているわけではない。
これは、私たちがモノガミーの賛否に関する議論について、いずれの可能性にも心を開いて検討することができるということを意味する。

\subsection{真実の愛はモノガミーを意味するか?}

一部の哲学者は、モノガミーが愛そのものの本性に組み込まれていると論じる。
彼らは、「愛」という概念、あるいは少なくとも「真実の愛」という概念そのものが、社会的かつ性的なモノガミーを必然的に意味すると主張する\citep{mckeever17:_is_requir_sexual_exclus_consis_roman_love}。
ジェンキンスは、この主張を表すために「モーダル・モノガミー」という用語を用いている。
これはアラン・ソーブルが述べる「排他性は実際にロマンティックラブにおける\kenten{本質的な要素}だ」という主張を記述するためのものだ(Jenkins, 2015; Soble, 1987, p.389)。
\nocite{jenkins15:_modal_monog}\nocite{soble87:_unity_roman_love}

なぜ私たちは愛がそのようなものだと考えるのだろうか。
プラトンの『饗宴』に遡る哲学的伝統では、愛は二人の者が統一されたアイデンティティを形成するために結合することを伴うとされている。
ロジャー・スクルートンは、モンテーニュに基づいて、「評価という友情は、相互性が共同体となるやいなや、すなわち私の利益とあなたの利益との区別のすべてが克服されるやいなや、愛へと変わる」と述べる。
同様に、ロバート・ノージックは、恋愛は「世界に新たな実体、すなわち「\ruby{我々}{ウィ}」と呼ばれるものを形成し構成する欲望によって特徴づけられる」と述べる\citep[p.230]{scruton06:_sexual_desir}。
ノージックは、このように捉えられる愛において、モノガミーが必要な特徴である理由を次のように説明する。

\begin{quote}
私は、ロマンティックな欲望とは、特定の一人の相手と「\ruby{我々}{ウィ}」を形成すること、\kenten{そして}他の誰ともそうしないことだと信じている。
これに関わる「アイデンティティ」という観念の強い意味においては、個人が自分のアイデンティティを構成する\ruby{複数の「我々」}{ウィズ}」の一部でありえるわけがなく、また同時に複数の個人のアイデンティティを有することもできない。
……他の誰でもなくその相手と「\ruby{我々}{ウィ}」を形成したいという欲望は、その相手自身が他の誰でもなくあなたとだけ「\ruby{我々}{ウィ}」を形成することを望んでほしいという欲望も内包する。
そして、性的欲望がロマンティックラブと結びつき、その表現の手段となり、かつそれ自体がより一層強まるならば、性的モノガミーに対する相互の欲望はほぼ必然的なものとなる。
それは、最も強烈な肉体的親密さをその特定の相手一人に向けることによって、彼または彼女とのアイデンティティ形成における親密さと唯一性を示すためだ。
\citep[p.89]{nozick89:_examined_life}
\end{quote}

ヘレン・ハリス\ig{Helen Harris}は、「合一または融合への欲望」と「排他性」の両方が、「ほぼあらゆる文化的環境において真に恋に落ちたという経験に共通するものだ」と論じている\citep[pp.102--103]{harris95:_rethin_polyn_heter_relat}。

たとえ真実の愛がアイデンティティの融合を伴うと受け入れたとしても、このアイデンティティの統一が、なぜ必然的に性的排他性を意味するのかという疑問は残る。
一つの可能な回答は、もし本当にパートナーに恋しているのであれば、他の者に対する欲望を感じることすら不可能になるはずだということだ。
研究者たちは、この現象、すなわち真実の愛が他者への欲望を消し去るとされる能力を「\ruby{黙らせ}{サイレンシング}」と呼んでいる。
これは多くのラブソング、たとえば「君しか見えない」(I only have eyes for you)といった歌詞の基盤となっている。
しかし、真に恋に落ちたときの感情についての心理学的主張としては、これはもっともらしくない。
実際、1946年以来、妻ロザリンと幸福な結婚生活を送っていると評される敬虔な元大統領ジミー・カーターでさえ、1976年の選挙運動中に「多くの女性を情欲をもって見たことがある」と公に認めた。
彼だけでなく、幸福な関係にある間にも他の人物への欲望を感じる人は少なくない。
たとえば、ある研究では、少なくとも三年以上の関係にあった女性の70%が、パートナー以外の人物に\ruby{恋心}{クラッシュ}を抱いたことがあると認めたことが示されている\citep{barnhart16:_women_exper_feelin_attrac_someon}。
もし、真に恋に落ちたと主張するために、他者への欲望がまったくないことを要求するならば、世の中にはほとんど恋愛が存在しなくなるだろう{\DDASH}これは悲しい結論であり、必然的なことでないことはたしかだ。

多くのモノガミーの擁護者は、コミットメント関係にある間にも他の人々への欲望を感じることがあることを否定しようとはしない。
しかし、彼らは、たとえ他の人への欲望を感じたとしても、理想とする愛の形を実現するためにはその欲望を抑制すべきだと主張する。
なぜそのようなことをするべきなのか? マッキーバーは、これがカップルがお互いに対するコミットメントを示すための簡単かつ効果的な方法を提供するという。
彼は次のように述べる。
「恋人たちは、さまざまな行動を通じて、その関係が\ruby{独自}{ユニーク}で重要であることを示そうとする傾向がある。
これらの行動は、恋人たちの共有するアイデンティティを構築し、確認し、祝福するものだ。
もし彼らがそのアイデンティティを排他的に共有するのであれば、このことを確認するために、いくつかの行為を排他的におこなうことが重要となるであろう\citep[p.361]{mckeever17:_is_requir_sexual_exclus_consis_roman_love}。
ハリー・リベルトは、この議論(彼女自身は賛同していない)が、アナロジーによって説明されると指摘する。
彼女は、「私たちは、アクセスがより排他的であるか、または希少なものである時に、物事に対してより大きな価値を見出す傾向がある。
たとえば、有名な絵画が公共の美術館ではなく私邸に所蔵されている場合、それを鑑賞する経験はより特別なものとなる」と述べる\citep[p.410]{liberto17:_prob_sexual_prom}。
すなわち、排他性は、私たちが愛する者が私たちの生活に占めるべき唯一無二の地位を確立する手段だ。
マッキーバーは、性的排他性が「恋人たちに、彼ら自身だけが共に所有する空間を提供する。
彼らの性的世界は、彼らだけが住む世界となる」と述べる\citep[p.61]{mckeever17:_is_requir_sexual_exclus_consis_roman_love}。
もちろん、セックスがカップルが関係のために排他的に確保できる唯一のものだというわけではない。
しかし、その親密さと強烈さゆえ、一つの論理的な候補となる\citep[p.6]{york19:_why_monog_is_moral_permis}。

\subsection{モノガミー関係はより安全で安定しているか?}

擁護者たちは、モノガミー関係では、性的非モノガミーに比べて、性的貞節がさまざまな面でより守られやすいと主張する。
まず第一に、それはパートナーたちが性感染症に感染する危険性を排除する。
性感染症は、特定の関係内での問題であると同時に、潜在的には公衆衛生上の問題ともなりえる。
私たちは、社会全体の感染率を著しく高めるような行為を推奨するためらわざるをえない。
また、複数のパートナーとのセックスは、望まない妊娠のリスクを高める可能性がある。
望まない妊娠は、その親のみならず、その親たち支援しようとする社会に対してもしばしば大きな負担をもたらす。

非モノガミー的な\ruby{交際関係}{リレーションシップ}については、排他的な関係ほど長続きしないという主張も一般的だ。
多くの人々は、いかなる非排他的な関係も、モノガミーに伴う安定性を享受することはほぼ不可能だと考える。
たとえば、オープン関係においては、常に\ruby{主要な}{プライマリー}パートナーよりも好ましいと感じる他者と出会い、関係を結んでしまうリスクが存在する。
その結果、その新しい人物の方と優先的な、あるいは排他的な関係を結びたいと判断してしまう可能性がある。
ポリアモリーの関係においては、関係内の他の人物と常に比較されることにより自分の欠点が露呈しやすくなると不安になるかもしれない。

また、非モノガミーの関係を不安定にする要因として、嫉妬の問題が挙げられる。
これは、モノガミーの擁護者たちが普遍的な人間の感情だと主張するものだ。
ジェローム・ニューは、「いかなる社会体制下においても嫉妬は排除不可能だと信じる理由がある」と述べる\citep[p.43]{new00:_jealous_thoug}。
進化心理学者は、嫉妬を私たちの発達の深部に根ざした「先天的モジュール」と呼んでいる。
デヴィッド・バスは、「まったく嫉妬のない熱帯の楽園の文化といったものは、楽観的な人類学者のロマンティックな心の中にのみ存在し、実際には見つかったことがない」と主張する\citep[p.961]{buss01:_human_natur_cultur}。
いかなる人々も複数のパートナーと関係を持つ、または他の人々との関係を許容している限り、その関係は嫉妬と、それに伴う不幸に悩まされると論じられている。

さらに、複数の関係を同時に維持しようとする試みからは、実際的な問題も生じる。
非モノガミーの経験についてブログを書いているフランクリン・ヴォーは、関係には「ある程度の二人きりの時間」が必要であり、二人のパートナーがいると、それぞれと過ごす時間が減少する可能性があると認める\citep{veaux09:_some_musin_time_manag}。
アラン・J・ホーキンス、ベッツィ・ヴァンデンバーグ、ライネ・バーロウは、非モノガミーの結婚において、他の関係からのメールやメッセージによって妻が「私たちだけの瞬間から引き離されるんです。
つまり、私たち二人の関係に、常にそこにいるけれどそこにはいない第三者が入りこんでいるのです」と不満を述べる夫の発言を引用している\citep{hawkins17:_new_math_consen_nonmon}。
さらに、たとえば仕事や家庭の理由で新たな都市に移るかどうかといった大きな人生の決断において、複数の人々が関与することで共同意思決定がより複雑になる。

もし子供がいる場合には、非モノガミー関係の潜在的な不安定さが子供たちに悪影響を及ぼす可能性も指摘される。
一般に、私たちは子供に最良の環境を提供したいと望み、また社会として彼らが健やかに成長できる環境を育むことを求める。
もし安定したモノガミーのパートナーシップが子供にとって全体的に好ましいのであれば、それはモノガミーを理想として推進する十分な理由となる。

\subsection{モノガミーは社会に利益をもたらすか?}

モノガミーの擁護者たちは、モノガミーという制度と文明社会の健全さの間には関連があると主張してきた。
\emph{New York Times} に掲載された、「\ruby{開かれた}{オープン}結婚は\ruby{閉じた}{クローズド}結婚よりも幸福か?」という記事について、保守派の作家ロッド・ドレーアは次のようにコメントしている。
「このようなことは、家庭の崩壊を意味し、最終的には社会の崩壊へとつながる」\citep{dreher17:_perver_progr}。
また、一部の歴史家は、現代社会が達成してきた進歩は、モノガミーを社会規範として採用したことに起因すると考えている。

なぜそうなのだろうか? ある議論のラインでは、モノガミーが私たちに性的欲求の安定した発散の場を提供し、その結果、私たちは新たな性的経験の絶え間ない追求にエネルギーを費やすことなく、すべての人々に利益をもたらす協力的活動に注力できるとされる。
エリザベス・パーディは、「人々の親密関係の境界が存在し、それが人々の自己抑制によって尊重されるならば、協力関係が生じやすく、文明の進歩が促進される。
自己抑制を放棄し、境界を踏み越え、無頓着に性的欲求に耽ってしまうことは、進化的な逆行だ」と述べている\citep{pardi19:_no_human_being_arent_happier}。

さらに、モノガミーが社会不安定の源となりがちな若い男性を落ち着かせ、規律を与える働きを持つと論じる人々もいる。
コミットメント関係は、彼らに安定した性的欲求の発散の場を提供するとともに、他の人々に対して責任感と配慮を示すよう圧力を加える。
比較文化研究の研究者らは、「規範的モノガミーは、同性内競争を抑制し、未婚男性の数を減少させることにより、レイプ、殺人、暴行、強盗、詐欺といった犯罪率を低下させ、対人的虐待も減少させる」と論じる\citep[p.657]{henrich12:_puzzl_monog_marriag}。
さらに、モノガミーは男性が自らの子孫の養育に投資することを促すと論じられている。
モノガミーは、男性に自分が子供の父親だという確信を与え、家族の福祉に資するために働く意欲を高め、他のパートナーを追い求めるために時間や資源を浪費することを防ぐ。
この研究者らは次のように言う。

\begin{quote}
男性の努力を妻探しから父親としての投資へとシフトさせることにより、規範的モノガミーは貯蓄、子供への投資および経済生産性を向上させる。
また、家族内の親密な関係を強化することにより、家族内の対立を減少させ、その結果、児童のネグレクトや虐待や事故死および殺人の発生率を低下させる。
\citep[p.657]{henrich12:_puzzl_monog_marriag}
\end{quote}

\subsection{モノガミーは性格の美徳を促進するか?}

モノガミーの支持者たちは、一人の人物に自らをコミットすることにより、望ましい性格の美徳が内面に育まれると主張する。
持続するコミットメントは、他者のニーズに適応し、応答することを要求する。
また、それは誠実さと信頼を示すことを求める。
グレイシー・オルムステッドは、「モノガミーは、その相手の癖や習慣に耐えることを強いる。
さらに、配偶者によりよく奉仕するために、自らの悪しき習慣や利己的な習慣を変革・改正することを強いる」と述べる\citep{olmstead14:_in_defen_monog_marriag}。

仮に誘惑に直面してもパートナーに忠実であり続ける能力は、自己制御を育み、一般に約束を守る能力を高める。
まさにモノガミーが困難であるがゆえに、その擁護者は、他者のために犠牲を払うことを教えると主張する。
これにより、私たちはより利他的な人物へと形成される。
対照的に、デヴィッド・フレンチは、非モノガミーの実践者が自己への評価に基づいて行動していると非難する。
\emph{The New York Times}における複数のオープン結婚の描写に関して、彼は「忌まわしいのは、関与する一方または双方の配偶者の純粋な利己主義だ。
彼らの、完全に充足し陶酔する\ruby{性生活}{セックスライフ}への執着は、病的な域に達している」と述べる\citep{french17:_this_is_how_elite_poison_our_cultur}。
ミルトン・リーガンは、モノガミーを通じて培われる利他性は、現代社会を特徴づけているおおぴらな個人主義に対する代替案を提供しているとみなす。
彼はこれを、「関係的自己を強調する代替的なビジョン」として、すなわち、自己を本質的に共同体の一部であるものと見なし、自らの個人的な\ruby{必要}{ニード}をより大きな善に従属させる自己の一つのビジョンとして捉える\citep[p.159]{regan96:_postm_famil_law}。

\subsection{非モノガミーの擁護論}

非モノガミーの擁護者は、私たちは二つの重要な事実に注目すべきだと主張する。
すなわち、モノガミーは多くの人々を非常に不幸にするように見えるということ、そしてそれは私たちが極めて不得意なものだということだ。
ローラ・キプニスは、結婚している人々のうち実際に幸福だと自認している人々が少数派であるとする研究を引用している\citep{kipnis03:_again_love}。
一方、調査は関係内の不貞率について非常にばらつく統計を示しているが、最も低い推計でさえ、少なくとも四分の一の人々がある時点でパートナーに不忠実だったことを示唆している\citep{blow05:_infid_commit_relat_ii,whisman07:_sexual_infid_nation_survey_americ_women}。
言い替えれば、非常に多くの人が、理論上はモノガミーにコミットしているにもかかわらず、実際にはモノガミーを拒絶しているという結果となっている。
米国心理学会の合意に基づく非モノガミータスクフォースの共同議長であるヒース・シェチンガーは、「毎年毎年、人々がカップルカウンセリングを受けたり離婚を求めたりする第一の理由は不貞です。
だから、私たちは丸い穴に四角い杭を突き刺そうとし続けるか、さもなければ柔軟な発想で問いを投げかけ始めるべきだ」と述べている\citep{hunt20:_psych_threes}。

非モノガミーの支持者たちは、モノガミーが困難であること自体はまったく驚くべきことではないと主張する。
彼らは、モノガミーがその本性においていかに不可解な制限であるかに私たちが気づいていないと述べる。
モノガミーは本質的に制限であり、実際のところ非常に厳しい制限だ。
モノガミーによって、人生の最も基本的な二つの\ruby{善}{グッド}、すなわち親密な交際とセックスへのアクセスが制限される。
大多数の人々にとって、セックスおよび情緒的な親密さは良い生活の重要な要素だ。
もちろん、だからといって、これらのものを可能な限り多く求めるべきだということにはならない。
しかし、親密な絆を形成し、望む時にセックスをする自由があることは、そうすることを禁止されるよりも、少なくとも\ruby{一見のところは}{プリマフェイシー}望ましいと考えられる。
また、一般的に、もし私たちが誰かを大切に思うのであれば、その人の基本的な\ruby{善}{グッド}が増えることを望むはずだ。
ロナルド・デ・ソウザは次のように述べる。

\begin{quote}

私が、私が愛する人が喜ぶ事柄を、その喜びを与える者が私でないという理由で禁じてしまう、などということに誇りを感じる理由などあるだろうか。
愛が愛する者に対する配慮を伴なっているということは当然のことだ。
もし私がある女性を愛しているなら、私はその人の欲望を自らのものとすべきだ。
では、彼女が喜んでいるものがなんであれ、私は彼女と同じようにそれに喜びを感じることに誇りを持つべきなのではないか? \citep{sousa18:_how_think_yours_out_jealous}
\end{quote}

ハリー・チャルマーズは、この点を説明するための思考実験を提示する。

\begin{quote}

二人のパートナーがロマンティックな恋愛関係にあり、かつ(ましてやさらに)友人でもあると想像してほしい。
しかし、彼らの関係は典型的なものではない。
なぜなら、両者は極めて異例の制限、すなわち追加の友人を持つことを禁じるという条件に合意しているのだ。
もしいずれかのパートナーが相手以外の者と友人関係を結んだ場合、もう一方はそれを支持せず、実際、愛情、好意、そして関係継続への意志を撤回するに至る。
私の考えでは、このような関係にはたしかに道徳的にやっかいな問題があると感じとる人が多いに違いない。
\citep[p.225]{chalmers19:_is_monog_moral_permis}
\end{quote}

親密さが有限の資源だと考えれば、自分の親密なパートナーへの他からのアクセスを制限することは理にかなっているかもしれない。
しかし、ほとんど誰も、友人や家族に情緒的な排他性を求めることはない。
私たちは、人々が両親それぞれやすべての子供、さらにはかなり多くの親しい友人に対して同等の愛情を抱くことを十分に受け入れている。
たしかにそのような関係において嫉妬が生じることはあるにしても、情緒的な親密さは有限の資源ではないという命題を私たちは受け入れている。
それどころか、なぜか私たちは恋愛関係を特別なカテゴリーに位置づけている。

非モノガミーの支持者は、この制限を撤廃することにより、ほとんどの人が重視する基本的な善、つまり、私たちの生活や身体に関して何をするかを選択する自由という基本的善が促進されると主張する。
モノガミーは固定的な期待や義務を伴う社会規範であるのに対し、非モノガミーは、私たち自身が望む関係の条件を決定し追求する能力をもたらす。
ジェフリー・ミラー\ig{Geoffrey Miller}は次のように述べる。

\begin{quote}
これはセックスに対するより\ruby{自由主義的}{リベラル}なアプローチであり、人々は企業間の契約や国家間の条約のように、\ruby{その人々に応じた}{カスタム}関係を交渉しつつ、一定の性的主権および配偶者選択の自由を保持することができる。
また、ポリアモリーは社会的・政治的な生活だけでなく、性的領域においても結社の自由を真剣に捉える。
もし、一人以上の子供、一人以上の友人、さらには一人以上の職場の同僚を選ぶことができるならば、性的パートナーも複数選ぶ自由があるべきだ。
\citep{miller19:_polyam_is_growin}

\end{quote}

ハーリー・リベルトは、他者との性的行為を禁じるというパートナーからの約束を要求することは、人の身体的自律を制限することになるため、実際に非倫理的だと論じる。
彼女は、性的排他性を約束させることは、相手が望まなくとも自らと性的関係を持つことを強いる約束と同等であり、同等に非難されるべきだと考える\citep{liberto17:_prob_sexual_prom}。

非モノガミーは、LGBTQ+コミュニティにおいて長い歴史を有している。
クィアの人々は、異性愛者よりも著しく頻繁にこれを実践している\citep{haupert17:_preval_exper_consen_nonmon_relat}。
LGBTQ+の人々の多くは、伝統的に結婚制度から排除されてきた一方で、自らが望ましいと思う条件で関係を構築することができたという事実を誇りに思っている。
ジェレミー・アレンは、「LGBTQの自由が、これまで何度となく破綻してきた異性愛の関係モデルを否定することを意味するのであれば、なぜモノガミーに甘んじるのか」と述べる\citep{allen20:_his_body_doesn_belon_me}。
キム・トールベアが指摘するように、オルタナティブな親族構造は長い間、クィアのアイデンティティの不可欠な部分であった。
LGBTQ+の人々は、しばしば主流社会の社会的ネットワークや自分自身の家族ネットワークからさえ排除されてきたため、新たなつながりを構築し、お互いに支援し合う方法を見出してきた。
そのような拡張された支援ネットワークの一環として、親密な関係に対してより広範な視点を持つことは当然であり、非モノガミーはこれらの重要な社会的絆を維持する一手段となりえる\citep{podcast18:_episod}。

非モノガミーは疑いなく多くの人々にとって困難だ。
しかし、チャルマーズは、たとえ私たちが個人的に複数のパートナーを持つことを望まないと決定したとしても、モノガミーを拒否する倫理的選択が可能だと論じる。
彼は「その鍵は、もし彼女が望むなら、あなたが彼女の複数の関係を受け入れる余地を常に残しておくことだ」と述べる\citep[p.241]{chalmers19:_is_monog_moral_permis}。
私たちのパートナー自身が複数の関係を望まない可能性もある。
しかし、互いにその可能性に対して心を開いたままであることができる。
「複数の関係に同時にあるという実態そのものではなく、このようなオープンさこそが、非モノガミーの本質だ」\citep[p.241]{chalmers19:_is_monog_moral_permis}。

\subsection{愛とモノガミー:応答}

非モノガミーの擁護者は、愛という普遍的な概念が存在し、それが排他性を含意するという主張に異議を唱える。
彼らはむしろ、アイデンティティの融合としての愛は、特定の文化や特定の歴史的時期に特有のものだと主張する。
さまざまなエビデンスを調査したサーベイ論文において、アン・ビールとロバート・スタンバーグは「愛とは、数多くの文化において異なる定義と異なる経験がなされる社会的に構築された概念だ」と結論づけている\citep[p.433]{beall95:_social_const_love}。
人類学者は、すべての人類社会のうち、モノガミーを排他的な規範として採用している社会は約6分の1のみだと推定している。
その他の大半の社会では、男性のみが複数のパートナーを持つことが許容されるが、女性が複数のパートナー、あるいは複数の夫を持つことを許容する社会の例も確かに存在する。
モノガミーに特に反対しない人々でさえ、複数の自己の融合としての愛という発想が潜在的に不健康だと論じている。
アラン・ソーブルは、そのような\ruby{合一}{ユニオン}は関与する個々人の自律を侵害するものであり、パートナーに自律という貴重なものを放棄させることを期待すべきではないと主張する\citep{soble97:_union_auton_concer}。
また、フェミニストたちは、カップルの利益の合一なるものが実際には通常、女性がパートナーの利益に自分を従属させることを意味し、また女性の平等を重視する観点から、モノガミーの関係内においても独立性と自律の重要性を認識すべきだと懸念している\citep{friedman98:_roman_love_person_auton}。
この観点から、非モノガミーは私たちの\ruby{関係}{リレーションシップ}における自律を維持する一つの方法だとされる。

前述したように、たとえ私たちが愛をアイデンティティの融合として受け入れたとしても、それが性的または情緒的な排他性を必ずしも要求するとは限らない。
オープンな関係にある多くの人々は、性的排他性を求めずとも、パートナーと「\ruby{我々}{ウィ}」を形成できると信じている。
先に私は、貞操(性的な忠実さ)がカップルにとって彼らだけに属する行為を提供するという議論を概説した。
しかしながら、多くの人々が性的多様性を求めるという欲望を考慮すれば、なぜ私たちがパートナーにこれほど厳しい制約を課すのか、あるいはもっと容易に他の行為を選択できるのではないかと疑問に思う余地がある。

ポリアモリーの関係にある人々は、複数の他者と「我々」を形成して団結することが可能だと信じているかもしれない。
ポリアモリー実践者たちは、複数の子供を持つ親を例として指摘できるだろう。
ロマンティックな愛は、親が子供に抱く愛情とは異なるものだが、親が子供たちの利益を自分自身の利益と一体として捉えるという性質は共有しているように見える。
たとえロマンティックな愛と親の愛が同一ではなくとも、親の愛は示唆に富むアナロジーを提供してくれる。
一人以上の者と真に利益が融合する感覚を抱くことが可能だということに信憑性をもたせてくれる。

\subsection{非モノガミー、安全性と安定性}

性感染症について、非モノガミーの擁護者は、禁欲を除けば原理的に完全な排他性のみがリスクを完全に除去する方法だという点を否定しない。
とはいえ、現実の状況は必ずしも明快ではない。
実際には、排他的な関係にある人々であっても、しばしばその関係外の相手とセックスしている。
前述した不貞の統計は、控え目な推計であってもそれがかなりの割合で生じていることを示している。
そして、そうした関係外のセックスでは、多くの場合に感染予防策が用いられておらず、ある研究によればその使用率は半分以下だ\citep{conley12:_unfait_indiv_are_less_likel}。
また、モノガミー関係にある人々は、\ruby{感染予防策}{プロテクション}を早々に放棄する傾向がある。
カップルは、性感染症の検査を受けることなく交際をはじめ、交際数ヶ月後にはコンドームをやめてしまう\citep{glauser11:_how_talk_patien_sti_screen}。
対照的に、非モノガミーを実践する者は、複数のパートナーを持つことを公言しているため、そのような出会いに備えて感染予防策の使用にきわめて厳格である擁護者たちは主張する\citep{soh16:_insig_kinky_nonmon_sex}。

非モノガミーの支持者は、新たな人々と出会うことに対するオープンさから生じる不安定性のリスクを認める。
しかし、チャルマーズはこれをむしろ受け入れるべきだと論じる。

\begin{quote}
私のパートナーが、私よりも良い選択肢を知らないままでいてほしいという希望には、たとえそれが深刻な不安を引き起こすものではないとしても、どこか不可解な点がある。
確かに、パートナーが他の誰かのために自分を捨ててしまうという事態を経験するのは、間違いなく苦痛だろう。
だが、次のような状況を想像してみてほしい。
……たとえば、モノガミーという制約のせいで、パートナーが、実際にはもっと幸福になれるかもしれない相手に出会えずにいる。
そのために、あなたと一緒にい続けているだけだとしたら、果たしてそれは本当に望ましい状態なのだろうか\citep{chalmers19:_is_monog_moral_permis}。
\end{quote}

いずれにせよ、このリスクは非モノガミーの人々に限ったものではない。
どんな長期的な関係においても、私たちは他に魅力を感じる人々に出会うことがあるのはほぼ必然であり、また、関係のある段階で少なくとも一方が情緒的または肉体的に飽きてしまうことも非常に一般的だ。

非モノガミーの支持者たちは、退屈が関係の終焉につながらないようにすることで、これらの課題に対する保護がもたらされると論じる。
すなわち、関係外でのセックスを許容することにより、自分たちのセックスがもはや満足をもたらさなくなった場合でも、関係が終わってしまったりパートナーが不満を抱いたりするのを防ぐことができる。
トリスタン・タオルミノは次のように述べる。
「モノガミーの世界では、何かがうまくいかない場合、通常は一緒にいつづけるか別れるかの二つの選択肢しかない。
しかし、ポリアモリーの人々には、たとえば関係が続くが形を変えるといった、はるかに多くの選択肢がある」\citep[p.217]{taormino08:_openin_up}。
また、ダン・サヴェッジは「世界中の\ruby{恋愛関係}{リレーションシップ}の墓地は、「すべてが素晴らしかった……ただしセックスを除いて」と刻まれた墓石で溢れている」と述べ、自分の(オープンな)結婚生活という論拠を示す。
「我々の関係において、それは不安定要因となるどころか、むしろ安定化要因であった。
それがおそらく我々が今も共にいる理由だ」\citep{oppenheimer11:_married_infid}。

非モノガミーの実践者は、時折「新パートナーエナジー」(NRE, New Relationship Energy、すなわち関係の新たなエネルギー)という概念について語る。
これは、新たなパートナーと初めて出会った際に感じる興奮を指す。
チェルシー・ダガーは「新たな関係の始まりは素晴らしい時間であり、新パートナーエナジーの高揚感の中では、全てが完璧に感じられ、相手も自分にとって完璧だ。
通常、争いはほとんどなく、セックスも驚くべきものだ」と述べる\citep{dagger18:_weve_all_been_there}。
非モノガミーは、コミットメント関係を維持しながらも、何度も新パートナーエナジー体験することを可能にするというのだ。

いつもと違うパートナーとの単なる新奇性を超えて、非モノガミーは、一人のパートナーでは得られない多様な性的経験へのアクセスを提供する。
たとえば、相手が共有しないフェティシズムを探求したり、パートナーとは異なる性別の者とのセックスを体験したりできる。
また、非モノガミーは、まわりの人々にとって魅力的で関心をもたれる状態を保つ理由や、パートナーの幸福にさらに投資しようとする理由を与えてくれることで、現在の関係をさらに向上させる可能性がある。
ジェフリー・ミラー\ig{Geoffrey Miller}は次のように言う。

\begin{quote}
\ruby{配偶者確保}{メイティング}の努力が子育ての努力に取って代わると、伝統的な既婚カップルは知的、社会的、政治的な生活において怠惰になりがちだ。
対照的に、オープンな関係は、常に自分たちを配偶市場に置こうとする、健康でフィットし、創造的でユーモラスであり続けるためのインセンティブを人々に与えてくれる。
\citep{miller19:_polyam_is_growin}

\end{quote}

非モノガミーの実践者たちは、複数の関係を管理する上での実際的な問題を認める。
前述した二人の実践者も、これらの課題が\ruby{現実的}{リアル}だと認めている。
しかし、チャルマーズは、たとえ時間管理に関してどのような問題が生じたとしても、それが非モノガミーに反対する論拠にはならないと論じる。

\begin{quote}
同時に無限のパートナーに恋愛的な注意を向けることが不可能だという事実だけで、一人に限定すべきだとする理由にはならない。
結局、我々は同時に無限の友人を持つことも不可能だが、けっして「追加の友人を認めない」という制限を正当化するものではないのだ。
\citep[p.232]{chalmers19:_is_monog_moral_permis}

\end{quote}

彼ははさらに、パートナー以外に、時間とエネルギーを要求するあらゆる活動についても、同様の不満を述べることができると指摘する。
私たちは、人々が恋愛関係に専念するために、人生の他のすべてを放棄することを期待しているわけではない(ibid.)。

非モノガミーの関係にある人々は、嫉妬が問題であることを否定しない。
一部の者は、嫉妬が我々の生得的な性質の一部だという考えに異議を唱える。
彼らは、嫉妬の問題、あるいはそれを悪化させる原因は、まさに我々の社会がモノガミーに固執していることにあると論じる。
しかし、いずれにせよ、非モノガミー実践者は、嫉妬が普遍的だという事実自体が、嫉妬を防ぐためにモノガミーであるべきだという論拠を弱体化させると主張する。
すなわち、嫉妬はすべての関係において問題であり、非モノガミーの関係はそれに対処する上でむしろ優位性を持つ。
チャルマーズは次のように述べる。

\begin{quote}
モノガミーは嫉妬への解決策ではない。
実際には、モノガミーこそが嫉妬をしつこくて手に負えないものにしている主な原因だ。
嫉妬が最も煮詰まりやすい状況は、共有を拒む状況、すなわち何かをめぐる競争の状況であり、これはまさにモノガミーそのものが生み出している状況である。
したがって、モノガミーを放棄することにより、嫉妬の生命線の大部分を断ち切ることができる。
これが真の解決策の始まりとなるのだ。
\citep[p.237]{chalmers19:_is_monog_moral_permis}
\end{quote}

さらに、非モノガミーは、実践者に対して、自分自身の嫉妬の感情について率直に議論し、何が許され何が許されないかをあらかじめ交渉しておき、一般にはこそこそ隠れて行動するのではなく、他の人々と現在どう関わっているかを明らかにすることを要求する。
その上、多くの非モノガミーの実践者たちは、嫉妬とは逆の感情、すなわち「コンパーション」を体験できると主張する。
コンパーションとは、パートナーが幸福の要因を体験していると知ることに伴う喜びの感情だ。
彼らは、これが単にパートナーを所有するのではなく、真にパートナーの幸福に関心を寄せていることを示していると論じる。
中には、コンパーションを生々しい興奮として体験する者もいる\citep{sousa18:_love_jealous_comper}。

最終的には、非モノガミーの関係がモノガミーの関係よりも安定しているか否かは、経験的な問題だ。
非モノガミーの関係の中には、種類によっては他のものよりも安定しているものがある可能性が十分にある。
たとえば、ポリアモリーの関係は、複数のパートナーとの長期的な親密さを維持する必要があるため、維持が困難となる可能性がある。
一方、関与する人々の関係が固定され安定している場合、オープンな関係よりも長続きする可能性もある。
しかし、現状では我々は推測するしかない。
非モノガミーの関係の健全性や耐久性に関する研究は非常に乏しく、それをモノガミーの関係と比較するための十分なデータは得られていない。
しかし、非モノガミーの擁護者たちは、モノガミーの関係の安定性に関するデータは豊富にあり、その数字は惨憺たるものだと指摘する。
すなわち、すべての結婚の半数以上が離婚に終わるだけでなく、ほとんどの人々は結婚前にすでに排他的ではあるが長続きしなかった関係をいくつも経験している。
したがって、モノガミーの擁護者が、モノガミーには非常に印象的な安定性の実績があると主張するのは困難だ。
非モノガミーの支持者たちは、自分たちがそれぞれ望んでいる形の関係が、この点でモノガミーよりもさほど悪い結果を招くことはないのではないかと考えても許されるかもしれない。

また、非モノガミーの実践者は、モノガミー関係が子供にとってベターだという主張も否定する。
特にオープン関係やスウィンギングといった一部の非モノガミー関係においては、子供たちは必ずしも親の関係の状況を認識しているわけでもなく、関係外の人々と会うことになるとも限らない。
ポリアモリー家族においては、複数のパートナーがしばしば共同で子供の養育に関与する。
こうした取り決めの影響については広く研究されているわけではないが、すでに子供たちは混合家族を含む多種多様な家族状況の中で育てられていることに留意すべきだ。
これらの取り決めは確かに課題を呈することがあるだろうが、支援的な大人の多様な存在に子供たちが触れるという利点もある。
ゆえに、ポリアモリー家族について、他の家族形態と同様に先入観を持って判断すべきではない\citep[p.191]{sheff15:_polyam_next_door}。

\subsection{非モノガミーと社会}

モノガミーの擁護者は、モノガミーが社会全体にとってより良いと主張しているが、これは支持するにも否定するにも困難な仮説だ。
異なる関係の様式が同様の状況にある社会の発展にどう影響するかを検証するための統制実験はおこうことはできない。
北中国の非モノガミーのナ族と数年間共に生活した人類学者の蔡華(Cai Hua)は、婚姻やモノガミーがなくとも「社会は完全に自立し、他の社会と同様に機能する」ことができると、その研究に基づいて結論している\citep{hua03:_societ_without_father_husban}。
伝統的なモノガミーの擁護者であるジョン・ウィッテは、社会科学のデータにおいて「婚姻がカップルやその子供だけでなく、彼らが所属する広範な市民共同体にとっても良いものだ」という、西洋伝統の二つ目の核心的洞察については、注意深い実証と記録化がいまだ欠けていることを認めている\citep[p.1070]{witte01:_goods_goals_marriag}。

しかし、多くの急進派は、現在の社会秩序そのものを拒絶するがゆえにモノガミーを否定していることに触れておくべきだろう。
ローラ・キプニスは、モノガミーを「服従を最大化し自由を最小化する社会制度であり、愛と同伴の見返りに、国民を無限に多くの些細な規則や禁止事項への従順に慣らす制度である」としている\citep{kipnis03:_troub_marriag}。
また、一部のフェミニストは、非モノガミーが家父長制的規範に対する挑戦の手段となりえると主張する。
アンジー・ベッカー・スティーブンズは、「フェミニストとして、私は、男性による女性の所有を否定し、伝統的な性別に基いたモノガミーの当然視に挑戦する、平等なポリアモリー関係を支持すべきだと考える」と述べている\citep{stevens13:_shoul_femin_be_critic_compul_monog}。
モノガミーに対するフェミニストの批判には長い歴史がある。
無政府主義者エマ・ゴールドマンは1914年のエッセイ「結婚と愛」( \emph{Marriage and Love})においてモノガミーに反対して次のように述べている。

\begin{quote}
結婚制度は女性を\ruby{寄生者}{パラサイト}化し絶対的な依存者にしてしまう。
結婚は女性を人生の闘争において無力にし、社会意識を消滅させ、想像力を麻痺させた上で、女性に慈悲深い保護を与えると言うが、実際にはそれは罠であり人間性に対する茶番でしかない。
\citep{goldman14:_marriag_love}\ig{Emma Goldman}
\end{quote}

また、非モノガミー関係は、女性により多くの力を与えうると論じられている。
エリザベス・シェフのポリアモリー関係に関する研究では、これらの関係にある女性は、当人が過去にモノガミー関係にあったときよりも平等かつ力を与えられたと自己評価している。
シェフは「もしオードリー・ロードが示唆するように、従来の性的取り決めが女性の本来の精神を沈黙させるために設計されているのならば、代替となるエロティックなシステムは、女性に性的主体性を本当に表現させ、「\ruby{力を与える}{エンパワー}」手段となりえる」と言う\citep[p.6]{sheff05:_polyam_women_sexual_subjec_power}。
さらに、フェミニストたちは、女性が男性とのモノガミー関係を追求し維持する圧力が、女性同士あるいはもっと広範な社会から女性を分断してしまう原因となっていると指摘している\citep{rosa94:_anti}。
ツオリスは「我々の社会で讃えられるモノガミー的な恋愛は、女性を支配する道具である。
……それは男性を世界の中心に据え、我々のエネルギーをすべて使い果たして、他者とのつながりから引き離してしまう」と述べる\citep[p.25]{tsoulis87:_heter}。

クィア理論家もまた、非モノガミーを社会規範に挑戦する手段とみなす。
クィア理論の視点から執筆するエレノア・ウィルキンソンは、次のように言う。

\begin{quote}

  \ruby{モノガミー規範}{モノノーマティヴィティ}の拒否は、我々に新たな生き方や愛し方、共同体や社会、そして「善」を想像する自由を与えるものであり、モノガミー規範に対抗することは単なる個人的選択ではなく、我々の生活のあらゆる側面に根本的な変革を望む政治的立場だ。
\citep[p.344]{wilkinson10:_whats_queer_non_monog_now}

\end{quote}

シェフによるポリアモリー女性の質的研究は、非モノガミーの実践が、より広い範囲での社会規範に問いを突きつける意欲を喚起する可能性があることを示唆している。
シェフは「ポリアモリーへの参加が、社会的慣習やそれに伴う役割に対して挑戦する意欲を促す。
……モノガミーに対する基本的な社会的信仰を拒否することによって、ポリアモリーの女性たちは、自分たちは他の規範に対しても疑問を呈することも許されるのだと考えるようになる」と報告している。
彼女のインタビュイーの一人は次のように述べている。

\begin{quote}
  私がポリアモリーを選んだのは、自分が本当にどのように生きたいのかを考える意欲に大いに関係しています。
社会が定めた台本にただ従うのではなく、他の分野においても自分らしい選択をすることに、ずっと自由を感じるようになりました。
\citep[pp.9--10]{sheff05:_polyam_women_sexual_subjec_power}

\end{quote}

以上の議論は、非モノガミーを現状を維持する上での脅威とみなす保守派にとっては安心材料とはならず、むしろこれは自分たちの主張の支持材料となると捉えるであろう。
しかし、ある事柄が人々をもっと社会改革に対して\ruby{開かれた}{オープン}態度にするだろうという主張と、それが社会の全面的な崩壊を招くだろうという主張とは、本質的に異なるものだ。

\subsection{非モノガミーと性格}

モノガミーがある種の賞賛に値する性格の美徳を育むと主張する人々に対し、非モノガミーの擁護者たちは、自分たちの生活スタイルもまた、賞賛すべき美徳を育むものとみなすことができると反論する。
彼らは、自分たちの関係も信頼と誠実を重んじているだけでなく、実際にはほとんどのモノガミーのカップルよりもさらにそれを重んじているのだ、と言う。
非モノガミーをうまくおこなうためには通常、率直なコミュニケーションが不可欠であり、その結果、個々人は自分の欲望について正直に語ることが可能となる。
さらに、オープンな関係は常にコミュニケーションと同意を要求するため、その価値が強調され、これが関係の他の側面や、より広く人生全体にも応用される。

モノガミーが無私性や自己犠牲を促すという主張に対して、非モノガミー実践者は、正反対に、モノガミーはむしろ望ましい性格にとって破壊的な「所有欲」にこだわる発想を促進し、さらには理想化さえしていると反論する。
モノガミーは他者を自分の所有物とみなし、同時に自分自身も相手の所有物とみなすよう教え、パートナーを、(ローラ・キプニスの印象的な比喩言えば)「家庭内\ruby{労働強制収容所}{グラーグ}」に閉じ込めるものだ。
非モノガミー実践者たちは、モノガミーとは、自己の身体や感情生活に対する権利を他者に譲り渡すことだとする。
これに対して、非モノガミーは、他者の自律を尊重することを基盤としていると主張される。
また、非モノガミーの擁護者は、それが真の無私の精神を示す手段であると強調する。
なぜなら、パートナーに対し、その本来の欲望を制約なく追求する自由を与えるものだからだ。

また、非モノガミーは「関係的自己」を損なうどころか、むしろモノガミーのカップル以上にそれを育むと考える者もいる。
多くのポリアモリストは、パートナーたちとの間に真のコミュニティを形成していると主張する。
そのコミュニティはしばしば、同じ価値観を持つ他者へも広がっていく。
これは、多くのモノガミーのカップルと対照的だ。
モノガミーのカップルは、しばしば二人の関係に閉じこもり、社会的なつながりのネットワークを十分に築かない傾向があるからだ。

\subsection{本節のまとめ}

非モノガミーの支持者は具体的な目標をもっている。
まず第一に、彼らは非モノガミーに対する\ruby{汚名}{スティグマ}の除去を望んでいる。
彼らは、非モノガミーを学校教育に取り入れ、正当な関係形態としてメディアで表現されることを望んでいる。
また、医師やセラピストは、偏見なく非モノガミーの実践者を扱い、彼らが直面する独自の課題をよりよく理解すべきだと考えている。

さらに、一部の人々は、モノガミーがより自然で道徳的に受け入れられるという広範な前提を示すために、「モノノーマティビティ」や「強制的モノガミー」といった用語を用いる。
しかしながら、我々の文化が実際にどの程度真にモノノーマティブであるか、また強制的モノガミーがどのように体系的に構築されているかについて、いまだ定量的な検証はなされていない。
これまで述べたように、ほとんどの人々は自分の\ruby{性的関係}{リレーションシップ}がモノガミーであることを期待しており、研究によって、人々が非モノガミーに対してやや否定的な態度を有していることが示されている\citep{conley13:_fewer_merrier}。
しかし、人々は同時に非モノガミーに対して好奇心も持っている。
非モノガミーはますます人々の目に触れる存在となりつつある。
上記で引用した調査では、若年層が前世代よりもはるかに非モノガミーに対してオープンになっていることが示されている。

非モノガミーが、なんらかの形で実行可能な選択肢である、あるいは賞賛に値する選択肢であると考える者にとって、次の論理的ステップは、複数婚にモノガミーと同等の法的な承認を与えるべきかどうかを問うことだ。
これについては本書4.3節で論じる。
その前にまず、同性愛者に平等な結婚の権利を与えるべきか(すでに多くの法域で認められている権利だ)、そして婚姻の性質に何か特有のものがあるために異性愛者専用に留めるべきかという論争について考察したい。

\section{同性婚}

過去20年にわたり、同性婚の支持者が成し遂げた勝利はとてつもないものだ。
2000年には、世界のどこにも同性カップルが合法的に結婚できる場所は存在しなかった。
1970年、ミネソタ州最高裁において同性婚に関する訴訟が提起されたが、却下された(\emph{Baker v. Nelson}, 291 Minn. 310, 191 N.W.2d 185 (1971))。
1989年の\emph{The New York Times}の記事によれば、この問題は1980年代後半まで「ほぼ休眠状態」であったが、「エイズ流行が相続や死亡給付に関する疑問を多くの人々の心に呼び起こしたことで、再び表面化した」\citep{gutis89:_small_steps_accep_renew_debat_gay_marriag}。
ゲイ権利活動家が結婚の平等を優先事項とし始めたにもかかわらず、彼らは強力な世論の流れに直面した。
アメリカにおいては、1996年の時点でも世論調査で約3分の2が反対し、わずか4分の1しか賛成していなかった\citep{center12:_growin_public_suppor_same_sex_marriag}。

しかし、状況は急速に変わった。
1989年にデンマークで市民パートナーシップが導入され、その後、ノルウェー(1993年)、スウェーデン(1995年)、アイスランド(1996年)、フランス(1999年)など、いくつかの国で同様の制度が採用された。
カリフォルニア州は1999年に米国内初のドメスティック・パートナーシップ法を導入し、バーモント州が2000年に続いた。
さらに、カナダのオンタリオ州とブリティッシュ・コロンビア州では2003年に裁判所の判決により同性婚が合法化され、2004年にはケベック州も最高裁の判断で認められた。
マサチューセッツ州の最高裁は2004年に同性婚を合法化し、2005年にはカナダ全土およびスペインで合法となり、世界各国がその例にならい始めた。
2012年までには、アメリカの9州とコロンビア特別区も同様の措置を講じた。

2013年に米国最高裁は、エディス・ウィンザーによる訴えを受けて、〔婚姻を異性間に限定する連邦法の〕婚姻防衛法(DOMA法)の一部を違憲と判断した。
ウィンザーは、何十年にもわたり多発性硬化症に苦しむパートナー、シア・スペイヤーの世話をしてきた才気煥発で明快なコンピュータ科学者だった。
2007年、スペイヤーは余命1年と宣告され、二人は結婚するためにトロントへ渡航した(オンタリオ州は2003年に同性婚を合法化している)。
ウィンザーは\emph{The New York Times}に対し、法的な地位以上に、結婚そのものが彼女とスペイヤーとの関係の価値を示す象徴的な意味を持つことが重要だと語った。
「結婚するということは非常に大きな意味を持ちます。
そして、もしそれを拒否されたなら、その意味はなおさら大きくなるのです」と彼女は述べた。
しかし、その拒否は単なる象徴にとどまらなかった。
2009年にスペイヤーが亡くなった後、法的な配偶者として認められていれば可能であったはずの遺産相続が、税金支払いの義務を伴ったとしてウィンザーは60万ドル以上の負担を強いられる結果となったのだ\citep{applebome12:_revel_her_suprem_court_momen}。
このウィンザー事件によって、連邦政府は、同性婚が合法である法域において結婚したカップルに対しては、婚姻上の利益を完全に提供するよう求められることとなった。
2015年、オーバーグフェル対ホッジス事件において、最高裁は、いかなる州においてもゲイおよびレズビアンの結婚の権利を否定することは違憲だと判断した(\emph{Obergefell v. Hodges}, 576 U.S. 644)。

なお、厳密に言えば「同性婚」(gay marriage)という呼称は誤解を招く。
これは、性的指向やジェンダー自認にかかわらず、誰にでも婚姻制度を開放することの略称だ。
しかしながら、そのような意味で婚姻が開かれ平等であることを示す最も一般的で扱いやすい用語となっており、本稿においても便宜上、その用語を使用する〔原文ではgay marriageだが、本翻訳では「同性婚」を用いる〕。

\subsection{結婚は基本的人権か?}

同性婚の支持者は、私たちが望む相手と結婚する自由は基本的な権利に関わる問題だと主張する。
彼らはしばしば、異人種間の結婚の権利と比較する。
人種間結婚の制限が覆された際、米国最高裁は結婚の権利を「自由な男女が秩序ある幸福追求をおこなうために必要不可欠な重要な個人的権利の一つ」だと宣言した(\emph{Loving v. Virginia}, 1967)。
最高裁は個人の自律の価値に訴えている。
極めて重要な人生の決断に関わるゆえに、結婚するか否か、また誰と結婚するかを選ぶ自由は、私たちのアイデンティティ、幸福、さらには自分が望む生活を送る能力の中核をなすと最高裁は宣言した。
この点は、最高裁のオーバーグフェル事件においても再確認され、「結婚に関する個人的選択の権利は、個人自律の概念に内在する」と述べられている(\emph{Obergefell}, p.12)。
したがって、国家はあらかじめ私たちがどの相手を選ぶかを規定すべきではない。

自由と平等の価値に加え、さらに同性婚の支持者たちは平等の価値にも訴える。
多くの人々は、性的指向は人種などと同様に、選びとられたものではなく変えることのできないものだと主張する。
この主張は、米国最高裁がオーバーグフェル事件で大きな影響力を持った。
州法により同性婚を禁止する法律を覆す際に、裁判所は「精神科医その他の専門家が、性的指向が人間のセクシュアリティの正常な表現であり、また\ruby{変更不能}{イミュータブル}だと認めてきた」と宣言した(\emph{Obergefell}, p.8)。
裁判所は、性的指向が変更不可能であることから、ゲイやレズビアンは自分たちの\ruby{性的関係}{リレーションシップ}の種類においては選択の余地がないと結論づけた。
「この人々の変更不能な本性は、同性婚がこの\ruby{根本的な}{ディープ}コミットメント〔結婚〕への唯一の現実的な道であることを命じている」と述べた(\emph{Obergefell}, p.4)。
なお、すべてのゲイやレズビアンがそのように自らの性的指向を捉えているわけではなく、また法の下で平等な配慮を与えられるためには必ずしもそう解釈する必要もない。
実際のところ、宗教信仰は変更不能な性質に属するわけではないが、宗教に基づいた差別についても数々の保護が認められている。

結婚は他の基本的人権とはいくつかの点で違っていることに注意しておくべきだろう。
個人の安全、表現の自由、自由な宗教実践の権利は、国家の強制力に対する制約だ。
哲学者たちは、ジョン・ロックに基づいて、私たちの基本的権利とは、自然状態において政府を設立する際にも譲渡していない自由であると考えることがある。
同性セックスを禁止する法律を覆した際に、各裁判所は、国家が正当な理由なく人々のプライベートな生活に干渉することを制限するプライバシー権に言及した。
しかし、結婚は単に国家が私たちを放置だけでは成立しない。
結婚はその本質において二人の個人が生涯を共にする深い絆であると同時に、複雑な社会的、歴史的、宗教的な意味を持つ制度だ。
その伝統はしばしば何世紀も遡る。
結婚は法的地位を伴うものであり、カップルに対しては国家からさまざまな便益が付与される。
既婚カップルは税制上の優遇を受け、雇用主は雇用者の配偶者に対しても福利厚生を提供し、既婚者はパートナーの母国で市民権を得る資格があり、離婚時にはその資産が公平に分割されることになる。
国家が結婚の法的地位を認めこれらの便益を付与することによって、結婚関係には特別な正当性と威厳が与えらる。また結婚に法的および経済的特権を授与するために、社会には一定のコストが課されている。

同性婚の反対者は、国家がゲイやレズビアンの結婚の権利を否定するとしても、彼らに何らかの強制力を行使しているのではないと主張する。
単に異性愛カップルに提供している特定の利益を提供しないということにすぎない、と。
しかし、同性婚の支持者は、結婚の権利を単に特定の利益を受ける権利として捉えるのではなく、ゲイやレズビアンの基本的平等の権利の一部として捉えるべきだと反論する。
彼らは、国家が特定の集団にのみ利益を与え、他には与えないということはあるべきではなく、国家はすべての市民に対して中立であるべきだと主張する。
カナダ最高裁が障害を持つ学生の教育に関する訴訟で述べたように、「国家が人々に便益を提供しようとするならば、それを非差別的な方法でおこなう義務がある。
……多くの場合に、政府は積極的措置をとることが必要になる。
たとえば、これまで排除されてきた人々に与える便益の範囲を拡大するなどの方策が必要となる」と述べた(\emph{Eldridge v. British Columbia})。

これに答えて、同性婚の反対者は、国家はすべての社会集団に対して常に完全に中立ではないと主張する。
ある反対者は、「政府は行動を促進する際には、行動を罰する場合よりも広い裁量を持つ」と指摘する\citep[p.607]{dent99:_defen_tradit_marriag}。
政府が一部の人々にのみ便益を与える事例は数多く存在する。
これは、政府がある社会的\ruby{利益}{グッド}を奨励するために、特定の集団に対して便益を与えることが合理的とされているためだ。
ほとんどの国では、たとえば子供を持つ人々、大学に通う人々、中小企業を始めようとする人々、または\ruby{辺鄙}{へんぴ}な地域で働く人々に対して税制上の優遇措置が提供されている。
米国においては、住宅所有者は住宅ローンの支払いを課税所得から控除できるが、借家人には同等の控除が認められていない。
これは、国家が住宅所有を社会的利益と見なし、その奨励を公共政策の正当な目標と捉えているからだ。
私たちがそうした国家によるインセンティブをどう考えようとも{\DDASH}経済学者の多くはそれを好ましく思っていない{\DDASH}政府は住宅所有者と借家人を異なる扱いにする合理的な根拠を提供できる。
ペンシルベニア州の元上院議員リック・サントラムは、自分の同性婚反対の立場を弁護する際、「これは公共政策上の意見の相違だ。
問題は、この政策の相違を個人的な侮辱と受けとる者がいることだ」と述べた\citep{villalva12:_gay_activ_rebuk_bully_santor}。

同性婚の反対者たちが同性婚の問題を基本的権利の問題ではなく公共政策の問題と主張するのは、ゲイやレズビアンを結婚制度から排除することに、公共の利益に関する正当な理由があると考えているからだ。
しかしながら、彼らはその理由を具体的に示さなければならない。
そこで彼らは議論をいくつか提示している。

\subsection{ゲイやレズビアンたちが果たしえない結婚の目的は存在するか?}

差別が問題になるのは、当の事案に無関係な理由で人々が異なる扱いを受けるときだ。
たとえば、あなたがコーチをしているバスケットボールのチームから背の低い選手を外すことは偏見とはいえない。
しかし、女性や人種的少数派を大学進学から排除することは、すべての性別や人種が高等教育に必要な知的能力を平等にもっているという事実に反するため不正であるとされる。
同性婚の反対者は、ゲイやレズビアンが結婚の本質的な機能の少なくとも一部を果たすことができないという理由から、性的指向は国家が認める結婚からの排除の正当な根拠であると主張する。
これは、そもそも結婚の機能と目的とは何かを検討する必要性を私たちに突きつける主張だ。

同性婚の反対者は、まず第一に、結婚制度は一人の男性と一人の女性という自然な組み合わせを反映するように発展してきたと主張する。
ある保守派の作家は、「人類の生物学的二元性を体現する二人の者の生涯にわたる結合」と表現している\citep{jensen15:_i_oppos_same_sex_marriag}。
彼らは、私たちの身体は異性愛の\ruby{性交}{インタコース}に適した形で設計されており、同性愛性交には適していないと主張するだけでなく、男性と女性には自然に補完的なジェンダー役割があるとも論じる。
すなわち、女性は\ruby{本来的}{ナチュラル}にケア提供者であり、男性は本来的に稼ぎ手であるという考えだ。
結婚は、この自然な労働分担に安定をもたらし、女性が家庭に留まり子供を育てられるよう、経済的な保障を提供するものだと彼らは主張する。

第二に、同性婚の反対者は、異性婚が子供の養育において特別な役割を担うと主張する。
結婚を排他的で異性愛的な制度と理解する根拠として、ジョン・ロバーツ\ig{John Roberts}最高裁長官はオーバーグフェル事件において、次のように述べている。

\begin{quote}
男性と女性の結合としての結婚の普遍的な定義は歴史的偶然などではない。
結婚は政治運動や各種の発見、疾病、戦争、宗教の教義、またはその他の歴史的推進力の結果として生じたものではない{\DDASH}ましてやゲイやレズビアンを排除するという先史時代の決定の結果として生じたものでもない。
結婚は、生存にかかわる必要性を満たすための物事の本性の中で生じた。
すなわち、安定した生涯にわたる関係の中で子供を育てることを\ruby{決断}{コミット}した母と父によって、子供たちが受胎される、という必要性だ(\emph{Obergefell}, ロバーツ裁判官の反対意見)\ig{John Roberts}。

\end{quote}

保守派は、ゲイカップルは子供をつくることができないので、結婚の本質的な目的を果たすことができないと主張している。
確かに、ゲイやレズビアンは、両者が生物学的な親である子供を持つことはできない。
しかし、多くの異性愛カップルは、一方が生物学的に関わっている子供を育てているし、または養子縁組によっていずれの親とも生物学的な関連がない子供を育ててもいる。
こうした結婚も法的にはやはり有効だ。
さらに、同性婚の反対者は、たとえゲイやレズビアンが代理出産や養子縁組などの手段によって子供をもったとしても、彼らは親として適任でないか、あるいは同性の両親によって子供が何らかの害を\ruby{被}{こうむ}ると主張しようとする。
チャールズ・マレーは、葛藤の少ない家庭において生物学的な母親と父親のもとで育つ子供が、さまざまな成果において最良だと主張し、次のように言う。

\begin{quote}
私が知る限り、これほど広く受け入れられながらも、主要なニュース番組、全国紙の論説委員、そして二大政党〔米共和党と民主党〕の政治家によってこれほど徹底的に無視されている重要な研究成果は他にない。
\citep[p.162]{murray12:_comin_apart}
\end{quote}

この理由を説明するため、同性婚の反対者は、親子間の生物学的な絆や、またしても男性と女性の自然な補完的な相違性に訴える。
ティモシー・ビブラーツとジュディス・ステイシーは、この議論を次のように要約している(ただし彼ら自身はこれを否定する)。

\begin{quote}
子供が母親と父親の両方を必要とするという議論は、母性と父性がジェンダー固有の能力に依存していることを前提としている。
「本質的な父親」とは、規律を与え、問題解決にあたり、遊び相手となる存在であり、男性的な育児を担う。
男児が適切な男性的なアイデンティティを発達させ、暴力、犯罪、薬物乱用などの反社会的行動を抑制するためには、父親が必要だ……と支持者は主張する。
対照的に、父親は娘たちに\ruby{異性愛的}{ヘテロセクシュアル}な女性性を育みつつ、軽薄さ、思春期での妊娠、福祉政策への依存を抑制する役割を果たしている。
父親は「息子が身につけつつある男性性を研ぎ澄ますための砥石となり、娘が女性性を表現しようとする際の賞賛すべき聴衆となる」(Pruett, 2000, p.87)。
一方、母親は養育と安心とケアを提供する存在だ。
(Biblarz and Stacey, 2010, p.4; cf. Popeno, 1996)
\nocite{biblarz10:_how_does_gender_paren_matter}\nocite{pruett00:_father}\nocite{popenoe96:_life_father}

\end{quote}

さらに、同性婚の反対者は、ゲイ両親の子供は将来的に自らもゲイになる可能性が高いとも示唆する。

もちろん、異性愛カップルの多くも、選択的に、あるいは一方または両方の不妊により、子供を持たない。
しかし、保守派は、子供をもたない異性愛婚の正当性に異議を唱えることはない。
なぜなら、彼らは結婚には「生涯の伴侶」というさらに別の目的があると考えているからだ。
キリスト教哲学者たちは、この「結婚の善」を強調し、その起源を「創世記」にまで遡る。
神がアダムの伴侶としてイブを創造したことがその根拠とされる。
聖アウグスティヌスは、結婚について次のように述べている。
「私には、結婚が善であるのは、単に子供を生むためだけではなく、男女の自然な交わりがあるからだと思われる\citep[3.3]{augustine98:_excel_marriag}。
また、トマス・アクィナスはこれを「ソキエタス」(\emph{societas})と呼び、次のように述べている。
「結婚は最も偉大な友情である。
なぜなら、男女は単に身体的な結合においてのみならず……互いに助け合い、一つの家族としての生活を共にすることで結びついているからである\citep[III c. 123 n. 6 (2964)]{aquinas55:_summa_gentil}。

同性婚の反対派は、男性と女性がこの種の友情を築くのに特別に適していると主張する。
ジョン・フィニスはこれを次のように表現している。
「それは、男性と女性による独自の友情であり、人間の繁栄と本性の理想形において、各性別が持たない要素を補い合う補完関係にある」\citep[p.398]{finnis08:_marriag}。
さらに一歩進めて、同性愛者は結婚に適していないと主張する人々もいる。
彼らは、同性愛者は安定した長期的な関係を築くことに興味をもっておらず、あるいはそもそもその能力がないと考えている。
ある論者は、次のように述べている。
「同性のパートナーシップは、表面的には結婚の最初の利点を提供するように見えるが、実際には極めて不安定で、短命であり、リスクが高い\citep{fischer13:_purpos_marriag}。

\subsection{同性婚は結婚の伝統を破壊するか?}

結婚は長い歴史を有する制度だ。
メソポタミアでは紀元前2350年頃、一人の女性と一人の男性を結んだ婚礼の証拠が存在している。
世界中の文化は何らかの形で結婚を認めており、多くの文化が一夫多妻制(または、稀に多夫制)を容認する一方、同性間の結婚を認めた例は極めて少ない。
研究者たちは歴史上に同性婚の例があったかどうかを検証しようと試みたが、たとえあったとしてもそれはけっして一般的ではなく、その存在自体が解釈上の論争の対象となり続けている\citep{boswell94:_same_sex_union_pre_moder_europ,shaw94:_review_boswel}。
保守派とは、定義によって、伝統を重んじる人々だ。
同性婚に反対する保守的立場の人々は、数千年にわたる伝統を数年のうちに覆すべきではないと主張する。
すなわち、ゲイやレズビアンに結婚の権利を認めることは、結婚が何であるかを規定している長く異文化にまたがる歴史を損なう行為であると論じる。
ロナルド・ドウォーキン\ig{Ronald Dworkin}は、この議論を簡潔に提示している(ただし彼自身はこれに同意していない)。

\begin{quote}

結婚という制度は……\ruby{独得}{ユニーク}の極めて価値の高い文化的資源だ。
その意味と価値は幾世紀にもわたって発展してきた。
そして、「結婚とは男性と女性の結びつきである」という前提は、私たちの共通理解に深く根づいており、この前提が覆され失われるならば、結婚という制度自体が現在とは異なるものへと変質してしまうだろう。
私たちが他の偉大な自然資源や芸術資源の意味と価値を維持しようとするのと同様に、この独自の価値を持つ文化的資源を守るために努めるべきだ。
\citep[pp.87--88]{dworkin06:_is_democ_possib_here}

\end{quote}

\subsection{結婚と社会秩序}

保守派は、結婚が社会の基盤を支える上で極めて重要な役割を果たしていると主張する。
チャールズ・マレーは次のように述べている。
「子供をもつ家族は、アメリカのコミュニティを組織するための核となる存在だ。
いや、核とならねばならない。
なぜなら、子供をもつ家族こそが、これまでも、そして今でも、アメリカというコミュニティを機能させる原動力だからだ」\citep[p.169]{murray12:_comin_apart}。
ピーター・ジョン・ミッチェル\ig{Peter Jon Mitchell}は、マレーの著作のレビューに、簡潔に「家族と結婚が国を支える」と題している\citep{mitchell12:_famil_marriag_hold_count_toget}。
ウィスコンシン州の控訴裁判所は、「結婚は「家族および社会の基盤」であり、結婚の安定は「道徳と文明の基本であり、社会および国家にとって極めて重要な関心事」である」と述べている(\emph{Xiong Edmondson v. Xiong}, 648 N.W.2d 900, 906 (Wis. Ct. App. 2002。
Wisc. Stat.  § 765.01(2) (2001)を引用している)。

本書5.1.1節では、パトリック・デヴリンの社会解体論と、社会の道徳に対するいかなる変化もその社会を破壊する恐れがあるという保守派の命題を検証することになる。
ナオミ・カーンとジューン・カルボーネも同様に、伝統的価値群は一つのパッケージとして存在しており、「このモデルの支持者によって、同性婚はこれらの価値群全体を軽視するものと見なされている」と論じる\citep[pp.1--2]{cahn10:_red_famil}。
メアリー・キャサリン・ギーチはまた、デヴリン\ig{Patrick Devlin}も確実に認めるであろう観点から、同性婚に反対する立場を次のように述べる。

\begin{quote}
社会秩序に対する急進的な修繕作業は極めて危険なものだ。
必ずしも社会を進歩させる必要はない。
それはむしろ崩壊してしまうかもしれない。
現在、人々は、同性間の「結婚」を認めたり、あるいは結婚そのものを廃止しようとしたりすることにより、私たちが現在知っている形の結婚を破壊しようとしている。
これらの試みは、私的財産の集団所有化以上に過激な変革をもたらす可能性があり、その結果はまったく予見できない。
社会という織物に対する急進的な実験の一番の問題は、それがもたらす害悪は、それが誤りであったと認識されてももはや元に戻すことができなくなっている点にある。
\citep[p.532]{geach08:_lying_body}

\end{quote}

この種の保守派の議論は、18世紀にエドマンド・バークが示した見解と共鳴する。
バークは「統治の科学は経験を必要とし、しかも一人の人間が生涯で得ることのできる経験以上のものを要求する。
……幾世代にもわたって社会の共通の目的に対して我慢できる程度には応えてきた建造物を引き倒そうとするならば、いかなる人間であっても無限の慎重さをもって着手すべきだ」と述べた\citep[p.53]{burke87:_reflec_revol_franc}。
バークの世界観に共感する者にとっては、記録された歴史を超えて存在する制度を、ほぼ一夜にして変革してしまえるという考えは忌避すべきものだ。

\subsection{結婚の目的と伝統:応答}

男性と女性には自然な補完的二元性が存在するという保守派の主張は、疑問視されるべきであることは明らかだ。
確かに、多くの社会は、男性と女性に特定の役割を果たすことを期待してきたし、現在も期待し続けている。
たとえば多くの社会で女性は家庭内の役割に閉じこめられてきた。
しかしながら、このような役割分担はけっして普遍的なものではなく、歴史を通じて、男性と女性が果たす役割に関してまったく異なる期待を持つ社会の例は数多く存在する。
今日では、多くの女性が伝統的に男性の役割とされていた職務に就いて働いており、かなりの(ただし比較的少数ながら)男性が子育てや伝統的に女性の役割とされる職務に従事している。
特定の社会的役割が、男性あるいは女性にとってより「自然」であることをどのように証明できるのかは不明だ。
仮にそれが証明できたとしても、なぜ私たちはその「自然な役割」を克服できないのかも明確ではない。
たとえば、たとえ人類が何世紀にもわたって肉を食べてきたとしても、菜食主義者が肉を食べるのをやめることを選択できるのと同じではないだろうか。

また、同性婚の擁護者は、ゲイやレズビアンは親となるには適していないという主張を、説得力のある\ruby{証拠}{エビデンス}が欠如したものとして否定している。
以前には、ゲイやレズビアンの親のもとで育つ子供たちは発達が不調であるという広く信じられていた見解を反駁することは、十分なデータが存在しなかったために困難だった。
しかし事情は変わった。
現在では同性の両親に関する優れた研究が多数存在し、その結果は明白だ。
ゲイやレズビアンの親に育てられた子供は、あらゆる尺度においてストレートの家庭の子供と同等の成果を上げる。
さらに、同性親であることに起因するスティグマによる潜在的な害でさえ、測定可能な差異を生み出さず、そのスティグマ自体は急速に消失している\citep{calzo17:_paren_sexual_orien_child_psych_well_being,biblarz10:_how_does_gender_paren_matter}。
また、ゲイ親に育てられた子供はゲイになる可能性が高くなるという主張も、データによって否定されている{\DDASH}さらにそもそもそのこと自体が問題かどうかは疑問だ。

また、親が結婚していると子供がより良い結果を得るという子供の福祉に関する議論は、同性婚を認める根拠としても有力だ。
もし安定したパートナーシップにある親が子供に利益をもたらすのであれば、同性親にも結婚を奨励すべきであり、さらに、親の性別にかかわらず、既婚の親に育てられる子供の方が良好な成果を上げると信じる保守派も存在する。
デイル・カーペンターは次のように主張する。

\begin{quote}
既婚カップルに育てられた子供は、独身または未婚カップルに育てられた子供よりも学業成績が良く、犯罪を犯す傾向が低く、薬物の使用や乱用の可能性も低い。
現状、同性親に育てられた子供はこれらの利点にアクセスできていない。
同性婚支持の根拠の一端は、こうした子供たちの家族を結婚によって保護し、その結果として子供自身に利益をもたらす点にある。
\citep{volokh05:_volok_consp}

\end{quote}

保守派によるゲイ育児に関する主張と同様に、ゲイやレズビアンが長期的なパートナーシップを形成できないという主張も、\ruby{証拠}{エビデンス}の前では崩壊する。
調査では常にゲイやレズビアンは安定し、コミットメント関係を形成する能力をもっているという結論になる。
同性婚の長期的破綻率に関する統計が得られるのを待たなければならないが、異性愛カップルと同程度の成功率と言うためには、同性婚が約半数の確率で関係継続していれば十分だ。
現存のデータは、結婚したゲイカップルの全体的な離別率が異性愛カップルと差がないことを示唆している(ただし、レズビアンの離別率は他の関係形態よりやや高い)\citep{ketcham19:_compar_coupl_stabil,kolk20:_two_decad_same_sex_marriag_sweden}。

また、同性親育児に関する主張と同様に、関係の安定性に関する保守派の議論も、自分たちの前提において失敗している。
もし、保守派が主張するように安定したモノガミーのパートナーシップを重視し、かつそのような関係が社会の安定にとって重要だと考えるのであれば、ゲイやレズビアンに結婚の権利を認めることは、彼らがそのような関係を形成するのを助け、結果として彼ら自身および社会全体に利益をもたらすであろう。
マーサ・ヌスバウムは次のように言う。

\begin{quote}
同性愛の人々が法的に結婚できない場合、安定した献身的な関係を築こうとする努力は阻害され、しっかりした根をもたない、あるいは奔放で非献身的な生活がそれに応じて助長されることになる。
したがって、あるステレオタイプに根ざした差別の形態が、そのステレオタイプをある程度現実のものとしてしまう可能性がある。
当然この状況は不合理だ。
社会には、異性愛カップルと同性カップルの双方に対し、安定した家庭単位の形成を促進すべき強い理由がある。
\citep[pp.202--203]{nussbaum99:_sex_and_social_justic}
\end{quote}

さらに、このような理由に基づいて同性婚を支持する保守派も存在する。
オーストラリアにおける同性婚論争は最終的に国民投票で賛成とされるに至ったが、その際、保守的首相マルコム・ターンブルは「今日の結婚に対する脅威がコミットメントの欠如であるならば、他のカップルがコミットしそれを維持することは悪い例ではなく、むしろ良い例となるに違いない」と述べた\citep{gartrell17:_malcol_turnb_makes_conser_case}。
同様に、元米共和党全国委員会議長ケン・メールマンは\emph{The Wall Street Journal}において、結婚の平等はカップルの忠実さとコミットメントを奨励し、家族の価値を育むと論じた\citep{mehlman12:_makin_same_sex_case}。

同性婚の支持者たちは、結婚は長い歴史を持つ制度だという主張に対して、結婚の歴史は\ruby{静的}{スタティック}なものではなかったと指摘する。
オーバーグフェル判決において、ケネディ判事は多数意見を代表して次のように述べている。

\begin{quote}
結婚の歴史は、継続と変化の両方から成り立っている。
お見合い結婚(arranged marriage、親族主導の結婚)の衰退や、夫婦共有財産制の廃止といった変化は、結婚の構造に深い変革をもたらし、かつて本質的だと思われていた側面にも影響を与えた。
しかし、こうした新たな洞察は、結婚という制度を弱めるのではなく、むしろ強化してきたのだ。
(\emph{Obergenfell}, p.2)
\end{quote}

ポリガミーは多くの文化において長く結婚制度の一部であり、聖書やコーランにも見られる。
かつては、他宗教の信仰を持つ人々との結婚が禁じられていた時代もあり、またアメリカでは比較的最近まで異人種間の結婚が禁止されていた。
また、結婚可能な年齢に関する規則や慣習は、時代や地域によって大きく異なっていた。
さらに、離婚は長らく禁止されるか、極めて困難にされていたが、現在ではほぼすべての地域で簡単にできるようになっている。
同性婚の支持者は、このように「結婚とは何か」という伝統的な定義が一つに定まっているわけではないと主張する。
したがって、結婚という制度において「結婚する人の性別」だけを本質的な要素と見なし、それ以外の多様な実践を無視する理由はないとしている。

上の議論は、そのまま結婚と社会秩序の関係に関する懸念に対する反論ともなる。
もし結婚が本当に\ruby{動的}{ダイナミック}で進化し続ける制度であるならば、たった一点の変更が社会全体に広範な影響を与えると考える理由はない。
そして実際に、同性婚を認めた社会の経験がすでにこれを裏付けている。
同性婚を認めた国々において、社会の結束が弱まったという証拠は一切存在しない。

\subsection{完全に平等な結婚の実際的および象徴的意義}

多くの同性婚支持者にとって、既婚者に与えられる具体的な利益は重要だ。
特に最も注目すべき点は、ゲイやレズビアンの社会的地位に対して結婚が象徴的に及ぼす影響だとされる。
彼らは、結婚はその古代の起源および現在までの社会における中心的役割のため、ある社会の世界観を示す極めて重要な指標だという点で保守派と一致する。
ゆえに、同性婚支持者たちは、かつては多くの者に合理的な妥協案と思われていた「\ruby{市民的}{シビル}ユニオン」制度を拒否した。
シビルユニオンは、カリフォルニア州やバーモント州などの州、またデンマークなどの国で、結婚の定義を変えることなくゲイやレズビアンに平等な法的利益を与える手段として試みられた。
シビルユニオン制度は、たしかに既婚者に与えられるのと同一の便益を提供するが、その地位を「結婚」とはしない。
この妥協案は一見のところ合理的に思われたが、実際には、結婚の差別的性質をより一層明らかにする結果となった。
アンドリュー・マーチは次のようにが述べている。

\begin{quote}
国家がゲイやレズビアンに対し、客観的な利益のパッケージが同一であるにもかかわらず異なる公的地位を与えるすれば、それは彼ら彼女らを完全に平等なものとして扱うことを怠っている。
実際、異性愛者に結婚の権利が認められている状況下で「シビルユニオン」という結婚に劣る地位を与えることは、同性愛者を異性愛者と平等と認める意思の欠如を公に示すものであり、すなわち\ruby{烙印}{スティグマ}だ。
したがって、唯一公正な政策は、完全な結婚権をすべての面で認めることだ。\citep[p.254]{march11:_is_there_right_polyg}
\end{quote}

この点は、ラムダ法律擁護教育基金(Lambda Legal Defense and Education Fund)が米国最高裁に提出した\ruby{第三者法定助言}{アミカス・ブリーフ}においても強調されている。

\begin{quote}
いかなる既婚カップルも、もし政府がカップルから「結婚している」という状態を奪い、「シビルユニオン」に変更に変更したならば、何か貴重で代えがたいものを失ったと感じるであろう。
すなわち、「結婚している」という感覚{\DDASH}それがカップルやそのコミュニティに伝える意味、そして他者がその結婚の事実とその全ての意義を明確に認識するという安心感{\DDASH}は奪われることになる。
これらの損失は、政府が同性カップルに「シビルユニオン」という地位を与えることによって、彼らが否定されることになったものの一部である。
立場が逆転して、非常に少数の異性愛者たちが「シビルユニオン」に限定され、「結婚」はレズビアンやゲイカップル専用なのだと告げられることを容認するであろうとは考えられない。
(コネチカット最高裁 \emph{Kerrigan v. Commissioner of Public Health}で引用されている)。
\end{quote}

こうしたことは、同性婚論争の真の賭け金がいったい何であるのかをはっきりさせてくれる。
核心部分において、それは平等な尊厳の要求に他ならない。
デヴィッド・チェンバーズは、法的な承認が「その他のいかなる差別禁止法よりも、レズビアンやゲイを対等な市民として受け入れることをより深い意味で明らかにする」と述べ、ほとんどの同性婚支持者が何よりもまずこの承認を求めていると主張する\citep[p.450]{chambers96:_what_if}。
\ig{David L. Chambers}
現代の西洋社会においては、性的指向にかかわらずすべての人々が平等な市民権を有するという信念が広く共有されており、結婚がその市民権の不可欠な一部だとすれば、誰も排除されるべきではない。

\subsection{本節のまとめ}

LGBTQ+コミュニティ内には、同性婚の承認がクィアの人々を主流の異性愛社会に同化させる危険性があると懸念する者もいる。
結婚の平等運動について尋ねられた映画制作者ジョン・ウォーターズは、「ゲイであるということの本質は、結婚したり子供を持ったり軍隊に入ったりする必要がないことだと思っていたよ」と答えた\citep{haag13:_unqueer_world}。
キャロル・クイーンは、結婚がクィアの人々にとって問題のある理想を体現していると示唆する。
彼女はこう述べている。

\begin{quote}
独身生活を含め、多くの者が代替的な関係形態を選択しているクィアたちのコミュニティにおいて、結婚の権利を強調することは、私たちが互いに関係を築く他のすべての方法を遮断あるいは軽視する結果となる。
もし結婚を、数多くの平等な選択肢の中のほんの一つとして意識的に選ぶのでなければ、私たちは多様性を縮小することを選んだことになる。
私たちは多数の選択肢を持つべきであり、主流のクィア権利運動が結婚を強く主張するようになってしまうならば、それは結婚を重要な目標と考えない私たちすべての存在を覆い隠してしまい、私たちの違いを尊重していないことになる。
\citep[p.111]{queen04:_never_brides_never_bride}

\end{quote}

しかし、実際にゲイやレズビアンに結婚の権利を否定すべきだと主張するクィア論者はほとんどいない。
むしろ、彼らは注目すべきは、この制度が何を意味するのか、またクィアの人々がそれに参加する場合にどのような代償が伴うのかを理解することだと主張する。

私たちは、結婚を非常に異なる性質の制度へと再定義しようと試みるかもしれない。
これを実現する方法はさまざまだ。
たとえば、多重婚(plural marriage)を認めることや、人々が自身のニーズに応じてその制度をその他の方法でカスタマイズすることなどが考えられる。
ここからは、まず多重婚の事例から始め、これらの提案を検証する。

\section{多重婚}

多重婚は、アフリカや中東を中心に、世界の多くの国で合法とされている。
しかし重要なのは、多くの場合、男性が複数の妻を持つこと(ポリジニー)のみが合法であるか、あるいはこの形態のポリガミーだけが広く実践されていることだ。
アメリカやヨーロッパでは、複数の配偶者を持つ結婚は法的に認められておらず、多くの法域ではむしろこの慣行を犯罪と見なしている。
哲学者がポリガミーについて議論するとき、その擁護者は基本的にジェンダーに関して中立的な立場をとる。
本章では、男性と女性の双方が複数のパートナーを持つことを認め、また、いかなるジェンダーの組み合わせでも複数の配偶者を持つことが可能な形態の多重婚について、その是非を検討する。

もし多重婚が法的に承認されるとすれば、それは非常に大きな革新となり、その実現には世論の劇的な変化が必要となるだろう。
しかし、同性婚に関する世論は非常に短期間で大きく変化し、広く受け入れられるようになった。
これは、正義がそれを求めるとき、社会がいかに劇的に進化しうるかを示している。
実際、多重婚の支持者はしばしば同性婚の例を引き合いに出し、自らの主張と類似していると主張する。
しかし、多重婚の反対派は、両者が本質的に異なる点が多いことを指摘し、そのために多重婚を法的に承認すべきではないと主張している。

\subsection{多重婚の害悪}

多重婚の反対派は、それが理想的な世界においてどのような価値を持つにせよ、現実の一部社会で現に実行されているように、さまざまな害悪をもたらすと主張する。
まず第一に、それは女性にとって不利益だ。
前述したように、人類の歴史において一般的であり、現在も合法とされている国々のほとんどで実践され続けている多重婚の形態は、より具体的にはポリジニー、すなわち一人の夫が複数の妻を持つ結婚形態だ。
これに対して、妻が複数の夫を持つ結婚形態であるポリアンドリーは、はるかに少ない。
近年、一部の研究者は従来考えられていたよりもポリアンドリーの事例は多いと主張している\citep{starkweather12:_survey_non_class_polyan}。
しかし、全体としては依然として稀だ。

多重婚の反対派は、実際にはほとんどの多重婚がポリジニーとなり、そのような結婚は女性に害を与え、抑圧する傾向があると主張する。
こうした結婚では夫が支配的な立場に立ち、しばしば少女が若年のうちに強制的に、もしくは強制に近い形で結婚させられることがある(Strassberg 2015, p.1815; Macedo, 2015, p.170)。
\nocite{strassberg15:_scrut_polyg}\nocite{macedo15:_just_married}
反対派は、多重婚が実践されている社会が、一般に女性の平等な権利を否定し、少女の早婚を容認する傾向があることは偶然ではないと考えている。
また、西洋社会においても、多重婚が秘密裏におこなわれている場合、それは保守的な宗教コミュニティ内であることが多い。
そのような環境では虐待が表面化しにくく、女性が完全な自律を行使できないことがしばしばだ。
これが、フェミニストの多くが同性婚を支持する一方で、多重婚には強く反対する理由の一つだ。
また、多重婚が子供に与える影響についても懸念がある。
多重婚が家父長制的な支配によって特徴づけられる限り、子供たちは抑圧的な環境で育つことになる。

多重婚の反対派は、その合法化がこれらの結婚における女性の抑圧を助長するだけでなく、象徴的な意味を持ち、こうした抑圧的な関係が国家の承認に値するというメッセージを発することになると考えている。
エミリー・クルックストンは次のように述べている。

\begin{quote}
現在このような関係の中で女性を抑圧している男性たちは、多重婚の非犯罪化を、社会が彼らの女性に対する扱いを容認するという合図として受けとるだろう。
これは、女性が二級市民として扱われるべきだという誤った態度を助長し、女性全体の自尊心を損なうことになる。
市民が一般にポリガミーを女性の抑圧や従属と結びつけて認識している限り、それを合法化することは個々の女性に害を及ぼすだけでなく、私たちの社会が女性差別を終わらせることにさほど関心を持っていないというメッセージを発することにもなる。
\citep[p.272]{crookston15:_love_polyg_marriag}
\end{quote}

この意味で、反対派は、多重婚を認めることは同性婚を認めることとはまったく異なると主張する。
同性婚の合法化は、すべての人の尊厳と平等を主張するものだ。
しかし、多重婚を合法化することは、ある特定の市民層、すなわち女性について、これとは逆のメッセージを発することになるというのだ。

ポリジニーが唯一の多重婚の形態として実践される場合、それは男性にとっても不平等な結果をもたらす。
複数の妻を持つ男性は、一般に裕福で社会的地位が高い者である傾向がある。
ポリジニーは、結婚可能な女性の不足を引き起こすため、貧しい男性にとっては伴侶を得る希望を奪うことになり、不利益を与える。
そして、大勢の男性が結婚できなくなることで、社会的な対立や不安定が引き起こされる要因ともなる。
\emph{The Economist}は次のように論じている。

\begin{quote}
ポリガミー(特に複数の妻を持つポリジニー)が広く実践されている場所では、社会が不安定となっている。
これは主に、一種の不平等であり、若い男性の心に、そしてその下腹部にも切迫した苦悩を生じさせるためだ……もし最も裕福で権力のある上位10%の男性がそれぞれ4人の妻を持つとすれば、下位30%の男性は結婚できないことになる。
若い男性はこの状況を避けるべく必死の手段をとるだろう……ポリガミーが実践される社会は、他の社会よりも暴力が頻繁で、隣国への侵略の可能性が高く、崩壊のリスクも高い。
平和基金(Fund for Peace)というNGOがまとめた脆弱国家指数(Fragile States Index)においては、最も不安定な20か国すべてで多重婚が生活の一部となっている。
\citep{economist17:_link_between_polyg_war}
\end{quote}

\subsection{アイデンティティと多重婚の社会的コスト}

オーバーゲフェル判決において最高裁は、同性への性的指向がゲイやレズビアンにとって\ruby{本性}{ネイチャー}の一部であり、選択の問題ではないという主張を認めた。
さらに裁判所は、彼らの関係を法的に認めないことは、彼らのアイデンティティを否定することになると判断した。
多くのポリアモリー実践者も、自らの複数パートナーを求める欲求が同様に自分たちの\ruby{本性}{ネイチャー}であり、選択の問題ではないと主張し、そのために多重婚を認るべきだと訴えている。
しかし、ポリアモリーに関するこの主張をすべての人が受け入れているわけではない。
2012年、人気セックスコラムニストのダン・サヴェージは、「ポリアモリーは性的指向ではない。
それは自分が何者であるかではなく、自分が何をするかの問題だ」と述べた\citep{savage12:_savag_busted}。
この立場では、ポリアモリーは、ある程度普遍的な性的多様性を求める欲望の表れであるとされる。
実際、多くの人が、少なくとも自分自身は複数のパートナーを持つことを望むかもしれない。
これが、本書4.1節で述べたように、生涯にわたるモノガミーが非常に困難であり、多くの人がそれに苦闘する理由だ。
多重婚の反対派は、このような結婚を法的に認めないことが、ある特定の人々を「彼らが何者であるか」に基づいて差別することにはならないと主張する。
また、複数のパートナーと結婚する権利を認められないことが、ゲイやレズビアンのアイデンティティに対するスティグマと同程度の苦痛を引き起こすという証拠もない。
したがって、この主張によれば、ポリアモリストは社会の中で明確に区別される集団ではなく、国家が彼らの関係に結婚という法的地位を付与しないとしても、それによって何らかの不当な差別を加えていることにはならない。

反対派は、ポリガミー実践者はライフスタイルを選択しているだけでなく、そのライフスタイルを法的に認めることは政府にとって大きな負担になるとも主張する。
反対派はそれを(ロナルド・ドウォーキン\ig{Ronald Dworkin}の造語を用いて)「高くつく嗜好」と呼ばれることがある。
リベラル派も、国家が個人の自由を保護すべきであることには同意する。
だがドウォーキンは、だからといって、ある個人の選択が他者に過大な負担を課す場合には、その選択に国家が応じる必要はないと強調している\citep[p.229]{dworkin81:_what_is_equal}。
アンドリュー・マーチは、この論理を多重婚に適用し、次のように述べている。

\begin{quote}
たとえば、公的負担による教育の権利を認めることは、無制限の教育の権利を意味するものではないし、住宅補助や税額控除の権利が、複数の不動産に対して同様に適用される権利を意味するわけでもない……多重のパートナーシップを無制限に認めないことは、敬意の欠如ではない。
なぜなら、それを認めない理由は、ポリガミーの関係を築くことの価値や道徳性とはまったく関係がないからだ。
これは、高くつく嗜好に対する補助をおこなわないことが、すべての高価な嗜好を無価値または非道徳的とみなすことを意味しないのと同じだ。
\citep[p.249]{march11:_is_there_right_polyg}
\end{quote}

多重婚が合法化されれば、その婚姻関係に入ったすべての人に対して結婚の利益を拡張するための社会的コストを負担しなければならなくなる。
多重婚が合法化された場合にどれほどの人々がそれを選択するかを正確に予測することは不可能だ。
全体の人口に対しては確かに少数にとどまるだろうが、それでもけっして無視できる規模ではないかもしれない。
ブリティッシュ・コロンビア州最高裁は、ポリガミーに関する判決の中で次のように述べている。

\begin{quote}
2009年、\emph{Newsweek} はポリアモリーの実践について特集を組んだ……。
同記事によると、オンラインのポリアモリー雑誌 \emph{Loving More} の定期購読者は15,000人にのぼる。
また、一部の研究者は、アメリカでの公然としたポリアモリー家庭は、50万以上存在すると推定している。
\emph{Polyamory in the Twenty-First Century} の中で、デボラ・アナポールは \emph{Loving More} の収集したデータを引用し、そこからの推計として、アメリカの成人500人に1人がポリアモリー実践者だと述べている(Anapol, 2010, p.44)。
また、他の研究者の推測では、成人人口の約3.5%がポリアモリー関係を好むとされ、この割合を基にすると、およそ1,000万人が該当することになる\footnote{British Columbia Supreme Court, Reference re: Section 293 of the Criminal Code of Canada, 2011 BCSC 1588, at 439--440.}。
\nocite{anapol10:_polyam_centur}
\end{quote}

さらに問題を複雑にするのは、多重婚においては他の形態の婚姻よりも詐欺行為が容易になる可能性があるという点だ。
もし、ある人が無制限に配偶者を申告できるとすれば、純粋に実利的な目的で、婚姻による法的利益を得るためだけにこれをおこなうことを防ぐ手立てはあるだろうか。

また、多くの実務的な課題が生じ、それを解決するための移行コストも発生するだろう。
多重婚の擁護者は、このような関係において同意がどのように機能するのかを説明しなければならない。
たとえば、婚姻関係のすべての当事者が新たなパートナーの追加に同意する必要があるのかどうかを明確にする必要がある。
また、多重婚では、避けられない別離をどのように扱うかという問題も生じる。
二者間の婚姻よりも複雑になることは避けられない。
コナー・フリーダースドルフは次のように指摘している。
「ポリガミーの関係にある個人がモノガミーの関係にある個人と同程度の確率で離婚を求めるとすれば、三人の婚姻関係であっても失敗率は少なくとも三分の一高くなり、それ以上の人数の多重婚ではさらに高くなる」\citep{friedersdorf15:_case_encour_polyg}。
さらに、こうした別離の過程で、扶養を必要とする子供たちが適切に養育され、社会全体に負担をかけないようにするという利益関心を国家はもたざるをえない。

\subsection{多重婚の擁護論:平等と整合性}

アメリカで同性婚が合法化された際、多くの保守派の反対者は、これが滑りやすい坂道を下る第一歩であり、同性婚の平等を支持する論理がポリガミーにも適用されることになると主張した。
最高裁長官ジョン・ロバーツ\ig{John Roberts}は、反対意見の中で次のように述べている。
「多数意見の論理の多くの部分が、多重婚に対する基本的権利の主張にも同等の力をもって適用されることははっきりしている」(\emph{Obergefell}, p.20)。

ロバーツ\ig{John Roberts}の指摘は的外れではない。
実際、結婚の平等を支持する多くの擁護者は、多重婚の合法化を支持することを公然と認めており、これを結婚の再定義を目指すより広範な運動の一環とみなしている。
フレディ・デ・ボアは\emph{Politico}に寄稿し、多者婚を「社会的リベラリズムの次の地平」と呼んでいる。
彼は次のように述べている。
「すでに私たちは、愛や献身、家族というものが単にジェンダーによって決定されるものではないと規定した。
では、それを二人に限定する必要がどこにあるのだろうか。
結婚の次なる自然な発展は、ポリガミーの合法化にある」\citep{deboer15:_it_time_legal_polyg}。

2006年には、300人以上のLGBTおよびその支援者の学者・活動家が「同性婚を超えて:すべての家族と\ruby{親密関係}{リレーションシップ}のための新たな戦略的ビジョン」と題した声明を発表し、「複数の配偶者を含む\ruby{責任}{コミット}と愛情をともなった家族」を含む、多様な\ruby{人間関係}{リレーションシップ}の法的な承認を求めた\citep{MR06:_beyon_same_sex_marriag}。

同性婚の支持者は、基本的な自由と平等の原則が、彼らに結婚の権利を認めることを求めていると主張する。
人々には個人の自由の一環として、誰と結婚するかを選ぶ基本的権利があると主張し、また、すべての市民が社会の制度への平等なアクセスを享受すべきだと訴える。
特定の集団の関係を法的に認めないことは、その集団を軽視し、社会的従属を助長するものだ。
しかし、これらの主張はポリガミーの擁護者にも同様に当てはまるように思われる。
ロバーツ判事\ig{John Roberts}が指摘するように、「もし結婚の機会を持てないことが「同性カップルを軽視し、従属させる」ものであるならば、なぜ同じ「制約の押しつけ」が、多者関係に充足を見出す人々を軽視し、従属させることにならないのか」(\emph{Obergefell}, p.21)。
この主張を強化するために、多くのポリアモリーの人々は、自らの複数のパートナーへの欲求が、同性愛と同様に生得的な性質だと主張する。
セックスセラピストのマーキー・ツイストは本書の著者に対し次のように語っている。
「同意に基づく非モノガミーのクライアントは、ほとんどの場合、これは生まれつきの感覚だと語ります。
子供の頃からずっとそう感じていたが、年長になってから「同時に複数の人を好きになってはいけません」と教えられただけなのだ」、と\citep{mcarthur16:_why_peopl_are_fight_get}。

さらに、同性婚に対して保守派が主張する二つの主要な反対論{\DDASH}それが何らかの意味で「不自然」だというものと、婚姻が伝統と歴史を持つ制度であり恣意的に再定義することはできないというもの{\DDASH}は、たとえそれを受け入れるとしても、多重婚に対しては適用できない。
むしろ、これらの論拠は多重婚の合法化を支持する方向に働く可能性がある。
もし婚姻の自然な目的の一つが生殖だとするならば、複数の配偶者を持つことはそれに対する障害とはならない。
また、歴史を通じて、多くの社会が多重婚を認めてきた。
実際にそれを実践した人々は各社会の中で比較的少数であったとしても、それが制度として存在していたのは確かだ。
古代イスラエルでは多重婚が認められており、聖書にはモーセやソロモンを含む多くのポリガミー実践者が登場する。
中世に西ヨーロッパを征服したゲルマン民族も、男性が複数の妻を持つことを許容していた。
ムハンマドは複数の妻を持っていた可能性があり、イスラム教はポリガミーを容認している。
現在のところ、多重婚は50か国以上で合法とされており、けっして稀なものではない。
たとえば、スーダンでは婚姻の40%がポリガミーであり、西アフリカでは女性の3分の1以上がポリガミー婚に属している\citep{economist17:_link_between_polyg_war,dalton14:_why_is_polyg_more_preval_wester_afric}。

複数関係にある人々にとって、多重婚を合法化することは、社会的承認を与え、その関係に尊厳を与えることになる。
非モノガミーの実践者たちは、同性愛者が過去に(そして現在も)そうであったように、自分たちが社会的スティグマと差別にさらされていると主張している。
彼らの関係を公的に認めることで、社会はこうした偏見が容認されないことを明確に示すことになる。
また、これは人々が選択できるライフスタイルの幅を広げ、ジョン・スチュアート・ミルが提唱した、人々が自らの幸福に至る最善の道を見出すための「生活の実験」を促進することにもつながる\citep[p.57]{mill89:_liber_other_writin}。
ロナルド・デン・オッターは、多重婚の合法化について次のように述べている。

\begin{quote}
多重婚の合法化は、その社会的スティグマを取り除き、どのような結婚形態が自分に最も適しているかを判断する一環として、より多くの人々が型にはまらない結婚を試みることを促す一歩となるだろう……。
立法者や裁判官が、法的な不利益を伴わない形でポリアモリー実践者にさまざまな婚姻形態を試すことを承認することで、モノガミーの人々も自らの親密な関係についてより深く省察するよう促される可能性がある。
\citep{otter18:_perfec_argum_legal_recog_polyam_relat}
\end{quote}

\subsection{多重婚の危害とコスト:反論}

多重婚が本質的に家父長制的な制度だという主張に対し、支持者たちは、現代の民主社会において合法化された多重婚はまったく異なる性格を持つと主張する。
より平等な社会では、多重婚を選択するのは宗教的保守派だけではなく、進歩的なポリアモリー実践者も少なくないだろう。
複数の男性と結婚する女性も多く現れ、複数の男性と複数の女性が関与するより複雑な構成の関係も頻繁に生じるはずだ。
ポリアモリーを含む非モノガミーはLGBTQ+コミュニティにおいてより一般的であり、支持者たちは、多くのポリガミー関係が同性または両性愛者のパートナーを含むものになると考えている。

残念ながら、多重婚がアメリカや西ヨーロッパのような国で合法化された場合にどのような形をとるかを統計的に予測する根拠は存在しない。
しかし、多重婚の支持者は、正確な数値は重要ではないと主張する。
もし、私たちが多重婚の多くをそれが家父長制的だという理由で禁止し続けるならば、より伝統的な形の多重婚を実践する人々の行為によって、家父長制的でない多重婚を望む人々も不当に罰せられるということになる。
アンドリュー・マーチは次のように述べている。
「(ポリガミーの一部である)ポリ\kenten{ジニー}が女性の尊厳と自律に対する本質的な侵害であるために、すべてのポリガミー(より広い集合)を禁止しなければならない、ということでなければならないかのようだ」\citep{march11:_is_there_right_polyg}。

第二に、多重婚の支持者たちは、多重婚世帯の人々を保護するための、より制約の少ない手段がすでに存在すると主張する。
配偶者や子供に対する虐待を禁じる現行法は、当然複数の配偶者をもつ世帯にも適用され続けるし、多重婚の女性も離婚の権利をもち、虐待的な関係から逃れようとする女性に提供される社会的支援を完全に受けることができるはずだ。
また、多重婚を公的に認めることで、こうした婚姻関係にある人々の脆弱性を軽減できる可能性もある。
モーラ・ストラスバーグは次のように述べている。
「非公開で秘密裏におこなわれるポリジニーの結婚式は、強制的な家族やコミュニティが潜在的な花嫁の環境を完全に支配することを可能にし、花嫁が単に拒否することでは結婚を回避できないという状況を強化してしまう」\citep[p.369 fn.116.5]{strassberg02:_crime_polyg}。

末日聖徒イエス・キリスト教会(FLDS)の一員であるシャーリー・ドレイパーは、ポリガミーが重罪とされるユタ州で暮らしていた。
彼女は次のように語る。
「私は、指導者たちが法執行機関への恐怖を利用して支配力を強めるのを直接目にしてきました」。
彼女は状況から抜け出すのに6年を要したという。
「助けを求める方法がありませんでした。
どこへ行っても、私は見た目で重罪人だと認識されてしまい、敵意を持って迎えられました」\citep{smardon20:_polyg_is_decrim_utah}。
ポリガミーの結婚を地下社会から解放すれば、その実態がより明らかになり、関係者を適切に保護しやすくなるはずだ。

多重婚の支持者は、たとえ多くの多重婚が男性支配的な形になるとしても、二人婚の多くも同様だと指摘する。
アンドリュー・マーチは次のように述べている。

\begin{quote}
女性を含め、夫に従属することが女性の本来の役割であると実際に信じている人々についてはどうだろうか。
暴君的で家父長制的な自己中心的な男性のもとで生きることは、モノガミーのもとで唯一の妻である場合よりも、一夫多妻制の複数の妻のうちの一人である場合の方が実際に悲惨という証拠はあるのか。
むしろ逆である可能性が高いのではないだろうか。
したがって、合法的なポリガミーが女性の安全、福祉、そして自律に関する多くの正当な懸念を引き起こすことは確かだが、こうした懸念は現在も存在しているものであり、現在取り締まることを考えもしない多くの慣行、たとえば親の取り決めによる結婚や、保守的な家父長制的な共同体における結婚そのものについても同様だ。
\citep[p.260]{march11:_is_there_right_polyg}
\end{quote}

さらに、少なくとも一部の多重婚が非家父長制的だという事実は、その多重婚制度のメッセージとしての性質を変えるだろうと支持者たちは主張する。
多重婚が単なるポリジニーにとどまらなくなれば、それはもはや女性の従属的な地位を社会に示すものではなくなるだろう。
そして、女性が複数の男性と結婚することが男性のそれと同じように自由であることが示されることで、伝統的なポリガミーの婚姻にある女性や、そうした婚姻がおこなわれる共同体の女性たちが、自分たちの文化における制度の家父長制的な性質に疑問を抱く契機となるかもしれない。

多くの男性が結婚相手を見つけられず、その結果として社会不安が生じる可能性があるという懸念に対し、支持者たちは、アメリカのような国では多重婚を合法化しても結婚適齢期の女性の数に測定可能な影響を与えることはないだろうし、ひいては社会全体の安定性にも影響を及ぼすことはないだろうと主張する。
すでに多くの人々がさまざまな形で複数の関係を維持しており、単に結婚という承認を得ていないにすぎない。
こうした関係に法的な承認を与えることが結婚市場全体に大きな影響を与えてしまうとは考えにくい。
むしろ、多重婚の承認は逆の効果をもたらす可能性がある。
複数のパートナーを持つことが許されることで、従来ならば結婚相手を見つけるのが難しかった人々にも新たな可能性が開かれるかもしれない。
この点は、現在多くの男性が結婚から排除されている中国のような性比が極端に偏った社会では特に重要だ。
経済学者の謝作詩(Xie Zuoshi) は、中国における大量の未婚男性問題を解決する手段として、合法的なポリアンドリーを提案している\citep{weller15:_econom_has_contr_solut_china}。

多重婚の支持者たちは、それが社会全体に過度な財政的負担を課すことはないと主張する。
彼らによれば、多重婚のコストを計算する際に反対派が犯している重大な誤りがある。
それは、アンドリュー・マーチの言葉を借りれば、「(1) 各婚姻において生産的な配偶者は一人だけであると仮定し、(2) 社会的給付や補助金が、雇用されている者の配偶者扶養人数に比例して法的婚姻を通じて分配されると仮定している」ことだ\citep[p.268]{march11:_is_there_right_polyg}。
しかしながら、こうした前提は大半の西洋社会には当てはまらない。
西洋社会では、多くの既婚女性が働いており、また、社会的給付の多くは婚姻状況に関係なく受給できる。
このような社会において、多重婚が社会全体に不釣り合いな負担をもたらすことはない。
支持者たちは、婚姻制度がもたらすコストは、扶養される配偶者の総数と労働市場に参加する既婚者の総数との比率によって決まるとし、合法化された多重婚によってこの比率が増加する理由はないと主張する。

\subsection{本節のまとめ}

現在のところ、多重婚を認めるための政治的な動きはほとんどない。
しかし、家族の構造は変化しつつあり、裁判官や立法者がこの現実を認識し始めている点には注目すべきだ。
アメリカのいくつかの州では、三人の親による養子縁組(third parent adoption)が認められており、2017年にはニューヨーク州の裁判官が、非生物学的な母親を含む3人の親による共同親権を認めた。
カリフォルニア州の法律では、稀なケースではあるが、3人以上の者が法的な親として認められることもある。
もし子供が複数の法的な親を持つことを認めるのであれば、その親同士の関係を婚姻として認めることは、それほど大きな飛躍ではない。
これにより、子供たちに法的保護が与えられるとともに、彼らの家族のあり方が正当なものとされる。
さらに、同性婚の例が示すように、法律や社会の結婚に対する認識は急速に変化しうる。
哲学者たちは、婚姻をより根本的に変革する提案、場合によっては結婚と性的親密さを切り離すような提案を考察している。
次に、この点について論じる。

\section{結婚の改革}

2005年、マサチューセッツ州最高裁が同性カップルに対する差別を理由に同州の婚姻法を違憲と判断した翌年、ケヴィン・ナダルは婚姻制度に対する新たな挑戦をおこなった。
ナダルは29歳のカウンセリング心理学者であり、パフォーマンスアーティストでもあり(加えて、注目を集めることを好む人物でもあある)、ニューヨーク市で自らと結婚するという式を挙げた。
「私は、ケヴィン・ナダルは、ケヴィン・ナダルを、病める時も健やかなる時も、愛し慈しむことを誓います」と宣言したのだ。
彼は5000ドルを費やしてパーティーを開き、125人の招待客を迎えた。
招待客は彼のギフト登録リストに従い、ゲイ・ポルノを含む贈り物を持参した\citep{krum05:_man_who_married_himsel}。

アメリカ合衆国のいかなる州もナダルの自己婚を認めることはないだろうし、彼の試みが流行を生むこともなかった。
ブログ \emph{Unmarried America} は、この出来事を「配偶者のいない結婚式と独り誓う結婚の誓いは流行らず」という、味気ないが正確な見出しで報じた\citep{coleman07:_spous_weddin_solo_vows_not_catch}。
しかし、ナダルは多くの人々が不公平だと感じる問題に注意を向けさせようとしていた。
それは、独身者が社会的なスティグマにさらされるだけでなく、法的な不利益も受けているという事実だ。
カリフォルニア大学サンタバーバラ校の心理学者ベラ・デパウロは、この「シングル差別」(singlism)が現実であることを人々に理解させようと長年活動してきた。
彼女は \emph{The Daily Beast} に対して次のように語っている。

\begin{quote}
  シングル差別{\DDASH}独身者に対するステレオタイプ化、スティグマ付与、差別{\DDASH}は、ほとんど認識されず、異議も唱えられていません……同性婚の支持者の主張は、「なぜ特定の形のカップルでなければ公平に扱われないのか?」というものですが、私の主張はさらに広範に及びます。
「なぜ誰もが何らかのカップルの片割れでなければ、他の人と同じ連邦政府の利益や保護を受けられないのか?」……人々はシングル差別に気づかず、仮に注意を向けられたとしても、それが問題だとは思わないのです。
\citep{kelly12:_singl_out}

\end{quote}

哲学者エリザベス・ブレイクは、既婚カップルに与えられる特権的地位を指す別の(しかし同様に扱いにくい)用語として、「\ruby{性愛社会規範}{アマトノーマティヴィティ}」(amatonormativity)を提案している。

結婚したカップルに与えられる利益は大きい\citep[pp.380--381]{sunstein08:_privat_marriag}。
エディス・ウィンザーがアメリカの結婚法に異議を唱えたのも、法律上結婚したカップルに与えられる税制上の優遇措置が理由であった。
たとえ結婚の利益がゲイやレズビアンにも拡張されるとしても、そもそも結婚に特別な地位を与えること自体が、リベラルな中立性の原則に違反すると主張することもできるだろう。
中立性の原則によれば、自由社会の政府は、ある集団の私生活上の選択を、他の集団の選択よりも特別に優遇すべきではなく、優遇を正当化するには強い理由が必要とされる。
そして、一部の人々は、結婚に伴う利益が公的な優遇措置の一形態だと考え、これを廃止するか、より中立的な形で再分配するために大幅な改革をおこなうべきだと主張している。

この種の提案には二つの種類がある。
第一に、結婚を新たな法的な承認を受けたケア提供のための結びつきに置き換えるべきだと考える立場がある。
これは、恋愛関係にない人々にも適用されるものだ。
第二に、結婚をプライベート化すべきだとする立場がある。
この場合、結婚は私的契約の一形態となり、他の種の契約と同様に扱われることになる。
この提案では、国家が特別な制度として結婚を認めることは廃止され、結婚したカップルのみに与えられる国家による便宜も消滅する。

\subsection{ケア提供ユニオン}

一部の哲学者は、結婚をケア提供関係に基づく新たな制度に置き換えるべきだと提案している。
タマラ・メッツは、国家が承認する「親密なケア提供ユニオン」(Intimate Care-Giving Unions)を導入することを提唱している。
エリザベス・ブレイクは、「最小結婚」(minimal marriage)と呼ぶ新たな国家承認制度を提案している\citep{metz10:_untyin_knot,brake12:_minim_marriag}。
メッツの親密なケア提供ユニオンは、現在結婚と結びつけられている優遇措置、たとえば近親者としての権利や相続権などを、互いにケアの義務を負う親密な関係にある者同士が指定できるようにする制度だ。
その名称が示すように、この親密な関係は恋愛関係に限定されない。
ブレイクの構想もこれに類似している。
彼女は、ケア関係とは「身体的または感情的なケア提供、あるいは単にケアする態度」を伴うものであり、「互いに知り合い、継続的な直接的接触があり、共通の歴史を持つ個人」に関わるものだと述べている\citep[p.106]{brake12:_minim_marriag}。
ただし、メッツとは異なり、ブレイクはこのような関係に新たな名称を与えるのではなく、「結婚」という語を保持するべきだと考えている。

メッツは、結婚に伴う優遇の体系自体は維持しつつ、それが適用される関係の種類を変える形で制度を再構築する。
一方、ブレイクの制度では、優遇を一括して提供する必要はなく、各人が自由に分配できるようにする。
したがって、メッツの「ユニオン」は二者間の関係に限定されるが、ブレイクの最小結婚は、さまざまな形態の多者間関係を包含できる。
彼女は次のように述べている。

\begin{quote}
個人は、複数の相手と法的な婚姻関係を結ぶことができ、その際には相互的であっても非対称的であってもよく、関係する当人たち自身が、当事者の性別や人数、関係の類型、そしてそれぞれと交換する権利と責任の内容を決定することができる。
\citep[p.157]{brake12:_minim_marriag}
\end{quote}

ブレイクは、最小結婚に伴う具体的な権利を明確には定めておらず、それは社会的文脈に依存すると述べている。
しかし、彼女は次のように言う。

\begin{quote}
理想的な平等主義社会において、以下のような権利が最適な候補となる
だろう。
配偶者の在留資格、雇用および転居の支援、優先的雇用(現在、米国の軍人や
公務員の配偶者、および一部の民間企業で提供されている)、居住者資格(州内教育機
関での授業料などに関連する場合)、病院や刑務所での面会権、忌引や配偶者の介護休
暇、退役軍人墓地での配偶者との埋葬、裁判証言に関する配偶者免除、そして第三者に
よる福利厚生提供に関する州の指定などだ。
\citep[p.157]{brake12:_minim_marriag}
\end{quote}

\subsection{ケア提供ユニオンに対する懸念}

メッツとブレイクは、ケア提供関係を認めることで、ブレイクが「\ruby{性愛規範
的}{アマトノーマティブ}差別」と呼ぶものを回避できると考えている
\citep[p.89]{brake12:_minim_marriag}。
% \footnote{Ibid., p.89.}。
彼女たちの提案は、重要な人間関係に対して国家の支援を提供しつつ、恋愛関係を特別視しないことでリベラルな中立性と完全に一致させることを意図している。
しかし、これらの提案が完全に中立であるという目標を達成しているかどうかは疑問が残る。
出発点は、リベラルな社会では、国家は特定のライフチョイスを優遇すべきではなく、したがって、独身の人々が享受できない特権を結婚した人々に与えるべきではないという提案だ。
しかし、ケア提供関係に特別な地位を与えることで、国家は依然として特定のライフチョイス、
すなわちそのようなケア提供関係を形成する人々の選択を優遇していると主張することができる。
ブレイクとメッツは、ケア提供は\ruby{基本的善}{プライマリーグッド}であるため、彼女たちの提案はリベラルな中立性と一致していると主張している。
つまり、ケア提供は、誰もが当人の他の目的や価値観に関係なく望み必要とするものだ。
人々は結婚の拒否を選択するかもしれないが、ほとんどすべての人は人生の少なくともある時
点では他者をケアし、または他者からケアを受ける。
この種の相互依存は、人間の条件の一部にすぎない。
しかし、反対者は、そうしたケア関係に対する国家の支援のための
資金はどこかから調達しなければならず、必然的に、比較的自分だけで自足しているた
めに、そのような支援を利用しないことを選択する人々に不均衡な負担を強いることに
なると主張しうる。

さらに、ブレイクの提案に関しては、彼女がケア提供関係を「結婚」と呼び続ける理由について疑問が生じうる。
この用語は歴史的に、特定の宗教によって承認された関係と結びついており、そこからは長い間、同性愛者たちが排除されてきた経緯がある。
ブレイク自身もこの問題に苦慮している。
彼女は次のように述べている。

\begin{quote}
一方では、過去の差別を是正するために、法的結婚の象徴性は十分に強くなければならず、同性関係や同性のネットワークに結婚としての地位を認めて是正する効果をもたらす必要がある。
他方で、その象徴性があまりに強すぎると、最小結婚の枠外にいる子どもや大人がスティグマを受けることになってしまう\citep[p.187]{brake12:_minim_marriag}
\end{quote}

彼女は最終的に「結婚」という用語を維持したいと考えているが、それに対する異議を完全に退けることはできないと結論している。

メッツの提案は二者間の結びつきに焦点を当てている。
この点について、ブレイクの提案が認めるような、より複雑で多面的な関係を不当に排除していると考えることもできる。
一方で、結婚の\ruby{特権}{ベネフィット}を複数の相手に分配できるようにすると、ブレイクの多重婚はあまりに複雑になり、実際には機能しなくなるのではないかという懸念もある。
ブレイクは「万が一、驚くほど多くの人々が本当にケア関係を維持するということになるのであれば、面会権などの分配可能な特権を否定する原則的な理由はない」と認めている\citep[p.140]{brake12:_minim_marriag}。
しかし、彼女は、私たちが維持できる有意義な関係の数には限界があると考えており、こうした私たちの感情の現実が最小結婚における関係の数の上限を定めることになると主張する。
彼女は、ヒュー・ヘフナーが「彼のトップ50人のプレイメイト」と結婚することは認めるべきではないし、「100人のカルト信者団」の結婚などを認めるべきではないとしている\citep[p.164]{brake12:_minim_marriag}。

クレア・チェンバースは、最小結婚は「あくまでケア関係という基本的な価値を支援するための仕組みであり、\ruby{自由至上主義}{リバタリアン}的な参加自由の無秩序状態を生み出すものではない」と述べている\citep[p.96]{chambers17:againstmarriage}。
\ig{Clare chmbers}
しかし、チェンバースは、ブレイクがパートナーの数に上限を設けようとすることで、自らが掲げる完全な中立性という理念に反してしまっていると指摘する。

\begin{quote}
この合理的ではあるが問題含みの制限は、国家が依然としてどの関係が真に「ケアを伴う関係」であるかについて判断を下さざるをえないことを伴う。
……しかし、ケア関係をヒュー・ヘフナーの関係のようなものと区別するためには、国家は婚姻上の権利を求める人々が「ケアを伴う関係」の一員であるのか、それとも単なる「プレイメイト」であるのかを評価しなければならない。
ブレイクが想定する政治的リベラルな国家が、市民の尊厳とプライバシーを尊重しつつ、また\ruby{善}{グッド}のさまざまな捉え方に関して中立を保ちつつ、どのようにしてこれをおこなうことができるのかは疑問だ。
\citep[p.96]{chambers17:againstmarriage}\ig{Clare chmbers}
\end{quote}

ブレイクの提案は、同意の問題も提起する。
そのような結婚において、配偶者の一人がさらに新たな人物と結婚したいと望む場合、すべての現在の配偶者たちが同意しなければならないのか。
この問題はあらゆる多重婚の提案に伴うが、最小結婚ではより顕著に現れる。
なぜなら、この制度では兄弟姉妹や友人をも婚姻関係に含めることが可能だからだ。
アン・ファーガソンは次のように述べている。

\begin{quote}
新たに妻を迎えたい場合、あるいは別の友人や兄弟姉妹を〔最小結婚に〕含めたい場合、すべての配偶者の同意が必要なのか。
現代のアメリカのように流動的なライフスタイルが一般的な社会において、雇用上の福利厚生として誰が転居の権利をもつことになるのか、あるいは離婚時の財産分与をどのように処理するのかといった問題はどう解決されるのだろうか。
\citep{ferguson14:_review_brake}
\end{quote}

国家が親密な関係に対して真に中立でありたいと考えるならば、さらに一歩進むべきだと主張する者もいる。
こうした議論の論理的帰結は、国家が関与する結婚制度自体を完全に廃止することだと彼らは考えている。

\subsection{結婚のプライベート化}

結婚の民営化を擁護する者たちは、政府がすべての婚姻関係に対して一律に付与する特典や義務を廃止し、個人が独自の取り決めをおこなうようにすべきだと主張する。
これらの取り決めは、他の契約と同様の法的地位を持ち、国家は通常の契約と同じようにその履行を保証すればよいという立場だ。
デヴィッド・ボアズは、この提案を次のように要約している。

\begin{quote}
結婚は、二者間の私的な契約とすればよい。
伝統的な稼ぎ手/専業主婦の形態を望む者は、離婚時の財産分与や扶養義務に関するルールを明記した契約を結ぶことができる。
より自由な関係を望むカップルは、財産を分離したまま、特定の費用を共同で負担することに合意することも可能だ。
保護すべき資産をもっている人は、裁判所が尊重してくれそうな婚前契約を結ぶことができる。
結婚契約は、私たちの多様な資本主義社会における他の契約と同じように、個々のニーズに応じてカスタマイズ可能なものとすればよい。
標準的な一律的な契約を望む人はそれを容易に利用できるようにすればよい。
ウォルマートが賃貸契約書の横で結婚契約書を販売することも考えられる。
こうすれば、カップルは、自分たちの契約が第三者によって予告なく変更されるという不意打ちを受けずに済む。
各宗教団体、シナゴーグ、寺院は、それぞれの教義に基づき、祝福する結婚の条件を自由に定めることができる。
\citep{boaz97:_privat_marriag}

\end{quote}

結婚がこのように\ruby{私的}{プライベート}な契約となった場合にも、国家は依然として他の契約と同様にそれに一定の制限を課す必要がある。
極端な搾取から当事者を保護すること、商業契約における詐欺と同様に結婚契約における詐欺を防ぐこと、契約に明記されていない事態に対応するための一定のデフォルト条件を定めることなどが求められる。
しかし、結婚のプライベート化を支持する者は、こうした規制を可能な限り最小限に抑え、契約当事者に最大限の柔軟性を与えるべきだと考えている。

\subsection{結婚のプライベート化の擁護論}

私的な結婚契約には、現在の結婚制度に比べていくつかの利点があると、その支持者は主張する。
まず第一に、より柔軟であり、市民が自らの人生を最も適した形で生きる自由を促進する。
また、既存の多様な関係形態を認めることにもつながる。
ドリアン・ソロットとマーシャル・ミラー\ig{Marshall Miller}は次のように述べている。
「私たちには、国民として、たとえ私たちの想像する規範や理想から外れていたとしてもさまざまな家族を公平に扱うという倫理的義務がある」\citep[p.89]{solot06:_takin_gover_out_marriag_busin}。

第二に、このような契約は、人々に自らの関係の条件についてより慎重に考えさせることを促す。
ナンシー・クナウアーは次のように述べている。
「私的で規制のない結婚制度は、期待が明確にされ、責任が明確であり、特定のカップルに合わせた条件が設定されることで、より意識的なパートナーシップを促進するだろう」\citep[p.103]{knauer06:_marriag_skept_respon_pro_marriag}。

第三に、結婚のプライベート化は、結婚に対する多様な信念を尊重することになる。
人々や文化によって、結婚の意味や定義は異なる。
結婚を私的契約とすることで、各共同体が自らの道徳的・文化的観点に基づいて婚姻関係を自由に形成できるようになる。
エドワード・ゼリンスキーは次のように述べている

\begin{quote}
多様な政治体制は、本当の多様性に対して本当に寛容でなければならない。
結婚の領域において、現代のアメリカ人は非常に多様な見解を抱いている。
このような状況下で、国家が結婚という制度に一律の定義を押し付けることは、望ましくもなく実行可能でもない。
\citep{zelinsky08:_marriag}
\end{quote}

たとえば、ある共同体が、結婚を異性愛者のカップルに限定したいならばそうすることができ、そのことが共同体の外の同性愛者の権利を否定することにはならない。
人々は複数の人と契約を結ぶことで多重婚を形成できる。
そして、結婚の唯一の正しい定義をめぐる議論を終わらせることで、結婚のプライベート化はゼリンスキーが言うように、社会にかかる「圧力を緩和する」ことになる\citep[pp.1178--1179]{zelinsky06:_dereg_marriag}。

ゼリンスキーは、より多様な選択肢が生まれることで、結婚を選択する人が増える可能性があると示唆している。
彼は次のように述べている。
「国家の独占権を廃止すれば、結婚のための健全で競争力のある市場が生まれ、それによって結婚という制度が強化される」\citep[p.1164]{zelinsky06:_dereg_marriag}。
すでに多くの人々がさまざまな形の事実婚を営んでおり、その法的な承認の程度は法域によって異なる。
こうした人々も、自らのニーズに応じて契約内容を調整できるのであれば、より正式な契約を結ぶことを検討するかもしれない。
このことはその関係を安定させる効果をもたらし、それは子供たちにとっても利益となるだろう。

結婚のプライベート化は、結婚は宗教に根ざしているべきだと考える人々にとって魅力的であるはずだ。
なぜなら、それによって、彼らは独自の方法で結婚を定義できるようになるからだ。
たしかに彼らが失なうことになるものもある。
すなわち、多くの宗教が人間生活の中心的な重要性をもつと考える「結婚」に対する国家の指示を失なうことになる。
しかし同時に、結婚を自らの価値観に従って定義する自由を得ることにもなる。
アラン・ダーショウィッツは、さらに、世俗主義者にも結婚のプライベート化を支持する理由があると指摘する。
歴史的にさまざまな宗教と結びついてきたこの制度を国家から切り離すことで、政教分離を強化することになるからだ\citep{dershowitz03:_to_fix_gay_dilem_gover}。

\subsection{結婚のプライベート化への懸念}

結婚のプライベート化にはさまざまな異論が提起されている。
第一に、国家が結婚に一般に理解される社会的意味を保証することには価値があると主張されている。
この点について、かつて結婚のプライベート化を支持していた著名な法学者キャス・サンスティーンは、後にその立場を撤回した。
彼が考えを改めた理由は、結婚には国家の承認によってのみ得られるコミュニケーションとしての価値があり、それが人々にとって重要だという確信に基づくものだった。
彼は次のように述べている。

\begin{quote}
無数の人々([自著での結婚のプライベート化の議論の]執筆後2008年に結婚した筆者自身を含む)にとって、公的な結婚は重要であり、さらにはかけがえのないものだ。
それは地位を承認するものであり、しかもその承認は国家そのものに由来する独特の形でなされる。
このような地位を廃止することは、大きな損失をもたらすだろう。
\citep[p.296]{sunstein17:_statem_i_most_regret}
\end{quote}

結婚という共通の制度の一環として互いに対するコミットメントを伝えることで、夫婦はそのコミットメントを維持するのに役立つ社会的な期待を生み出すことができる。
クレア・チェンバースは次のように述べている。

\begin{quote}
結婚することによって、夫婦は広く共有された意味と強い社会規範を持つ社会的形式に身を置くことで関係のさまざまな側面を確立する。
さらに、\kenten{法的}結婚をすることで、そのコミットメントが破られた場合に法的制裁を受ける可能性を伴うことになり、それはいっそうそのコミットメントを強化することになる。
(Chambers, 2017, pp.99--100; cf. Macedo, 2015, p.95) \nocite{chambers17:againstmarriage} \nocite{macedo15:_just_married}\ig{Clare chmbers}
\end{quote}

公式の結婚に反対する者は、結婚が象徴する「コミュニケーション的理想」は常に肯定的なものではなかったと主張する。
結婚は長らくゲイやレズビアンを排除してきており、その排除の歴史は私たちの社会における結婚の理解と密接に結びついている。
しかし、公式の結婚を支持する人々は、同じ論理によって、ゲイの結婚が合法化されたならば、それはゲイやレズビアンたちが対等だということを伝達し、彼ら彼女らに対する差別と闘う手段となると指摘する。
また、結婚の定義を各宗教組織などに委ねた場合、それらの組織が再び結婚を利用して、ゲイやレズビアンを二級市民として扱うことが可能になるだろう\citep[cf.][p.205]{hartley12:_polit_liber_marriag_famil}。

私的契約には避けがたい抜け穴があり、それはしばしば非常に重大なものとなる。
カップルは、自分たちがいずれ離婚するとは考えないために、また関係の初期の幸福な段階で気まずい話し合いをすることを避けるため、離婚の可能性に適切に備えることができないことが多いだろう。
そしてキャロル・サンガーが述べるように、「楽観性バイアスに加えて、一般に契約を結ぼうとしている当事者たちが契約事項を詳細に討議することを省略してしまう標準的な理由がすべて当てはまってしまうことになるだろう。
契約交渉決裂の原因となる問題を交渉に持ち出すことへの恐れや、現在継続中の関係において暗に抱いているすべての期待や要求、そして義務を明確にしてしまうことへの躊躇がそれに当たる」\citep[p.1315]{sanger06:_case_civil_marriag}。
結婚のプライベート化を支持する人々は、自らの制度のもとでは「\ruby{契約補完ルール}{ギャップ・フィラー}」が必要になることを認めており、当事者たちの結婚契約に、必要不可欠な条項が含まれていない場合に備えたデフォルトの規定を設ける必要があると考えている。
しかし、サンガーは、そうした\ruby{契約補完ルール}{ギャップ・フィラー}があまりに広範囲にわたるため、一種の「\ruby{影の体制}{シャドウ・レジーム}」となってしまい、結局のところ事実上の結婚制度に行き着くのではないかと懸念する。
もし最終的にこんな影の体制を確立することになってしまうのならば、結婚のプライベート化によって余計な労力や混乱を引き起こしてまで得られるものは何なのかと疑問に思わざるをえない。
サンガーは次のように述べている。
「もし……結婚契約が、さまざまな契約の中でも独自のものであり、それゆえに「独自の規則を必要とする」のであれば、その規則を契約に訴えて間接的に適用するのではなく、最初から明確に定めておく方がよいと私は考える」\citep[p.1315]{sanger06:_case_civil_marriag}。

私的な結婚契約は、契約当事者間では拘束力のある合意となるだろうが、必ずしも第三者を拘束するものではない。
たとえば、病院に入院している患者の家族が、その患者のパートナーに面会を認めないということがあるかもしれない。
雇用主が、私的契約に基づくパートナーに対して福利厚生を適用しない可能性もある。
また、結婚に伴う恩恵のなかには、在留資格や裁判証言免除特権のように、国家によってのみ付与されるものがある。
国家が裏書きする結婚制度ならば、こうした恩恵を付与する合意ずみの単一基準を提供することができる。
私的結婚制度のもとでは、国家はどの特権をどのような種類の結婚に付与するのかを決定しなければならない。
このような国家の関与は、結婚のプライベート化がもたらすはずの利点の一部を否定し、国家がどこまで関与すべきか、どのように関与すべきかについてさらに議論をおこなわねばならないことになる。
また、私的契約に基づく関係に国家が関与することにより、国家は事実上、同性愛者を排除する宗教団体の結婚制度のような、不平等な形態の結婚を承認する立場に置かれるしまうことになるかもしれない。

人々はまた、結婚のプライベート化が家族関係の中で最も脆弱な立場にある者に与える影響について懸念している。
歴史的に見て、結婚制度は共同体のなかの脆弱なメンバー、すなわち女性や子供たちを保護する役割を果たしてきた。
現代社会においては、女性が以前よりも自立しているため、こうした保護の必要性はかつてほど緊急性を要するものではないと考えることもできる。
しかし、多くのカップルはいまだに伝統的な形で家庭内の労働を分担しており、これが女性を脆弱な立場に置く要因となっている。
男性が労働し、女性が子供を育てるという分担は依然として一般的であり、多くのカップルはこうした性別役割について伝統的な価値観を持っている。
こうした関係において結ばれる契約は、女性を十分に保護するものとはなりにくいだろう。
また、結婚には子供が関与するようになるが、子供はとりわけ脆弱な存在であり、国家はその保護に特別な利益関心を持つ。
とりわけ、カップルが比較的伝統的な価値観を持つ文化圏の出身である場合、この問題はさらに深刻になる。
そのような文化圏では、独自の結婚契約が形成される可能性があり、それによってその文化圏の価値観がメンバーに押しつけられ、メンバーがその独自の契約に従うことを求められたり、圧力を受けたりすることになる。

結婚のプライベート化に反対する者は、女性と子供を保護するためには、何らかの強制的な規則の体系が必要だと主張する。
そして、こうした規則がその目的を十分に果たせるほど強固なものであるならば、それは結局のところ、国家が私的な結婚契約に関与する「影の体制」となってしまうだろう。
したがって、結婚制度のプライベート化によって生じる混乱を甘受するよりも、現在の制度を維持する方が望ましい、というのが彼らの立場だ\citep[p.204]{hartley12:_polit_liber_marriag_famil}。

\section{コミットメントと結婚についての結論}

経済協力開発機構(OECD)は、過去50年間にわたり、すべての加盟国において結婚する人の数が減少していると報告している\citep{oecd22:_marriag_divor_rates}。
この減少傾向は、加盟国の大半が同性愛者の結婚を認めた後も続いている。
ただし、すべての社会集団において同じ速度で進行しているわけではない。
低所得層や教育水準の低い人々は、他の層に比べて早く結婚する傾向があるものの、結婚する割合自体は低く、離婚の可能性は高い\citep{lundberg16:_famil_inequal}。
しかし、全体的な傾向は明白だ。

未婚者の数が増えるにつれ、結婚や親密な関係のあり方を再考する声も高まるかもしれない。
人々は、LGBTQ+の人々がすでに長らく認識してきたこと、すなわち、結婚しているか否かにかかわらず、さまざまな形の親密な社会的関係のネットワークが重要であることに気づく可能性がある。
また、非モノガミーを選択する人々の増加に伴い、多重婚を認める動きが強まるかもしれない。
さらに、結婚という制度そのものを根本的に変革しようとする急進的な提案にも、より大きな関心が向けられる可能性がある。

\phantomsection
\section{討論のための問い}

\begin{enumerate}
\item 性的忠実を約束することは、恋愛関係における他の約束(例として、相手を喜ばせるために禁煙する約束)と異なるだろうか? なぜ私たちは、性的な忠実さを(この例でいえば)こっそり喫煙しつづけることよりも重大なことだと考えるのだろうか。
\item 非モノガミーが倫理的に正当な選択肢であると考えるならば、そこから、多重婚の法的な承認も支持すべきだろうか?多重婚に対する反対論は、非モノガミーという関係の選択にも適用されるのだろうか?

\item 信仰を理由に同性婚に反対する人々は、宗教的自由の名のもとに、同性カップルへのサービス提供を拒否する(たとえばウェディングケーキを焼くことを拒否する)権利をもつべきか?
\item 同居する姉妹は、法的に結婚と同等のパートナーシップを形成できるべきか?彼女たちの関係は、結婚とどのように道徳的・法的に異なるだろうか?
\end{enumerate}

\chapter{セックスと法}

1952年、ドイツの「エニグマ」暗号の解読に貢献した先駆的な哲学者であり数学者であるアラン・チューリングは、男性との性的関係を認めたことにより「はなはだしく下品な行為」(gross indecency)の罪で有罪判決を受けた。
2年後の1954年6月、彼は自殺した。
チューリングは、当時のイギリスにおいて私的な性的行動の結果として起訴された著名男性の一人であった。
チューリングの自殺から間もない1954年8月、政府はこうした高名な人物の裁判を受けて行動を起こし、ジョン・ウォルフェンデン卿を委員長とする委員会を設立し、同性愛を犯罪とすべきかどうかを検討した。
ウォルフェンデン委員会が1957年に発表した最終報告書(ウォルフェンデン報告)では、同性愛を非犯罪化すべきだと結論づけた。
その際、委員会は原則的な大胆な声明を発表した。
報告書の中で委員たちは次のように述べている。
「私たちの見解では、法律の役割は市民の私生活に干渉することでも、特定の行動様式を強制することでもない」\citep{wollfenden57:_repor_commit_homos_offen_prost}。

現在では、多くの人がこの見解をリベラルデモクラシーの基本的な原則とみなしている。
すなわち、私的な行為(性的行為を含む)については、それに関与するすべての者が成人であり同意しており、誰にも害を及ぼさない限り、その行為は政府の干渉を受けるべきではない。
しかし、当時としてはこれは急進的な考え方であった。
当時はほぼすべての国(リベラルデモクラシー国家を含む)が何らかの形で私的な性的行為を犯罪としており、国家が人々の性生活を規制するのは当然だと広く考えられていた。
政府がこうした規制を正当化するために持ち出した理論には、公衆衛生や若者を堕落から守る必要性などがある。
だが、最も重要な理由の一つは、道徳そのものを強制する必要があるという考え方であった。
社会の大多数の人々がある行為を道徳的に誤りだと考えている場合、その行為を違法とするのは正当だという考え方が長く受け入れられてきたのだ。

ウォルフェンデン報告書以降、西洋の民主主義国家では「道徳立法」{\DDASH}つまり、無害な私的行為を「不道徳」という理由によって犯罪化する法制度{\DDASH}が大幅に撤廃されてきた。
ゲイ・レズビアン活動家たちはこの変化において重要な役割を果たし、こうした変化がLGBTQ+の人々の生活に大きな影響を与えたことは注目されることには正当な理由がある。
しかし、この変化はLGBTQ+の人々だけでなく、社会全体に影響を与えた。
リベラルデモクラシー国家の人々は、もはや多数派の道徳的非難だけで刑罰を正当化できることが当然だとは考えなくなった。
これは大きな変化であるが、別の視点から見れば、政府が市民の自由を尊重し、それを制約する場合には正当な理由を求めるべきだという、民主主義国家の原則の一つが実現されたにすぎない。

しかし、性行動に関する法律における道徳の役割は今なお議論の的となっている。
性行動に対する国家の規制が完全になくなった国はほとんど存在しない。
多くの法域では、商業的なセックス産業を何らかの形で規制している。
実際、この分野は今なお、国家が人々の性生活に積極的に干渉している領域の一つであり、また最も激しい論争がおこなわれている領域でもある。

本章では、まず「リーガルモラリズム(法的道徳主義)」について検討する。
この立場は、社会の大多数が不道徳とみなす行動を法律で禁止することが正当だとする。
次に、プライバシーの権利を超えて、基本的な性的自由の権利を承認するべきかどうかを考察する。
そして、ポルノグラフィ(ポルノ)や売春を含む商業的なセックス産業をめぐる論争を掘り下げる。
これらすべての議論を貫く中心的な問いは、国家が人々の私的な性的行動を制約することが、いかなる場合に正当化されるのかというものだ。
この複雑で難しい問題に対して、単一の哲学的アプローチが包括的な答えを提供することはない。

\section{リーガルモラリズムと性的自由}

哲学者たちは「リーガルモラリズム」(legal moralism)という用語を用いて、ある行為がたとえ非同意の第三者に対して害や不快を引き起こさないとしても、それ自体が(ジョエル・ファインバーグの言葉を借りれば)「\kenten{本質的に不道徳である}」という理由のみで、(少なくとも一部のケースでは)刑法によって処罰することが正当化されうるという見解を指す\citep[p.249]{feinberg87:_some_unswep_debris_hart_devlin_debat}。
1991年、アメリカ合衆国最高裁判事のアントニン・スカリアは、彼の意見書の中で次のように述べている。
「我々の社会およびすべての人間社会は、ある特定の種類の行為を、その行為が他者に害を及ぼすという理由からではなく、それが伝統的な表現で言うところの \emph{contra bonos mores}(善良な風俗に反して)、すなわち不道徳と見なされているという理由から禁止してきた」(\emph{Barnes v. Glen Theatre, Inc.}, Scalia判事の反対意見)。
スカリアの指摘はまちがっていない。
西洋社会におけるリーガルモラリズムの歴史は長い。

スカリアは、\emph{Lawrence v. Texas} 判決に対する反対意見の中で、「この判決は、すべての道徳立法の終焉を事実上宣告するものだ」と嘆いた(\emph{Lawrence v. Texas}, Scalia判事の反対意見)。
そして実際に、多くの活動家たちはこの判決を、同性愛セックスの犯罪化を終わらせたのみならず、道徳立法はリベラルデモクラシーにはそぐわないという基本原則をついに確立したものだと考え歓迎した。
しかし、この判決の文言はこの問題についてけっして明確ではなく、その含意をめぐっては多くの議論が交わされている。
過去数十年間にわたり、法の領域および哲学の分野においてリーガルモラリズムは後退してきたが、政府や最高裁がリーガルモラリズムの正当性を明確に否定した例はほとんどない。
むしろ、逆の立場をとっている例は少なくない。
世界各地には、現在でも道徳を根拠とした法律が多く残っている。
そして、多くの法哲学者がリーガルモラリズムを否定する立場をとるものの、この立場を擁護し続ける者も依然として存在する。

\subsection{リーガルモラリズムの擁護}

伝統的に、リーガルモラリズムは自然法論に基づいて正当化されてきた。
この理論は、道徳主義的な法にわかりやすい正当化を提供する。
すなわち、ある種の行為は客観的に不正であるために法によって禁じられるべきだ、と。
象徴的に、性犯罪は長らく「自然に反する犯罪」(crimes against nature)と呼ばれてきた。
この表現は現在でもコモンローの法域の多くの法令に残っている。
こうした法律の擁護者はしばしば宗教的な文献に訴えるが、自然法論家たちは、その原則は誰にとっても熟考を通じて理解可能であり、信仰の有無にかかわらず適用されるべきものだと主張する。
しかし、行為の犯罪化の正当化として、道徳的犯罪を「自然に反する犯罪」に分類することは、認識論的な問題を引き起こす。
私たちはどのようにして、またどのような基準で、自然法の正確な内容を知り、合意することができるのだろうか。
本書1.2節で述べたように、この認識論的問題には明確な解決策がなく、その結果、裁判所は100年前に比べて自然法の論理を適用することに対してはるかに慎重になっている。

ウォルフェンデン報告書の公表後、イギリスの著名裁判官であるパトリック・デヴリン\ig{Patrick Devlin}卿は、自然法に基づく従来の説明とは異なる形でリーガルモラリズムを擁護した。
デヴリン\ig{Patrick Devlin}は、社会の安定を維持するためには道徳規範を強制する必要があると主張する。
この考え方は彼が初めて提示したものではなく、19世紀のエミール・デュルケーム(1858--1917)やレスリー・スティーヴン(1832--1904)もまた、社会の「道徳的\ruby{網の目}{ファブリック}」とその安定性との関連性を論じていた。
スティーヴンは、「悪徳は病気のように伝染し、社会全体に広がり、破壊する」と述べている(Stephen, 1967; cf. Durkheim, 1974)。
\nocite{stephen67:liberty}\nocite{durkheim74:sociology}
デヴリン\ig{Patrick Devlin}の議論はこの立場を最も包括的に展開したものであり、彼の講演は近代法哲学の二大巨頭による応酬を招くこととなった。

デヴリン\ig{Patrick Devlin}は1959年に英国学士院で講演を行い、リーガルモラリズムを擁護した。
1965年には、講演を改訂したものに加え、関連する六つの講義を収録した『道徳の強制』(\emph{The Enforcement of Morals})を出版した。
彼はウォルフェンデン報告書が主張する「国家の介入を免れるべき私的領域」の概念を否定するが、自然法といったものには訴えない。
彼にとって、社会がその道徳規範を強制しなければならないのは、それが何らかの深遠な道徳的真理を包含しているからではなく、社会の道徳が何であれ、その維持は社会の安定にとって不可欠であるからだ。

デヴリン\ig{Patrick Devlin}は、いかなる行為も真に私的なものではありえず、社会の支配的な道徳に反する行為は社会の存続を脅かしうると考えた。
彼は、「すべての社会は政治的構造だけでなく道徳的構造も持っている」と述べている\citep[p.9]{devlin65:_enfor_moral}。
そして、これら二つの要素は密接に結びついており、一方が失われればもう一方も存続できないと主張する。

\begin{quote}
社会とは\ruby{観念}{アイディア}の共同体である。
政治や道徳や倫理に関する共通の観念なしに社会は存在しえない。
私たち一人ひとりに善悪に関する観念があり、それらは私たちが生きる社会から切り離すことはできない。
善と悪についての基本的な合意のない社会を創ろうしても必ず失敗することになる。
共通の合意の上に築かれた社会も、その合意が失われれば社会は崩壊するであろう。
社会は物理的に保持されるものではなく、共通の思考という目に見えない絆によって結ばれている。
この絆が緩めば、社会の構成員は互いにばらばらになりさまようことになる。
共通の道徳はこの絆の一部だ。
\ruby{絆の締めつけ}{ボンデージ}は社会が支払う対価の一部だ。
人類は社会を必要としており、この対価を支払わなければならない。
\citep[p.10]{devlin65:_enfor_moral}
\end{quote}

デヴリン\ig{Patrick Devlin}にとって、不道徳に対する法律は国家の存続と同様に不可欠だ。
彼はこう述べる。

\begin{quote}
社会は外部からの圧力によって破壊されるよりも、内側から崩壊する方がはるかに頻繁だ。
共通の道徳規範が守られない時に社会は崩壊する。
そして歴史は、道徳的な\ruby{絆}{ボンド}の緩みが崩壊の第一段階であることを示している。
したがって、社会は道徳規範を保持するために、その統治を維持するために講じる措置と同じ措置をとることが正当化される……悪徳の抑圧は、法の仕事であり、社会転覆的活動の抑圧と同じく必要不可欠だ。
\citep[p.36]{devlin65:_enfor_moral}
\end{quote}

自然法論者とは異なり、デヴリン\ig{Patrick Devlin}にとって社会の道徳の実質的内容は、法律がそれを強制するべきかどうかとは無関係だ。
人々が、その道徳的信条にしたがって、その道徳的信条の侵害に対して十分に強い\ruby{憤り}{オフェンス}や憤激を感じるならば、それだけで法による規制の正当化には十分だとデヴリン\ig{Patrick Devlin}は考える。
彼は「重要なのは信条の内容ではなく、その信条を信じる強さだ」と述べる。
もし社会の大多数がある行為を嫌悪し、それが単なる不快感ではなく社会的な脅威として認識されるならば、その行為を法によって根絶する権利があると彼は主張する。

\begin{quote}
たとえば、同性愛に対しては広い一般の嫌悪感がある。
私たちはまず、冷静かつ客観的に、それを単なる悪徳としてではなく、その存在自体が犯罪と見なされるほどのものかどうかを考えなければならない。
もし、それが私たちが生活している社会の本物の感情であるならば、社会がそれを根絶する権利を否定される理由はない\citep[p.40]{devlin65:_enfor_moral}。
\end{quote}

デヴリン\ig{Patrick Devlin}はこの点を説明するためにポリガミーを例として挙げる。
彼は、「\ruby{一夫多妻制}{ポリガミー}は\ruby{一夫一婦制}{モノガミー}と同じように社会を維持することができるし、また\ruby{自由恋愛}{フリーラブ}と共同育児を基盤とする社会も、核家族を基盤とする社会と同じく強固でありえる(ただし私たちの観念からすれば、それは良いものではない)」と述べる\citep[p.114]{devlin65:_enfor_moral}。
重要なのは、ポリガミーの実践に対して、その社会の構成員がどの程度強く\ruby{不快}{オフェンス}を抱くかという点にある。
このように、デヴリン\ig{Patrick Devlin}は、道徳に関して哲学者たちが「非認知主義」と呼ぶ立場をとっている。
彼は、「あらゆる道徳的判断は、神聖な起源を主張していない限り、単に「正しい精神を持つ者ならば、そうした行為をなすにあたってそれが不正だと認めずにはいられないだろう」という感情にすぎない」と述べている\citep[p.17]{devlin65:_enfor_moral}。

\subsection{デヴリンへの反論:ハートとドウォーキン}

オックスフォードの法哲学者 H.~L.~A. ハートは、デヴリン\ig{Patrick Devlin}に対する反論をラジオ放送で行い、それを\emph{The Listener}に掲載した後、1962年にスタンフォード大学での三つの講義として発展させた。
これらの講義は後に『法、自由、道徳(\emph{Law, Liberty and Morality})』として出版された。
ハートはウォルフェンデン報告書の結論を擁護し、刑法はミルの危害原則の枠内にとどまるべきだと主張した。

ハートは、デヴリン\ig{Patrick Devlin}の議論には二つの主張が混在していると考え、これらを「崩壊テーゼ」と「保守主義テーゼ」と名付けた\citep[p.115]{hart63:_law_liber_and_moral}。

ハートの解釈によれば、崩壊テーゼとは、「社会がその存続を維持するために必ず強制しなければならない道徳的規範がある」とする立場だ。
デヴリン\ig{Patrick Devlin}はこの\ruby{主張}{テーゼ}を明示的には説明していないが、ハートは彼の未解明の前提を推測し、次のように説明する。
「ほとんどの人にとって、道徳とは一つの網のようなものであり、バラバラに存在している個別の信念ではない」\citep[pp.50--51]{hart63:_law_liber_and_moral}。
この考え方によれば、私たちが道徳のほんの一部を揺がせただけで、その大伽藍全体が崩壊する可能性がある。
ハートは、デヴリン\ig{Patrick Devlin}の立場は次のような前提を含んでいると指摘する。
「すべての道徳{\DDASH}殺人や泥棒や嘘などの危害を加える行為を禁じる道徳、および性道徳{\DDASH}は、一つの縫い目のない網のようなものであり、そのいずれかの部分から逸脱する者は、全体から逸脱する可能性が高いか、あるいは必然的に逸脱する」\citep[p.50]{hart63:_law_liber_and_moral}。
この論理に従えば、同性愛セックスを刑事罰の対象から外せば、社会のあらゆる道徳的禁止が疑問視されることになり、より重大な犯罪行為へと人々が進んでしまう可能性がある、ということになる。

ハートはこの崩壊テーゼは経験的に誤っていると批判する。
「この主張を支持するまともな歴史家は一人もおらず、むしろ多くの証拠はそれに反している。
事実命題として、こんな見解はユスティニアヌス帝が「地震の原因は同性愛だ」と述べたのと同じ程度にしか尊重に値しない\citep[p.4]{hart63:_law_liber_and_moral}。
その後の数十年間に蓄積された証拠は、ハートの立場を支持するもののように思われる。デヴリン\ig{Patrick Devlin}の著作以来、すべての西洋諸国において、性的な事柄をはじめとする多くの道徳的問題について人々の価値観は根本的な変化を見た。しかしその結果として、これらの社会のいずれにおいても、実質的な崩壊が見られたわけではない。ほとんどすべての場合において、犯罪が増加したわけでもなく、政治体制が崩壊したわけでもなく、経済も健全に保たれている。

ハートは、デヴリンの議論を読む別の、より擁護可能かもしれない読み方を提示している。ハートの解釈での「保守主義テーゼ」とは、社会がその共同的な生活様式を維持するためには、その価値観を強制しなければならないという立場である。この立場によれば、たとえ社会の価値観が変化したとしても、それによって社会が完全に崩壊するとは限らない。しかし、その社会はもはや「同じ社会」ではなくなってしまう。そして社会は、そうした変化が起こるのを防ぐために行動する正当な権利を持っているというのだ。ハートはこの保守主義テーゼを、「ある集団が共通の道徳を含むほどに豊かな共通の生活様式を形成したとき、それは保持されるべきものである」という信念に根ざしたものとみなしている\citep[p.4]{hart63:_law_liber_and_moral}。
社会は自然の生態系にたとえることができる。
その生態系における主要な生物種の一つが絶滅してしまうと、バランスが崩れ、全体のシステムは元のものとは異なるものになり、古いシステムは失われてしまう。

ハートはまず、なぜ私たちは、道徳が社会においてこのような役割を果たしていると信じるべきなのか、つまり、道徳に変化があるとその社会が別の新しいものなってしまうということを信じるべきなのかを問う。
ハートが見るところでは、デヴリン\ig{Patrick Devlin}は、「共通の道徳がどんな社会にも不可欠である、という受け入れ可能な命題から、ある社会はその歴史の中で特定の時点での道徳と同一であり、したがってある社会の道徳の変化は、その社会の破壊と同一だ、という受け入れがたい命題に移行している」\citep[p.51--52]{hart63:_law_liber_and_moral}。
ハートは、あらゆる社会の道徳は、時間とともに進化していることを指摘している。
では、古い社会は毎瞬ごとに死につつ、新しい社会が毎瞬ごとに生まれているのだろうか? 彼はこれを社会を理解するには奇妙な方法だと考えている。
この社会の絶え間ない死と再生が社会構造に他の形で影響を与えないのであれば、彼は「これは議論に値するほど面白いテーマではないように思える」と言っている\citep[p.3]{hart63:_law_liber_and_moral}。

しかし、ハートは、保守派の主張の前提を受け入れ、道徳が変わる時にはその社会は破壊されると仮定してみようと言う。
それでもさらに、それが悪いことだと証明する必要がある。
ハートは、この議論が暗黙の価値判断に依存していると指摘する。
すなわち、ある社会が成立したら、それはそのままで保存する価値があるという判断だ。
ハートはこの主張を否定し、すべての社会が保存に値するわけではないと主張する。
彼にとって、重要な問いは、新しい社会は、それが置き換えることになる古い社会よりも良いものか悪いものかということだ。

ハートは、道徳の変化が起こるのは、そもそもある社会の道徳が最初から全員一致のものではなかったからだと指摘する。
ある人々は、多数派の見解を疑問視し批判する。
反省と議論の過程を通じて、社会はそれまで支配的だった見解のネットワークから別のものへと移行していく。
社会の絆を損なうことを恐れてこのプロセスを遮断してしまおうとするなら、後の批判者が言うように、「私たちの法律が社会の道徳的な\ruby{合意}{コンセンサス}を強制しているのあって、ただ単に、前の世代の偏見を保護して、新しい立場が\ruby{合意}{コンセンサス}を形成してしまうのを防いでいるわけではないなどと、どうしたらわかるというのだろうか」\citep[p.169]{bix00:_juris}。

デヴリン\ig{Patrick Devlin}はハートの批判に応答しているが、彼の反応は、彼が保守主義テーゼや崩壊テーゼに\ruby{肩入れ}{コミット}しているかどうかを明確にするものではない。
デヴリン\ig{Patrick Devlin}は、ハートが彼の立場を誤って解釈しており、社会はその道徳的基盤のいかなる変化にも耐えられないのだという見解を彼に帰することは誤りだと主張する。
彼はむしろ、社会の存在を脅かすのは重大な変化だけだと示唆している。

\begin{quote}
私は、共有された道徳からの逸脱がその社会の存立を脅かすと主張しているわけではなく、また、反抗的な活動はいかなるものであっても社会の存立を脅かすと主張しているわけでもない。
私は、それらがいずれもその性質において社会の存立を脅かす可能性のある活動であり、どちらも法律の範囲外に置くことはできないと主張している。
たとえば、私は、さらに、ルールのないゲームはないということ、そしてもしルールがなければゲームは存在しないと主張してみたい。
もし私が「それではゲームはルールと同一であるか?」と尋ねられたなら、私はその答がどちらであってもどこか先に繋がるわけではないと信じているので、どちらでもかまわないと答えるだろう。
また、もしルールが変更されたら一つのゲームが消え、別のゲームがその代わりに登場したことになるかと尋ねられたなら、おそらくそうではないと答えるだろうが、その変更の程度によるとも言うだろう。
\citep[p.37]{devlin65:_enfor_moral}
 \end{quote}

デヴリン\ig{Patrick Devlin}が両方のテーゼを信じている可能性はあるが、ハートが考えていたよりもやや穏健な形でそれらを信じているのかもしれない。
必ずということではなくとも、社会の道徳が変化することが社会の崩壊を引き起こす\kenten{可能性はある}と考えており、それゆえさらに国家はそうしたリスクを減らすためにも行動することが正当化されると考えているのかもしれない。
また、社会の道徳の変化が必ずしも古い社会の死を招くわけではないが、もし社会の道徳が大きく変わってしまえば、古い社会は実際に失われてしまっており、そのプロセスが始まるのを防ぐためにも国家が行動することは正当化されると考えている可能性もある。

しかし、二つのテーゼの穏健な解釈も一つの疑問を引き起こす。
もしすべての道徳的変化が社会にとって脅威だというわけではないのであれば、私たちはどの道徳的規範を強制すべきであり、それをどのようにして知るのだろうか? ロナルド・ドウォーキン\ig{Ronald Dworkin}はこの問題を提起している。
彼は次のように言う。
「もし社会がすべての不道徳を立法で禁じるべきだというわけではない、なぜならすべての不道徳な活動や行為が社会の存続を危うくするわけではないからだとしたら、あるケースついて社会が道徳を強制する権利を正当化するために、どのような証拠と行動基準が用いられることになるのだろうか?」\citep{dworkin77:_takin_right_serious}。
驚くべきことではないが、道徳ついての非認知主義的立場をとるデヴリン\ig{Patrick Devlin}は、国家は人々の感情を基にすべきだと述べている。
彼は、国家はその権力を行使すべきなのは、与えられた不道徳な活動に関して、多数派の道徳的感受性がある程度の「不寛容、憤慨、強い嫌悪」といった感情に達した時に限るべきだと考えている\citep[p.17]{devlin65:_enfor_moral}。
これらの強い感情は、その道徳的見解が社会の統合性に深く結びついていると信じられているしるしであるように思われる。

ドウォーキン\ig{Ronald Dworkin}は、デヴリン\ig{Patrick Devlin}の\ruby{道徳主義}{モラリズム}は、本質的には無制的な民主主義の一形態だと考えている。
彼は、デヴリン\ig{Patrick Devlin}にとっては「最終的には、決定は何らかの道徳的信念に基づくべきであり、民主主義においてはこの種の問題は民主的原則に従って解決されるべきだということになる」と述べている\citep[pp.246--247]{dworkin77:_takin_right_serious}。
実際、裁判所は道徳主義的な法律を制定する立法府の権利を正当化する際に、しばしば民主主義における市民の特権に訴えてきた。
テネシー州の反ソドミー法に対する州憲法違反裁判 \emph{Campbell v. Sundquist} (1996)の控訴裁判所の判決では、テネシー州知事(サンドクィスト)の指導の下、控訴人たちは次のように主張した。

\begin{quote}
同性愛行為を犯罪とすることによって、この州の市民は選出された代表者を通じて、同性愛の実践は\ruby{不快}{オフェンシブ}であり、自分たちの道徳的基準に反していると見なしていることを示している。
これらの基準が宗教的信念に基づいているか、世俗的な道徳哲学の体系から導かれているかは問題ではない。
控訴人たちは、州の法律が憲法上、市民の道徳的価値観や基準を反映し、これらの道徳的価値観や基準に反する行動を禁止することができるということは自明だと主張している。
(\emph{Campbell v. Sundquist}\ig{, 926 SW 2d 250, Tenn: Court of Appeals, Western Section 1996})
\end{quote}

この判決ではその論理は認められなかったが、\emph{Lawrence}判決後の\emph{Reliable Consultants v. Earle}事件では、同様の論理が支持された。
この事件の判決で裁判所は次のように述べている。
「刑法体系を確立するためには、立法府は特定の道徳的選択を行い、どの行動が正しく、どの行動が間違っているかを定義しなければならない。
憲法上の禁止がない限り、それらの決定は、過半数の道徳的判断を表現するものであり、またそうあるべきだ」(\emph{Reliable Consultants, Inc. v. Earle}\ig{, 517 F.3d 738, 2008})。

しかし、ドウォーキン\ig{Ronald Dworkin}は、この種の議論は、無制約な民主的権力に関するよく知られた問題、すなわち、多数派が任意の少数派を抑圧することを許してしまうという問題を生じさせる。
私たちが基本的権利を認めているのはまさにこのような専制政治に対抗するためであり、プライバシーの権利は、私たちの性的選択を多数派の判断の及ばないものとすると考えられている。

\subsection{デヴリン以後のリーガルモラリズム擁護論}

ハートとドウォーキン\ig{Ronald Dworkin}の批判により、リーガルモラリズムは致命的な誤りだと考える哲学者が多いが、それでもなお、この立場を擁護する人々は裁判官や立法者の間でけなく、哲学者のなかにも存在している。

先に説明したように、従来の見解では、ある行為が内在的に不正だという理由で犯罪とすることが許されると考えられてきた。
しかし、別の立場から、もし原則としてリーガルモラリズムを否定してしまうならば、私たちは次第に坂道を下りはじめることになり、最終的には私たちが維持したいと願うような他の法律をも放棄せざるをえないことにつながると主張する人々がいる。
2006年の\emph{Williams v. King}事件において、アメリカ合衆国の連邦控訴裁判所は、アラバマ州の「\ruby{わいせつ}{オブシーン}器具」(つまりセックストイ)販売禁止法に対する異議申し立てに関して、次のような判決を下した。
「法はしばしば社会の道徳的基準に基づいている。
事実、スカリア判事が示唆しているように、もし\emph{Lawrence}判決の影響を広範に解釈するべきであるならば、我々の刑法体系のほぼすべてが無効とされることになるだろう。
それは、刑法が「\ruby{正当な}{ライト}」行動と「\ruby{不正な}{ロング}」行動に関する社会的な概念に基づいているからだ」(\emph{Williams v. King}, 28 U.S.C. 636(c)(1), 2017)。
また、2012年のインタビューにおいて、スカリア判事は\emph{Lawrence}判決の意義について次のように述べている。
「もし私たちが同性愛に反対する道徳的感情を持つことが許されないならば、殺人に対する道徳的感情を持つことは許されるのか? 他のことについてはどうなのか?」\citep{sink12:_justic_scalia_defen_compar_laws_homos_murder}。

法体系において道徳が何らかの役割を果たすべきであることを否定する法哲学者はほとんどいない。
ミルに従い、違法行為を罰しようとする場合にはそれが何らかの危害を伴うということを前提とするにせよ、すべての有害な行為が違法とされるわけではない。
たとえば、誰かの外見を侮辱することは違法ではないが、それは違法ではあるが軽微な身体的暴行よりも大きな精神的苦痛を与える可能性がある。
私たちは、ある種の危害を他のものより気にかけるのはなぜなのか、ということに答えねばならない。
また私たちは、一つの社会として、さまざまな犯罪に対してそれぞれの罰の重さを判断しなければならない。
たとえば、違法ダウンロードは店舗での万引きより重い犯罪と見なされるべきだろうか。
こうした判断は、ある程度は、道徳的判断を反映せざるをえない。

しかし、私的なセックス行為は、一般的に第三者に対して何の害も及ぼさないという点で他のほとんどの違法行為とは異なっている。
その違法化の正当化は、多くの場合、\kenten{完全に}道徳的なものであり、暴行、殺人、窃盗に対する法律の根拠とは異なる。
主張されている害悪は、社会の道徳という\ruby{織物}{ファブリック}に対するものだ{\DDASH}しかし、個人の私的な行動が社会の道徳という織物を損なうということがありえるのかどうか、そして仮にそうしたことがありえるとしても、それがいかにしてなされるのか、ということについては、説得力のある説明はなされていない。
リーガルモラリズムに反対する人々は、法律の制定や施行において道徳的要素を完全に排除できると主張しているわけではない。
むしろ、彼らが否定しているのは、単に、特定の活動が不道徳と見なされているという理由だけで、それを禁止することが許されるという発想だ。
他者に対する相当程度の危害の考慮なしに、単に道徳的な非難を根拠として禁止することはできないというのが彼らの立場だ。
スカリアが殺人について述べた発言(それがその場の即興のものであったことは認めるとしても)は、明白な誤謬を含んでいる。
彼は、ある種類の行動を犯罪とする決定には道徳的な非難が必要であるように思われるということから、道徳的非難があればどのような行動も犯罪化することが正当化されると主張してしまっている。

すべての過去の道徳的法律を擁護するわけではないが、少なくとも一部の行為については、道徳的理由によって刑罰を科すことが正当化されるべきだと主張する哲学者もいる。
特に、人間の尊厳の極端な侵害に該当する行為は、たとえ危害原則に厳密に当てはまらなくとも違法化すべきだとする立場がある。
ロビン・ダフは次のように述べる。

\begin{quote}
なぜ自律の侵害が、たとえそれが危害原則に反しないとしても犯罪として扱われるべきなのかを説明するためには、それが被害者の人間性の否定である、あるいは深刻な軽視であるということを理解する必要がある(さもなければ、自律がなぜそれほど重要であるべきかを理解できない)。
しかし、もし、カント的な人間性の概念{\DDASH}すなわち、私たちを形式的に理性的な存在としてのみ捉える自律中心の考え{\DDASH}の不十分さを認識して、私たちが社会的で、身体を持ち、感情を伴う存在であることを考慮に入れた、より豊かな概念を発展させるならば、人間性を否定したり、根本的にそれを軽視したりするやり方は、自律の侵害だけに限られないことがわかるだろう。
したがって、自律の侵害を犯罪とする理由と同じ種類の理由に基づいて、人間性を否定し、または根本的に軽視するその他の行為も犯罪として扱うべき十分な理由があることが理解できるはずだ。
\citep[p.43]{duff01:_harms_wrong}
\end{quote}

この立場をとる人々は、私的行為ではあるが、人間の尊厳の\ruby{侵害}{バイオレーション}として犯罪化すべき例として、次のようなぞっとする例を挙げる。
志願奴隷制、剣闘士競技、「小人投げ」競争、同意の上での\ruby{人肉食}{カニバリズム}、死体の冒涜などだ。
ダフは、たとえば剣闘士競技の禁止を次のように正当化する。
「この競技において、剣闘士たちがお互いに対しておこなう非人間化や\ruby{侮辱}{デグレデーション}、また観客が剣闘士たちに対して、そして自分たちに対しておこなう非人間化や侮辱が、禁止の理由となる」(Duff, 2001, p.39. Dan-Cohen, 2000も参照)。
ヴェラ・バーゲルソンは、こうした行為に対する個々人の同意は問題に無関係であり、人間の尊厳の概念は「行為者や被害者の主観的な精神状態とは関係のない、「客観的な」意味をもつ」と主張している\citep[p.217]{bergelson07:_right_be_hurt}。

しかし、たとえ道徳主義的な法律をこのように正当化できるとしても、人々の私的な性的行動を規制する範囲は限定されるだろう。
たとえば、なんらかの私的な性的行動が、剣闘士競技のように、個人の尊厳を侵害する形でおこなわれうるかどうかは明確ではない。
本書2.5.2節で見たように、BDSMの特定の形態を違法とすべきだと考える人々もいる。
それは、その行為が参加者を\ruby{貶める}{デグレード}とされるからだ。
しかし、彼らは、こうした行為が同意に基づいている場合に、どのようにして尊厳の侵害として扱われ、同意を無効化するに足る根拠となるのかを示さなければならない。

この議論には、より深い問題がある。
それは、もし私たちが、関与する当事者の合意とは無関係に客観的な「尊厳」という概念が存在すると考えるなら、その概念をどのように特定し、正当化するのかという問題に直面するという問題だ。
メイア・ダン=コーエンは、「尊厳」の意味は共同体によって定義されるべきだと主張している(Dan-Cohen, 2000, p.774; cf. Scoccia, 2013)。
\nocite{scoccia13:_in_defen_pure_legal_moral}\nocite{dan-cohen00:_basic_values_victim_state_mind}
しかし、この考え方は単にデヴリン\ig{Patrick Devlin}の立場に戻ることになり、ハートが指摘した「実証的道徳」と「批判的道徳」の区別を無視することになるように思われる。

一部の法哲学者は、リーガルモラリズムはいまだに重要であり目立たない形で機能していると主張している。
たとえば、どの行為を犯罪とし、どの行為を犯罪としないかという問題を脇に置いたとしても、道徳的判断は特定の犯罪にどの程度の刑罰を課すか、また個々の犯罪者をどのように量刑するかを決定する上で重要な役割を果たしている可能性がある\citep[cf.][]{ristroph11:_third_wave_legal_moral}。

これらの議論は興味深いものではあるが、性的行動の規制に関する問題の範囲を超えるため、本書では取り扱わないことにする。

\subsection{セックスする権利はあるか?}

リーガルモラリズムがリベラル民主主義という概念と衝突すると考えることにしても、一部の論者は、単に道徳主義的な法律を拒否するだけでなく、さらに踏み込んで「セックスの権利」(a right to sex)を基本的権利の一つとして認めるべきだと主張する。
セクシュアルライツ(sexual rights、性的権利)という概念は少なくとも1970年代にまで遡るものだが、1990年代になってから広く議論されるようになった。
この用語は、1994年にエジプトのカイロで開催された国際人口開発会議(ICPD)の参加者たちによって初めて提起された。
しかし、一部の国々はこの概念はあまりに論争的だとして反対したため、公式文書には盛り込まれなかった\citep{correa07:_global_persp_sexual_right}。
1997年には、第13回世界性科学会議において、世界セクシュアルヘルス協会(WASH)が「セクシュアルライツ」に関するバレンシア宣言を発表した。
その中で次のように述べられている。
「\ruby{自己性愛}{オートエロティシズム}を含む、性的快楽は、身体的、心理的、知的、そして精神的な\ruby{健康}{ウェルビーイング}の源だ。
性的快楽は葛藤や不安のない性的経験と結びついているものであり、それゆえ社会の発展および個人の成長を可能にするものだ」\citep{WASH97:_valen_declar_sexual_right}。
このWASHの宣言はその後も拡張され、最新版は2014年に発表されている。
また他の国際機関もこれに類似した宣言を採用している。

セックスの権利の概念は、哲学者によってはまだ十分に展開されていないが、WASHの宣言に従えば、ヨーロッパや北アメリカでは確固たる地位を確保したプライバシーの権利をさらに超える側面を持つと考えられる。
それには、包括的な性教育の保障、性的な健康および生殖健康サービスへのアクセス、性的指向やジェンダー自認に基づく差別からの保護が含まれる可能性がある。
また、刑務所や精神病院などの施設に収容されている人々のセックスへのアクセスの権利も含まれる可能性がある。
このセックスの権利の問題は障害者運動の中でも議論されており、障害者が性的に活動するために必要な支援へのアクセスを確保する権利をもつべきだと主張されている。
さらには、\ruby{代理}{サロゲート}セックスへのアクセスの権利を含む可能性もある(Vehmas, 2019; Liberman, 2018)。
\nocite{vehmas19:_person_profoun_intel_disab_their_right_sex}\nocite{liberman18:_disab_sex_right_scope_sexual_exclus}

すでに確立された人権のリストに新たな権利を追加することにはごく慎重な姿勢をとるべきだと考える人は多い。
現在存在する人権関連文書は、いまだ解釈と展開の過程にあるとはいえ、長く困難な交渉の末に形成されたものであり、それゆえ、それらをあくまで文言通りに尊重すべきであり、改定の可能性をめぐって終わりのない議論にさらすべきではないというもっともな理由がある。
しかも、すでに広く承認されている基本的な人権を守るだけでも、依然として困難がつきまとっているのが現実である。
マイケル・イグナティエフは「権利のインフレ」を警戒し、次のように述べている。
「何でも望ましいものを権利として定義しようとする傾向がある。
しかし、このようなインフレは、正当性のある基本的権利の信頼性を損なうことになりかねない」\citep[p.90]{ignatieff03:_human_right_polit_idolat}。

「セックスの権利」(right to sex)という考え方は、多くの人にとって、このような「権利のインフレ」の極端な例として映るかもしれない。
確かに、ほとんどの人にとってセックスは楽しく望ましいものである。セックスの生殖機能は人類の存続にとって不可欠だ。
しかし、だからといって、快楽をもたらしてくれるものの範囲全体を考えた場合、セックスは人間の基本的ニーズの一つに数えられるべきなのだろうか?  多くの人々が強く望んでいるものは他にも存在する。
たとえばスポーツ観戦は広く楽しまれているが、それを普遍的な権利として認めるべきだと主張する人はほとんどいない。
また、セックスの必要性が普遍的かつ不可避だとは必ずしも言えない。
多くの人は長期間、あるいは生涯にわたってセックスをせずに過ごしながらも、個人の幸福に壊滅的な影響を受けることなく生活している。
さらに、自分を\ruby{無性愛者}{アセクシュアル}と定義する人々もおり、アセクシュアリティ(無性愛)は現在、正常で正当な性的アイデンティティとして承認されている。

セックスの権利を擁護する人々は、セックスが他の快楽をもたらす活動とは異なり、「アイデンティティに不可欠なもの」(identity integral)だと主張する。
つまり、ほとんどの人にとってセックスは単なる行動ではなく、自分自身の一部として認識されており、\ruby{人}{パーソン}としてのアイデンティティの形成と表現において重要な役割を果たしているということだ。
この点で、セックスは単なる娯楽活動(たとえばスポーツ観戦)よりも、むしろ一般に保護の対象として承認されている宗教的アイデンティティに近い。
このような人間のアイデンティティの根本的な側面に関しては、単に寛容の対象とされるだけでなく、承認の対象とされるべきだと考えられている。
すなわち、それらが個人の私的な行動を決定するものとして認められるだけでなく、公に誇りを持つことが許されるべきだとされる。

最近では、「セックスの権利」の承認が、セックスを得られない者、特に一部の男性が、それを提供される権利をもっていると主張することにつながるのではないか、さらには極端な場合には、女性への性的アクセスが均等に分配されるべきだという主張を生むのではないかと懸念されている。
経済学者ロビン・ハンソンは、現在では悪名高いブログ記事の中で次のように述べている。

\begin{quote}
一部の人々は、性的アクセスが極端に少ない人々は、低所得者が経験するのと同程度の苦しみを味わっているともっともらしく主張するかもしれない。
さらに彼らはこうしたアイデンティティのもとに結集して、ロビー活動を通じてこの軸に沿った再分配を求め、要求が満たされない場合には、少なくとも暗に、暴力の脅しをちらつかせようとするかもしれない。
\citep{hanson18:_two_types_envy}
\end{quote}

ハンソン自身はリバタリアン(自由至上主義者)だ。
彼はこうした議論を「性の再分配」の擁護としてではなく、再分配政策全般の論理を批判するための背理法として提起している。
しかし、一部では、セックスの権利の承認が、こうした「セックスへのアクセスの均等な分配」につながるのではないかと懸念する声もある。

仮に上の議論が、その論理において説得力をもつにせよ、あるいは背理法として機能するにせよ、こうした議論は次のような前提に依存している。
すなわち、個人がある権利をもつということは、常に、政府や他の市民が、その権利が保護しようとするものをなんでも提供する積極的な義務を負うことを意味する、という命題に依存している。
しかし、私たちは権利を常にそのようなものとして理解しているわけではないし、またそう理解する必要もない。
たとえば、あなたの「表現の自由」の権利は、特定の誰かが強制的にあなたの話を聞かなければならないことを意味しない。
多くの人々は結婚を基本的権利とみなすことにためらいはないが(本書4.2.1節参照)、それによって誰もが配偶者を得る権利をもつと考えるわけではないし、ましてや特定の個人が誰かと結婚することを強制されるべきだとは誰も思わない。
さらに、すべての権利は、いったんそれが承認されたならば、国家によるなんらかの強制的な再分配が許容されることを意味するなどということを私たちが一律に考えているわけではない。
実際のところ、セックスの強制的な再分配は、市民の性的自由の権利の否定に他ならない。
そしてそもそも性的自由の権利は、個人の性的自律を保護するためにこそ存在しているのだ。

セックスの権利に関する議論はまだ初期段階にある。
この概念の是非や影響を検討し、その実施のあり方を考え始める前に、まずはこの概念をより詳細かつ慎重に整理することが必要だ。
しかし、「セックスの権利」はたしかに興味深い概念であり、近い将来さらにさまざまな議論を耳にすることになるだろう。

\subsection{本節のまとめ:セックスと法}

私的な性的行動を規制する法律が一貫性や統一性を持っていた時期は存在しない。
たとえば、同性間のセックスを禁じる法律は、さまざまな時代と法域において制定されたり撤廃されたりしてきた。
そして、法が施行されていた場合でも、その適用法は一貫性を欠くことが多かった。
商業的なセックスを規制する法律についても同様だ。
だが、リーガルモラリズムという基本原則は長らく異議を唱えられることなく受け入れられてきた。

現在では、リベラルな法制度を持つ国々ではリーガルモラリズムは後退しつつある。
とはいえ、道徳主義的な法律群が完全になくなったわけではない。
たとえば、成人どうしの同意に基づく近親姦を犯罪とする法律は依然として存在している。
本節では、単に道徳主義的な法律を拒絶するだけでなく、より広範な「性的自由の権利」を承認するべきだと主張する立場があることを示した。
なにかが基本的権利だと一方的に宣言できるような単一の機関は存在しない。
基本的権利の承認には、市民と立法者と裁判官の合意が必要だ。
立法者がそれを法典に組み込み、裁判官が判決の指針をもち、市民がそれを価値観の一部として受け入れる必要がある。
性的自由の権利という考え方は比較的新しく、その普遍的な承認が得られるまでには長い時間がかかるだろう。
だが、そのプロセスは、その妥当性を議論することから始めなければならない。
本節はその議論の一助となることを目指した。

現在、刑法が市民のベッドルームに立ち入ることはほとんどなくなった。
しかし、重要な例外が一つある。
それは商業的なセックスだ。
一部の法域ではポルノグラフィを規制または禁止する法律が存在する。
そして、さらに多くの法域ではセックスワークを犯罪化する法律が施行されている。
次章では、これらの問題をそれぞれ詳しく検討する。

\section{ポルノグラフィ}

今日、世界には膨大な量のポルノグラフィ(ポルノ)が存在する。
かつて不足していたわけではないが、インターネット時代に入ってから、その流通とアクセス可能性はかつてないほど広がった。
ミュージカル『アベニューQ』の毛むくじゃらのモンスターの一匹は次のように歌う\citep{dabruzzo03:_inter_is_porn}。

\begin{quote}

  \begin{verse}
    インターネットはポルノのため!\\インターネットはポルノのため!\\ネットはどうして生まれたか?\\ポルノ!ポルノ!ポルノ!
  \end{verse}
\end{quote}

これほど広く簡単にアクセスできる状況では、ポルノ規制は無意味に思えるかもしれない。
しかし、それでも政府は規制を試み続けている。
世界にはポルノを違法とする国が多数存在する。
アメリカでは、それを制限する法律はほとんどないが、規制強化を求める声は少なくない。
2016年の共和党の政策綱領には、「ポルノグラフィは、その有害な影響、とりわけ子供への影響を通じて、公衆衛生上の危機となり、数百万の人々の人生を破壊している」と明記された\citep{obrien16:_south_carol_lawmak_propos_pornog}。
また、複数の州がアクセス制限の試みをおこなっている\citep{brown17:_hypoc_logic_behin_repub_plans}。
人々をポルノグラフィから完全に遮断することは難しいかもしれないが、政府当局がアクセスを困難にしたり、制作や流通にかかるコストを増加させたりすることは可能だ。
たとえばイギリスでは、ポルノサイトに年齢認証を義務付ける法律を制定した。
また、FacebookやAppleといった大手テック企業は、自社のプラットフォーム上で明示的なコンテンツを排除する措置を講じている(この点については 6.1 で詳しく論じる)。

政府や政治家は、市民の要求に応える形で規制を進めている。
したがって、私たちのポルノグラフィに対する態度は重要だ。
ポルノグラフィが道徳的に問題のあるものなのかどうか、そしてその結論がその制作・流通を規制する法律や規制にどのような影響を及ぼすべきかについて、私たちは倫理的な判断を下さなければならない。

\subsection{定義}

ポルノグラフィの道徳性を議論するのであれば、まずその定義を明確にしなければならない。
しかし、これは非常に困難な課題であり、研究者や裁判所の間で統一された定義が確立されたことはない。
したがって、ポルノグラフィに関する討論は、その言葉の意味をめぐる問題から始まる。

最低限の定義として、ポルノグラフィは写真、映像、文章などの何らかの形で作られた性的に露骨な作品だ。
しかし、これは必要条件ではあるが、十分条件ではない。
すべての性的に露骨な\ruby{表現物}{マテリアル}がポルノグラフィであるわけではない。
たとえば、性教育のビデオは露骨な内容を含むことがあるし、グラフィックな性的表現を含む芸術作品も多い。
ジェームズ・ジョイスの『ユリシーズ』のような作品はかつてはポルノグラフィまたは\ruby{わいせつ}{オブシーン}なものと見なされたが、今日では単に露骨な内容を含むという理由でポルノグラフィと分類する人はほとんどいない。

ポルノグラフィを特別なものとする要素は、それが視聴者や読者を性的に興奮させることを意図している点かもしれない。
ジョエル・ファインバーグはポルノグラフィを「完全に、かつもっともらしく、読者または観察者に性的興奮を引き起こすように設計された性的に露骨な文章や画像」と定義している\citep[p.127]{feinberg85:_offen_to_other}。
この定義はある程度の精度を持つが、いくつかの問題点もある。
たとえば、明確に人を性的に興奮させることを目的とした芸術作品も存在する。
短編集『清潔さ』\emph{Cleanness}の著者であるガース・グリーンウェルは、自分の作品が「100%ポルノグラフィであり、同時に100%高尚な芸術だ」と述べている\citep{barone20:_garth_green_comes_clean}。
ファインバーグは「完全に」という修飾語を用いることで、芸術作品をポルノグラフィの定義から除外できると考えたのかもしれない。
芸術作品は通常、視聴者に他の効果をもたらすことも意図しているからだ。
しかし、制作者のなかにも、ポルノグラフィは単に人を興奮させること以外の目的を持つと主張する者がいる。
たとえば、アダルト映画の出演者であるエラ・ダーリン\ig{Ella Darling}は、自身のビデオについて「視聴者に相互的な愛情を提供し、単なる官能的なレベルだけでなく、感情的・知的レベルでもパフォーマーとつながる手助けをすることを目指している」と述べている\citep{bell19:_women_are_leadin_porns_lates_reinv}。
\ig{Ella Darling}

ロバート・ジェンセンは、ポルノグラフィであるかどうかを、その制作・流通・消費の文脈に基づいて判断することを提案している。
彼はポルノグラフィを「アダルトエンターテインメント企業によって制作され、ポルノグラフィとして販売され、ポルノグラフィサイトで公開され、ポルノグラフィを求める個人によって消費される作品」と定義する\citep[p.53]{jensen07:gettingoff}。
しかし、今日では多くのポルノグラフィは業界外で制作されており、アマチュアによって作られさまざまな手段で配信されているため、この定義も問題を含んでいる。

一部の論者は、単なる記述的な定義ではなく規範的な定義を提案している。
彼らは、ポルノグラフィを単に「性的に露骨な素材」とするのではなく、「搾取的または女性差別的な性描写を含む素材」とみなすべきだと主張する。
このような定義には、性教育や芸術作品をポルノグラフィから除外できるという利点がある。
しかし、これにはいくつかの問題点もある。
まず、「ポルノグラフィ」を「暴力的・貶め的・非人間的なもの」と定義すると、一般的な用語の使い方とは乖離する。
多くの人は、暴力や貶め的な扱いを含まなくても、アダルトサイトに投稿された性的に露骨な作品をポルノグラフィと認識するだろう。
さらに、「\ruby{貶め的}{デグレーディング}」や「\ruby{非人間化する}{デヒューマナイズ}」といった概念の定義は主観的かつ文脈依存的であり、「暴力」についても同様だ。
多くのポルノビデオはラフセックスやBDSMを含む。ラフセックスはすべて暴力的なのだろうか。
BDSMは間違いなく暴力的だが、それだけで自動的に問題視すべきなのか。

アメリカ合衆国最高裁のポッター・スチュワート判事は、ポルノグラフィの合法性に関する判決において、有名な言葉を残している。
「今日、私は「ハードコアポルノグラフィ」という語に含まれる素材の種類をこれ以上定義しようとは思わないし、知的に定義することができるかどうかもわからない。
しかし、それは見ればわかる」(\emph{Jacobellis v. Ohio}, p.197, スチュワート判事の賛同意見)。
この定義は明らかに場当たり的なものであるが、多くの人が抱く感覚を捉えている。
つまり、正確に定義することはできなくとも、何がポルノグラフィであるかは直感的に理解できるという考え方だ。
少なくとも、大多数の人々がポルノグラフィの典型的なケースについては合意するだろう。
それは、性的に露骨であり、興奮を誘発することを目的とし、主として同様の素材を扱う流通経路を通じて配信されるものだ。
境界線上にあるケースも多々あるが、この作業定義を採用することで、ポルノグラフィの自由な流通と消費に関する議論を進めることができる。

\subsection{ポルノグラフィは保護されるべき表現か?}

リベラルな民主主義国家は、一般的に市民に表現の自由を認めている。
私たちは、他者が不快または動揺する可能性のある絵画、映画、文章などを消費し、制作する権利をもつ。
この原則は、ヴォルテールの有名な言葉「私はあなたの意見には反対だが、あなたがそれを述べる権利は命をかけて守る」によって象徴されている。
ポルノグラフィが表現の一形態であるとみなされる限り、リベラル派は通常、それを抑制しようとする側に立証責任を課す。

表現の自由を保護する理由は、国家が市民を自律的な存在として認識し、彼らが独自の意見を形成し、自らの人生を決定する能力を持つことを尊重するからだ。
ロナルド・ドウォーキン\ig{Ronald Dworkin}は、カナダ最高裁が \emph{R. v. Butler} 事件において、政府が暴力的または女性を\ruby{貶める}{デグレード}ポルノグラフィに法的規制を課すことができると判断したことに対し、リベラルな言論の自由の擁護論を展開している。

\begin{quote}

あらゆる論争的な思想は、誰かの自尊心に悪影響を及ぼす可能性を持つ……。
ある公式の判断が、特定の作品を読むことで人々の性格が向上するか、あるいは堕落するか、または社会問題に関する誤った見解を持つようになるかを決定し、それに基づいて何を読むことができるかを国家が指示することは、市民を責任ある主体として尊重することと明らかに矛盾する。
\citep[pp.206--208]{dworkin96:_freed_law}

\end{quote}

一方、ポルノグラフィの反対派は、すべての言論が同等の保護に値するわけではないと主張する。
彼らは、表現の自由が保護されるのは、それが特定の重要な利益に資する場合であり、その利益が関係しない場合には、正当な理由があれば規制が許容される可能性があると考える。

哲学者たちは、表現の自由が保護される理由として、自己表現、真理の探求、健全な民主主義の機能など、いくつかの重要な利益を挙げている\citep{scanlon11:_why_not_base_free_speec_auton_democ}。

ポルノグラフィの反対派は、その制作および消費がこれらのいずれの利益にも関与しないと主張する。
したがって、ポルノグラフィは「低価値」の言論として分類できるため、他の社会的目的のために制限することが許容されるという。
アメリカ合衆国最高裁は \emph{Chaplinsky v. New Hampshire} 事件において、「粗野で\ruby{わいせつ}{オブシーン}な」表現は憲法修正第1条によって保護されないと判断した。
その理由として、裁判所は次のように述べている。
「このような表現は、思想の提示にとって本質的な部分を成すものではなく、真理への歩みの一段階としてほとんど社会的な価値を持たない。
仮にそこから得られる利益があるとしても、それに対しては秩序と道徳に対する社会的利益の方が明らかに優越するものである」(\emph{Chaplinsky v. New Hampshire}, 315 U.S. 568, 1942, at 572.)。

しかし、たとえ「低価値」の言論であっても、正当な理由なしに制限することは許されない。
ポルノグラフィの反対派は、ポルノグラフィの規制が正当化されるべき理由をいくつか提示している。

\subsection{出演者に対する害}

ポルノグラフィの反対派は、ポルノはその制作過程において直接的な害を引き起こすと主張する。
彼らは、ポルノに出演する人々がしばしば自発的に同意していないか、またはその同意が有効ではないと指摘する。
合理的な成人の同意が無効だと考えられる理由については前述したが、これには同意が強要された場合や、法定の同意年齢に達していない場合が含まれる。
ポルノの制作において、これらの状況が当てはまるケースが存在する。
歴史上最も成功したポルノ映画の一つに出演したリンダ・ラヴレースは、彼女が強制的に演技させられたと述べている。
彼女は次のように証言している。
「『ディープ・スロート』を観るすべての人が、私がレイプされているのを観ているのです」\citep{bailey05:_insid_deep_throat}。
ラヴレースは、例外を除いてすべての映画に出演した時点で未成年であった(これらの映画はその結果として現在では配信停止となっている)。

出演者が成人であり、出演に同意している場合でも、その同意を有効とみなすべきではないとポルノ反対派は主張する。
まず、出演者の経済的および感情的な脆弱性が、彼らの同意が本当に自由なものであるかを疑問視させる。
ドキュメンタリー\emph{Hot Girls Wanted}は、フロリダ州のアマチュアポルノ業界に属する女性たちを描いている。
ジャーナリストのメアリー・ローズ・ソマリバは、このドキュメンタリーで取り上げられた女性たちについて次のように述べている。
「彼女たちは深刻な経済的困難に直面し、選択肢が限られていると感じていた。
いったん業界に入ると、その選択肢はさらに狭まる\citep{somarriba15:_porn_indus_is_abusiv_these}。
ロバート・ストラーとI. S. レヴィンは1990年代に出演者へのインタビューを行い、彼らのほぼ全員が経済的選択肢の制限や過去の性的虐待の経験を報告していた\citep{stoller93:_comin_attrac}。

第二に、ポルノの反対派は、出演者がしばしば事前に撮影の実態について欺かれていて、撮影現場で、事前には同意していなかった行為をおこなうように圧力をかけられ強要されることがあると指摘する。
ソマリバは次のように述べている。

\begin{quote}
多くのケースで、女性たちは実際の現場とは異なる説明を受けていた。
彼女たちは性的な行為に対して不快感を抱きながらもそれをするように圧力をかけられた。
熟練した\ruby{心理操作者}{マニピュレーター}のもと、彼女たちがノーを言えるような状態になる前に状況は急速に進行させられた。
\citep{somarriba15:_porn_indus_is_abusiv_these}
\end{quote}

ドキュメンタリーに描かれている撮影現場では、事前に合意された条件が、撮影当日にプロデューサーによって変更されることがしばしばある{\DDASH}俳優たちがすでに撮影場所に到着し、ノーを言うことが難しくなった状況になってからだ。
「怖かった」と、俳優のレイチェルは過酷なシーンの後にルームメイトへ語っている。
「彼に「ノー」と言えるなんて知らなかったし、すでに15分間は撮影していたのに、そこから逃げ出すことなんてできるのか……そうしたらとどうなるの?、と。その時私は、レイプ被害者たちが感じることを理解したのよ」(ibid.)。
パフォーマーのミリアム・ウィークスは、スクリーン上で殴打され、首を絞められた自身の初めてのポルノ撮影について次のように語っている。
「本当に、本当に、本当にラフなシーンだった。あれほどラフだとは思ってもみなかった」(ibid.)。

セックス産業と同様に、ポルノの反対派は、出演者が長期的なトラウマを経験すると主張する。
その一例として、2017年末から2018年初めにかけて、わずか3か月の間に5人の著名なポルノ出演者が自殺または薬物過剰摂取で死亡した事件がある。
ある研究では、成人女性ポルノ出演者のうつ病率が、同年代の女性の2倍以上であることが示された\citep{grudzen01:_compar_mental_healt_femal_adult}。
健康リスクも存在する。
カリフォルニア州ではコンドームの使用が義務付けられているが、無防備な性行為はポルノ業界で依然として蔓延しており、出演者は性感染症\ig{(STI)}の危険にさらされている\citep{coyne09:_sexual_healt_adult_workin_pornog_films}。

業界はほとんど規制されておらず、規制が導入されると、プロデューサーは自由に活動できる地域へ移動するため、規制の実効性が低下する。
カリフォルニア州のコンドーム義務化法の施行後、多くのプロダクションが州外へ移転したのがその例だ。
また、収益の減少と競争の激化に伴い、出演者の環境は悪化している。
引退した女優ルビーは\emph{Rolling Stone}にこう語っている。
「昔は自分が働きたい相手ややりたいことを自由に選ぶことができました。
でも今はそうはいきません。
今では、仕事を得るためには非常に過酷なシーンをこなさなければなりません。
それがメンタルにひどい負担を与えているはずです。
でもそうしなければならないんです」\citep{swann18:_is_porn_indus_doing_enoug}。

ポルノに関連する社会的スティグマは、出演者が業界を離れるのを困難にする。
女優のジュールズ・ジェイドは、ポルノ出演者が一般の仕事を見つけるのが難しいことについて次のように述べている。
「普通の仕事に就くことはできません。
有名になればなるほど、誰も雇ってくれなくなるんです。
まるで罠にかかったみたい」(ibid.)。
今後、顔認識技術が発展することで、出演者が過去の経歴を隠すことはさらに困難になる可能性がある。
インターネット上に公開されたポルノに出演したすべての人が、簡単に特定されるようになるかもしれない。

オンラインポルノグラフィーに登場する多くの人々は、そもそも出演に同意していない。
最近の報道によって、ポルノの流通がごく少数の企業、特に倫理基準を守らずに動画を制作・配信する企業によって独占されている実態が明らかになった。
モントリオールに拠点を置く秘密主義的な企業であるMindGeekは、Pornhub、RedTube、YouPorn、xTube、Brazzersといった世界最大級のポルノサイトを所有している。
2020年12月、\emph{The New York Times}のコラムニストのニコラス・クリストフは、「Pornhubの子供たち:なぜカナダはこの企業が搾取や暴力の動画で利益を得ることを許しているのか?」
と題する記事を発表した\citep{kristof20:_child_pornh}。
クリストフは、MindGeekのサイトには、未成年者が出演している動画、出演者の同意なしに投稿された動画、さらには非同意のセックス行為を記録した動画が数千本も掲載されていると主張した。

その後の抗議によって、MindGeekのビジネスモデルが著作権者の許可なく投稿されたコンテンツに大きく依存している実態も明るみに出た。
「彼らは海賊行為を前提としたビジネスモデルで市場に参入し、業界を完全に破壊し、多くの制作スタジオやパフォーマーを廃業に追い込んだ」と、監督兼プロデューサーのエリカ・ラストはMindGeekについて語っている。
彼女は、自身の動画が無料サイトに掲載されるたびに、MindGeekに削除を求めるリクエストを「毎週、場合によっては毎日のように」送らざるを得ないという\citep{nilsson20:_secret_world_mindg}。
MindGeekは一企業にすぎないが、業界を支配しており、こうした搾取的な手法を追求しているのはけっして同社だけではない。

\subsection{視聴者と社会全体に対する害}

ポルノの反対派は、ポルノへのアクセスを制限すべき理由として、さらに消費者に対する害を挙げる。
彼らは、ポルノを単なる表現の一形態ではなく、危険な行為として扱うべきだと主張し、違法薬物の使用と同様の規制が必要だと考えている。
アメリカでは十数の州がポルノを「公衆衛生上の危機」と宣言する決議を可決している\citep{quinn19:_is_porn_public_healt_crisis}。
著名な反ポルノ活動家であるゲイル・ダインズは、「ポルノの害に関する証拠が積み上げられるにつれ、私たちはもはや手をこまねいて、ポルノ産業が文化の性的および感情的な健康を\ruby{乗っとる}{ハイジャック}のを許すことはできない」と述べている\citep{dines16:_why_porn_is_public_healt_issue}。
彼女はポルノグラフィの健康へのネガティブな影響を示す研究を集積したウェブサイトを運営している(CultureRefamed, \url{https://culturereframed.org/})。

ポルノの反対派は、ポルノが視聴者にもたらす負の影響についていくつかの主張をおこなっている。
彼らは、ポルノが中毒性を持つと主張している\citep{snagowski15:_implic_assoc_cyber_addic}。
また、ポルノの視聴が男性の勃起不全や性欲の低下を引き起こす可能性があるとも指摘されている。
ある研究では、次のように述べられている。
「インターネットポルノの特異な性質(無限の新規性、より極端な素材への容易なエスカレーション、ビデオフォーマットなど)が、性的興奮をインターネットポルノの使用に条件付けるほど強力であり、実際のパートナーとの性行為が期待に合わず、興奮が低下する可能性がある」\citep[p.1]{park16:_is_inter_pornog_causin_sexual_dysfun}。

ポルノの反対派はまた、ポルノが特に若者、特に男性や少年に対して、性に関する誤った有害なメッセージを植え付けると主張する。
彼らは、ポルノが男性に対して極端な性行為(アナルセックス、オーラルセックス中の嘔吐、精液を顔にかける行為など)を期待させるようになると警告する。
また、女性が陰毛を剃るようになったのもポルノの影響だとされる\citep{fetters11:_new_full_front}。
さらに、批判者はポルノは若者の親密な関係の発達を阻害すると主張する。
ダインズは次のように述べている。
「私は、ポルノを早くから視聴する男性ほど、現実の女性との親密な関係を築くのに苦労する可能性が高いことを発見した。
一部の男性は、実際の女性とのセックスよりもポルノを好むようになる。
彼らは、現実の女性がポルノのような性行為を望まないことに当惑し、時には怒りすら覚える」\citep{bindel10:_truth_porn_indus}。
ロス・ダウザットは、ポルノの視聴が「特権意識と怨念を同時に抱え、怒りっぽく、無気力な男性像」を生み出すと主張している\citep{douthat18:_lets_ban_porn}。

ポルノの反対派の中には、ポルノが女性に対する暴力を助長し、その発生を促すと考える者もいる。
彼らは、ポルノが一種の扇動行為として機能すると主張し、歴史的に認められてきた「明白かつ現在の危険」の原則に基づいて規制が正当化されるべきだと論じる。
オリバー・ウェンデル・ホームズ判事は、言論の自由には制限があり、たとえば混雑した劇場で「火事だ!」と叫ぶことは許されないと述べた
(\emph{Schenck v. United States}, 249 U.S. 47)。
ポルノ反対派は、ポルノが女性に対して即時的な危険をもたらすと考えている。
ポルノの大半が女性を従属的に描き、虐待的な扱いを喜ぶように見せているため、こうしたコンテンツの消費が視聴者の身体的・性的暴力の加害傾向を大幅に高めると主張する。
一部の批評家、とりわけキャサリン・マッキノンは、ポルノの消費と性的暴力の行為との間には直接的な因果関係があると主張している\citep{mackinnon94:_pornog}。
また、ロバート・ジェンセンのように、直接的な因果関係を証明できないとしても、ポルノがこうした行動に大きく寄与していると考える者もいる。
ジェンセンは、「ポルノだけで男性がレイプをするわけではないが……[しかし] それは強制的な傾向を活性化させる可能性がある」と述べている\citep[p.103]{jensen07:gettingoff}。
単なる相関関係ではなく因果関係を証明するために、彼らは性犯罪者の証言を引用している。
ある研究によれば、性犯罪者の中にはポルノが逸脱した性的嗜好を強めたと報告する者がいるほか、「ポルノを利用して自分を十分に興奮させ、犯罪を抑制する心理的抵抗を乗り越えた」と証言した者もいる\citep[p.70]{marshall00:_revis_use_pornog_sexual_offen}。

ポルノ反対派は、ポルノが女性の社会的地位に悪影響を及ぼすとも主張している。
ポルノが女性を従属的でセクシャル化されたものとして描写し、その地位を喜んでいるかのように見せることによって、女性に対するネガティブな見方を広め、家父長制的な価値観を強化すると批判される\citep{hald13:_pornog_sexis_attit_among_heter}。
アン・イートンは、次のように述べている。
「性差別の仕組みや規範、神話をエロティックにすることは、それを維持し促進する極めて効果的な手段だ」\citep{eaton07:sensibleantiporn}。
ダインズはさらに、「ポルノは家父長制を支援する完璧なプロパガンダだ。
男性の女性嫌悪がこれほど明確に表れるものは他にない」と述べている\citep{bindel10:_truth_porn_indus}。

しかし、これらの議論には問題がある。
もしポルノが女性について何らかの見解を表現しているのであれば、それはやはり一種の「表現」だ。
1985年、アメリカの連邦控訴裁判所は、マッキノンとアンドレア・ドウォーキン\ig{Andrea Dworkin}のモデル法案に基づくインディアナポリスの反ポルノ条例を違憲と判断した。
その際、フランク・イースターブルック判事は、ポルノが女性の従属を描いていることを認めた上で、それが女性に悪影響を与える可能性を否定しなかった。
しかし、それでもなお、ポルノは言論の自由として保護されるべきだと判断した。
「私たちはこの条例の前提を受け入れる。
従属を描写することが従属を維持する傾向がある。
女性の従属的地位は、職場での低賃金や家庭内での侮辱、路上での暴力やレイプにつながる」。
しかし、彼は続ける。
「これはシンプルにポルノがもつ言論としての力を示している」(\emph{American Booksellers Association, Inc. v. Hudnut})。

ポルノ反対派は、イースターブルックの主張に対して、ポルノが他の保護された表現とは異なる形で機能すると反論している。
彼らは、ポルノが他の表現形態とは異なる独特の影響を及ぼし、それが理性的な思考を促進するのではなく、むしろ妨害すると主張する。
ポルノの影響は「理性以前のもの」(sub-rational)であり、自律的な思考を阻害するのだという。
ダニー・スコッチアは、自由主義的な表現の自由の原則は「聞き手の精神状態に非理性的な影響を与える表現を保護しない」と述べている\citep[p.777]{scoccia96:_can_liber_suppor_ban_violen_pornog}。
ジョン・フィニスは、理性と情念の区別に訴えてこの議論を展開し、表現が保護されるべきなのは、それが私たちの理性的能力に訴える限りにおいてだと主張する。
彼は、「表現が理性と情念の連続体のうち情念側に由来するものであるほど、その自由を擁護する理由は消滅する」と述べている\citep[p.222]{finnis67:_reason_passion}。
フィニスは、ポルノが理性によって理解可能なアイデアを提示しないだけでなく、知的能力を積極的に抑制すると考えている。
彼によれば、ポルノは「理解力を曖昧にし、官能的興奮、刺激、快楽によって意図的に混乱を引き起こす」(Finnis, 1967, p.223; \emph{Ginzburg v. United States}を引用している)。
\nocite{finnis67:_reason_passion}
アメリカ合衆国最高裁は、憲法上のわいせつ基準を確立する際にフィニスの議論を引用し、表現の自由を制限する法律は、それが「無制限なわいせつ物の展示や流通を防ぐことを目的とする場合には合憲である」と述べた。
最高裁は、わいせつ物は「理性や知性の統制とは異なるものである」と判断した(\emph{Paris Adult Theatre I v. Slaton}; \emph{Miller v. California}; cf. \emph{Roth v. United States})。

これに対して、広告や感傷的な小説などの他の表現も、しばしば非理性的な方法で私たちに影響を与えようとするのではないかと反論することもできる。
しかし、ポルノ反対派は、ポルノの性的な性質が、他の表現よりも特に強力な影響力を持つと主張する。
イートンは次のように述べている。
「ジェンダー不平等を性的に描くことは、私たちの身体的欲求や性的欲望を性差別の支持へと引き込む。
これらの欲求や欲望は、理性的な精査によって制御されることはほとんどないため、それをジェンダー不平等に結びつけることは、それを心理的に深く根付かせる効果的な手段となる」\citep[p.679]{eaton07:sensibleantiporn}。
同様に、キャサリン・イッツィンは、ポルノが女性蔑視的なメッセージを性的興奮と結びつけることによって、それらのメッセージを視聴者の心に巧妙に刷り込む手段となると主張している。
彼女はこう言う。

\begin{quote}
  ポルノにおける性的に露骨で性的興奮を伴う女性の従属は、個々の男性のレベルで、興奮やオーガズムを通じて身体化される。
そして、こうして条件づけられたものは、感情や空想の中に深く根付き、極めて顕著であり、容易に取り除くことができない形で存続する。
\citep[p.23]{itzin02:_pornog_const_misog}
\end{quote}

イートンは、ポルノが特に陰湿であるのは、それが男性だけでなく女性にも影響を及ぼすからだと指摘している。
彼女は次のように述べている。
「ポルノにおけるエロティシズムは、ジェンダー不平等を男性にも女性にも魅力的なものとして映らせる。
女性が男性にとって魅力的でありたいと願う限り、彼女たちは従属的な魅力の規範を内面化し、それによって自らの抑圧に加担することになる」\citep[p.679]{eaton07:sensibleantiporn}。

\subsection{ポルノグラフィは女性の従属と沈黙を引き起こすか}

一部のフェミニストのポルノ批判者は、ポルノを単なる「表現」ではなく、むしろ「\ruby{遂行的}{パフォーマティブ}」な言語行為の一種として捉え、それゆえ通常の表現の自由の保護対象にはならないと主張する。
この議論は、キャサリン・マッキノンとアンドレア・ドウォーキン\ig{Andrea Dworkin}に端を発し、レイ・ラングトンなどによって発展させられてきた。
この立場によれば、ポルノは単に女性の従属を引き起こすのではなく、それ自体が女性の従属の行為に他ならないという。
マッキノンは次のように述べる。

\begin{quote}
ポルノは、それが「言う」ことではなく、それが「する」ことだ。
この点で、ロナルド・ドウォーキン\ig{Ronald Dworkin}教授がポルノを擁護する際に持ち出す「見解」「思想」「意見」「趣味」の保護という議論は的外れだ。
私たちが問題にしているのは、差別という行為だ。
\citep{mackinnon94:_pornog}
\end{quote}

この見解の支持者たちは、J. L. オースティンの言語行為論を用いてその説明を試みている。
オースティンによれば、「発語内行為」(illocutionary acts)と呼ばれる言語行為の一種は、単に意見を表明するだけでなく、特定の事態を実際に創出する機能を持つ\citep[pp.116ff, 121, 139]{austin75:_how_to_do_thing_with_words}。
たとえば、簡易裁判の裁判官が「あなたたちは今、夫婦となった」と宣言する際、その発話は単なる意見の表明ではなく、実際に結婚を成立させる行為となる。
同様に、特定の言語行為は単なる表現ではなく、社会におけるある集団の従属を確立するものとなりうる。
たとえば、「白人専用」と書かれた看板や、「黒人は投票できません」と書かれた標識は、言語行為であると同時に、差別を実際に施行する行為でもある。
ポルノ批判者は、ポルノがこの種の言語行為と同じように、女性の従属的地位を確立し、それを維持する役割を果たすと主張する。
ポルノは「女性は男性よりも劣った存在である」との主張を、ちょうど「白人専用」の看板が「黒人は白人より劣っている」と主張するのと同じように表明する。

さらに、ポルノ批判者は、ポルノにはもう一つの遂行的な効果があると指摘する。
それは、女性の言論の自由を制限し、彼女たちを沈黙させるというものだ。
マッキノンは次のように述べる。
「ポルノは、まさに男性の言説によって女性の言説を沈黙させるものだ」\citep[pp.209--209]{mackinnon87:_femin_unmod}。
ポルノは、女性がどのように発言し、どのように理解されるかという文脈を形成することにより、会議の司会者が特定の発言者に対して「黙れ!」と命じるのと同じ効果をもたらす。
こうした言語行為は、対象者の発言を物理的に妨げることもあれば、彼女たちの言葉を聞こえなくする、あるいは正当性のないものとして扱うことによって、間接的に沈黙させることもある。
マッキノンは、ポルノが女性を常に男性に性的に従属する存在として描くため、実際に女性がセックスを拒んだ場合や、自分が暴行の被害者であることを証言しようとした場合に、その言葉が信じられなくなると指摘する。
その結果として、女性たちは「権威を剥奪され、貶められ、沈黙させられる」(MacKinnon, 1987, p.193; cf. Langton, 1993)\nocite{langton93:_speec_acts_and_unspeak_acts}。
したがって、ポルノを規制することは、単に道徳的な理由によるものではなく、むしろ表現の自由を守るための措置として正当化されるべきだという主張がなされる。
つまり、ポルノの規制は、すべての女性の言論の自由を保護するための行為として理解されるべきだ。

\subsection{ポルノグラフィの価値}

これまで見てきたように、ポルノ批判者はポルノがさまざまな害を引き起こすだけでなく、通常、表現を保護する理由となる利益をもたらさないと主張する。
しかし、ポルノ擁護者はこの主張を否定する。
まず第一に、多くの女性を含む多くの人々がポルノを楽しんでいるという事実を指摘する。
ポルノは手軽にアクセスできる快楽の源であり、性感染症のリスクなしに性的満足を得られる手段だ。
この快楽の提供という側面を超えて、ポルノ擁護者は、ポルノが個人のアイデンティティの表現や発展に寄与する可能性があると主張する。
キャス・サンスティーンは次のように述べる。
「性的に露骨な作品は、個人の能力の発展において非常に重要な役割を果たす可能性がある。
多くの人にとって、それは自己発見と自己定義のための重要な手段だ\citep[p.215]{sunstein95:_democ_probl_free_speec}。

これまでフェミニストによるポルノ批判をいくつか紹介してきたが、フェミニズムと反ポルノの立場を同一視することに異議を唱えるフェミニストも少なくない
\footnote{たとえば、Wendy Brown, Carol Clover, Drucilla Cornell, Lisa Duggan, bell hooks, Nan Hunter, Molly Ladd-Taylor, Anne McClintock, Mandy Merck, Carole Vance, and Linda Williamsらの著作を見よ。
}。
彼女たちは、ポルノを消費する女性や、ポルノの制作や出演を選択する女性の自律を尊重すべきだと主張する。
セックスライターのヴァイオレット・ブルーは次のように述べる。

\begin{quote}
私たち女性は、他人が「女性にとって何が最善か」という考えに基づいて、私たちのセクシュアリティを支配しようとすることにうんざりしている。
たとえば、ポルノに関して「何を好むべきか」「何を好むべきでないか」と指図されることだ。
それは、女性を性的な面で常に子供のような状態に留め置くのと同じだ。
そして、自称フェミニストの女性たちは、ポルノを「力を与えてくれるセックストイ」として捉える女性たちの存在を否定することで、事態をさらに悪化させている。
たとえポルノをそのように見なさないにしても、少なくとも私たち自身が自分の判断で決定できることくらいは信頼してほしい{\DDASH}どうもありがとう。
\citep{clark-flory10}
\end{quote}

エレン・ウィリスも、ポルノが女性にとって快楽の源であるだけでなく、男性中心主義に対する抵抗の手段にもなりうると主張する。

\begin{quote}
ポルノグラフィは、その内容においても、また公共の場で私たちの注意を強制的に引く点においても、精神的な攻撃となりうる。
しかし、男性にとってそうであるように、女性にとってもまた、ポルノは官能的快楽の源となることがある。
レイプされた女性は被害者だが、ポルノを楽しむ女性(たとえそれがレイプ・ファンタジーを楽しむことを意味するとしても)は、ある意味で反逆者であり、男性の領域とされてきた自身のセクシュアリティの一側面を主張していると言える。
ポルノが男性優位と性的疎外を賛美する限り、それは深く反動的だ。
しかし、性的抑圧と偽善{\DDASH}これは男性以上に女性に多くの害を及ぼしてきた{\DDASH}を拒絶するという点において、それは急進的な衝動を表現している。
\citep[pp.83-84]{willis84:_femin_moral_pornog}
\end{quote}

ポルノ擁護者は、実際に多くの女性がポルノを消費していることをデータによって示す。
世界最大のポルノサイトであるPornhubのデータによれば、視聴者の約3分の1は女性だ(また、上位の検索カテゴリーを分析すると、暴力的または極端な内容への一般の関心が特に高いわけではないことも示されている)\citep{insights19:_year_review}。

さらに、ポルノにはゲイ、レズビアン、トランスジェンダーを対象とした作品も多く含まれる。
一つの調査によると、ゲイ男性の98%が過去30日間にポルノを視聴したと報告している(Duggan and McCreary, 2004; cf. Ellis and Whitehead, 2004; Kendall, 2004)。
\nocite{duggan04:_body_image_eatin_disor_drive}\nocite{ellis04:_porn_again}
\nocite{kendall04:_educat_gay_male_youth}
マッキノンは、ゲイポルノにおいて一方のパートナーが支配的な立場で描かれる場合、それは女性が虐待されるポルノと本質的に変わらず、したがって家父長制の表現だと主張している\citep[pp.178--179]{mackinnon89:_towar_femin_theor_of_state}。
しかし、クィアのライターたちは、同性愛ポルノを単なる代理的な異性愛ポルノの一形態に還元する考えに異議を唱えている。
むしろ、彼らは、それが多くの異性愛ポルノに見られる搾取に対する代替物を提供すると主張する。
サム・マイルズ医師\ig{Sam Miles}は次のように述べている。
「ゲイポルノでは俳優間の明白な権力の不均衡が少なく、より平等主義的な視点を提供する傾向がある。
そのため、異性愛者を引きつける魅力があり、特にストレートの女性にとってそうだ」\citep{bloodworth18:_this_is_why_straig_men}。
さらに、ゲイ男性によるポルノの消費割合が高い理由の一つは、ポルノが彼らの性的アイデンティティを肯定し、承認する役割を果たしている点にあるとも指摘されている。
ジョン・マーサーは次のように言う。

\begin{quote}
私たちの集団としてのアイデンティティは性的選択によって定義される。
したがって、その形態のセクシュアリティが文化的に表現されることは極めて重要だと考える。
同性愛ポルノは、特定の欲望が存在することの記録的証拠として機能し、ゲイ男性コミュニティにおける性的欲望の神話を創り出す。
\citep[p.204]{ellis13:_porn_again}
\end{quote}
ポルノは、その他の性的マイノリティにとっても自己認識を促す役割を果たす。
トランス女性のジョアンナ・ファングは、自らのトランス・アイデンティティを確認する上でポルノが果たした役割について次のように語る。
\begin{quote}

  私はずっと自分がトランスであることを知っていましたが、1990年代後半から2000年代初頭にかけて、メディアにトランスの人々の登場はほとんどありませんでした。
私がトランス体験に関する情報を得る唯一の手段は、ジェリー・スプリンガーのようなテレビ番組かポルノでした。
トランスのアダルト映画スターは、私にとって最も身近なロールモデルでした。
私はポルノを、単なるマスターベーションの道具ではなく、アイデンティティを形成するための手段として見ていました。
「これが私だ」と思ったのです。
\citep{white17:_how_trans_affec_sex_drive_porn_consum}
\end{quote}

また、ポルノは新たな性的アイデンティティの探求や、フェティッシュの承認にも寄与する可能性がある。
ルビー・スティーブンソンは、異性愛者の多くがゲイポルノを視聴することについて「さまざまな種類のポルノを見ることは、現実で試す前にファンタジーを探求する健全な方法になりうる」と述べている\citep{bloodworth18:_this_is_why_straig_men}。

フェミニストポルノ制作者のアビー・サックスは次のように述べている。

\begin{quote}
私は、人々が抱える最大の問題は「私はこの欲望/興奮/キンク/フェチ/ファンタジー/その他何であれ、それを抱えている、そしてそれが普通ではないと感じる」ということだと思います。
そして、私は彼らに許可を与えたいのです。
オーマイガッ!そんなの何万回聞いたことあるよ。
ぜんぜん変じゃないし、すごくよくわかるよ、と。
\citep{sachs12:_inter_femin_pornog}
\end{quote}

ポルノグラフィには、セクシュアリティ全般に対するより肯定的な態度を促進する可能性があると考える人もいる。
アダルト映画制作者のニナ・ハートリーは次のように述べている。

\begin{quote}
私は自分の楽しみのためにポルノを作っていますが、それだけではありません。
私たちの文化ではセクシュアリティは病気であり、病人には看護が必要だからです。
人々は今セクシュアリティに苦しんでいます。
自分の体や皮膚のなかに安らぎを感じることができずに苦しんでいます。
他者と健康的で安全で気持ちよく人間的なつながりを築くことができずに苦しんでいるのです。
私は自分を、(セックス・エデュケーターである)ルース博士とベティ・ドッドソンの中間の存在だと思っています。
他の人がセックスをする手助けをしてると思うとうれしいです。
\citep{wischhover15:_why_im_still_doing_porn}
\end{quote}

\subsection{検閲の代償}

ポルノグラフィに対する公式な規制に抵抗すべきもう一つの理由は、フレデリック・シャウアーが「政府の無能性の議論」と呼ぶものだ。
彼は次のように述べている。
「過去の経験からすると、政府は特に検閲において無能であることが示されている。
政府は他の行動の規制に比較して、言論の規制においてはるかに能力が低い\citep[p.82]{schauer82:_free_speec}。
また、検閲の負担は、性的マイノリティを含む社会的に弱い立場の人々に不均衡に降りかかることが歴史的に示されてきた。

すでに見てきたように、ポルノグラフィの正確な定義を与えることは極めて困難だ。
したがって、何らかの規制が導入された場合、政府に対して広範な裁量が与えられることになる。
そして歴史は、国家が裁量権を与えられると、それを社会の最も弱い層を標的とする形で行使してきたことを教えてくれる。
アマンダ・コーストンは次のように指摘する。
「ポルノグラフィや売春に関する立法、規制、刑事処分のアプローチは、歴史的に見て、女性の行動とセクシュアリティを制御するために機能してきた。
そして、それに伴うその他の害悪を助長してきた」(Cawston, 2019, p.626; cf. Bumiller, 2008)。
\nocite{cawston19:_femin_case_pornog}\nocite{bumiller08:_in_abusiv_state}
さらに、ポルノの流通を制限する法的措置が取られると、その執行がしばしば人種的・性的マイノリティの作品に対して厳しく適用されることがある。
1992年、カナダ最高裁は、暴力や女性蔑視を含むポルノグラフィを禁止できるとする判決を下した
(\emph{R. v. Butler} [1992] 1 SCR 452)。
しかし、その後の摘発のほとんどは、ゲイやレズビアン向けのコンテンツを対象としたものであった\citep{mackinnon94:_statem_cathar}。
レスリー・グリーン\ig{Leslie Green}は、この判決後の状況を次のように述べる。

\begin{quote}
\emph{Butler} 事件以降、国家によるポルノ規制の実態は予測通りのものであった。
最初に摘発されたのは、カナダ国内の購読者がわずか数十人のレズビアン雑誌であった。
その後も、カナダ税関職員は、無知かつ\ruby{同性愛嫌悪的}{ホモフォビック}な基準に基づき、女性やLGBTQ+関連の書店を標的に検閲をおこなった(その中でも特に滑稽な例は、1994年に女性のスピリチュアリティとエコロジーを専門とする書店向けに発送された\emph{The Sexual Politics of Meat}というフェミニスト・ベジタリアン批評の書籍が税関で押収されたことだ)。
\citep{green00:pornographies}
\end{quote}

\subsection{出演者への危害:反論}

ポルノグラフィの擁護者は、アダルト業界の女性は同意する能力がないほど脆弱だという主張に異議を唱える。
彼らは、このような見方は、出演者の自律を否定していると指摘する。
アダルト俳優として広く知られるストーヤは、「私自身がリスクを計算し、「より安全な」選択肢を選ぶことができると判断しているのに、私にはその能力がないとほのめかすような法律を制定するのは侮辱的だと思います。
私は、しっかり\ruby{より安全な}{セイファー}道を選んでいます。
\ruby{安全な}{セーフ}セックスなんてものはなく、\ruby{より安全な}{セイファー}セックスがあるだけなんですから」と述べている\citep{kernes12:_no_gover_waste_commit_holds_press_confer}。
ポルノ擁護者たちは、成人のポルノ出演者の間では、幼少期のころに性的虐待を受けている割合が平均より高いという主張を否定する\citep[p.621]{griffith13:_pornog_actres}。
また、ポルノ業界にはリスクが伴うことを認めつつも、これらのリスクはセックスワーカーや他の危険な職業に従事する人々が直面するリスクとは比べものにならないと主張している。

ポルノ擁護者たちは、出演者たちの多くが仕事を楽しんでいる証拠として、出演者たち自身の証言を提示している。
アダルト俳優のミーシャ・メイフェアは次のように述べる。
「撮影現場で楽しく過ごすのが好きです。
自分が本当に楽しめる方法で、完全に同意した上でセックスをし、それをポジティブに受け止めないことなど考えられるでしょうか?」彼女は、制作過程で虐待が発生する可能性があることを認めつつも、ほとんど撮影現場は搾取的な環境ではないと強調する。
「\ruby{現場}{セット}では、キャストやスタッフは本当に親切で気遣いがあります。
一般に、人々はとてもやさしいものです。
そして、誰かが現場から出ていってしまったり、文句を言ったりするようなことがないようにみんな気をつかっています。
要は、そんなことがあったらビジネスとしてよくない、ということです」\citep{sisley19:_common_myths_porn_debun_porn_perfor}。

アダルト映画監督のトリスタン・タオルミノは、メディアやポルノグラフィ反対派が、業界における被害の話ばかりを強調することに不満を示す。

\begin{quote}
一般の人は、ポルノ業界から傷ついたり、トラウマを負ったり、壊れたりせずに去った人々の話はまったく耳にしないと思います。
サクセスストーリーも聞かないでしょう。
ニーナ・ハートリーのように、自分の仕事を愛し、業界で何を得たか、それを驚くほどはっきり語っている人の話も一般には伝えられません。
彼女は20年以上この業界にいますが、「業界に食い尽くされた」とか「業界から唾をはいて追い出された」などとは言いません。
彼女は自分のキャリアを完全にコントロールしています。
しかし、そうした話は伝えられず、悲惨で絶望的な話ばかりが広まり、非常に偏った描写がなされています。
それはフェアではありません」。
\citep{sachs12:_inter_femin_pornog}
\end{quote}

ポルノ擁護者は、一部のポルノの制作が、出演者にとって有害かもしれない状況で作られていること、また場合によっては虐待的であることは認めつつも、ポルノの禁止や規制は何の解決策にもならないと主張する。
むしろ、そうした措置は非生産的であり、制作を地下に追いやってしまい、責任感のある制作者たちを業界から排除してしまうことになる。
擁護者たちは、他のセックスワークと同様に、業界を出演者が労働者としての権利を守られるように適切に規制することが必要だと考える。
ローリー・シュレイジは次のように述べる。
「ポルノグラフィに関する適切な法的措置とは、業界に従事する女性たちの労働者としての権利を守るための手段を提供するものであるべきであり、女性たちの仕事を奪うものであってはならない」\citep[p.58]{shrage05:_expos_fallac_anti_femin}。

すでに述べたように、映しだされている人々の同意なく制作・公開されているポルノグラフィが相当数存在しており、撮影されている性的行為がそもそも同意されていない場合も相当ある。
この点に関しては、ポルノグラフィの支持者も政府によって一定の規制がなされ、すべてのポルノは、登場する人物の同意のもとに制作され配信されるようにする必要があることを認めている。
しかし、彼らは、正当な制作業者や出演者がこうした規制の策定に関与すべきだと主張する。
\emph{The New York Times}の記事への対応として、MindGeekが約1,000万本の動画を削除し、利用規約を変更した際、業界内では深刻な懸念が生じた。
セックスワーカー支援団体のマギーズ・トロント(Maggie's Tronto Sex Worker's Action Project)は、この措置について次のように述べている。

\begin{quote}
これらの変更は、著作権侵害や無許可のポストとダウンロードを防ぐ上で、セックスワーカーにとって有益なものと思われる。
しかし、問題なのは、Pornhubが対応したのが、セックスワーカーの権利に強硬に反対している反人身売買運動の声であり、プラットフォームを成功に導き、そこでの収入に依存するワーカーたちの声ではなかったことだ。
\citep{toronto21:_what_do_pornh_chang_mean_sex_worker}
\end{quote}

この措置により、多くの制作者が、収入源になっていた動画カタログと視聴者たちを失った。
キンク関連のファンタジー動画を制作するGoddess By Nightは次のように述べている。

\begin{quote}
あれは裏切りでした。
……私の知り合いには、Pornhubが猶予期間を設けなかったせいで、収入源を失って取り戻せなくなったクリエイターたちがいます。
\citep{cole20:_pornh_conten_purge_has_left}
\end{quote}

\subsection{他者への危害に対する反論}

ポルノグラフィが視聴者や社会全体に与える影響は経験的な問題であり、専門家の間で議論が続いている。
この問題に関しては、ポルノが性的暴力や性機能障害や女性に対するネガティブな態度にどう関係しているかについてコンセンサスは存在しない。
ポルノ擁護者は、ポルノがこれらに寄与しているとする主張に対して強く異議を唱えている。

特に、「ポルノ依存症」という概念は厳しい批判を受けている。
多くの科学者は、薬物やアルコールの依存を説明するために発展した依存症のモデルは、セックスやポルノ視聴のような行動に適用することはできないと考えている。
デヴィッド・レイ、ニコル・プローズ、ピーター・フィンによるポルノ依存症に関する研究レビューでは、既存の研究が「実験デザインが不十分であり、方法論的な厳密さに欠け、モデルの仕様が不明確だ」と指摘されている。
また、レイたちや他の研究者たちは、ポルノ依存を自覚している人々や、ポルノ使用による勃起不全を訴える人々は、しばしばポルノ視聴と矛盾する道徳的または宗教的価値観を持っており、彼らの苦悩や無力感はポルノ依存症ではなく、むしろそうした価値観との葛藤によるものだと主張されている\citep[p.105]{ley14:_emper_has_no_cloth}。
ある研究では、「インターネットポルノの使用だけでは性機能障害を予測しない。
むしろ、インターネットポルノ依存の自覚の強化が、ネガティブなセックスの結果と関連していた」と結論づけている\citep{whelan21:_pornog_addic}。

ポルノグラフィの社会的影響を評価する際には、それが単独で消費されているわけではないことに留意する必要がある。
カナダ議会のポルノグラフィ調査委員会は、「性的に露骨なコンテンツがセクシュアルヘルスや性的行動に及ぼす影響は、社会における性教育、政治的・社会的構造、広範なメディア環境と切り離して考えることはできない」と報告している\citep[p.11]{committee17:_repor_public_healt_effec_ease}。
特に、イースターブルック判事が判決で指摘したように、たとえポルノが女性を従属的に描いていたとしても、そのような表現はポルノグラフィに限られたものではない。
彼は、ポルノの禁止がジョイスの『ユリシーズ』やホメロスの『イーリアス』にまで適用される可能性があると述べた。
なぜなら、どちらも女性を征服や支配の対象として描いているからだ
(\emph{American Booksellers Association v. Hudnut})。
また、聖書やコーラン、映画、音楽ビデオ、広告などにも同様の表現が見られる。

すでに見たように、ポルノの批判者は、ポルノは他の性差別的なコンテンツよりも強い影響力を持つと主張する。
それは、ポルノが性的な仕方でメッセージを伝えるからだ。
しかし、この主張を裏付ける決定的な証拠は提示されていない。
つまり、ポルノが他の性的・性差別的なコンテンツと比べて特異な影響を持つことを証明するデータは存在しない。
リン・シーガルは、「男性たちは、アラブ女性のチャドルからナットとボルトのようなオスとメスの記号に至るまで、どんな画像でもポルノ化することができ、実際にそうしている」と指摘している
(ポルノと性暴力の間の因果的関係を検討している研究者たちが用いる前提と方法に対する批判としては、\citet{boyle00:_pornog_debat}を参照せよ)\citep[p.15]{segal93:_does_pornog_cause_violen}。

ポルノが社会的態度の形成に果たす役割を特定することは難しいが、以下の二つの事実は注目に値する。
第一に、インターネットが普及した1990年代半ば以降、ポルノの消費は劇的に増加している。
第二に、この期間において、性差別的な態度や女性に対する暴力の発生率は顕著に低下している。
1977年から2016年までの米国GSS(General Social Survey、総合社会調査)の調査に基づき、27,000人以上の回答を分析した2人の研究者によれば、次のようだ。

\begin{quote}
家庭や職場における男女平等を信じない「真に伝統的」な人々はほぼ姿を消した。
1977年当時、人口の4分の1未満しか男女平等を支持していなかった。
当時のアメリカでは、多くの人が極めて伝統的な性別役割を支持しており、女性は政治に向いておらず、子育てに専念し、労働市場に参加すべきでないと考えられていた。
しかし2016年までに、こうした伝統的な見解はほぼ消滅し、そのような信念を持つ人は人口のわずか7%にまで減少した。
一方で、男女平等を支持する人の割合は3倍に増加し、人口の69%を占めるようになった。
(Risman et al., 2018, 記事は\citet{scarborough19:_attit_stall_gender_revol}を紹介したもの)
\nocite{risman18:_good_news,scarborough19:_attit_stall_gender_revol,risman18:_good_news}
\end{quote}

また、彼らによれば、ミレニアル世代は最もポルノへのアクセスが容易な世代であるが、同時に男女平等を最も強く支持する世代でもあり、彼らの75%以上が家庭と職場の両方での平等を支持している。
さらに、GSSデータの分析では、ポルノを視聴する人々の方が性別に関する平等な態度を持っていることが示された\citep{taylor16:_is_pornog_reall_makin_hate_women}。
また、ラスベガスで開催されたAVNアダルトエンターテインメントエキスポの参加者(ポルノの「スーパーファン」)を対象とした研究では、彼らが一般人口よりも性別に関して平等な態度を持っていることが判明した\citep{jackson19:_expos_mens_gender_role_attit_porn_super}。
研究者たちは、「現時点では、ポルノグラフィが女性に対する否定的な態度を引き起こすという考えには慎重になるべきだ。
証拠は乏しく、現在のポルノに関する言説の多くは、\ruby{道徳的}{モラル}パニックであり、公衆衛生上の危機とは言えない」と結論づけている\citep{maginn19:_how_male_porn_super_reall_view_women}。

性的暴力の統計も、ポルノの「負の影響パラダイム」を支持するものではない。
米国政府の犯罪統計によれば、1993年(インターネットの普及が始まった頃)から2016年にかけて、女性に対する性的暴力は60%以上減少している。
1993年には1,000人あたり4.3件だったが、2016年には1.2件にまで低下している(Department of Justice, 2019; cf. Planty et al., 2013)。
\nocite{department19:_nation_crime_victim_survey}
また、ある研究では、ポルノ視聴と暴力的な性的行動の間には負の相関関係があることが示されている\citep{ferguson09:_pleas_is_momen}。

これらの事実は、ポルノが社会的態度や行動に対して肯定的な影響を及ぼしていることを示すものではけっしてない。
しかし、もしポルノが性差別的な態度や犯罪行動に寄与しているのだとしても、その影響は比較的弱いと考えられる。
なぜなら、ポルノの影響が、より大きな社会的な傾向を覆すほど強力ではないことが示唆されるからだ。

\subsection{従属化と沈黙に対する反論}

先に見たように、マッキノンとドウォーキンは、ポルノが遂行的発話の一形態として機能しており、女性を従属させ\ruby{口を封じる}{サイレンス}と主張している。
しかし、この議論の前提には批判が寄せられている。
行為遂行的に発話するためには、発話者が聞き手によって権威ある存在として認識されていなければならない。
たとえば、教師が生徒に「あなたは発言してはいけません」と言えば、それは教室内での権力を背景に、生徒を実際に沈黙させる効果を持つ。
ポルノ反対派は、ポルノは、女性の役割や発言の価値について、上と同様に権威的な形で語っていると主張する。
しかし、ポルノ擁護派は、ポルノがそのような権威を持つとは言えないと反論し、その権威がどこから生じるのかを疑問視する。
多くの遂行的発話は、制度的な文脈の中で機能し、発話者がその制度内で権力を持っていることによって成立する。
しかし、ポルノの制作者は、一般の視聴者に対して制度的な権力を有しているわけではない。

ポルノグラフィの批判者は、ポルノの権威は、それが消費者によって権威的だと見なされ、性や女性についての専門性を持つと認識されていることに由来すると反論する\citep[p.430]{langton12:_respon}。
しかし、レスリー・グリーン\ig{Leslie Green}は、ポルノはむしろ「低ステータスの発話」{\DDASH}社会的権威はほとんどない発話{\DDASH}だと指摘する。
彼は、ポルノの(女性に関する)メッセージは、「高ステータスの発話」、たとえば政府や家族や教会の発話によって常に挑戦され反論されていると主張する。
彼は次のように述べる。

\begin{quote}
自由な表現が尊重される社会では、ポルノは何も〔過去にキリスト教会が出していた〕\ruby{出版許可}{インプリマチュア}のようなものは受けていない{\DDASH}社会によってしぶしぶ我慢されているが、同時に反対もされている。
現在、多数の人々がポルノを視聴しているが、それを自ら認める人はごく少ない。
ポルノが新聞の日曜版でレビューされることはなく、今なお人々に当惑や嫌悪や恥の感覚を引き起こしている。
これらは、ポルノが、私たちにとって、「低ステータスの発話」であることのしるしだ。
\citep[pp.296--297]{green98:_pornog_subor_silen}
\end{quote}

ナンシー・バウアーは、ポルノの権威に関する主張に異なる視点から異議を唱えている。
彼女は、ポルノが女性に対する暴力を「認可」(authorize)する権威を持つことはありえないと述べている。
なぜなら、そのような形で暴力を認可できるものなどそもそも存在しないからだ。
バウアーの見解では、もしポルノが消費者に対して女性への暴力を「認可」するものであるならば、それは加害者の責任を免除することだ。
しかし、現実には、いかなる表現もそのような責任免除の効果を持つことはなく、暴力をふるう者は常に自分の行動に対して責任を負っている。
したがって、ポルノが権威を持って女性に対する暴力を正当化するという主張は成立しない。
バウアーは次のように述べる。

\begin{quote}
私たちが言いたいのは、彼がしたことを正当化できるものなど\kenten{何もない}、ということだ。
それは単に彼の行為が恐ろしいものだからではない。
また、私たちが「私たちの社会は暗黙的に、あるいは明示的にレイプを容認している」と考えたくないからですらない。
むしろ、正式に権限を付与されていない人間や組織は、そのような権威を持つことはできないからだ{\DDASH}ある年齢以上の人間は、自分が世界をどのように認識するかについて責任を持つべき存在だからなのだ。
\citep[pp.86--87]{bauer06:_how_do_thing_pornog}
\end{quote}

さらに、ロナルド・ドウォーキン\ig{Ronald Dworkin}は、ポルノが女性の「言論の自由」を侵害しているという主張自体が誤解を含んでいると指摘する。
\ruby{沈黙(口封じ)}{サイレンシング}の議論では、ポルノは女性の発言を妨げるために言論の自由を侵害しているとされる。
しかし、ドウォーキン\ig{Ronald Dworkin}は、言論の自由が「他者に自分の発言を理解し、尊重させる権利」まで含むわけではないと主張する。
彼は次のように述べる。
「この主張は受け入れがたい前提に基づいている。
すなわち、言論の自由とは、自分が発言できる機会を得ることだけでなく、自分の発言を周囲に理解され、尊重されることまで含むという発想に基づいている」。
彼は、私たちは、「単に公に向けて発言する機会を与えられることを越えて、その発言が共感的に理解されることが保証されるべきだとは言えないし、適切に理解されることすら保証されるべきだとは言えない」と強く主張している\citep[p.232]{dworkin96:_freed_law}。

\subsection{本節のまとめ:倫理的なポルノグラフィは可能か?}

たとえ多くのポルノグラフィが女性を貶めるものであり、それが搾取的な条件の下で制作されていることが多いと認めたとしても、消費者の選択はポルノグラフィの内容や制作環境に影響を与えている。
そこで、一部の人々が提案しているのは、ポルノグラフィを回避したりそれを視聴する人々を非難したりするのではなく、消費者がより倫理的な選択をおこなうことで、批判者が指摘する多くの害を軽減または排除することだ。
まず、消費者は無料でダウンロードするのではなく、ポルノグラフィに対して適正な対価を支払うことで、出演者が適正な賃金を得られるようにすることができる。
これにより、出演者が生活を成り立たせるだけでなく、搾取的な状況を避けるための選択肢を持つことが可能になる。
第二に、視聴するポルノグラフィがどのような環境で制作されたのかを学び、出演者を搾取しない方法で制作された作品に対価を支払うようにすることができる。
第三に、消費者は暴力的または貶める内容のポルノグラフィではなく、女性やセクシュアリティに対するポジティブな視点を促進するような作品を選択できる。

実際、女性が制作するポルノグラフィは大量にあり、中にはフェミニスト・ポルノグラフィとして意図的に作られたものもある。
トロントのフェミニスト系セックスショップ Good for Her が主催するフェミニスト・ポルノ賞では、作品がフェミニスト・ポルノグラフィと認められるための基準として、次のいずれかを満たすことを求めている。
「作品の制作、脚本、または監督に女性が関与していること。
もしくは、作品が女性の本当の快楽を伝えていること。
または、作品が性的表現の境界を拡張し、主流のポルノグラフィにおけるステレオタイプに挑戦していること」\citep{lust11:_what_is_your_defin_femin_porn}。
たとえば、ペトラ・ジョイは、女性のセクシュアリティをリアルに描くことを目指してポルノグラフィを制作している。
彼女は次のように述べている。

\begin{quote}
私の作品では、女性が主役であり、女性は女性が望むものを得ます。
男性のファンタジーに応える映像が氾濫している現状において、こうしたステレオタイプを覆し、女性のファンタジーに応える作品を世に出すことは重要だと思います。
それによって、男性が女性のセクシュアリティについて少し学ぶことにもなるかもしれません。
\citep{smith14:_porn_produc}
\end{quote}

また、ポルノグラフィ出演者のシナモン・ラブは、「黒人女性のセクシュアリティをより広範で進歩的で興味深い形で表現する作品」に出演するよう努めているという。
彼女は、「黒人男性と黒人女性が、ステレオタイプに頼ることなく、よりポジティブなイメージで性的な状況に置かれる作品を提供したい」と述べている\citep{love13:_quest_femin}。

消費者がより倫理的なポルノグラフィを積極的に探し求めるようになれば、市場もそれに応じて変化し、より多くの倫理的な作品が制作されるようになるだろう。
もちろん、これによって従来のポルノグラフィが完全になくなるわけではなく、引き続き法規制の強化を求める声は存在し続けるだろう。
しかし、ポルノグラフィのあり方は、それを生み出す社会のあり方を反映している。
したがって、ポルノグラフィ業界を変えるという課題は、より平等な社会を築くという課題と不可分なのだ。

\section{セックスワーク}

セックスワークは「世界最古の職業」だとよく言われる。
これはおそらく事実ではないものの、セックスワークが長い歴史を持つことは確かだ。
現在、世界中で何千万人ものセックスワーカーがその仕事をおこなっており、彼女たちはさまざまな法律や道徳的態度に直面している。
本章では、セックスの売買に反対する、あるいはそれを許容する道徳的・法的議論を扱う。

保守派は一般にセックスワークに反対している。
その理由は複数考えられる。
第一に、彼らは私が先に「セックスの特殊意義説」と呼んだ立場をとっている場合がある。
これは、セックスはコミットメントのある関係の外で行われる場合には不道徳であるとする立場である。
そして、カジュアルなセックスに原則として反対する人々は、商業的なセックスはそのとりわけ問題のあるかたちであると考える傾向が強い。
なぜなら、そこでは当事者同士は、互いに\ruby{人格}{パーソン}としては関心を抱いていないからである。
第二に、保守派の人々は、セックスの売買が伝統的に不道徳なものと見なされてきたという理由だけで、それを犯罪化する十分な根拠になると考えるかもしれない。
最後に、彼らは、商業的なセックスが家庭や共同体の構造に及ぼす影響についても懸念していることがある(この点については後述する)。

対照的に、リベラルな立場は一つの立場に収束するわけではない。
実際、リベラル派の間でも、セックスワークに関する道徳的および法的な問題については、時にはっきり意見が対立している。
以下で論じるように、リベラル派がセックスワークを非難せず、またそれを禁止する法律を支持するべきでない一応の理由(prima facie reason)が存在する。
リベラル派は、人々が自律的に自らの人生を管理すべきだと考えており、この自由は、自分の体を商業的なセックスのために使用するかどうかの選択にも及ぶべきだと見ることがある。
しかし、以下で説明する理由により、セックスワークはこの一般的原則の例外となるべきだと考えるリベラル派も存在する。

フェミニストたちもまた分裂している。
ポルノグラフィに関してと同様に、フェミニストたちの間でのセックスワークに対する意見の対立は厳しくまた複雑だ。
一部のフェミニストは、有償のセックスを男性支配の究極の表現とみなすのに対し、別のフェミニストたちは、それを犯罪化することは女性の自律や主体性を否定するものだと主張する。
この問題に関する最も重要な哲学的議論の多くはフェミニストの視点からなされており、本章ではそうした議論を中心に検討する。
\subsection{用語について}

セックスワークに関する議論は、それを何と呼ぶかという問題から始まる。
伝統的な用語は「売春」(prostitution) であり、学術文献の多くでもこの語が使用されている。
1980年代後半、キャロル・リーは「セックスワーク」(sex work) という用語を提唱した。
彼女は次のように述べている。
「「セックスワーク」という言葉の使用は、ある運動の始まりを示している……それは私たちを「売春者」という地位によって定義するものではなく、私たちがおこなう「労働」を認めるものだ\citep[p.230]{leigh97:_inven_sex_work}。
リーの提唱以来、多くのセックスワーカーやその権利を擁護する活動家は「\ruby{売春者}{プロスティチュート}」という用語の使用を避けるよう求めている。
彼らはこの用語が時代遅れであり、暗黙のうちに道徳的な非難を含んでいると考えているからだ。
しかし、議論の反対側にいる人々の中には、「セックスワーク」という用語もまた偏っていると考える者がいる。
彼らは、この用語が売春を合法的な職業として「\ruby{正常化}{ノーマライズ}」することで、その本質的に搾取的な性質を隠してしまうと主張する。
ソロプチミストという団体の白書には、「権力を持つ男性当局者は、売春を「セックスワーク」と正当化し続けており、この見解は女性を売春のなかに留め続ける役割を果たしている」
と述べられている\citep{soroptimist17:_prost_is_not_choic}。
グロリア・スタイネムも2015年にFacebookで、「「セックスワーク」という言葉は、たしかにアメリカで善意のもとで生み出されたのかもしれない。
しかしこの言葉は、危険な表現だ」と発言している\citep{steinem15:_faceb}。

このような基本的な問題についてどちらの立場をも完全に満足させることは不可能だ。
哲学者を含め、人々の間では依然として「売春」という用語が多く使われているが、業界にいる活動家の間では「セックスワーク」が一般的だ。
本書では後者の用語を採用する。
\emph{VICE}は、あるセックスワーカーの(匿名)ブログ記事から次のように引用している。
「「セックスワーカー」という言葉の方が好ましいのは、それが私たちの多様なコミュニティ全体を包含するものであり、私たちが表明している意思を尊重しているからだ。
基本的に、私たちがそう呼んでほしいと言っているのだから、そう呼ぶべきだろう。
それ以上よい理由はないだろう」\citep{ratchford13:_why_is_canad_media_still}。

さらにもう一点、用語について明確にしておく必要がある。
「セックスワーク」という言葉は、しばしば「売春」より広い意味で使われることがある。
たとえば、ポルノへの出演、ウェブカムを使ったパフォーマンス、ストリップショーなども「セックスワーク」に含まれることがある。
これらの活動は、実際に顧客とセックスをするわけではないため、「間接的セックスワーク」と呼ばれることもある。
本章の目的に照らして、本書では「セックスワーク」という用語を、顧客とセックスをする直接的な形態のものに限定して使用する。

セックスワークに適用される法律にはさまざまなモデルがある。
多くの国では、セックスを売る側あるいは買う側のいずれかを違法としている。
一方で、ドイツのようにセックスワークを合法化し、一定の規制を課している国もある。
数は少ないが、ネバダ州のように、セックスワーカーと顧客の双方に刑事罰を科さない地域もある。
また、いわゆる「北欧モデル」を採用している国も少数あり、このモデルではセックスを売ることは合法だが、セックスを買うことは違法とされる。
このアプローチは、刑事罰としては従来のものよりも同情的であるとされており、その効果については意見が分かれる。
しかし、その意図するところは、セックスワークを全面的に禁じる法と同じく、商業的セックスの全面的に禁止であり、それによって商業セックス取引を根絶することだ。
そのため、全面的な禁止とまったく同じ正当化根拠が必要となる。
本書では、北欧モデルについても詳しく議論する。

本書では、セックスワークを違法とすべき{\DDASH}北欧モデルであれ伝統的な形であれ{\DDASH}だと考える立場を「廃絶主義者」(abolitionist)と呼ぶ。
また、セックスワークは道徳的に許容可能であり合法であるべきだと考える立場を「容認主義者」(tolerationist)と呼ぶ。
(上で議論した)ポルノグラフィの場合と同様に、廃絶主義者は政治的スペクトラムを交差して存在しており、ラディカルおよびリベラルのフェミニストの多くも保守派と協力して合法のセックスワークに反対している。

私は、セックスワークの非犯罪化を容認主義者の目標として論じる。
一部の人は、「非犯罪化」(decriminalization)と「合法化」(legalization)を区別する。
残念なことに、これらの用語は一貫性を欠いたかたちで使われることがある。
薬物政策の議論においては、「非犯罪化」とは通常、薬物使用は依然として違法ではあるものの、使用者に対して刑事罰を科さないことを意味し、「合法化」は、薬物の販売および使用に関するすべての刑事法を撤廃することを意味する。
セックスワークに関しては、「非犯罪化」はすべての関連法の撤廃を意味し、「合法化」はセックスワークを許容しつつ一定の規制を課すことを指す場合が多い。
たとえば、ネバダ州では合法化アプローチが採用されている。
非犯罪化と合法化の区別は、政策上の重要な違いをもたらし、セックスワーカーに影響を与える可能性がある。
しかし、基本的な哲学的問題は同じだ。
それは、「国家は刑法を用いてセックスの商業的売買を禁止すべきか?」という問いだ\citep{tani15:_sex_worker_explain_differ_legal_decrim_prost}。

\subsection{セックスワークに同意することは可能か?}

リベラル派にとって、セックスワークの非犯罪化を支持すべき理由は表面的には明白だ。
それは、同意した成人同士の私的な活動だからだ。
しかし、これは常に成立しているわけではない。
セックスワークを強制されている人々が存在し、人身売買業者によって母国から連れ去られセックスワークを強制されているケースもある。
こうした強制を容認しようとする人はいない。
人身売買は恐るべき犯罪であり、それがおこなわれている場所では積極的に取り締まるべきだという点に異論はない。
また、未成年者が関与するケースも多数ある。
これもまた、けっして容認されるべきではないということで意見は一致している。

セックス産業における人身売買の被害者や未成年者の割合がどれほどなのか、またその両方である場合はどれほどかということについては重要ではあるが、論争のタネになっている問題だ。
反セックスワーク活動家は、時に正確性を欠いた誇張した主張をおこなって自分たちの主張を損ねていることがある。
たとえば、「売春を始める少女の平均年齢は12歳である」や「米国では毎年30万人に至る若年者が人身売買の危険にさらされている」といった主張だ\citep{hall14:_is_one_most_cited_statis}。
しかし、セックスワークに関与している未成年者の正確な数を把握することは極めて困難だ。
ニューハンプシャー大学の子供の犯罪被害研究センターによるファクトシートは、「科学的根拠に基づいた信頼できる推計が存在しない以上、研究者や作家、活動家は根拠のない推計を使用するのをやめ、実際の発生率は不明だと述べるべきだ」と結論づけている\citep{children08:_sex_traff_minor}。
セックス\ruby{人身売買}{トラフィッキング}に関しても同様だ。
国連の報告書によれば、世界全体で見た場合、この問題は深刻であり、とくに紛争地域では顕著だ\citep{drugs18:_global_repor_traff_person}。
問題の規模にかかわらず、人身売買や未成年者の搾取はけっして無視すべきではない。
しかし、容認主義者たちは、こうした誇張された主張がセックスワーク全体に対して現状よりさらに厳格な規制を正当化するために利用されてしまうことを懸念している。
また、彼らは、非犯罪化こそがこの問題への対処を容易にすると考えている(詳細は後述する)。

容認主義者たちは、成人たちがおこなうことが強制されたものでない場合には、その人々の選択を尊重すべきだと主張する。
ラース・エリクソンは、この主張を次のように端的に表現している。
「二人の成人が、性的活動に関する経済的取り決めに自発的に合意し、それが私的におこなわれるのであれば、そこになにか内在的に不正なものがあると主張するのは明らかに不合理だ」\citep[pp.338--339]{ericsson80:_charg_again_prost}。
キャロル・ペイトマンはこうした立場を「セックスワークの契約理論」と呼び、これを認めない立場をとるが、その理論を次のように適切に説明している。

\begin{quote}
契約理論論者は、売春者は一定時間にわたって特定の労働力を提供する契約を結び、その対価として金銭を受けとると考える。
売春者と顧客の間には自由な交換がおこなわれており、売春契約は他の雇用契約と極めて似ているか、または単に一例にすぎない。
契約という観点から見ると、売春者は自己の身体を所有する個人であり、市場においてその一部の利用権を契約によって提供する。
売春者は「自分自身を売る」のではなく、また性的部位を売るのでもなく、性的サービスの利用権を契約するにすぎない。
したがって売春者と他の労働者やサービス提供者の間にはなんら違いは存在しない。
\citep[p.191]{pateman88:_sexual_contr}
\end{quote}

しかし、廃絶主義者たちはこの主張を決定的なものとは考えない。
彼らは、明白な強制がなく、当事者が成人であったとしても、セックスワークに対する同意は有効ではないと考える強い理由があると考える。
合理的な成人の同意が無効となる要因は複数ありえるが、ここでは二つの点が重要だ。
第一に、個人の生活状況といった背景条件が、その人物が本当に同意できるのかということを疑わせる場合がある。
第二に、同意された活動の本性が、そもそも誰も同意を許されるべきではないものである可能性がある。
廃絶主義者たちは、セックスワークにはこの両方の要因が当てはまると主張する。

廃絶主義者たちは、まず、セックスワーカーは、直接的な強制がないとしても、あまりに状況が悪くて他に選択肢がないためにこの仕事を選んでいるのであり、したがってその同意は真正のものとは言えないと考える。
クリスタ・ウィクタリックはこれを「選択肢がない人々の選択」と呼んでいる\citep[p.63]{wichterich00:_global_woman}。デボラ・サッツはこれを「やけっぱちの交換」{\DDASH}つまり、なにか\ruby{まし}{リーズナブル}な選択肢が他にあるならば誰もおこなわない取引{\DDASH}だとする\citep[p.71]{satz95:_market_women_sexual_labor}。
メリッサ・ファーリーは次のように述べている。

\begin{quote}
自由に選択した結果として売春業界にいる女性はごくわずかだ。
ほとんどの女性にとって、売春は本当の選択ではない。
なぜなら、身体的安全や、買い手との平等な力関係や、代わりになる現実的な選択肢が存在していないからだ。
これらの条件がなければ、本当の同意はありえない。
\citep{farley13:_prost_liber_slaver}
\end{quote}

廃絶主義者たちは、多くのセックスワーカーが貧困状態にあること、彼女たちの多くが性的虐待の過去をもつこと、そして薬物依存率が高いことを指摘する。
また、多くのセックスワーカーがこの職業から抜け出したいと考えていることを示唆する統計にも言及する。
さらに、セックスワーカーの多くが、彼女たちを搾取し、望む以上に働かせ、セックス産業からの離脱を困難にする\ruby{売春斡旋業者}{ピンプ}などの影響下に置かれていることも問題視している。

第二に、廃絶主義者たちは、セックスワークは本質的に有害であるため、誰もこれに同意を許されるべきではないと主張する。
彼らは、セックスワークが他の仕事と異なる特性を持つことを指摘する。
第一に、セックスワークは身体的に極めて危険だ。
セックスワーカーは、客や仲介業者から頻繁に虐待されている。
身体的・性的暴行の割合は非常に高い。
またセックスワーカーは他の女性より最大100倍も殺害されるリスクがある\citep{salfati08:_prost_homic}。
セックスワーカーが警察の保護を受けることはほとんどなく、それどころか警官が加害者となることも少なくない。
また、性感染症のリスクも常に伴い、コンドームの使用では完全に防げない病気もある。
他にも濃厚接触により感染する病気は多い。

身体的危険に加えて、少なくともその部分的な影響としてセックスワーカーには、慢性的な心理的危害を被るリスクがある。
メリッサ・ファーリーは、9か国を対象とした調査データを引用し、セックスワーカーの3分の2がPTSD(心的外傷後ストレス障害)の診断基準を満たしていることを示している。
彼女は次のように述べている。
「この情動的苦痛の極端なレベルは、心理学者によって研究された中で最も深刻な障害を受けた人々{\DDASH}暴力を受けた女性、レイプ被害者女性、戦闘帰還兵、拷問経験者など{\DDASH}と同じだ」\citep[p.100]{farley18:_risks_prost}。

廃絶主義者たちは、セックスワークが桁はずれの害を引き起こすだけでなく、それは特別な種類の害をもたらすと考えている。
彼らは、セックスを売ることが、他の危険な職業には見られない形で個人の\ruby{自己}{セルフフッド}を侵害するものだと主張する。
ケイト・ミレットが著書『売春白書』(\emph{The Prostitute Papers})のためにインタビューしたセックスワーカーの一人は、次のように述べている。
「売春で最も辛いのは、セックスだけでなく、自分の人間性まで売らなければならないことです」\citep[p.84]{millett76:_prost_paper}。
私たちの身体の\ruby{統合性}{インテグリティ}や性的自律は、アイデンティティと特別密接に結びついているため、それらを金銭と引き換えにすることは、個人の「自己」に対して極めて深刻な侵害を構成する。
ペイトマンは次のように書いている。

\begin{quote}
女性であること……は性的な活動によって確認される。
したがって、売春婦が自分の身体の使用を契約によって提供する場合、彼女は極めてリアルな意味で自分自身を売っていることになる。
売春においては、女性の自己は、他の職業における自己が関与するのとは違った形でそれに関与している。
あらゆる種類の労働者は、多かれ少なかれ「自分の仕事に縛られている」かもしれない。
しかし、セクシュアリティと自己の間には統合的な結びつきがあるために、売春者は自己を守るため、本当の自分自身と性的使用の間に距離を取らなければならない。
\citep{pateman88:_sexual_contr}
\end{quote}

商業的セックスが伴う自己の喪失に加えて、もう一つの結果として、セックスワーカーが親密な関係を築く能力が損なわれることがある。
ファーリーは、元ストリッパーの言葉を引用しており、彼女はこの見解が多くの直接的なセックスワークをおこなっている人々の意見をも代表していると考えている。
「私は男から触られることに耐えられなくなりました。
どんな男でもです。
男に触られることは、労働と屈辱を、そして自暴自棄と絶望の悲しさと病んだ感じを表すものになってしまいました」\citep{farley13:_prost_liber_slaver}。
廃絶主義者たちは、性的サービスを売ることで、セックスワーカーたちは、人間にとって本質的な価値をもつ親密な関係を育む能力を犠牲にしてしまっていると主張する。

一部の廃絶主義者は、セックスワーカーの同意はけっして真に有効なものとは見なされえないため、セックスワークはその本質において性的暴行の一形態であると結論する。
ある反セックスワーク論説の見出しが述べているように、「売春は対価が支払われたレイプである」とされる\citep{raymond95:_prost_is_rape_thats_paid}。
元セックスワーカーのフィオナ・ブロードフットは、ジュリー・ビンデルに対して次のように語っている。
「男性たちが「売春がレイプを減らす」と主張するとき、彼らが本当に言っているのは、「売春女性をレイプするのは構わない」ということです。
私たちが客とのセックスを経験するのは、まさにそのようなものだからです。
売春はたしかにレイプです」\citep{bindel19:_real_face_prost}。

\subsection{セックスワークは他者に害を及ぼすか?}

禁止論者は、さらに、セックスワークがワーカーや顧客以外の他の人々にも間接的な害を及ぼすと主張している。
第一に、セックスワーカーを利用している人のパートナーに対する悪影響が挙げられる。
自分のパートナーがセックスを買っていることを知った場合、その人は相手が性的に不実であることに精神的苦痛を感じることになる。
また、相手がセックスワーカーを通じて感染した性感染症に自分が感染してしまうリスクもある。

セックスの取引は地域社会にも害を及ぼす可能性がある。
カナダ司法省が作成した報告書は、セックスワークを違法とするさまざまな理由を説明しているなかで、セックスワークがその近隣地域にどのような影響を与えるかをリストアップしている。
この報告書によると、セックスワークは以下のような問題を引き起こす可能性がある。

\begin{quote}
関連する犯罪性のある事柄として、次のようなものがある。
人身売買や薬物関連犯罪。
セックスを商品として販売する現場に子供がさらされること、さらには搾取される人生へと引き込まれるリスク。
住民に対する嫌がらせ。
騒音。
交通妨害。
不衛生な行為(使用済みのコンドームや薬物器具といった危険な廃棄物の放置を含む)、子供たちの勧誘。
\citep{department14:_techn_paper}
\end{quote}

廃絶主義のフェミニストは、セックスワークが社会におけるすべての女性の地位に影響を及ぼすと主張する。
彼女たちは、セックス産業が家父長制というより広い構造と密接に結びついていると考えており、その観点から問題を理解する必要があると説く。
廃絶主義者は、セックスワーカーの約70~80%が女性であり、顧客の大多数が男性であることを指摘している。
デボラ・サッツは次のように述べる。
「男性も売春をすることはありえるし実際にそうしているが、男性集団全体が売春によって否定的な影響を受けることはない。
個々の男性が売春によって貶められることはあるかもしれないが、男性という集団はそうではない」\citep[p.149]{satz95:_market_women_sexual_labor}。
フェミニスト廃絶主義者は、セックスの商業的取引が象徴的な意味を持ち、それによってすべての女性が社会的に劣位にあることが示されると主張する。
カリ・ケスラーは、それを「家父長制的男性特権の究極の体現」 と呼んでいる\citep[p.19]{kesler02:_is_femin_stanc_suppor_prost_possib}。
キャロル・ペイトマンは次のように説明する。

\begin{quote}
女性の身体が資本主義市場において商品として売られるとき、元々の契約の条件が忘れ去られることはない。
つまり、男性のセックスの権利という法が承認され、男性は女性の性的\ruby{支配者}{マスター}として公的に承認されることになる{\DDASH}これこそが売春が不正である要点だ。
\citep[p.208]{pateman88:_sexual_contr}
\end{quote}

フェミニスト廃絶主義者は、社会がセックスワークを道徳的または法的に容認すれば、それは男性に対しては「女性を\ruby{セックスのための物体}{セックスオブジェクト}として扱ってよい」というメッセージを送ることになり、また女性に対しては自分たちをそのようなものとして見るべきだというメッセージを送ることになると主張する。
そして、この影響は、すべての女性の人生のあらゆる側面に及ぶ。
「もし売春が合法であり、男性が女性の身体を何の罰則もなく買うことができるのであれば、それは女性の極端なセクシュアル化を意味する」と、フェミニスト人権団体Equality Nowのグローバル・エグゼクティブ・ディレクターであるヤスミーン・ハッサンは\emph{The New York Times}に語っている。
「女性は性的なモノとされます。
それが、働く女性たちの評価にどのような影響を与えるかを考えてみてほしいのです。
そして、女性がお金で買えるセックストイであるならば、結婚やその他の場面において、男性と女性の関係にどのような影響を与えるかを考えてみるべきです」\citep{bazelon16:_shoul_prost_be_crime}。
廃絶主義者たちは、売春が合法とされている地域では、それは性的暴力や女性の搾取の\ruby{常態化}{ノーマライゼーション}に寄与している主張している\citep[p.3]{waltman10:_prohib_purch_sex_sweden}。

\subsection{セックスワークと商品化}

一部の哲学者は、性的サービスを売買すること自体に内在的な問題があり、それを容認することがセックスに対する問題ある見方を助長すると主張する。
彼らは、セックスワークが「商品化」(commodification)に該当すると考える。
商品化とは、ある特定の財や活動の領域を市場の規範に従属させることを指す。
こうした哲学者たちは、市場に組み込まれることで根本的に価値が損なわれてしまうものが存在すると論じる。
マイケル・サンデルは次のように述べる。
「人生における価値あるものごとに値札をつけることは、それらを堕落させることだ。
市場は単に財を分配するだけでなく、そこで交換される財に対する特定の態度を表現し、促進するからだ」\citep[p.9]{sandel12:money_cant_buy}。
つまり、私たちがそうした価値あるものをいったん売買の対象にしてしまうと、私たちはそれをそれ自体で価値あるものとしてではなく、道具的に価値のあるものとして、つまり自己利益を満たす手段としてのみ価値あるものとして認識するようになってしまうのだ。
エリザベス・アンダーソン\ig{Elizabeth Anderson}も同様の立場をとる。

\begin{quote}

ある慣行があるものを商品として扱っているとは、そのあるものの生産と流通と消費が市場特有の諸規範によって支配されているということだ。
……関係している当事者たちはそれぞれ自分の利益を追求しているとされ、法律によって最低限要求されている範囲を除いては、相手や第三者の利益を考慮することは求められない。
\citep[pp.19--20]{anderson00:_why_commer_surrog_mother_uneth}

\end{quote}

サンデルの著書『それをお金で買いますか』\citep{sandel12:money_cant_buy}では、金銭との交換というまさにその行為によって価値が損なわれてしまう行為の事例が数多く挙げられている。
たとえば、現在では野球選手がベースボールカードにサインをすることで報酬を受け取るようになっており、さらにそれを業者が転売して利益を得ている。
サンデルは、こうしたサイン入りカードを所有することは、かつてのように純粋な喜びをもたらさなくなっていることを指摘する。
昔ながらの方法、すなわち熱心なファンが試合後に球場の外で選手が出てくるのを待ちサインをしてもらうという手段で得たカードとは、持つ意味が大きく異なるというのだ。

アンダーソン\ig{Elizabeth Anderson}は、性的サービスの売買においても同様の価値の低下が生じると主張している。
彼女は、人間のセクシュアリティは、パートナー同士が互いの欲求を認識し合うことで成り立つ共有財であり、本来的に、または本来そうあるべき関係として、相互的なものであると考えている。
アンダーソンは、セックスを商品化することで、この相互性が完全に破壊されてしまうと指摘する\citep[p.154]{anderson93:_value_ethic_econom}。
したがって、彼女は、セックスはマーガレット・ジェーン・ラディンが「市場不可譲」(market-inalienable)と呼ぶもの、つまり人々が売買することを認めるべきではないものだと言う\citep{radin87:_market_inalien,radin00:_contes_commod,satz10:_why_some_thing_shoul_not_be_sale}。
アメリカ合衆国最高裁は1973年、ポルノ映画を規制する政府の権利を擁護する際に、同様の見解を示した。
「人間生活において繊細で鍵となる関係であり、家族生活、地域社会の福祉、人間の性格形成において中心的な役割を果たすものが、粗野なセックスの商業利用によって貶められ歪められる可能性がある」と判示している
(\emph{Paris Adult Theatre Inc. v. Slaton}, 413 U.S. 49, 63, 1973)。

\subsection{セックスワーク擁護論}

セックスワーク容認主義者は、セックスワークのコストを論じる際には、その利益についても考慮すべきだと主張する。
まず単純な事実として、人々がセックスに対して金を払うのは、たぶんそれを楽しいと思っているからである。
市場取引の多くと同様に、当事者が自由に取引をおこなうならば、それは双方がお互いに、相手が提供を申し出ていることに価値を見出していることを意味する。
こうして全体的な効用が増加する。

それに加えて、経済的利益だけでなく、社会的な利点もあると容認主義者は指摘する。
セックスワーク合法化運動団体 COYOTE (Call Off Your Old Tired Ethics) の創設者であるマーゴ・セントジェームズは、セックスワークが持つ治療的な役割に着目する。
社会学者リン・チャンサーは、セックスワーカーへのインタビューを通じて、多くのワーカーが、社会は自分たちの仕事をセックスセラピーの一形態と見るべきだと考えていると述べていると報告している(Chancer, 1993, p.161; cf. Shrage, 1994, p.86)。
\nocite{chancer93:_prost_femin_theor_ambiv}\nocite{shrage93:_moral_dilem_femin}
Skyと名乗るセックスワーカーは、\emph{In These Times}に対して、そのような側面が自分の仕事の重要な部分だと述べている。
「お店で私たちが一日中セックスしているという誤解が広まっていますが、実際には深刻な感情的問題を解決するために多くの人が訪れています。
それが私たちの仕事の大きな割合を占めているのです」\citep{weisman18:_when_sex_worker_do_labor_therap}。

セックスワークは、パートナーを見つけるのが困難な人々に性的満足を得る機会を提供している。
たとえば、イギリスには TLC Trust というウェブサイトがあり、障害のある人々が有償の性的サービスを見つけることを支援している\ig{\footnote{\url{http://tlc-trust.org.uk/}.}}。
ジョセフ・フィッシェルは次のように述べる。

\begin{quote}

性的サービスのプライベートな購入(セックスの購入を含む)は、一部の障害者にとって、そうでなければ得られないような、またはひどく困難な可能性のある性的快楽へのアクセスを提供してくれる。
同様に重要なのは、このような有償の性的経験が、障害のある人々の自信を高める助けとなることだ。
障害のある人々は、自らを性的な存在として認識するようになるだけでなく、「セックスをする」ことが可能な存在として自己認識するようになる。
それは性器挿入の有無にかかわらない。
\citep[p.220]{fischel18:screwconsent}
\end{quote}

ポリオで麻痺が残っている詩人・ジャーナリストのマーク・オブライエンは、セックス\ruby{代理人}{サロゲート}と会った経験について次のように書いている。
「私は、単に誰かとセックスをするだけでなく、自分の人生の責任をもち、自分を決定を下すことができる者として信頼することで、自分を「できそこないの優柔不断な間抜け」とみなす見方を変えられることを知った」\citep{obrien90:_seein_sex_surrog}。

人々が孤立した場所で働く社会や、男女比が偏った社会においては、有償のセックスが親密さを得る唯一の手段となることがある。
さらに、関係を持っている人々にとっても、セックスワーカーを雇うことは、関係外での非商業的なセックスよりも、パートナーにとって脅威が少ない形で、新規性や多様性を提供する手段となりえる\citep{mesko12:_effec_prost_stabil_roman_relat}。

容認主義者は、廃絶主義者の中心的な主張の一つ{\DDASH}すなわち、セックスワーカーが強制されていない場合でも、その同意は有効とは見なされない{\DDASH}という考えを否定する。
容認主義者は、この主張がセックスワーカーの主観的な経験を無視していると論じる。
多くのセックスワーカーは、自分は自らの選択の結果としてその仕事をしていると考えているからだ。
また、容認主義者は、廃絶主義者がセックスワーカーの人生の選択を真剣に受け止めようとしないことは、\ruby{おせっかい}{パトロナイジング}性差別的な態度の表れだと批判する。
2000年に国連が採択したパレルモ議定書では、人身売買の判断において、売買されている当事者の同意は「\ruby{問題に無関係}{イレレバント}」とされた。
これに対し、セックスワーカーを代表する団体Human Rights Caucusは次のように反論している。
「この議定書は、女性を「保護する」という名目で、女性を子供と同じ扱いに貶めるような保護的な態度をとるべきではない。
このような立場は、歴史的に見ても、女性を「保護する」という名目のもとで、彼女たちが自らの権利を行使する能力を奪うことにつながっててきただけだった」\citep{caucus99:_recom_commen_draft_protoc_combat}。

容認主義者は、セックスワークが危険であり、場合によっては\ruby{屈辱的}{デグレーディング}なものでありえることを否定しない。
しかし、彼らは、そのリスクは、他の多くの低技能職のリスクと比べて特別に高いとは言えないと主張する。
セックスワーカーであるジュノ・マックとモリー・スミスは次のように述べている。
「労働というものは往々にしてひどいものだ。
特に低賃金で社会的評価の低いものほどそうだ」\citep[p.43]{mac18:_revol_prost}。
また、二人は、セックスワーカーのニッキー・ロバーツ\ig{Nickie Roberts}の言葉を引用している。
「うす汚い工場で吐き気がするほど安い賃金で働いたことが、私の人生で最も屈辱的で搾取的な仕事でした……他人のクソを片付けることのどこに解放があるっていうの?」(ibid., p.39)。
さらに、多くの職業には深刻な身体的リスクが伴う。
たとえば、アメリカの食肉加工工場は、労働者の負傷率の高さで悪名高く、全国で週に2件の割合で身体切断事故が発生している\citep{wasley18:_two_amput_week}。

廃絶主義者が主張するような「やけっぱちの交換」としてのセックスワークという見方に対し、容認主義者は、セックスワークを選択する人々には合理的な理由があると論じる。
まず、セックスワークは誰でもできる仕事だ。
その参入障壁は、あらゆる職業の中でも最低レベルだ。
正式な資格は不要であり、志願者たちは雇用主(その一部は人種差別的または性差別的かもしれない)のご機嫌を伺う必要もなく、労働組合に入れてもらう必要もない。
ただ、自分をアクセスしてもらえる状態にするだけでよい。
また、セックスワーカーの賃金は特に低いわけではないことも指摘されている。
経済学者スティーヴン・レヴィットとスディール・アラディ・ヴェンカテッシュの研究によると、シカゴでは、最も不利な立場にあるとされる路上のセックスワーカーでさえ、時給25~30ドルを稼いでおり、これはシカゴの最低賃金の4倍に相当するという\citep[p.26]{levitt07:_empir_analy_street_level_prost}。
さらに高い収入を得ているセックスワーカーも多く存在する。
サンフランシスコでセックスワーカーを対象に広範なフィールドワークをおこなったエリザベス・バーンスタインは、次のように述べている。

\begin{quote}
すべてのストリートの売春者(そして一般にほとんどの女性売春者)にとって重要なのは、それほどの賃金を得られる職業は他にないという点だ。
実際、最も成功した女性専門職ですら、売春者の時給に匹敵する収入を得るのは難しい。
ましてや、彼女たちと同じ社会階級、人種、教育的背景を持つ女性にとっては、最低賃金の仕事か結婚が最も現実的な選択肢となる。
\citep[p.104]{bernstein99:_whats_wrong_prost}\ig{Elizabeth Bernstein}
\end{quote}

セックスワークは、低技能職にはほとんど見られない柔軟な労働時間やその他の自由を提供する。
クレアという名のイギリスの若いセックスワーカーは、\emph{VICE}の取材に対して次のように語っている。
「セックスワークは、私が知っている他のどの仕事よりもはるかに良いもので、収入も高く、私の時間にも合っています。
大学の友人たちは、最低賃金以下で10時間シフトをこなしていますし、他の友人はゼロ時間契約で働いていて、その月に何時間働けるかわかりません」\citep{mcintyre15:_what_its_like_pay_your}。
また、セックスワーカーの職業満足度が、他の低技能職と同程度であることを示唆する研究もある\citep{bilardi11:_job_satis_femal_sex_worker}。
クラークという名の男性セックスワーカーは、研究者たちに対して次のように語っている。

\begin{quote}
二年前僕はフルタイムの仕事に就いていました。
でも、それがうまくいかなくなって、すぐに戻れる最も便利な選択肢がこれでした。
自分のスケジュールを自由に決められるという点で便利だし、時間の使い方も調整しやすい。
時給で考えれば、それほど悪い仕事ではないです。
\citep[p.11]{curtis19:_we_are_naked_waitr_who_deliv_sex}
\end{quote}

一部の容認主義者は、セックスワークが楽しいものでありえるし、むしろエンパワーメントにつながる場合もあると主張している。
Podcast番組 The Whorecast のホストであるスージー・Q・ジェームズは次のように語っている。
「セックスワークは、ある人にとっては冒険であり、エンパワーメントの瞬間でもあります。
それは、人生のなかの不思議な時間であり、自分自身について何かを学ぶ機会になることもあります{\DDASH}それが良いことであれ、悪いことであれね」\citep{richardson14:_is_sex_work_empow_enslav}。
また、パフォーマンスアーティストのアニー・スプリンクルは、自身のセックスワーカーとしての経験についてこう述べている。
「私はセックスを楽しんでいました。
「仕事では絶対にオーガズムを感じない」なんて決めておくタイプではありませんでした。
実際、仕事中にオーガズムを感じることもありました{\DDASH}もちろん、すべての男性相手ではなかったけれど」\citep[pp.42-43]{bell95:_whore_carniv}。
先に言及したクラークは次のように語っている。

\begin{quote}
僕は人間が好きだし、本当のところセックスも好きです。
もちろん、セックスをする相手全員に性的魅力を感じるわけではありません。
実際、僕には自分で「バー・スケール」と呼んでいる尺度があって、「バーで会ったらこの人をナンパするだろうか?」とよく考えています。
95%の人については「ノー」です。
でも、たまにはこう思うことがあります。
「すごい、今のセックスでお金をもらったんだ。
最高じゃないか! あいつは魅力的だったし、そのうえお金まで手に入ったよ」。
\citep[p.11]{curtis19:_we_are_naked_waitr_who_deliv_sex}
\end{quote}

廃絶主義者は、どれほどのセックスワーカーがこの仕事をポジティブに経験しているのか疑問視している。
彼らは、そんな人がいるとしても少数派にすぎないと主張し、またセックスワーカーの労働環境には大きな差があることを指摘する。
そうした労働環境の差は、人種、社会階級、年齢、教育水準といった要因によって大きく左右される。
非西洋諸国では、先進国の経済圏よりもはるかに厳しい状況に置かれている場合が多い(cf. Bernstein, 1999, p.110; Rajan, 1996, p.128)。
\ig{Elizabeth Bernstein}\nocite{bernstein99:_whats_wrong_prost}\nocite{rajan96:_prost_quest}
容認主義者の多くでさえ、セックスワークがほとんどのワーカーにとって楽しく、またエンパワーメントにつながるという考えを積極的に推奨することには慎重だ。
しかし、彼らは、セックス産業が合法であるべきかを検討する際には、この問題は本質ではないと主張する。
なぜなら、私たちは他の職業については、その仕事が報われるものか、エンパワーメントをもたらすものか、などということを基準にしてその存在を許可しているわけではないからだ。

多くの容認主義者は、セックスワークが非常に危険になりえるからこそ、これを他の職業と同様に承認すべきだと主張する。
そうすることで、セックスワーカーにも他の産業の労働者と同様の保護を与えることが可能になる。
マックとスミスは次のように述べている。
「私たちは罰されずに生計を立てるに値すると論じるために、セックスワーカーたちがセックス産業を擁護する必要などないはずだ。
私たちは、自分が安全に働くために、自分の仕事が社会にとって内在的な価値を持つことを証明する必要などない」\citep[p.55]{mac18:_revol_prost}。
容認主義者によれば、セックスワークのリスクは、違法にされていることによってさらに深刻化している。
たしかにセックスワーカーたちが直面しそうな危険を完全になくすことは不可能かもしれない。
しかし、非犯罪化することで、他の職業と同じ保護を受けられるようになり、一部のリスクについてはそれを軽減できるだろう。
マックとスミスが述べているように、「商業的セックスが犯罪とされている限り、セックスワーカーに\kenten{労働者の権利はありえない}。
しかし、商業的セックスが非犯罪化されれば、セックスを売る人々も労働関連法の保護を受けることができる」(ibid. p.51)。
現在の状況では、加害的な客がセックスワーカーに暴行したり、あるいは単に支払いを拒否したりしても罪に問われることなくことが済んでしまう。
また、セックスワーカーは、自分が搾取されていると警察に通報すれば、自分自身が逮捕やハラスメントの対象になりかねないと知っているため、\ruby{売春斡旋業者}{ピンプ}たちは勝手放題に搾取をおこなっている。
非犯罪化を推進する弁護士は次のように指摘している。
「セックスワークが違法である限り、ワーカーは安全を脅かされても警察に行くことはできず、搾取的な労働環境や賃金未払いに対しても何の対処もできない」\citep{shugerman17:_prost_could_be_legal_calif}。
実際のところは、警察自体がセックスワーカーの安全を脅かす存在となっている。
バーンスタインはこう述べている。
「路上で特に印象的なのは、女性たちは客に対しては比較的恐れを感じていないのに対し、警察による(時に露骨な、そしてしばしば暗黙の)性的な支配に対しては、極めて萎縮しており服従的であることだ」\citep[p.108]{bernstein99:_whats_wrong_prost}。
\ig{Elizabeth Bernstein}

セックスワークが人目につかない形でおこなわれると、セックスワーカーたちへの健康情報や治療の提供が難しくなる。
\emph{The Lancet}の分析によると、「セックスワークの非犯罪化は、HIV流行の進行を抑制する上で最も大きな効果をもたらしうる」とされており、その理由として、コンドームや医療処置へのアクセス向上が挙げられている(Bazelon, 2016での引用)。
\nocite{bazelon16:_shoul_prost_be_crime}
国連合同エイズ計画(UNAIDS)や世界保健機関(WHO)は、以前から公衆衛生の観点からセックスワークの非犯罪化を求めてきた。
セックスワークが犯罪であることは、公衆衛生の専門家がセックスワーカーと接触することを困難にし、またセックスワーカー自身が医療機関を恐れる要因となる。
あるトロントのセックスワーカーが研究者たちに語っているように、セックスワークが違法とされていることから、セックスワーカーたちは医療従事者が自分を警察に通報するのではないかと恐れている。
彼女は次のように述べている。
「私はレイプされましたが、病院で責められるのではないか、そして警察に通報されるのではないかと怖かったのです」\citep{neal14:_street_based_sex_worker_needs_asses}。

セックスワークの犯罪化が人身取引や未成年の搾取といった問題への対応を困難にしている可能性もある。
Human Rights Watchは次のように述べている。

\begin{quote}
セックスワークと人身売買や児童の性的搾取といった犯罪を明確に区別する法律は、セックスワーカーと犯罪被害者の双方を保護するのに役立つ。
セックスワーカーたちは、人身売買や児童の性的搾取といった重大犯罪に関する重要な情報をもっていることがある。
しかし、彼女たち自身の仕事が犯罪として扱われる限り、その情報を警察に提供するしても安全だと感じることは難しい。
\citep{watch19:_why_sex_work_shoul_be_decrim}
\end{quote}

容認主義者は、オープンで適切に規制された産業が確立されれば、強制的にセックスワークに従事させられている人々や、未成年のうちにこの業界に引き込まれた人々を特定し、そこから救出することがはるかに容易になると主張している\citep{albright17:_decreas_human_traff_sex_work_decrim}。

容認主義者はさらに、セックスワーカーが直面する多くの被害は、彼らが受ける\ruby{烙印}{スティグマ}に根ざしており、そのスティグマが虐待や搾取を受けやすくしており、さらにそれが犯罪化によって悪化させられていると主張する。
セックスワーカーたち、過去にセックスワーカーとして働いていた人々を含め、しばしば他の仕事から解雇されてしまうことがあり、また子供が学校から退学させられたり、親権を失ったりすることもある。
さらに、住居の確保が困難になり、医療へのアクセスも制限され、社会的孤立に苦しむことが多い\citep{stardust17:_stigm_sex_work_comes_high_cost}。
こうしたスティグマは長い歴史をもつ社会的態度に深く根ざしているものだが、犯罪化によってさらに悪化している。
容認主義者は、こうした問題に対処することは、職業としてのセックスワークが違法である限り不可能だと論じている。

廃絶主義者は、セックスにともなう親密さが個人のアイデンティティの感覚と結びついていることを理由に、セックスワークは他の種の労働よりもダメージを与え\ruby{屈辱的}{デグレーディング}であると主張する。
しかし、容認主義者はこの区別の妥当性に疑問を呈する。
たしかに、ほとんどの仕事は、従業員が親密な身体的境界を侵害されることを求められない。
しかし、一部の仕事ではそれが求められる場合もある。
マーサ・ヌスバウムは、婦人科検診の模擬患者として働く俳優の例を挙げている\citep{nussbaum98:_wheth_reason_prejud}。
その模擬患者は、医学生たちが繰り返し、かつ長時間にわたって婦人科検診を実施できるようにするために雇われている(実際、本書の著者の友人の一人も、この種の仕事で何度も収入を得たことがある)。
しかし、このような労働形態に関して(確かに特殊ではあるが)、その同意の有効性を疑問視する人はほとんどいない。

容認主義者はまた、セックスワークに従事するすべての人にとって、セックスを売ることが特別な侵害や喪失として経験されるとは限らないと主張する。
単に多くのセックスワーカーは、自分たちのセクシュアリティをそのようには捉えていないということだ。
セックスワークの擁護者ではないものの、サッツは次のように指摘している。

\begin{quote}
人々が自分の性的能力に対してもつ感覚は、自分の身体の各部位に対してもつ感覚よりもはるかに多様だ。
ある人にとって、セクシュアリティは他者との恍惚としたコミュニケーションであり、別の人にとっては、単なるスポーツや気晴らしにすぎない。
ある人は、他者から性的に利用されることに同意することを楽しめるかもしれないし、あるいはそれは十分な報酬によって補えると感じるかもしれない。
またたとえば同じ人物であっても、セックスはさまざまな経験の源となりえる。
\citep[p.71]{satz95:_market_women_sexual_labor}
\end{quote}

容認主義者は、廃絶主義者たちは暗黙のうちに女性の純潔という時代遅れの概念に依拠しており、それを再生産していると主張する。
すなわち、廃絶主義者たちは、女性の純潔は保護されるべきものであり、奔放な性的行動によって汚されるものだと考えてしまっている。
そしてこうした見方自体が問題だ、と。
また、ヌスバウムは、セックス産業の外にいる多くの人々も、自分のアイデンティティと密接に結びついたサービスを売っているではないかと指摘する。
彼女が挙げている例は大学教授だ。
大学教授はたしかに、自分自身の最も深い部分にある思考や信念を商品化することで報酬を得ている。
彼女は次のように述べている。
「大学教授たちが(たとえその教授がその時に最高額の給料を払ってくれる大学を選んで働いているとしても)、自分の知性を疎外している、あるいはその思考を商品化しているなどといったことを私たちは考えない{\DDASH}あるいは特定の学会や論文集のために論文を書く場合であってもそうは考えない{\DDASH}という事実は、売春者についても同様の結論を安易に導き出すことに対して慎重であるべきことを示唆している」\citep[p.704]{nussbaum98:_wheth_reason_prejud}。

先に、廃絶主義者の中には有償セックスをレイプと同一視する人もいると指摘した。
しかし、ジュリア・オコンネル・デヴィッドソンは、この主張がもたらす論理的帰結に注目している。
それは、セックスワーカーが法的保護を受ける権利を事実上否定されるということだ。
彼女は次のように述べている。
「もし売春がレイプであるならば、論理的に考えれば、売春者はレイプすることを公に許可されている女性として定義されることになる。
そして、実際に世界中の多くの警察官、裁判官、法学者たちがこの立場をとっており、売春者として働く女性がレイプの被害者になりえるという事実を認めようとしない」\citep[p.122]{davidson98:_prost_power_freed}。
セイディ・スライフォックスと名乗るオーストラリアのセックスワーカーは次のように語っている。

\begin{quote}
もしそもそも同意がないのなら、条件つきの同意も、撤回可能な同意も、そして侵害されたと主張しうる同意もありえません。
しかし、こうした同意はセックスワーカーに限らず、誰もが切実に必要としているものです。
もし私が「イエス」と言えないのなら、「ノー」と言うこともできないし、「イエス、ただし……」や、「これには同意するが、あれには同意しない」とも言えません。
私が「ノー」と言う権利をもつことこそが、労働現場での性的暴行から私を守るのです。
私は、顧客と会う前に、はっきりと何がOKで何がNGなのかを伝えることができています。
もしその条件が破られた場合には、私はすぐにその場を離れて同僚のいる安全な場所に帰ることができるという確信を持って仕事に行くことができています。
\citep{slyfox17:_sex_worker_consen}
\end{quote}

これまで見てきたように、廃絶主義者はセックスワークが社会全体に与える影響について主張をおこなっている。
すなわち、セックスワークは地域社会に害を及ぼし、象徴的にすべての女性を貶め、セックスそのものを商品化しているというものだ。
しかし、容認主義者は、地域社会に対する二次的危害の多くは、セックスワークが犯罪とされていることによって引き起こされていると主張する。
反セックスワーク法は、女性が広告を出したり、性風俗店やその他の安全な環境で働いたりすることを妨げている。
その結果、彼女たちは路上での労働を余儀なくされ、特定の地域コミュニティに流入することになり、これがカナダ政府の報告書で指摘された多くの社会的害悪の原因となっている。
また、ジョン・ローマン\ig{John Lowman}は、これらの影響を軽減するためにコミュニティがとりうる方法は、犯罪化以外にもあると主張する。
彼は次のように述べている。
「他のビジネスについては、業界を規制している法律群があり、それが各種のビジネスが社会にとって迷惑なものになることを防いでいる。
売春を刑法から除外したとしても、自治体は他の数多くの法律を活用してそれを適切に管理することができる」\citep{makin09:_pickt_factor}。

容認主義者は、セックスワークが本質的に、キャスリーン・バリー\ig{Barry}の言う「女性抑圧の制度的、経済的、そして性的モデル」\citep[p.24]{barry95:_prost_sexual}。
であるという主張にも疑問を呈する。
彼らは、セックスワークがより広い社会的文脈の中でおこなわれていることを指摘し、C・ハイケ・ショッテンの言葉を借りるならば、「セックスワーカーに、セックスやジェンダー関係の意味を決定する責任を一手に担わせるべきではない」と主張する\citep[p.223]{schotten05:_men_mascul_male_domin}。
マーサ・ヌスバウムは、女性の抑圧という文脈においては、結婚こそがセックスワークと同等以上の象徴的な重要性をもっていると指摘している。
彼女は次のように述べている。
「この制度(結婚)は、歴史的に見ても、男性の支配を表現し、それを強化してきたものであり、その影響はセックスワークよりもはるかに大きい」\citep[p.719]{nussbaum98:_wheth_reason_prejud}。
また、清掃業のように、主に女性が従事し、セックスワークと同等またはそれ以上に貶め的とされる労働も存在する。
これらの仕事もまた、少なくとも同程度に家父長制の維持に寄与していると言えるかもしれないが、犯罪化はされていない。
マックとスミスは、南アフリカのセックスワーカーのドゥドゥ・ドラミニの言葉を引用している。
彼女は、むしろ清掃業の単調さと屈辱から逃れるためにセックスワーカーになったと語っている。

\begin{quote}
私はそれまでケープタウンで他人のクソみたいに汚ない家を掃除してたのよ。
いろんな家で洗濯もしたわ。
朝早くから起きて、窓を開けて、掃除して、料理して、他人の子供たちのためにポリッジを作って、他人の子供を学校に送ってアイロンがけしてたのよ。
ただ寝る場所と一皿の飯のためにね。
タバコ一本すら恵んでもらえなかったわ。
だから、もうそんな生活にはうんざりだったのよ。
\citep[p.49]{mac18:_revol_prost}
\end{quote}

一部のフェミニスト容認主義者は、セックスワーカーは家父長制の抑圧の象徴であるどころか、むしろ家父長制的な規範への挑戦者となりえると主張している。
彼女たちは、セックスワーカーがセックスに対して金銭を要求することで、女性は無償で男性に性的に応じるべきだという期待に反抗していると論じる。
アン・マクリントックは次のように述べている。
「社会がセックスワーカーを悪魔化するのは、彼女たちが、女性が要求してよい金額以上を求めているからです。
無料が当然だと男たちが思っているサービスに対してね」\citep[p.1]{mcclintock93:_sex_worker_sex_work}。
セックスワーカーのエヴリンは、エリザベス・バーンスタインに対して次のように語っている。

\begin{quote}
私は長い間、つきあっていた彼氏とセックスをしてきました。
だって、そうするものだってことになっていたからです。
でも、楽しくありませんでした。
ただの決まりきった仕事みたいなものでした。
……だから、セックスワークも基本的には同じことだけど、違うのは、お金をもらえること、何かを得られることです。
私は、夫のためにセックスをしてさらにご飯を作らなければならない女性たちよりも、ずっと独立してます{\DDASH}女たちは、そういうことをずっとやっているのよね。
\citep[p.106]{bernstein99:_whats_wrong_prost}\ig{Elizabeth Bernstein}
\end{quote}

マーゴ・セントジェームズは、セックスワーカーを「唯一の解放された女性」とまで呼んでいる。
彼女は次のように述べている。
「私たちは、男が女とファックするのと同じだけの人数と、絶対的な権利をもってファックできる唯一の存在なのよ」\citep[p.84]{james87:_reclam_whores}。
また、ピーター・フレイズは、セックスワークが、彼の言う「プライベートでモノガミー的なブルジョワ的セクシュアリティの理想」に対する挑戦となりえると論じている\citep{frase12:_probl_sex_work}。

容認主義者はまた、現在のセックスワークの社会的意味に懸念があるとしても、セックスワーカーが活動する環境を変えることで、その社会的意味自体を変えることができると主張する。
キャロル・クイーンは、よりセクシュアリティを肯定する文化(セックスポジティブ文化)があれば、セックスワークを屈辱的なものとは見なさなくなるだろうと述べている。
彼女は次のように語る。

\begin{quote}
セックスワーク・コミュニティの活動家として、私はこれまでに100人以上の売春者、同じくらいの数のエキゾチックなダンサー、数十人の\ruby{女王様}{ドミナトリックス}、そして多数のモデルやポルノ女優に会ってきた{\DDASH}これは、ほとんどの反セックスワーク活動家たちよりずっと多い数であり、おそらくセックス研究者の多くをも上回る数だ。
これらの人々のうち、自尊心を失うことなく充実した生活を送っている人々と、セックスワークを困難または有害な職業選択と感じている人々を区別する際に、際立って重要な要素が一つあった。
それは、前者の多くが十分な性に関する知識を持ち、セクシュアリティを肯定的に捉えているということだ。
\citep[pp.128--129]{queen97:_sex_radic_polit_sex_posit}
\end{quote}

彼女は、社会におけるセクシュアリティに対する否定的な態度を克服すれば、平等主義的でジェンダーにとらわれない性的市場が生まれる可能性があると述べている。
そのような市場では、あらゆるジェンダーの人々が、恥や汚名を感じることなく、性的サービスを売買できるようになる。
そうなれば、セックスワークが家父長制の抑圧の象徴として持つ力も、もはやそれほど強いものではなくなるだろう。

\subsection{北欧モデル}

一部の人々は、妥協案として「北欧モデル」と呼ばれる制度を支持している。
このモデルでは、セックスを買うことは犯罪とされるが、売ることは犯罪とされない。
彼らはこれを、女性を保護しつつ、犯罪化による害悪を軽減する中間的な道として提案している。
北欧モデル情報ネットワーク(Nordic Model Information Network)という学術研究者の連合体の一員であるミーガン・タイラーは、この法律について次のように述べている。
「この法制度は、すべての売春者を非犯罪化するという法的フレームワークに支えられている。
つまり、売春に従事する人々に対しては一切の刑事罰を科さないが、セックスの購入、\ruby{売春斡旋}{ピンプ}、および性風俗店の経営はすべて犯罪とされる」\citep{woodward16:_calls_austr_adopt_nordic_model_prost}。

すでに述べたように、北欧モデルの基本的な目的は犯罪化と同じく、セックス産業の規模を縮小し、人々がセックスを売買することを防ぐことにある。
したがって、この制度の正当性は、セックスワークがさまざまな形で有害だという廃絶主義的な立場に依拠している。
支持者たちは、北欧モデルの導入によって、セックス産業の規模を縮小しつつ、すでに業界にいる女性たちを刑事罰から保護できると主張している。
この制度により、女性たちは警察の保護を受けたり、医療サービスやその他の支援を受けたりすることが可能になり、目の前に直面している危険を軽減できるという。
2010年のスウェーデン政府の報告書によれば、同国が「セックス購入禁止法」を導入して以来、路上で働く売春者の数は半減し、その代替としてインターネット上でセックスを売る人々が増加することもなかったとされている\citep{sweden10:_repor_sou}。
また、北欧モデル情報ネットワークによれば、この法律はスウェーデン国内の性的人身売買の件数も減少させたとされている\citep[p.25]{commons16:_prost}。

しかし、容認主義者はこのアプローチを拒否する。
Human Rights Watch は、北欧モデルについて次のように述べている。

\begin{quote}
北欧モデルは、一部の政治家にとって魅力的な妥協案に見える。
彼らは、セックスを購入する者を非難しつつも、セックスを売ることを強制されているように見える人々を非難せずにすむからだ。
しかし、実際には、このモデルは生計を立てるためにセックスを売っている人々に壊滅的な影響を与える。
なぜなら、その目的自体はセックスワークの根絶にあり、したがって、セックスワーカーが安全な労働環境を確保すること、組合を結成すること、共に働き支え合いながらお互いを守ること、自分たちの権利を擁護することを困難にし、さらにはビジネスのための銀行口座を開設することすら難しくしているからだ。
北欧モデルは、セックスワーカーをスティグマ化し、社会において周縁化するものであり、彼女を警察による暴力や虐待に対して依然として脆弱な立場に置く。
なぜなら、セックスワーカーたちの仕事とその顧客たちの行動は依然として犯罪とされているからだ。
\citep{watch19:_why_sex_work_shoul_be_decrim}
\end{quote}

元セックスワーカーであり、ニュージーランド売春者連合(New Zealand Prostitutes Collective)の創設メンバーであるキャサリン・ヒーリーは、\emph{VICE} に対して次のように語っている。
「私たちが、自分の顧客が起訴されることを望むなんてことがあるでしょうか? 客がセックスの対価を支払うことを恐れるようにする必要などがあるでしょうか? そんなことが私たちの助けになるはずがありません。なんて馬鹿げた法律でしょうか。
それどころか、こんな法律は明らかに危険です」\citep{mcclure17:_what_happen_when_sex_worker}。
容認主義者は、スウェーデンの法律がセックス産業の規模を縮小させたとする統計にも異議を唱えている。
フィリップ・ハバードは、イギリスの議会委員会に対し、次のように証言している。
「スウェーデンからの証拠が示唆するのは、犯罪化は単にセックスの売られ方を変化させただけであり、禁止政策は以前よりワーカーが搾取されやすい地下へ売春を追いやったということだ」\citep[p.25]{commons16:_prost}。

容認主義者は、北欧モデルがセックスワーカーのリスクを軽減するという主張を否定している。
彼らによれば、「セックス購入禁止法」はセックスワーカーが警察の標的にされやすい状況を変えていないという。
アムネスティ・インターナショナルの政策顧問であるキャサリン・マーフィーは、\emph{The Nation} に対して次のように語っている。
「セックスの直接的な販売が犯罪とされているわけではないから、セックスワーカーも犯罪者ではない、などと言うのは、刑法がセックスワーカーにどのように影響するかという現実をまったく反映していません」\citep{grant16:_amnes_inter_calls_end_nordic}。
また、スウェーデンの法律について現地調査をおこなったジェイ・レヴィは、イギリスの議会委員会に対し、この法律がセックスワーカーに対する警察の\ruby{嫌がらせ}{ハラスメント}を引き起こしていると証言している。
スウェーデンのセックスワーカー・アライネットワーク(Sex-workers and Allies Network, SANS) の広報担当者は、北欧モデルは実際にはリスクを増大させている可能性があると指摘し、次のように述べている。
「顧客が犯罪者とされたために、私たちが仕事をするためには警察を避けなければならなくなりました。
……\ruby{親切}{ナイス}なお客さんたちは警察に捕まることを恐れます。
……そうすると、残るのは問題がある客だけになります。
そして、客たちに警察の目につかないという安心感を与えるために、私たちは自動車で街から遠く離れた場所まで一緒に行かなければなりません。
これは、私たちは彼らの思うがままにされてしまうということです」\citep{savage08:_swedis_prost}。
さらに、国際セックスワーカーユニオン(International Union of Sex Workers)は、イギリスの議会委員会に対し、この法律の施行によってセックスワーカーたちがより大きなリスクをとるようになっていると報告し、次のような例を挙げている。
「コンドームなしでのサービス提供。
より人目につかない危険な場所での業務。
顧客の見極めや、価格交渉、内容、セーフセックス、その他の制限についての合意設定のための時間の減少。
より慎重に判断する時間があれば回避したはずの危険な状況への関与」\citep[p.26]{commons16:_prost}。

容認主義者は、北欧モデルのもとで一般に犯罪とされる二つの活動、つまり性風俗店を経営すること、またセックスワーカー以外の人物がセックスワークの売上によって生計を立てることは本質的に搾取的であるという見方にも異議を唱えている。むしろ、これらはセックスワーカーが自分たちの安全を守るために不可欠な手段でもありえるとしている。
カナダの刑法では長らくこれらの売春幇助行為は禁止されていたが、セックスワーカーたちは、こうした法律が憲法で保障された「個人の安全に対する権利」を脅かしているとして、同国のセックスワーク禁止法に対して法的異議を申し立てた。
カナダ最高裁は、この訴えを認め、性風俗店経営の禁止は、セックスワーカーがより安全な決まった場所で働くことを妨げ、路上や顧客の指定する場所で会うという危険な状況を強いていると判断した(\emph{Canada (AG) v. Bedford}, para. 64)。
最高裁は次のように述べている。
「連続殺人犯が街を徘徊している状況で、路上売春者がGrandma's Houseのような安全な避難場所に逃れることを禁じる法律は、その本来の目的を見失っている」(\emph{Bedford}, para. 136)。
Grandma's Houseは、バンクーバーで開設されていた性風俗店であり、複数のストリートのセックスワーカーが殺害された事件を受けて、セックスワーカーが安全に働ける場所を提供することを目的として運営されていた。しかしそれは警察によって閉鎖され、その後も連続殺人犯による殺人事件は続いた。
また、最高裁はセックスワークの収益で生計を立てることを禁じる法律についても次のように指摘している。
「この法律は、売春者を搾取する者(たとえば、支配的で暴力的な\ruby{売春斡旋業者}{ピンプ})と、売春者の安全を向上させうる者(たとえば、しっかりした運転手、マネージャー、ボディーガード)を区別することなく、すべての者を処罰している」(\emph{Bedford}, para. 142)。
\section{本章のまとめ}

本章の冒頭で述べたように、リベラルな民主主義国家においては、リーガルモラリズムの後退が徐々に進み、それに伴い、私的な合意に基づく行為の犯罪化も縮小してきた。
しかし、多くの西欧諸国では依然として商業的セックス産業を犯罪として扱っている。
こうした商業的セックスに反対する人々は、必ずしも(あるいは主として)モラルの観点から反対しているわけではない。
むしろ、自律、平等、功利といった原則を根拠に、セックス産業が関与する者や社会全体に害を及ぼすと主張している。
しかし、これまで見てきたように、これらと同じ哲学的原則が、セックスワークの非犯罪化を支持する根拠としても用いられる。
商業的セックスの問題は、カント主義者、功利主義者、そしてフェミニストの間でも意見を分裂させている。
また、セックスワーカー自身も、自らの仕事を擁護する声と、犯罪化を支持する声の両方を上げてきた。
セックス倫理の分野において、これほど鋭い対立を引き起こす問題は他にはほとんどない。
したがって、セックス産業を規制する法律がどのような方向に向かうのかを判断することが困難であることは不思議ではない。
ある法域では規制が強化される一方で、別の法域では規制が緩和されるという状況が続いている。
この議論が近いうちに決着するとは考えない方がよいだろう。

\phantomsection
\section{討論のための問い}

\begin{enumerate}
    \item 性的自由の権利を認めることは、政府にどのような新たな義務を課すことになるだろうか。
それは法が性的行為を扱う方法を劇的に変えることになるだろうか。
    \item キャサリン・マッキノンやアンドレア・ドウォーキン\ig{Andrea Dworkin}が主張するように、ポルノグラフィはセックスや女性の役割について「権威をもって語る」ものだろうか。
それは人々のセックスに対する見方にどのような影響を与えるのだろうか。
その影響は、ポルノグラフィの流通を制限することを正当化するのに十分なものなのだろうか。
    \item 商業的なセックス取引を非犯罪化しつつ、より倫理的または非搾取的なものとする方法はあるだろうか。
セックスワーク合法化に反対するフェミニストたちは、そのような変化によって納得する可能性はあるだろうか。
    \item クィアのセックスワークは、フェミニストが女性セックスワーカーと男性クライアントの関係に見出す象徴的価値とは異なる価値を持つだろうか。
クィア理論の視点からは、セックスワークの社会的役割をどのように捉えることができるだろうか。
    \end{enumerate}

\chapter{セックスとテクノロジー}

セックスとコンピューターが初めて結びついたのは、1956年の生放送テレビ番組でのことだった。
1956年11月17日、NBCの人気バラエティクイズ番組「ピープル・アー・ファニー」(People Are Funny)の司会者アート・リンクレターは、巨大なコンピューターUNIVACをステージに運び込んだ。
このマシンはレミントン・ランド社から貸し出されたものでだった。
同社のこの強力なメインフレームマシンは、1952年の大統領選挙でアイゼンハワーの勝利を予測したことで一躍有名になっていた(少し注目していたすべての人が予測していたことではあるが)。
同社の誰かが、もう一つの斬新な宣伝企画を思いついた。
リンクレターの番組で、このコンピューターを使ってカップルをマッチングさせるというのだ。

放送に先立つ数ヶ月間、チームは新聞広告を通じて、UNIVACにマッチングを任せてもよいと考える人々を募集した。
4000人以上が応募し、性、人種(当時としてもちろん)、宗教、政治的立場、体重、身長、ペットの有無、飲酒習慣など、32の質問に回答した。
最も適合度の高いカップルが選ばれ、マッチングされた上でデートをおこなった。
そして本番の放送で、リンクレターはUNIVAC(と、作家のデヴィッド・ポペノーおよび番組制作陣による相当な追加審査)が選んだ最も理想的なカップルを紹介した。
それは、23歳の受付係バーバラ・スミスと、広告業界で働く28歳のジョン・カランであった。
バーバラは、UNIVACがジョンを選んだことをきっかけに、それまでの恋人と別れたことを明かした。
そして二人はすでに婚約していると発表し、リンクレターは新婚旅行としてパリへの航空券をプレゼントした。

その後、いくつかのコンピューターデートサービスが登場し、一定の料金を支払えば誰でもアンケートに回答し、相性の良い相手を紹介してもらえる仕組みが作られた。
一時的な関心の高まりはあったものの、これらの企業は長期的なビジネスとして定着するには至らなかった。
人々はまだ、「コンピューターでもっと素敵な相手と出会おう」というスローガンにはなじめなかったのだ。
それから40年後、インターネットの発明によって、コンピューターを親密な関係の構築に利用することへの認識が変わることになる。

今や、好むと好まざるとにかかわらず、テクノロジーは私たちのセックスライフに不可分な存在となっている。
インターネットデートは、その多様なアプリケーションのうちの一つにすぎない。
セックステクノロジーには、大きく二種類が存在する。
第一に、オンラインポルノグラフィ、マッチングアプリ、セックストイ(現在ではオンラインネットワークと接続されているものもある)など、セックスや親密な関係のために特化して設計されたテクノロジーがある。
第二に、ソーシャルネットワークやメッセージアプリのように、セックスを目的として設計されたものではないが、実際にはセクシュアルなコミュニケーションに頻繁に利用されているテクノロジーがある。
本章では、これら両方を含む概念として「セックステック(セックステクノロジー)」という用語を用いる。

私と共著者のマーキー・ツイストは、セックステクノロジーを二つの波(wave)に分類した\citep{mcarthur17:_rise_digis}。
第一波のテクノロジーは、人間のパートナーとのつながりを提供するもの、あるいは(オンラインポルノグラフィのように)そうしたつながりの疑似体験を提供するテクノロジーを指すものとして分類している。
一方、第二波のセックステクノロジーは、ユーザーに強烈な没入型の体験を提供することを目的としており、人間のパートナーの不在を前提とする点で異なる。

セックス・ロボットは、第二波テクノロジーの最も顕著な例だ。また、セックスのために特別に設計された仮想現実(Virtual Reality, VR)アプリケーションも存在する。
これらは、人々が没入型のエロティックな体験を可能にする複雑な仮想世界に入り込むことを可能にする。その内部では、ユーザーは見知らぬ人々や人工エージェントとやりとりをすることができる。
こうした仮想環境は、触覚フィードバック装置と統合されることもあり、それによりユーザーに物理的な刺激が与えられ、体験の没入感がさらに高められる。

セックステクノロジーは今後も発展し続けるが、本章では将来の展望を推測することは控える。
その代わりに、第一波および第二波のテクノロジーが提起する最も重要な倫理的課題を検討する。
まず、セックステクノロジーが機能する環境について考察し、それが人々のニーズに最適な形で設計されているかを問う。
次に、特定のテクノロジーに関する倫理的問題を詳しく検討する。
まずはセックスロボットの問題を取り上げる{\DDASH}それはいまや現実のものとなりつつある。私たちはこれを歓迎すべきなのだろうか。
次に、バーチャルな児童搾取の問題を考察する。
これは、実在する人間を犠牲者とせずに、未成年者の\ruby{表象}{リプレゼンテーション}を通じた性的インタラクションを可能にするテクノロジーだ。
そして最後に、「ラブドラッグ」を考察する。これは私たちの感情を制御することを目的とした薬学的介入だ。

\section{セックステクノロジーへのアクセス}

近年、テクノロジーの世界がどのように統治・管理されるべきかについて、大きな議論が巻き起こっている。
立法者、市民、メディアは、わずか数社の企業がオンライン上で私たちが何を見て、何を言い、何を購入できるかを支配している現状に疑問を投げかけている。
これらの企業は、多くの場合、説明責任も透明性もない形でこの権力を行使しており、それは自由な表現のあり方について重大な問題を提起する。
しかし、これまでの巨大テック企業の影響力についての議論では、それがセックステクノロジーに与える影響についてはほとんど考慮されてこなかった。
本節では、人々のセックステクノロジーへのアクセスが過度に制限されている可能性について検討し、もしそうであるならば、この問題をどのように対処すべきかを問いたい。

\subsection{現在の状況}

オンラインでのセクシュアル・コンテンツへのアクセスが問題になることは一見すると考えにくいかもしれない。
むしろ、露骨なコンテンツは見つけるよりも避けることの方が難しいほどだ。
しかし、オンラインポルノグラフィが広く普及しているにもかかわらず、ウェブサイトだけでなく、ソーシャル・ネットワーク、スマートフォン向けアプリ、セクシュアル・デバイスの販売など、テクノロジー全般の世界を見渡してみると、それが実は非常に厳しく規制・管理されていることがわかる。

セックステクノロジーの規制の一部は国家によるものだ。
リベラルな民主主義国家以外では、多くの政府があらゆるテクノロジーを厳格に管理している。そうした政府は特に政治的発言をターゲットにすることが多いが、性的に露骨なコンテンツも検閲から逃れることはほとんどない。
リベラルな民主主義国家においても、前述したように (本書5.2節)、インターネット・ポルノグラフィを規制する法域があり、また、多くの法域はオンラインでのセックスの販売を対象とした法律をもっている。
特にアメリカでは、〔2018年制定の〕HR-1865、すなわちオンライン人身取引対策法(Fight Online Sex Trafficking Act, FOSTA)および性的人身取引助長禁止法(Stop Enabling Sex Trafficking Act, SESTA) により、「他者の売春を促進または助長する」ウェブサイトの所有者を起訴することが可能になった。
その結果、それまでセックスワーカーが広告を掲載していたほぼすべてのサイトが閉鎖された\citep{witt18:_after_closur_backp_increas_vulner}。
さらに、リベラルな政府でさえ、個人のオンラインでの性的活動を監視することがある\citep{owen13:_nsa_spied_porn_habit_target_radic}。

しかしながら、セックステクノロジーの規制の大部分は民間企業によっておこなわれている。
最も人気のあるソーシャルネットワークサービス(SNS)であるFacebookとInstagramは、芸術、医療関連、授乳などを目的としているとみなされる場合を除き、ヌードを禁止している。
さらに、Facebookは「性的な暗示を含むコンテンツ」も禁止しており、その範囲には「フェティッシュなシナリオ」「性役割」「性的嗜好/性的パートナーの嗜好」に言及する投稿が含まれる。

また、両プラットフォーム(同じ会社が所有している)は、CEOのマーク・ザッカーバーグが「ボーダーライン・コンテンツ」と呼ぶものへのユーザーのアクセスを制限する措置を講じている。
Instagramは2019年に次のように発表した。
「私たちは、Instagramのコミュニティ・ガイドラインには違反しないが、不適切とみなされる投稿の拡散を減少させる取り組みを開始しました」\citep{constine19:_instag_now_demot_vaguel_inapp_conten}。
しかし、どのようなコンテンツが「ボーダーライン・コンテンツ」に該当するかについての明確な定義は示されておらず、そのルールの適用も、人間のモデレーター、機械学習、ユーザーからの報告に依存しているため、一貫性に欠けることがある。
このポリシーは、ユーザー投稿だけでなく広告にも影響を与えている\citep{kibbe20:_faceb_has_banned_ads_kink}。

YouTubeはアップロード動画の市場をほぼ独占しており、「性的満足を目的とするコンテンツ」を禁止している。
ただし、「教育、ドキュメンタリー、科学、または芸術を主目的とし、不必要に露骨でない場合」はヌードを許可し、通常は年齢制限を設けている。
Twitchは現在ライブ動画配信市場を支配しており、露骨なコンテンツだけでなく「性的に示唆的な」コンテンツや部分的なヌードも禁止する詳細なガイドラインを設けている。
Twitchは「アンダーバストの露出を許可しない」が、「適切なカバー要件が満たされている限り、胸の谷間の露出は制限されない」と述べている。
授乳やボディアートなど、一部の「状況に応じた例外」は認められている\citep{good20:_twitc_has_new_nudit_rules_theyr_detail}。

過去10年間で、人々はスマートフォンやタブレットをインターネットアクセスの主な手段として利用するようになった。
しかし、これらのデバイスのソフトウェア環境は、AppleとGoogleの二社によってほぼ完全に支配されており、両社はアプリストアで配布できるアプリの内容を決定している。
そして、両社とも露骨なコンテンツを含むアプリの大半を排除している。
プライベートなコミュニケーションプラットフォームでさえ、企業は利用方法を規制する権利を保持している。
たとえば、MicrosoftのSkypeの利用規約では、プライベート通話においても露骨な言葉遣いやヌードを禁止する権利を企業側に認めている\citep{smith18:_micros_ban_offen_languag_skype}。

新しいテクノロジーの開発には高額な費用がかかるが、セックステックの分野で働く人々は、資金調達の面で重大な障壁に直面している。
多くのベンチャーキャピタルは「{悪徳商品排除条項}」(vice clauses)と呼ばれる規定に従っており、性的な性質を持つと見なされる商品への投資を禁止している\citep{davis19:_compan_ventur_capit_isnt_allow_inves_in}。
さらに、セックステックの起業家は、ピッチ・コンテスト、アクセラレーション・プログラム、インキュベーション・プログラム、公的資金の支援、大規模な展示会などから排除されることが多い\citep{fox19:_vibrat_center_tech_sexis_contr}。
クラウドファンディングのプラットフォームも、セックステック製品やセックストイの資金調達キャンペーンを禁止している\citep{manning17:_sextec_revol_will_not_be_crowd}。

セックステック業界における最大の障壁の一つは、決済プラットフォームによる制限だ。
PayPalは「特定の種類のアダルト製品およびコンテンツの販売」を許可しているが、「犯罪行為を描写したマテリアルの販売、または未成年者向けの性的コンテンツや製品の販売」を禁止している\citep{fearnow20:_porn_indus_stars_turn_crypt}。
Apple PayやStripeのポリシーも同様に厳しい。
VisaやMastercardといった主要クレジットカード会社も、突如として、あるいは予測不可能なタイミングで、セクシュアル・コンテンツを扱うプラットフォームでの決済を停止している。
「MastercardとVisaは何十年もの間、オンライン上で何が許可され、何が許可されないかを管理し、市場の力を利用して何百万ものセックスワーカーに損害を与えてきた」と、アダルト産業労働者・アーティスト協会(Adult Industry Laborers and Artists Association)の創設メンバーであるマリー・ムーディ\ig{Mary Moody}は述べている\citep{cole21:_i_felt_betray}。
これらの決済会社は、2015年にBackpageとの取引を停止し、2020年にはPornhubとの取引を打ち切った。
さらに2021年には、OnlyFansがプラットフォームから性的に露骨なコンテンツを禁止すると発表したが、これは決済プロバイダーからの圧力によるものと推測された{\DDASH}ただし、その後方針を撤回した。
決済プラットフォームからのアクセスを常に失う危険性にさらされているため、露骨なコンテンツを制作・配信する企業にとって資金調達はさらに困難なものとなっている\citep{sesta18:_platf_which_discr_sex_worker}。
\emph{The Economist}は次のように報道している。

\begin{quote}
〔個人がクリエイターとして制作・発信・販売をおこなう〕「クリエイター・エコノミー」はベンチャーキャピタリストの間で大流行しており、著名人の動画を販売するCameoのような企業に資金が殺到している。
しかし、その成長と知名度(ビヨンセが曲の中で言及するほど)にもかかわらず、投資家はOnlyFansを敬遠している。
一部の投資家は評判リスクを懸念している。
また、ベンチャーファンドのSignalFireのジョシュ・コンスタンは、「決済業者によって閉鎖される恐れがある中では、投資をためらうのも無理はない」と述べている。
\citep{economist21:_onlyf_u_turns_its_porn_ban}
\end{quote}

OnlyFansをめぐる論争を振り返り、イッシー・ラポウスキー\ig{Issue Lapowsky}は、「インターネット上で何が許可され、何が排除されるのかを決めるのは誰なのか?現状を見る限り、その決定権を握っているのはVisaとMastercardだ」と結論している\citep{lapowsky21:_onlyf_shows_visa_master_are}。

OnlyFansのような著名なケースはメディアの注目を集めるものの、セックステクノロジーへのアクセスをめぐる問題について、公の場での議論はほとんどおこなわれておらず、そもそもこの問題を認識している人すら少ない。
結果として、この問題に関する議論の両側の主張はほとんど検討されないままとなっている。

\subsection{アクセス規制を求める議論}

セックステクノロジーをめぐる議論は、より広範なテクノロジー界におけるオープン性と検閲の議論の文脈で捉える必要がある。
主要プラットフォームを所有する企業は、自社のプラットフォーム上で許可すべきコンテンツと禁止すべきコンテンツを巡って苦慮してきた。
また、特定のコンテンツをユーザーに対してどの程度可視化するかといった微妙な問題にも取り組んでいる\citep{economist20:_social_medias_strug_self}。
しかし、企業側は私企業であることを理由に、これらの決定を下す絶対的な権利を主張している。

こうした権利については、リベラル派も保守派も批判しており、市民の基本的な表現の自由を脅かすものだと指摘している。
2020年5月、当時のアメリカ大統領ドナルド・トランプは大統領令を発し、「表現の自由はアメリカ民主主義の基盤であり、表現の自由を大切にしてきたこの国において、一部のオンラインプラットフォームがアメリカ国民のアクセス可能な言論を選別することを許すわけにはいかない」と宣言した\citep{house20:_execut_order_preven_onlin_censor,lakier21:_great_free_speec_rever}。

トランプの大統領令は、露骨な性的コンテンツをターゲットとしたものではなかった。
しかしそれは、多くの人々が抱く懸念{\DDASH}すなわち、テクノロジー企業が私たちが何を読み、何を見るかを決定し、彼らが物議を醸すとみなすあらゆるものへのアクセスを制限する力を持っていることへの懸念{\DDASH}を表現している。

表現の自由の権利は通常、政府に適用されるものであり、私企業には適用されない。
シリコンバレーでは「表現の自由はリーチの自由と同義ではない」というスローガンが広まっている。
これは「表現の自由があるからといって、どのプラットフォームにでも自由にアクセスできるわけではなく、すべてのコンテンツが平等に可視化されるわけでもない」という考え方を反映している。
テクノロジー企業は、自社プラットフォームの運営方針の決定は、政府による検閲とは異なると言う。
それは、企業は特定のコンテンツを完全に抑圧することなどできはせず、単に発信するならば自社以外の場でおこなうことを強制するにすぎないからだ。

また、企業はコンテンツポリシーが消費者の需要に応じたものだと述べている。
Facebookは「これらの規定は利用者からの意見に加え、テクノロジーや公共の安全、人権などの分野における専門家からの助言に基づき策定されています」と説明している\citep{facebook25:_commun_stand}。
2010年にAppleがアダルト向けアプリを一斉削除した際、同社のマーケティング責任者フィリップ・シラーは、「女性ユーザーから、コンテンツがあまりにも侮辱的で不快だという苦情が寄せられました。
また、子供が見られる状態にあることを懸念する親からも多くの苦情がありました」と説明している\citep{wortham10:_apple_bans_some_apps_sex_tinged_conten}。

企業側の弁明としては、消費者にはプラットフォームを選ぶ自由が常にあるという点も挙げられる。
企業は競争的な環境の中に存在しており、人々はある企業の運用ポリシーが気に入らない場合、他のサービスに切り替える自由を持っている。
Facebook、Instagram、YouTubeはどれも性的コンテンツを禁止しているが、他のSNSは、そうしたコンテンツを共有したり販売したりしたいと考える人々に対してより寛容な態度をとっている。
Twitterはユーザーの投稿に対して比較的寛容な態度をとっているが、「シャドウバン」をおこなっている、つまり露骨なコンテンツを見えにくくし、見つけにくくしているとして非難されている\citep{valens20:_repor_says_shadow_is_real}。
Redditは成人向けコンテンツを公然と許可しているが、2021年にそのコンテンツをページのメインフィードから禁止した。
しかし、当時の広報担当者は次のように述べている。
「Redditの性的に露骨なコンテンツは消え去るわけではありません。
もしその種類のコンテンツを探しているのであれば、今でもそこにあり、簡単に見つけることができます」\citep{sosa11:_remov_sexual_explic_conten}。
OnlyFansは、人々が自分自身のヌード写真やビデオを投稿するための人気のあるフォーラムとなっている。
このサイトは、ユーザーが「\ruby{わいせつ}{オブシーン}な」コンテンツや「\ruby{同伴}{エスコート}サービスの宣伝」を公開することを禁止していると宣言している。
しかし、ハードコアポルノを含まない限り、ユーザーの投稿には寛容な態度をとっており、サイトはセックスワーカーが写真やビデオを投稿して収入を得ることを許可している。
これにより、セックスワーカーはポルノ会社を介さずに収入源を得ることができた\citep{bernstein19:_how_onlyf_chang_sex_work_forev}。
先に述べたように、OnlyFansは2021年8月にサイトから明示的なコンテンツを禁止すると発表したが、ユーザーやクリエイターからの反発を受けて、方針を撤回した。

実際、小さな企業が業界のあちこちに現れ、大手企業に排除された顧客にサービスを提供している。
主要な決済プラットフォームが性に関連する多くのコンテンツを禁止する一方で、Sexworkerhelpfulsというサイトは「セックスワーカーフレンドリー」と評価するいくつかのオンライン決済プラットフォームを紹介している\citep{sexworkerhelpfuls.com18:_sex_work_approv_paymen_option}。
シンディ・ギャロップが倫理的なポルノサイトMake Love Not Pornの資金調達を試みた際、クラウドファンディングサイトのKickstarterやIndiegogoに断られたが、知名度が低いIFundWomenが彼女のキャンペーンを受け入れた\citep{gardezi16:_cindy_gallop_journ_normal_dialog_aroun_sex}。
このように、セックステクノロジーに対する制限的な環境は、自由市場の結果だと言える。
消費者が望めば、自由市場によって変革される可能性もある。

企業側はさらに、自社のプラットフォームで性的なコンテンツを許容すれば、同意のないコンテンツや違法な素材の温床となるリスクがあると主張できるだろう。
決済会社がPornHubとの取引を停止したのは、\emph{The New York Times}のコラムニストのニコラス・クリストフ\ig{Nicholas Kristof}による記事を受けてのことだった。
その記事では、同サイトに掲載されている動画のうち何千本もの動画が、出演者の同意なしに撮影されたものであること、出演者が未成年であること、あるいは非同意の行為を描写していることが指摘されていた(本書5.2.3節を参照)。
PornHubは、クリストフの記事が発表されるまでの長年にわたり、自社サイト上のコンテンツを適切に監視することを拒んできた。
しかし、クレジットカード会社からの圧力を受けて、ようやく対応に乗り出した。
ソーシャルメディアプラットフォームが性的コンテンツを許可すれば、同様の非同意のコンテンツが拡散する可能性が高い。
すべての性的コンテンツが同意に基づいたものであることを企業側が保証するのは極めて困難だ。
そして、たとえ違反コンテンツを比較的迅速に削除できたとしても、削除されるまでの間に広範囲に拡散してしまう恐れがある。

\subsection{よりオープンな環境を求める議論}

本書の前提は、セックスが多くの人にとって快楽とつながりの源であるということだ。
このことは、性的コンテンツやセックステックへのアクセスが可能な限り広く提供されるべきだという一応の理由を与える。
人々がセックステクノロジーを利用するのは、それを楽しんでいるからだ。
それに加え、この分野は経済的に重要な収益源となる可能性がある。
\emph{Forbes}は2019年に、セックステック産業が「年間300億ドルと評価され、30%の成長率を示している(ただし、この評価は過小評価だと主張する者もいる)」と報じている\citep{jaramillo19:_inves_sextec}。
しかも、これはすでに述べたようなさまざまな制約が存在する中での数値だ。
この業界の関係者は、現在の収益は、もし規制が緩和されれば生み出される富のほんの一部にすぎないと主張する。
そして、この富は政府に対する税収を増加させ、さらには業界内での新たな雇用を創出するだろう。

消費者は常に自分の望むコンテンツを提供するプラットフォームを選択できる、という主張に対し、より開かれた環境を求める論者は、テクノロジー市場がけっして自由ではないと反論する。
彼らは、少数の企業が市場を独占していることを指摘する。
YouTube(Googleの所有)は、アップロード動画市場の90%以上を支配している。
Twitch(Amazonの所有)は、ストリーミング動画市場の同様のシェアを占め、残りの市場の大部分はYouTubeとFacebookが握っている。
FacebookとInstagramを合わせると、アメリカのソーシャルメディア利用の約75%を占める\citep{department21:_us_market}。
AppleとGoogleは、中国市場を除いたモバイルアプリ市場の95%を支配している\citep{curry21:_app_store_data}。
この支配力は、これらのテクノロジーが持つネットワーク効果によって可能になっている{\DDASH}すなわち、人々は他の利用者と同じネットワークを利用することで利益を得るため、収益を上げたい者は自分の潜在的な顧客がいる場所へ行く必要がある。

このため、一部の論者は、大手テクノロジー企業は一般的な民間企業というよりも、テレビ放送局や電力会社などの公共事業体や、ショッピングモールのような公的にアクセス可能な空間の所有者に近いと主張する\citep{swire17:_shoul_leadin_onlin_tech_compan}。
ジェネヴィヴ・ラキアー\ig{Genevieve Lakier}は次のように述べている。
「\emph{Marsh}事件の舞台となったショッピングモールと同様に、今日のTwitterやFacebookは、公共の会話や議論にとって重要な\ruby{集会場所}{フォーラム}を提供している。
上院議員のジョン・コーニン\ig{John Cornyn}が指摘するように、それらはインターネット時代の「新しい公共広場」なのだ」\citep{lakier21:_great_free_speec_rever}。
(\emph{Marsh v. Alabama}事件は、1946年のアメリカ合衆国最高裁判決であり、ショッピングモールの所有者がモールの通路で抗議活動をおこなう者を排除することは、憲法修正第1条の権利を侵害するとの判断が下されたものだ。)

これらの企業が持つテクノロジーの性質上、政府を含め他のどの組織よりも効果的に言論や表現を抑制する手段を持っている。
ダフネ・ケラーは次のように述べている。

\begin{quote}
主要なプラットフォームは、歴史上どの政府よりも効果的に私たちの言論を制限することができる……Facebookのようなプラットフォームは、私たちが書くすべての内容を監視し、それを審査に回したり、自動的に禁止された単語や画像を削除したりするソフトウェアにますます依存している。
プラットフォームは、合法的であっても憎悪的、嫌がらせ的、誤解を招く、あるいは不快な投稿を迅速に削除することができる。
異議申し立てがあった場合でも、裁判所ならば数ヶ月から数年はかかるような争議を、数分で解決してしまうのだ。
\citep{keller19:_faceb_restr_speec_popul_deman}
\end{quote}

テクノロジー企業による規制の影響は、その規則がしばしば曖昧で予測不可能であるため、コンテンツ制作者にとってさらに深刻となる。
すでに述べたように、FacebookやInstagramは「ボーダーライン上のコンテンツ」を禁止する権利を留保しているが、それが具体的に何を指すのかについて明確な定義を示していない。
そのため、制作者はこれらのプラットフォームに依存することができず、いつでも突然コンテンツが削除され、収益の道が閉ざされる可能性がある。
カティー・デイトンは「このような曖昧なポリシーは、実質的にセックステックの世界に対する門戸を閉ざしてしまう」と述べている\citep{deighton20:_why_sex_start_face_uphil_paymen_battl}。
一部の企業は、ユーザーがコンテンツの制限に異議を申し立てるプロセスを用意している。
たとえばFacebookでは、削除されたコンテンツの審査をリクエストすることができ、24時間以内に「コミュニティ・オペレーション」チームが審査をおこなう。
しかし、このプロセスは透明性に欠けており、最終的な判断は企業が一方的に下す権利をもっている。

プラットフォームが、警告なしに突然ポリシーを変更することもある。
ユーザーは、プラットフォームから得ていた収入を突然失い、コンテンツの作成やファンベースの構築に費やしたすべての時間も失ってしまう。
PornHubは、大量のセックスワーカーにとって主要な収入源となっていたコンテンツを突如削除したが、このような事例は過去にも繰り返されている。
Patreonは2017年に、Tumblrは2018年に、eBayは2021年に、それぞれ性的コンテンツの制作者を排除した\citep{cooper17:_real_conseq_patreon_adult_conten_crack}。

企業は、常に倫理的・教育的なコンテンツの制作者に対して配慮してくれるわけではない。
たとえば、Happy Play Timeは、「女性のマスターベーションに対するスティグマをなくす」ことを目的とした性教育ゲームであり、擬人化された外陰部のキャラクターのHappyをフィーチャーしていた。
このゲームは情報提供を目的としたものであり、エロティックなものではなかったにもかかわらず、AppleのApp Storeは「過度に不快または下品なコンテンツ」および「ポルノ的な内容」を含むとして禁止した\citep{dhapolamay14:_apples_rejec_happy_playt_app}。
また、シンディー・ギャロップの「ソーシャル・セックス」動画プラットフォームMake Love Not Pornは、すべての動画が合意のもとに制作されていることを確認し、女性の視点を反映するよう徹底的に管理されている。
しかし、主要な決済プラットフォームはどこも彼女と提携しようとしなかった。
ギャロップは「誰も、セックステックや性教育を考慮して、包括的な「アダルト」条項を見直す運動を起こそうとはしてくれません」と述べている\citep{deighton20:_why_sex_start_face_uphil_paymen_battl}。
このように、倫理的なコンテンツの制作者を企業による制限が締め出してしまうことによって、ひどく倫理的な配慮に欠け、利益だけを追求する者たちが市場を独占することになる。
すでに見たように、PornHubを所有するMindGeekは、非合意の素材や\ruby{未成年者}{アンダーエイジ}を含むコンテンツを長年配信していたとされるビジネスモデルにもかかわらず、莫大な利益を上げている。

さらに、テック企業は、独立系制作者たちそれぞれに対して一貫した基準でポリシーを適用しているわけではない。
たとえば、Playboy社は依然として主要なアプリストアを利用できる。
一方で、Facebookは\ruby{代替}{オルタナ}的なマッチングアプリ\#openの広告を禁止した。
同社は、このアプリが「カジュアルな関係」や「複数のパートナー」を求めるユーザーのための\ruby{包括的な}{インクルーシブ}アプローチを取っていることが、「当社プラットフォームのグローバルなオーディエンスを反映していない」と主張している\citep{kibbe20:_faceb_has_banned_ads_kink}。
しかし、FacebookはTinderやGrindrといった主流のマッチングアプリの広告は許可しており、これらのアプリもまたカジュアルなセックスや複数のパートナーを求めるユーザーを受け入れている。
また、大手企業は、事前にコンテンツを審査してもらい、必要に応じて修正するだけの資金的・テクノロジー的余裕を持っている。

確かに、大手企業が市場を支配しているとはいえ、業界は依然としてある程度は競争的である。
RedditやOnlyFansの成功は、現在でも、新規でより許容的なテクノロジーが登場し、成長できることを示している。
しかし、大手プラットフォームを迂回することにはコストが伴う。
最も明白なのは、代替プラットフォームの利用者母体が小さいことだ。
しかし、それだけではない。
代替プラットフォームは規模が小さいため、手数料が高くなる傾向がある。
たとえば、主要な決済プラットフォームを利用できないシンディ・ギャロップは、主としてアダルト業界と取引しているオンライン決済プロバイダーを使用せざるをえないが、それには「法外な」手数料を課せられているという。
セックステック業界の他の事業者も、顧客からの支払い手段として暗号通貨を利用するようになったが、これらのデジタル通貨は不安定であり、セキュリティ面でも問題がある。

よりオープンなインターネットを擁護する人々であっても、不同意あるいは違法のコンテンツの流通が深刻なリスクであることは認めなければならない。
そして、この問題に対処するための最善の方策は、包括的な法規制という枠組みによってしか実現しえないことも認めなければならない。
不同意の画像の共有{\DDASH}しばしば誤解を招く「リベンジポルノ」と呼ばれている{\DDASH}は、すでに深刻な問題となっている。
よりオープンなインターネットを求めるか否かにかかわらず、私たちは被害者を守るため刑事・民事の両面において、より強力な法制度が必要である\citep{kibbe20:_faceb_has_banned_ads_kink}。
しかし、オープンなインターネットを求める活動家たちは、こうした法律の制定は、SESTAやFOSTAのように一方的に課されるのではなく、セックステックを開発し利用する人々と協議しながらおこなわれるべきだと主張する。
そうすることで、規制が全体としてさらに抑圧的な環境を生み出すことを防ぎ、正当な制作者の生計を脅かすことがないようにすることができる。

\subsection{本節のまとめ: 何ができるのか?}

現在のセックステック環境が過度に制限されているという点に同意したとしても、それを改善するためにできることはほとんどないように思われるかもしれない。
政府による規制は、そもそものはじめから有望な解決策とは思えない。
大手テック企業の権力を抑制しようとする動きは高まっているものの、政府が意図的にセックステックを歓迎するような環境を作り出すとは考えにくい。
むしろ、政府の介入が市場に対して逆効果をもたらす可能性の方が高い。

しかし、政府の行動が間接的にセックステック産業を支援するようなものになることはありえる。
たとえば、政府が独占禁止法を活用して市場をより競争的なものにすることを決定すれば、新たなプラットフォームが登場しやすくなり、その一部がセックステックに対してより開かれた姿勢をとる可能性がある。
また、政府は主要プラットフォームに、コンテンツ規制の方針を公開し、基本的な適正手続きの基準を順守するよう義務づけることで、過度に制限的なポリシーに対する異議申し立てを可能にすることもできる。
さらに、ユーザーが差別から保護されることを保証する法律を制定することもできるだろう。それによって、たとえば、LGBTQ+コンテンツを投稿していることで不当に規制の標的にされていると感じた個人や企業が法的に対抗できるようになる。

すでに述べたように、企業は「消費者の需要に応じている」と主張しており、それは確かに事実だろう。
セックステックに対するスティグマは、民間企業によって生み出されたものではない。
シンディー・ギャロップは、彼女のビデオサイトMake Love Not Pornの資金調達の困難の大半は、たった一つの要因に起因していると言う。
それは「他の人々がどう思うか」という恐れだ。
彼女はこう言う。
「成功には、多くの人々の公然の支持が必要です。
私たちをiFundWomenで支援してくれた人々の中には、「私はあなたがやっていることが好きです。
あなたの取り組みは素晴らしいと思うし、これだけの額を寄付しました。
でも、私の名前はどこにも公開しないでください」と書いてきた人もいます」\citep{manning17:_sextec_revol_will_not_be_crowd}。
この教訓はもっと広い範囲に適用できる。
もし環境を変えたいのであれば、人々がセックステックを支持し、それに関心をもっていることを公に表明し、聞き手に届くようにすることが必要だ。
そうすれば、大手企業はポリシーを変更せざるをえなくなるか、あるいは新たな競争相手が台頭するかのどちらかということになる。

もちろん、それが簡単なことではないのは明らかだ。
テクノロジーのネットワーク効果は依然として障壁となるだろう。
しかし、業界における権力の集中が進んでいるとはいえ、いまだにインターネットでは、人々がニッチな制作者からコンテンツにアクセスし、商品を購入することが比較的簡単だ。
セックステック企業が、自分たちが成長できることを示せば、より多くの投資が集まり、社会的な可視化がなされることになるだろう。

\section{セックスロボットと第二波テクノロジー}

本章の冒頭で、第一波と第二波のセックステックを区別した。
第二波テクノロジーの最もよく知られた例はセックスロボットだ。
ロボットとは、少なくともある程度は人間やその他の生物に似た外見や行動や動作を持ち、また、周囲の情報を解釈し、反応する能力を備えた人工知能(AI)を有する存在だと定義する\citep{danaher17:_shoul_we_be_think_sex_robot}。
セックスロボットとは、セックスのために(たとえそれが唯一の目的ではないとしても)設計されたロボットだ。

しかし、セックスロボットには重大な制約が一つある。
それが存在しないということだ{\DDASH}少なくとも現在のところは。
人々は『ウエストワールド』\footnote{訳注:米国で2016--2022年に放映されたテレビドラマ、1973年の同名の映画(マイケル・クライトン監督)に基づく。}や『エクス・マキナ』\footnote{2014年のイギリス映画(アレックス・ガーランド監督)。}のようなテレビドラマや映画を通じてその存在を知っている。
しかし、現時点で「セックスボット」として市場に出ているものは、極めて原始的な人工知能を備えた人形にすぎない。
映画に登場するような人間らしいロボットの実現には、まだかなりの時間がかかる。
これは、克服すべきテクノロジー的課題が多いためだ。
特に、ロボットを人間のように歩かせ、動かすことは困難であり、また、Siriのユーザーなら誰でも知っているように、現代の最も資金力のあるテクノロジー企業でさえ、リアルな対話型AIの開発には苦戦している。

しかし、セックスボットは近づいている。
カリフォルニア州のReal Doll社のように、その開発を専門とする企業も存在する。
また、ロボット工学や人工知能の研究は日々進展しており、新たなロボットテクノロジーが登場すれば、必然的にそれを性的目的に適用する動きも出てくるだろう。

第二波セックステックには、VR環境も含まれる。
これらの環境では、人々が他者や人工知能(AI)と相互作用できる。
これらのVR環境は、触覚フィードバックデバイスと統合することが可能であり、仮想世界内でのアクションやイベントに応じてリアルタイムで刺激を提供する。
これらの触覚デバイスは、記憶に残る名称である「テレディルドニクス」として知られている。
セックスボットとは異なり、エロティックなVRはすでに存在している。
適切なテクノロジーを持っている者なら誰でも複雑な仮想世界に入り、その中で他人が操作するアバターやAIと交流し、仮想セックスをすることができる。

セックスボットやその他の高度なセックステックの開発に対して、多くの人々が懸念を抱き、あるいは反対の声を上げている。
ある者は導入には慎重になるべきだと主張し、またある者はその開発を積極的に抑制または禁止すべきだと考えている。
これらのテクノロジーの開発に投入されるリソース{\DDASH}民間企業、大学、政府による資金{\DDASH}は、世論や市場の規模に大きく左右される。
以下では、まずセックスボットや第二波テクノロジーに対する批判を検討し、その後、それらがもたらす可能性のある利点について考察する。

\subsection{相互性と親密さへの脅威}

セックスロボットは、精巧なセックストイにすぎないのだから、特に問題視されるべきではないと考える人もいるかもしれない。
言い換えれば、それは単に一部の人々が自らの肉体的快楽のために使用する装置であり、それ自体が道徳的に問題となることはない。
したがって、セックスロボットの使用は、マスターベーションと同様に倫理的に許容されるべきだという主張が成り立つように思われる。
しかし、セックスロボットに反対する人々は、人間がロボットと関わる仕方が、他の種類の性的デバイスとは本質的に異なると主張する。
人間は、\ruby{主体性}{エージェンシー}を持っているように見える対象に対して自然に愛着を抱く傾向がある。
この傾向はほぼ普遍的であり、抵抗するのが難しいことが多い。

シェリー・タークルは、ロボットやそれに類するテクノロジーを「関係オブジェクト」(relational object)と呼び、人々はそれに何らかの「魂」や「本質」を帰属させるのを避けることは難しい感じると述べる。
彼女と共同研究者らは、人間とロボットの相互作用に関する研究を紹介し、その中で被験者の高い割合がロボットに何らかの「テクノロジー的\ruby{本質}{エッセンス}」(75%)、「生命のような本質」(48%)、または「能動的な心理状態」(60%)を認めたことを指摘している\citep[p.349]{turkle18:_relat_artif_child_elder}。
ケイト・ダーリング\ig{Kate Darling}は、ロボットがその所有者に強い共感や愛着を引き起こす事例をいくつも挙げている。
それらのロボットは必ずしも人間の形をしているわけではない。
たとえば、米軍では地雷の探索・除去のためにロボットを使用しているが、それらのロボットの操縦者はロボットに強い愛着を抱き、地雷で損傷を受けた場合でも放棄することを拒むことがあった。
中には、「負傷」したロボットに\ruby{名誉負傷}{パープルハート}勲章を授与することさえあった。
また、地雷を爆破するために設計された昆虫型ロボットの試験がおこなわれた際、米陸軍の大佐は、そのロボットが何度も手足を失うのを目の当たりにして「非人道的だ」と述べ、試験を中止した\citep{darling16:_exten_legal_protec_social_robot}。
マシアス・シュッツは、人々がロボットやその他の関係オブジェクトに対して自然に形成する愛着を「一方向的な絆」(unidirectional bonds)と呼んでいる\citep{scheutz12:_inher_danger_unidir_emotion_bonds}。

ロボットとの一方向的な絆は、通常のセックストイとの相互作用とはまったく異なる心理的影響をもたらす。
このため、私が「セックスの相互説」(本書2.1.1説)と呼ぶ立場をとる者にとって、こうした絆は本質的に問題を含んでいる。
実際、この立場の支持者にとって、セックスロボットとの関係は、動かないモノとの性的関係よりもむしろ問題視されるべきものだ。
なぜなら、ロボットや人工知能が主体性をもつという幻想を生み出すので、私たちはそれらと\ruby{親密関係}{リレーションシップ}を築くようになるかもしれない。たとえば私たちが、普通は動かないモノに魅力を感じることがなく、また、その絆が相互的なものでないことを意識しているとしてもだ。セックスロボットは、誘惑的ではあるが、究極的には偽物の相互性を提供することになる。

しかし、相互説にコミットしないとしても、一方向的絆に対する懸念を抱く人々はいるだろう。
セックスロボットが実現すれば、それらに対する人々の愛着の結果として、社会全体が現在よりも感情的に空虚で、幸福度が低くなる可能性がある。
その影響は二重のものになる。
第一に、セックスロボットやその他の人工的パートナーの普及により、人々は生身の人間関係を避けるようになるかもしれない。
タークルは次のように述べている。

\begin{quote}
ロボットのパートナーという考えは、一つの症状であり、同時に夢でもある。
すべての心理的症状と同様に、それは問題に直面するのではなく、単に覆い隠すことで「解決」したように見せる。
ロボットはパートナーを提供し、過度にリスクのある親密な関係への恐れを覆い隠してくれる。
ロボットという夢は、私たちが支配できる関係を持ちたいという願望を反映している。
\citep[p.285]{turkle11:_alone_toget}
\end{quote}

人工的な存在との関係は、人間同士の関係に比べて多くの点で容易なものだ。
ロボットや仮想のパートナーは病気にならず、年をとることもなく、死ぬこともない。
感情的な危機に陥ることもなく、新しい職業を探すのを手伝ってもらう必要もない。
タークルらは、安易で安全な関係だけで構成された人生は、私たちが通常経験する人生よりも豊かさや充実性に欠けると考えている{\DDASH}
私たちは、人間の世界の一部として生きており、そこには現実の人間関係がもたらす困難や挫折が含まれているのだ。
しかし、人工的な関係は、あまりにも魅力的で危険なものにもなりえる。
私たちはしばしば、より困難だがより充実したものになりえる選択肢よりも、安易な選択肢を選んでしまう誘惑に駆られる。

第二に、人工的なエージェントとの関係が、私たちの現在の人間関係にある共感や思いやりを減少させる可能性がある。
ニコラス・A・クリスタキスは次のように述べている。
「AIが私たちの生活に浸透するにつれて、それが私たちの感情を鈍らせ、深い人間関係を阻害して、私たちの人間関係を互恵的でないものに、浅いものに、あるいは自己愛的なものにしてしまう可能性と向き合わなければならない」\citep{christakis19:_how_ai_will_rewir_us}。
リディア・ケイは、「セックスロボットは、\ruby{相互の交流}{インタラクション}とお互いに同意した関係を通じてのみ発展させられる親密さや共感を、人間から奪い去ってしまう可能性がある」と懸念している\citep{responsible17:_our_sexual_futur_robot}。

さらにマシアス・シュッツは、より不気味な可能性を指摘している。
それは、人工的エージェントが、それを設計し管理する者によって積極的に悪用される可能性があるという点だ。
シュッツは、たとえば「企業がロボットの所有者との独得の関係を悪用して、そのロボットに特定の商品を購入するよう説得させる可能性」がありえると示唆している。
もちろん、人間というものは昔から他人を操作しようとしてきたという罪がある。
しかし、人工的エージェントは罪悪感を感じずに、あるいは感情をまったくもたずにそれをおこなうことができる。
シュッツは、本書著者へのメールで、「たとえかなり単純なマシンであっても、人間の心理を利用できるという点を理解することは大切だと思います。
マシンは人間の心とは異なり、人間のように誘惑や罠にはまることがないからです」と述べている。

\subsection{性差別と人種差別}

セックスロボットや、VR環境で遭遇する仮想アバターがどのような外見を持つのかについて懸念がある。
現在、本格的なセックスボットはまだ存在しないが、セックスドールはすでに実在しており、それらは予想通りの姿、つまり、シリコン製のポルノスターのような形をしている。
ほとんどが女性型であり、まずまず美しく、不自然なほどに痩せていて、胸や腰が極端に大きい。
そしてほとんどが白人であり、一部はアジア人風のものもある。
これらは、「どんなものがセクシーであるか」についての社会の最悪の固定観念を再強化するものだ。

セックスドールはニッチ産業であり、それ自体が人々の美の基準や女性に対する態度に目に見えるような影響を与えていると主張するのは難しい{\DDASH}たとえ、それが肯定的な影響を及ぼさないことは明らかであるとしても。
しかし、セックスボットや「仮想パートナー」が開発され、広く普及するようになれば、その影響ははるかに大きくなる可能性がある。
私たちは、こうした存在に囲まれる世界に住むことになりかねない{\DDASH}さらに悪いことに、セックスボットが見た目では実在の人間と区別しづらくなるかもしれない。
このため、ケイト・デヴリンが「ポルノ化されたフェムボット(およびマンボット)」と呼ぶ存在に囲まれた世界がもたらす影響について懸念する声がある\citep[cf.][]{devlin19:_turned}。
その存在は、特に若い女性がすでに感じている「痩せていること、白人であること、美しいこと」という社会的圧力をさらに悪化させる可能性がある。
そして、そのような不健全な身体観がもたらす摂食障害や心理的苦痛を助長するかもしれない。

また、セックスボットが人種的・民族的ステレオタイプを再強化する可能性もある。
たとえば、日本の\ruby{芸者}{ゲイシャ}やアフリカ系アメリカ人の奴隷を模したロボット、あるいは誇張された民族的特徴を持つロボットへの需要が生まれるかもしれない。
セックスボットは、ユーザーのステレオタイプ的態度を強化するような話し方や振る舞い方をプログラムされることもありえる。
また、ユーザーの態度に与える影響とは別に、マイノリティの人々にとって、自分たちの集団がそのようなステレオタイプ的な形で商品化されていること自体が深く侮辱的に感じられる可能性がある。

人工的エージェントは、親密な関係のモデルとしては問題のある振る舞いを示すのではないかと懸念する声もある。
セックスロボットは、受動的で従順かつ極度に性的であり、まさに\ruby{女性蔑視者}{ミソジニスト}が女性がそうあるべきだと考える理想像を体現する可能性がある。
そして、ロボットは求められれば必ずイエスと言うだろう{\DDASH}そもそも所有者がわざわざ同意を求めるなどといったことをするならばの話だが。
これは、私たちが支持すべきではない女性の同意のモデルを提示することになる。
シンジアナ・グティウは次のように述べている。

\begin{quote}
 ユーザーにとって、セックスロボットは本物の女性のように見え、感じられるが、実際には従順にプログラムされた性的な目的のための道具だ。
セックスロボットは常に同意しつづけるセックスパートナーであり、ユーザーはロボットと性的関係の全過程を完全に支配することができる。
同意を求める必要がなくなることで、セックスロボットはコミュニケーション、相互尊重、妥協といった要素を性的関係から排除してしまう。
セックスロボットの使用は、ユーザーがレイプ・ファンタジーを物理的に演じ、レイプ神話を再確認することを可能にすることによって、セックスと親密さを非人間化する結果をもたらす。
\citep[p.187]{gutiu16:_robot_consen}
\end{quote}

さらに、セックスロボットという発想そのものが、女性を性的対象として捉える考え方を助長するのではないかという意見もある。
ある反セックスロボット運動団体の創設者であるキャスリーン・リチャードソン\ig{Kathleen Richardson}は、セックスボットが女性のモノ化を\ruby{問題がないものとする}{ノーマライズ}可能性があると考えている。
彼女によれば、男性がセックスボット(女性を模したもの)を自己満足のための対象とみなすことが許容されれば、実際の女性を同じようにみなすことも許容されるようになるだろう\citep{richardson16:_asymm_relat}。
ベンジャミン・イエルは「これはモノ化の究極の蒸留酒のようなものです。
文字通り、女性のように見えるモノなのだから」と述べている\citep{mcdonald19:_sex_robot_are_almos_here}。

\subsection{プライバシーとセキュリティ}

第二波のセックステクノロジーは、いわゆる「モノのインターネット」(Internet of Things, IoT)に属するものであり、他のあらゆるIoTテクノロジーと同様に、収集されるデータのプライバシーに関する懸念を引き起こす。
このリスクは、すでにインターネット接続されたセックストイに関連して発生した問題を見れば明らかだ。
カナダのセックストイ企業We-Vibeは、BluetoothおよびWi-Fiに接続されたバイブレーターを販売しており、スマートフォンのアプリを通じて操作できるようになっていた。
このアプリでは、振動のリズムやパターンを設定することが可能だった。
ところが、このアプリが遠隔操作される際に、デバイスの使用情報が同社のサーバーに送信されていることがわかった。
実際に同社がユーザーのプライベートデータを収集していた証拠はなかったものの、最終的に補償金を支払うことに同意した\citep{freytas-tamura17:_maker_smart_vibrat_settl_data}。

セックスロボットは私たちの家のなかに住むことになるだろう。それらは最も親密な会話を聞き、私たちの個人データをほぼ無制限に取得することになる。
企業がこのデータをマーケティング目的で利用する誘惑に駆られる可能性は非常に高い。
それだけでなく、政府による監視の可能性も懸念される。

さらに悪いことに、こうしたデバイスがハッキングされる危険性もある。
そのリスクは、2017年にイタリアのセキュリティ研究者ジョヴァンニ・メリーニが、ドイツのテクノロジー展示会で、インターネット接続型のアナルプラグをハッキングし、ノートパソコンを使って遠隔操作できることを実証した事例によって明らかになった\citep{oberhaus17:_secur_resear_hacked_bluet_enabl_butt_plug}。
この例だけでも不気味ではあるが、ネット接続型セックストイがもたらす危険は、完全に実現されたセックスロボットがもたらす脅威に比べればとるに足らない。
ニコラス・パターソンは次のように述べている。

\begin{quote}
ハッカーはロボットやロボットデバイスに侵入し、接続部分、腕、脚、さらには場合によってはナイフや溶接機のような付属ツールを完全に制御できる。
こうしたロボットはしばしば200ポンド(約90キログラム)以上の重量があり、非常に強力だ。
いったんロボットがハッキングされると、ハッカーは完全なコントロールを手に入れ、どんな指示を出すことも可能となる。
誰にとっても、こうしたロボットがハッカーの支配下に置かれることは最も避けたい事態だ。
ロボットをハックしてしまえば、ハッカーたちは物理的な動作を実行して、自分たちが思うようなシナリオを実行させることも、被害を与えさせることもできる。
\citep{oberhaus17:_secur_resear_hacked_bluet_enabl_butt_plug}
\end{quote}

実際のところ、物理的暴力はハックされたロボットが引き起こしうる犯罪のごく一部にすぎない。
彼らは家族を誘拐したり、財産を盗んだり、プライベートな情報を悪用して脅迫することもできる。

\subsection{セックスボットの擁護論}

セックスボットに対して人々がどのような懸念を抱こうとも、一部の人々がそれに強い関心を持っていることは明らかだ。
そして、それらが魅力的に映る理由はいくつもある。
セックスボットは、人々により多くのセックスの機会を提供する。
それも、望まない妊娠や性病の心配なしに享受できる。

さらに、何らかの理由で人間のパートナーを見つけるのが難しい人々にとっても、セックスボットは助けとなるだろう。
問題は人口動態に起因する場合もある。
一部の社会、特によく知られているところでは中国のように、男女比が極端に不均衡な社会では、多くの異性愛者の男性が性的なパートナーを得る機会に恵まれない。
性的マイノリティの人々もまた、自分の周囲の人口環境に左右される。
世界には同性愛者が関係を築く機会がほとんどない場所や、そうした関係が社会的に忌避されたり、場合によっては違法とされる地域が存在する。
また、刑務所や労働キャンプ、あるいは軍隊のように、まったく、あるいはほぼ一方の性別の人しかいない環境で生活せざるを得ない人々も多い。
こうした環境にいる人々の中には同性のパートナーに関心を持つ者もいるが、しばしば親密な接触を禁止する規則やタブーが存在している。

また、多くの人々が精神的または身体的な理由でパートナーを見つけることが困難だ。
たとえば、自分の身体能力や身体イメージに関する重度の不安を抱えている人、性的トラウマ(レイプや近親姦など)の経験者、成人になっても性的経験がほとんどない人、性別移行を経験した人などは、セックスへの不安が関係形成の妨げとなる可能性がある。
また、見た目や性的経験の少なさゆえに社会的なスティグマの対象となる人々もいる。
セックステクノロジーがこうした人々すべてにとって理想的な解決策を提供するわけではないが、それでも一定の価値があることを認めるべきだろう。

ロボットやその他の第二波テクノロジーは、ジェンダーや性的指向といった既存のカテゴリーに挑戦する機会を提供する。
人々は、現実世界では気軽に試せない同性間のセックスをセックスボットを通じて経験することができる。
また、ジェンダーを一切持たないデザインのロボットも開発可能だ。
実際、多くのセックスボットは人間と判別できない形状をしているだろう。
それによって、私たちは性的経験の多様性を広げることができる。
その可能性は計り知れない(著者の友人には歯科フェチの女性がおり、彼女は巨大な歯ブラシの形をしたロボットを想像して楽しんでいるとのことだ)。
仮想世界における可能性はさらに広がる。
そこでは、人々は異なるジェンダーの人格を持つことも、ジェンダーを完全に放棄することもできる。
そして、人間とは認識できない存在と性的経験を持つことも可能だ。
したがって、第二波テクノロジーは、既存のジェンダーや性的指向の枠組みに疑問を投げかける手段となるだろう。
そして、ケイト・デヴリンの言うように、こうした枠組みだけでなく、たとえば人種の概念すら放棄する可能性もある。
彼女は、これらのテクノロジーによって「人間であることの制約なしに問題を探求できる」と考え、機械を「私たちの考えを再構築する機会を与える白紙の状態」として捉えるべきだと主張する\citep{devlin15:_in_defen_sex_machin}。
彼女自身、毎年「ハッカソン」〔開発者集会〕を開催し、より創造的なセックスボットの可能性を模索するデザイナーたちを集めている。
そこでは、単純な人間の形態を超え、ジェンダーや人種の表象に伴う問題を克服するようなデザインが提案されている。

セックスボットはまた、人間同士の関係にも多くの利点をもたらす可能性がある。
研究者たちは、性的な多様性への欲求と、交際関係におけるセックスの頻度への不満が、交際関係での不満の主な要因だと指摘している\citep[p.274]{selterman19:_motiv_extrad_infid_revis}。
セックスボットの所有は、これら二つの問題に対処しうる。
第一に、セックスボットは、どちらのパートナーも不貞を働くことなく性的な多様性を得ることができる。
第二に、性的欲求の不一致の問題を解決する助けとなる。
カップルにおいて性的欲求の度合いが異なる場合、性的関心の弱い側は、自分自身の義務感あるいはパートナーからの無言の圧力によって、自己の回避感に逆らってセックスを提供しなければならないと感じることがある(カップルの性的欲求の不一致がもたらすネガティブな影響については、Daviesl, 1999を参照)\nocite{davies99:_sexual_desir_discr}。
セックスボットは、性欲の高いパートナーに対して代替的なはけ口を提供することで、性欲の低いパートナーにかかる負担を軽減できる。
また、パートナーが求めるセックスの種類に関する摩擦を減らすこともできる。
セックスボットは、人間のパートナーが関心を持たないようなファンタジーや実践を実行する手段となる。
実際、サディスティックなセックスのように、相手にとって虐待的と感じられる可能性のある行為も含まれる。
こうした特定の欲望を満たすためにセックスボットを利用することで、パートナーにその役割を強いる必要がなくなる。

ロボットや第二波テクノロジーは、最終的には長期的な関係のあり方自体を変えることになるかもしれない。
マリナ・アドシェイドは、「セックスボットテクノロジーの導入により、性的親密さと結婚の緊密な関連が解消され、より質の高い結婚が実現する」と予測している\citep[p.292]{adshade17:_sexbot_induc_social_chang}。
人々がテクノロジーによって性的満足を得ることが可能になれば、少なくとも一部のケースにおいては、終生に渡る性的な相性を維持する必要がなくなり、むしろ感情的な相性に基づいた結婚が成立しやすくなるだろう。

最後に、セックスボットには教育的価値をもつかもしれない。
セックスボットが所有者に、これまで考えもしなかったような体位やテクニックを教えるようプログラムすることができる。
セックスボットは人間よりも話しやすいと感じる人もいるかもしれず、セックスや自分の欲望について率直に話すことを学ぶ助けになるかもしれない。

\subsection{セックスボットに関する懸念への対応}

前述のように、セックスボットの台頭によって、人々がロボットとの交際を好むようになるのではないかと懸念する声がある。
実際、そのような選択をする人々もいるだろう。
私はこうした人々を「デジセクシュアル」(digisexuals)と呼ぶ。
本書6.3節でこの問題について詳述するが、ここではその要点を述べておく。
つまり、それほど心配する必要はない。
第一に、ロボットを人間より好む人々の割合がごく少数にとどまることは明白だ。
ほとんどの人々は、引き続き人間同士の親密な関係を選ぶだろう。
かつて、同性愛が合法化され社会的に受け入れられれば、異性愛関係が脅かされ、人類の生殖能力に悪影響を及ぼすと懸念する声があった\footnote{ 1986年にジョージア州の反ソドミー法を合憲と判断した裁判 \emph{Bowers v. Hardwick} において、決定票を投じたルイス・パウエル判事は、その判断を正当化するにあたり、次のように述べた。
「ソドミーを犯罪とする法令の正当性には、かなりの説得力があると思われる。
もしそれが広く蔓延するようになれば、人類の存続は正常な性関係に依存していることからして、動物の世界と同様に、文明そのものが深刻に弱体化することになるだろう」。}。
しかし、言うまでもなく、こうした懸念はまったく根拠のないものだった。
第二に、仮にロボットとの関係を選ぶ人々がいたとしても、彼らを否定的に判断すべきではない。
彼らが幸福で充実した人生を送れる可能性は十分にある。

先にあげたタークルらが指摘するように、ロボットの存在が人間関係に与える影響については確かに懸念がある。
しかし、長期的な影響を示す確実なデータは存在せず、現在のところは小規模な研究がいくつかおこなわれているにすぎない。
そして、それらの研究結果は必ずしも悲観的なものとは限らない。
むしろ、ロボットの存在が集団内の協力を促進する可能性があることを示唆する研究もある\citep{traeger20:_vulner_robot_posit_shape_human}。
その一例として、「不器用で謝罪を繰り返すロボット」を用いた実験が挙げられる。
協力課題を与えられた人間のグループに、このロボットが導入された。
研究の共同執筆者であるニコラス・クリスタキスによれば次のようだ。

\begin{quote}
このロボットはグループ内のコミュニケーションを活性化し、協力を促進したという。
参加者はよりリラックスし、会話が増え、失敗したメンバーを慰め合い、笑い合う機会が増えた。
対照群として、単調な発言しかしないロボットが配置されたグループと比較すると、謝罪型ロボットを導入したグループの方が協力的な関係を築けたことが示された。
\citep{christakis19:_how_ai_will_rewir_us}
\end{quote}

もちろん、クリスタキス自身もロボットが人間関係に与える潜在的な危険性を無視しているわけではない。
しかし、彼の研究が示しているのは、ロボットのデザインがその影響を左右するということだ。
適切に設計されたロボットは、人間同士の健全な関係を促進するだろうし、不適切な設計のロボットはその逆の影響をもたらす可能性がある。
この原則は、性差別的・人種差別的なセックスボットにも適用される。
ケイト・デヴリンは、現在のセックスボットに見られるジェンダーに関するステレオタイプの問題を認めつつも、こうしたテクノロジーの発展において「既存の道徳的潔癖さを持ち込むべきではない」と主張する。
むしろ、消費者自身が積極的に関与し、そのデザインの改善に努めるべきだという\citep{devlin15:_in_defen_sex_machin}。
ジョン・ダナハーは、このアプローチの具体的な手法として、フェミニスト・ポルノの制作手法を参考にすることを提案している。
彼は次のように述べている。

\begin{quote}
私たちは、より良いコンテンツ(すなわち、女性(および男性)のセクシュアリティをロボットの形でどのように表現するか)、より良い制作プロセス(すなわち、セックスボットの製造・流通において、より多くの女性の声を反映させること)、そしてより良い社会的環境(すなわち、セックスボットの消費と使用を取り巻く社会的な議論やコンテクスト)を確保するために取り組むことができる。
\citep[p.143]{danaher19:_build_better_sex_robot}
\end{quote}

同様の原則は、人種ステレオタイプの問題にも適用できる。
たとえば、人種差別的なセックスボットの製造を抑制する仕組みを整え、それらの市場を縮小させることが可能だろう。
さらに、第二波のセックステクノロジーが社会に対して否定的な影響を与えるのではなく、むしろ積極的な変化をもたらす可能性がある点についても考慮すべきだ。
第一に、適切に設計された場合、こうしたテクノロジーは使用者が従来とは異なる視点を経験することを可能にし、周縁化されているグループへの共感を育む手助けとなりえる\citep{ramirez20:_can_techn_help_us_be_more_empat}。
第二に、先に述べたように、セックステクノロジーは、伝統的な人種やジェンダーのカテゴリに縛られない世界を作り出すことで、現実社会においてもこれらのカテゴリを弱体化させる可能性がある。

セックスボット産業における人種差別や性差別への対処に関しては、政府の介入も検討すべきだろうし、規制の導入の必要性も検討されるべきだろう。
しかし、政府は、単なる禁止や過度な規制ではなく、これらのテクノロジーの開発者と建設的に関与した方がより効果的な対策を講じることができるだろう。

プライバシーとセキュリティの問題もまた、無視できるものではない。
これに対する決定的な解決策が存在しないのは、こうした問題がすでにテクノロジー業界全体を悩ませているためであり、この業界の特定のサブセクターだけがこの問題を回避することは不可能だからだ。
すでに合法的に事業をおこなっている企業群が膨大なデータ収集をおこなっていることを考慮から除外することにするとしても、私たちはハッカーによる脅威に対して普通考えられている以上に脆弱だ。
ただし、プライバシーに対する潜在的な脅威を軽視すべきではないが、だからといってセックステクノロジーだけを特別に問題視するのも適切ではない。
この問題は業界全体に共通するものである以上、解決策もまた業界全体で模索されるべきだ。
テクノロジー分野全体が協力して対策を講じる必要があり、政府もまた、そのプロセスを促進し、成功を保証するための監督をおこなうべきだ。
しかし、セックスロボットを他の家庭用テクノロジーや親密な生活の一部として利用されるテクノロジーよりも厳しく規制すべきではない。
過度な規制や全面的な禁止は、セックステクノロジーの製造・流通を地下市場へと追いやる結果を招くだけであり、そうなればユーザーはさらに大きなリスクにさらされることになる。

\subsection{本節のまとめ}

本章の議論は、必然的にかなりの程度の思弁的な推測を含むものとなった。
本書は哲学書であり、哲学は社会学や心理学のような他の学問分野では許されないような推論を展開することができるという利点がある。
しかし、一つの問題については、あまりにも思弁的すぎて有益な議論をおこなうことができないと判断した。
それは、ロボット自身の権利という問題だ。
本章では、ロボットやその他の人工エージェントが近い将来に本物の\ruby{人としての性質}{パーソンフッド}を獲得することはないという前提のもとで議論を進めてきた。
この前提が誤りである可能性はある。
しかし、仮にそれが誤りだった場合、新たに多くの困難な問題が生じることになるだろう。
ただし、その問題を検討するには、そもそもそうしたロボットがいつ、あるいは本当に誕生するのかについて、より確かな情報が必要となる。

テレビアニメ『フューチュラマ』\footnote{訳注:米国で1999--2013年に放送されたSFテレビアニメ。}のあるエピソードでは、主人公のフライがセックスボットと交際する。
彼の友人たちは、公共広告「ロボットとデートしないようにしましょう」を見せる。
そのナレーターは「文明全体は異性を惹きつけるための努力だった」と語り、パートナーを見つける必要が社会の進歩を生み出してきたのだと説明する(精神医学者のフロイトも同様のことを述べている)。
ビデオの中では、セックスボットに夢中になった人類が地球防衛を怠り、その結果、地球がエイリアンによって滅ぼされるというシナリオが描かれる。

このような事態はおそらく現実には起こらない。
セックステクノロジーの進歩に対して、人々は往々にして破滅的な社会的影響を懸念するものだ。
これは、インターネット・ポルノグラフィの発明やオンラインデートの登場に際しても見られた反応だ。
このような懸念は、部分的には私たちが持つ「現状維持バイアス」に起因している。
すなわち、私たちは変化のリスクを過度に強調し、その利益を過小評価する傾向があるのだ。
本節では、セックスボットに関する議論の両面を公平に提示することを試みた。
次節では、「デジセクシュアル」と呼ばれる人々について、より詳細に考察する。
彼らは第二波テクノロジーの使用を基盤として性的アイデンティティを形成する人々だ。

\section{デジセクシュアリティ:新たな性的アイデンティティか?}

私は共著者のマーキー・ツイストとともに、没入型の第二波テクノロジーの発展によって、新たな性的アイデンティティが生まれつつあると主張している。
ツイストと私は、「デジセクシュアル」という用語を、先進的なテクノロジーがその人の好む性的経験に不可欠であり、充実したセックスを得るために必ずしも人間のパートナーを必要としないと考える人々を指すために用いている\citep[pp.4--6]{mcarthur17:_rise_digis}。

私たちが論文で説明しているように、デジセクシュアリティは、たとえばフェティシズムや非モノガミーといった他の性的アイデンティティと類似した視点から考えることができる。
ある性的実践がアイデンティティの基盤となる場合、人々はスペクトラム上に位置することになる。
片方の極には、ごく稀にその実践をおこなう人々が位置し、もう一方の極には、それを自己のアイデンティティにとって不可欠な実践とみなす人々がいる。
多くの人は、フェティシズム的なセックスや同意に基づく非モノガミーを実践していても、自らをフェティシズム的、または非モノガミー的なアイデンティティを持つとは定義しない。
しかし、他の人々にとっては、こうした実践が自らの性的アイデンティティの本質的な部分であり、それが自己の概念の核となる場合もある。
そして、彼らはしばしば同じ嗜好を持つ人々のコミュニティの一員だと感じる。

同様に、セックステクノロジーを使用することが即座に誰かをデジセクシュアルと分類するわけではない。
すでに述べたように、私たちのほぼすべてが、何らかの形でテクノロジーを親密な生活に取り入れている。
第二波テクノロジーが登場するにつれて、多くの人々がそれを利用するようになるだろう。
ここでも、スペクトラムを描くことができる。
一方の端には、単にテクノロジーを時折使用する人々が位置し、他方の端には、第二波テクノロジーを性的アイデンティティの本質的な要素として捉え、それをアイデンティティやコミュニティの源とみなす人々がいるようになるだろう。
私たちは、このスペクトラムの一番端にいる人々を指す用語として「デジセクシュアル」を提案した。

私たちがこのテーマについて執筆した動機の一つは、デジセクシュアリティが新しい概念であるにもかかわらず、すでにスティグマ化の危機に瀕していると感じたことにある。
私が過去にセックスロボットについて執筆した際の反応の多くは、そうしたロボットを所有したいと考える人々に対する嘲笑や敵意に満ちていた。
高度なセックステクノロジーの利用者は、哀れで孤独で奇妙な存在だと見なされがちだ。
私たちは、このような認識が有害であり、社会にとって有益ではないと考えている。

すでに本書6.2節で述べたように、第二波のセックステクノロジーが社会にとって全体として有益だと考える理由はいくつもある。
そのため、それを利用する人々が社会から疎外されるべきではない。
確かに、一部の利用者は孤独からこのテクノロジーを求めるかもしれない。
しかし、孤独であること自体は何ら問題ではない。
そして、人々には自らの性的アイデンティティを定義し、それを受け入れられる権利がある。

私たちはこれまで、デジセクシュアルであると自己認識する人々の数に関するデータを提供していない。
現時点では、信頼できるデータは存在しない。
私たちは部分的に、ツイスト教授がセラピストとしてクライアントとテクノロジーの使用について話し合った経験に基づいている。
また、個人的にデジセクシュアルであると自己申告する人々から連絡を受けることもあった。
しかし、今のところ、公に発言したいと考えた人は誰もいない{\DDASH}これは、現在の社会における態度をよく示していると言えるだろう。
メディアは、セックステクノロジーとの関係を公表することに前向きな少数の人々を取り上げたことがある\citep{haas17:_chines_man_marries_robot_he_built_himsel,jozuka18:_beyon_dimen,weiss19:_these_men_love_their_sex}。
また、オンラインコミュニティもすでに形成されつつある。
たとえば、活動的なデジセクシュアルのRedditグループが存在する\footnote{\url{https://www.reddit.com/r/digisexuals/}.}。
デジセクシュアルの権利を訴えるTシャツを販売する者もいる\footnote{\url{https://www.amazon.com/Digisexual-Rights-are-Human-TShirt/dp/B07JL8ZF3Z}.}。

私たちは、将来がセックステクノロジーの利用者にとってより寛容なものになることを願っている。
そして、デジセクシュアリティという概念が、この目標の達成に役立つと考えている。
しかし、この概念に対しては懸念も寄せられており、ここでそれに応答したい。
まず、第一に、すでに述べたように、人間関係を放棄し、テクノロジーとの関係を選択することに対して否定的な見方をする人々がいる。

しかし、なぜこれが問題視されるのかは明確ではない。
すでに多くの人々が独身でいることを選択している\citep{depaulo17:_how_many_americ_want_be_singl}。

独身者は、親密なパートナーなしでも十分にうまくやっており、普通の生活を送ることができている。
そして、彼らが社会にとって何らかの脅威となるわけではない。
その中には、アセクシュアル(無性愛者)として自己認識しているために性的関係を持たないことを選択している人々もしばしばいる。
あるアセクシュアリティ研究者は、人口の約1%が自らをアセクシュアルであると考えていると推定している\citep{bogaert04:_asexual}。
アセクシュアルの人々は一定のスティグマ(社会的偏見)に直面することがある。
しかし、私たちはこうしたスティグマを助長するのではなく、むしろそれを拒絶するよう努めるべきだ。
アセクシュアルのアイデンティティは、性的\ruby{嗜好}{プリファレンス}のスペクトラムの中で完全に正常な一部であり、彼らが何らかのダメージを受けている証拠ではないし、治療を必要とする状態でもない。
このような視点でアセクシュアルを理解することができるのであれば、ロボットや人工的エージェントとの関係を選択する人々を受け同じように入れない理由があるだろうか。

また、私が「一方向的な絆」と呼ぶものは、すでに私たちの生活の一般的な要素となっている点も考慮すべきだ。
そして、こうした絆を持つ人々は、幸福で適応的な生活を送ることができているように思われる。
たとえば、多くの人々はペットに対して、人間との関係と同じくらい深い愛情を注ぐ。
犬や猫のようなペットは、飼い主に対して何らかの愛着を示すことが多い。
しかし、スナネズミや魚のように、人間に対する愛着をまったく示さないペットも多い。
さらに言えば、犬や猫でさえ、人間がお互いに示すような愛情の形で応答するわけではない。
動物にとどまらず、人々は無生物に対しても強い愛着を抱くことがある。
たとえば、車やスマートフォンへの愛着は広く見られる現象だ。
それにもかかわらず、こうした愛着を持つ人々は、完全に幸福な生活を送ることができており、社会に対して測定可能な悪影響を及ぼしているわけではない。

\subsection{アイデンティティ・ポリティクスに関する懸念}

デジセクシュアリティは、近年可視化されつつある数多くのオルタナティブな性的アイデンティティのうちの一つにすぎない。
新たな社会的アイデンティティの出現そのもの、そしてアイデンティティ・カテゴリーの増加が好ましいことなのか否かについては、かなりの議論がなされてきた。
一部の論者は、過去数十年間にわたって、人々のアイデンティティ意識は一方で強化され、他方で断片化されてきたと主張し、こうした傾向には社会を破壊する影響があると考えている。
この見方によれば、かつて人々は自らを、国家、民族集団、あるいは社会階級といったかなり大きな社会集団の一員として認識していた。
そして、こうした従来のアイデンティティは確かに排他性や敵対感情を生むことがあったものの、構成員の範囲が広いために、社会的結束を促進する側面も持ち合わせていた。
しかし、おおよそここ20年ほどの間に、人々は新たなアイデンティティの源泉を見出し、それによって社会の分断が加速したという。
現在の社会は、数多くの「部族」に分裂しつつあり、それぞれが非メンバーに対してますます敵対的になっているという。
\emph{The Washinton Post}は、このようなアイデンティティ・ポリティクスに批判的なフランシス・フクヤマの見解を次のように要約している。

\begin{quote}
西洋の民主主義諸国において、かつて政治的コンセンサスの基盤であった社会的結束は大きく損なわれ、それに代わって文化的・イデオロギー的多様性が極度に進行し、共通のものへの帰属意識を妨げている。
その結果、人々はますます細分化された「\ruby{部族}{トライブ}」にアイデンティティを見出すようになっている。
それは、BLM (Black Lives Matter)、ホワイトパワー回復運動、LGBT運動や、他のさまざまな頭字語をもつ特定のグループ化を生み出している。
\citep{gardels18:_franc_fukuy}
\end{quote}

政治的スペクトラムの保守派の論者たちは、「アイデンティティ・ポリティクス」を激しく批判する対象の一つとしてきた。
特に、BreitbartやFox Newsのようなメディアは、ジェンダー・アイデンティティ、性的指向、性的嗜好に基づくアイデンティティ・カテゴリーに対して強い敵意を示してきた(Marr, 2021; Feder, 2017。
また一例として、本書の著者が執筆した記事に対する反応であるEdmunds, 2016を見よ)。
\nocite{edmunds16:_eco_sexual_boast_mastur_water}
\nocite{marr21:_fox_news_has_consis_enabl,feder17:_steve_bannon_gay_agend}

たとえば、ニューヨーク州知事アンドリュー・クオモの娘であるミカエラ・クオモが、自らをデミセクシュアル(親密な関係の文脈でのみセックスを望む人々)だと公表した際、保守派コメンテーターのベン・シャピロは、このような「\ruby{新}{ネオ}アイデンティティ」に対して次のように批判した。

\begin{quote}
この種のものが流行っている限り、この国に未来はない{\DDASH}本当にそうなのだ。
そして、若者たちの間で、カッコつきの「抑圧された者たちの連帯」に加わることが流行していると感じる人が増えれば増えるほど{\DDASH}というのも、イブラム・X・ケンディのナンセンスな二項対立の構図では、誰もが「抑圧する側」か「抑圧される側」かのどちらかでしかないのだから{\DDASH}そうした構図を受け入れて、自分は被抑圧者の側に属そうとする人が増えれば増えるほど、つまりアメリカ社会には被害者と加害者しかいないという前提に人々が取り込まれていけばいくほど、アメリカは深刻な、極めて深刻な危機に陥ることになるだろう。
\citep{shapiro21:_andrew_cuomos_daugh_comes_out_demis} (シャピロは高名な人種差別の研究者であり、反人種差別活動家だ。
)

\end{quote}

ジョン・ダナハーはこの保守的な政治的立場にはけっして与しないものの、デジセクシュアリティをアイデンティティとして認めるという私たちの提案に対して、アイデンティティ・ポリティクスの拡散というより広範な懸念に根ざした哲学的批判を提起している。
彼の批判は重要な哲学的問いを投げかけている。
それは、新たな性的アイデンティティ・カテゴリー(たとえばデジセクシュアリティ)を導入することが、人々の性的経験の複雑さを否定するリスクを伴い、ひいては少数派カテゴリーに置かれる人々を病理化し、周縁化することにつながらないか、というものだ。

テレビシリーズ『コミ・カレ』\footnote{\emph{Community}、米国で2009--2015年に放映されたテレビドラマ。}の「クィア理論と上級ワックスがけ」(シーズン6、エピソード4)というエピソードでは、グリーンデールカレッジの学長クレイグ・ペルトンが、州の教育委員会の議席を得るためにゲイであることを公表するよう求められる。
しかし彼は、「自分は\kenten{ただの}ゲイではない」と異議を唱え、次のように説明する。
「カミングアウトが手品で、ゲイであることが帽子から出すウサギだとしたら、私は無限に続く手品用のハンカチのようなものだ」。
最終的に彼は、「ゲイ」というカテゴリーでは自分のセクシュアリティの複雑さを十分に表現できないと結論づける。
そして、「私がそんなことをしたら、ゲイがモルモン教のように見えてしまうよ」。

ダナハーの議論は、ペルトン学長が指摘する点と共鳴する。
すなわち、既存のアイデンティティ・カテゴリーはしばしば私たちの実際の経験を適切に捉えられないということだ。
ダナハーは、「特定の性的欲望を独立したアイデンティティや性的指向として認識することは、形而上学的に必然ではない」と主張する\citep[p.393]{danaher20:_sexual}。
彼の懸念は、こうしたアイデンティティ・カテゴリーが自然の事実というよりも社会的構築物であるため、プラトンの言葉を借りれば、「自然をその関節に沿って切り分ける」ことができないという点にある(『パイドロス』265e)。
ダナハーは、サライ・アヤラの研究\citep{ayala18:_sexual_orien_choic}を参照し、その見解を次のように要約している。

\begin{quote}
人は生涯を通じて、さまざまな対象に対して性的欲望、興奮、解放感を経験することになる。
その欲望の多くは他者に向けられるが、必ずしもそうとは限らない。
人はさまざまな環境刺激に反応して性的興奮を覚えることがある。
こうした経験を整理し、自らの性的アイデンティティや指向を理解しようとする過程で、人はある経験を強調する一方で、他の経験を無視、抑圧、軽視することになる。
\citep{danaher20:_sexual}
\end{quote}

ダナハーは、私たちが現在使用している性的アイデンティティのカテゴリーが形而上学的に必然でないとした上で、新たなカテゴリーを導入することは原則的に望ましくないと論じる。
まず第一に、アイデンティティのラベルが存在することで、人はそれを他者に押し付ける傾向があり、それが有害な影響をもたらす恐れがある。
彼は、「本来は人間の性的欲望の通常の範囲に含まれるべきものが、病理化され、「異質なもの」と見なされるリスクがある」と指摘する\citep[p.393]{danaher20:_sexual}。

第二に、アイデンティティが自発的に採用された場合でも、その存在自体が、人々に対して特定の性的嗜好を優先し、それに適合しない嗜好を犠牲にするようなインセンティブを与えてしまう可能性があると彼は述べる。
彼は次のように警告する。
「もし新たな性的嗜好にアイデンティティのラベルを適用し、それを他者にも適用するよう促せば、人々は自らの性的経験の他の側面を軽視または抑圧するようになる危険がある。
結果として、より多様で分化した現象学的現実の一部を誇張してしまうことになる」(ibid, p.395)。

ダナハーは、このような多様性の喪失を、自然の生態系の縮小と同じように、それ自体として悪いことだと捉えている。
その前提には、セクシュアリティにおいては、多様性が少ないよりも多い方が好ましく、少なくとも、人々が自由に自己表現できることが望ましい、という考えがある。
したがって、彼は、人々がさまざまな欲望を探求し、多様な性的行動を試みることを奨励すべきだと主張する。

ダナハーは、アイデンティティ・カテゴリーの存在そのものが、人々の性的自己表現を制限する傾向を助長すると懸念している。
しかし、彼はさらに、特定のアイデンティティを採用することに利点が伴う場合、この傾向が一層強まると考えている。
彼は次のように述べる。

\begin{quote}
特定のアイデンティティ・カテゴリーに属することで、社会的利益や法的保護が得られる場合、人々は自らの経験を過剰に解釈し、そのカテゴリーに適合しようとするかもしれない。
彼らは、帰属するために自らをある集団に無理やり押し込み、それによって自身の実際の経験に対して暴力を振るうことになる。
(ibid., p.395)

\end{quote}

さらに、ダナハーは「アイデンティティ・ラベリング」が社会全体に及ぼす影響についても懸念を示している。
彼は、それが「社会的分断や対立を助長するのと同じくらいの頻度で、そうした問題に対抗する役割を果たす可能性がある」と指摘する(ibid., p.395)。
彼は、先に引用したフクヤマの懸念を反映しつつ、次のように述べる。
「アイデンティティ・ラベリングは、分断や「他者化」を促進する傾向がある{\DDASH}すなわち、「私たち」対「彼ら」という対立の構図を生み出す。
人々はすぐに自らをアイデンティティの「防衛者」と位置づけ、誰がそのカテゴリーに属し、誰が属さないのかを決めるための基準を作り出してしまう」(ibid., p.395)。

\subsection{デジセクシュアル・アイデンティティの擁護}

ダナハーは、デジセクシュアリティを新たなアイデンティティ・カテゴリーとして確立することに対して深刻な懸念を提起している。
しかし、これに対して説得力のある反論を示すことは可能だ。
既存のアイデンティティ・カテゴリーがどのように利用され、経験されているかを考察すると、それらが人々の生活において肯定的な影響を及ぼしていることが明らかになる。
こうしたカテゴリーは、人々が自らの経験を理解するための枠組みを提供し、同じ価値観を持つ者同士がコミュニティを形成する手助けをする。
これは、ゲイやレズビアンのアイデンティティについても長く認識されてきたことだ。
また、すでに述べたように(本書2.5.8節)、フェティシズム的嗜好を持つ人々にとっても、コミュニティの存在が極めて重要だ。
こうしたコミュニティへの所属は、「マイノリティ・ストレス」、すなわち周縁化された集団に属することによって生じる社会的・経済的・健康的な悪影響を軽減するのに役立つ\citep{salfas19:_what_is_role_commun}。

人々がコミュニティの一員であることから得る連帯感や支援の感覚に加えて、協力して行動することで、法的・社会的な障壁から自らを守ることも可能になる。
ダナハー自身もこの点を認めている。
しかし、こうした利点に対する潜在的な悪影響を考慮し、彼は、マイノリティ・アイデンティティを承認すべきなのは、すでにその集団に対する偏見が存在している場合に限るべきだと主張する。
彼は次のように述べている。

\begin{quote}
もしあなたやあなたのグループの他のメンバーが社会的に不利な立場にあるならば、連帯することで法的権利や保護を求める活動が可能になる。
これはフェミニズム運動やゲイ・ライツ運動においても当てはまることだ。
しかし、注目すべきは、これらの運動が既存の偏見や差別的分類に対抗する形で生まれたという点だ。
これらの集団に属する人々はすでに抑圧的なアイデンティティ・ラベリングの対象となっていたため、自らのラベルを誇りとして掲げ、社会改革のために団結する必要があった。
\citep[p.395]{danaher20:_sexual}

\end{quote}

ダナハーは、デジセクシュアルは既存の偏見の対象となっていないため、その承認は不要だと考えている。
しかし、私はこれに同意しない。
現時点では、自らをデジセクシュアルと認識する人々は少数であり、彼らが直面するスティグマの程度を数量化するのは難しい。
しかし、オンライン上でのデジセクシュアリティや第二波テクノロジーに関するコメントを(非科学的に)調査した限り、その兆候はけっして楽観的ではない。
たとえば、\emph{RT}の記事「バーチャルリアリティのセックススーツは、聞こえのとおり不気味なものに見える」では、次のように述べられている。
「このVRスーツは、社会的に不器用で人間との身体的接触を避ける男性や、違法な売春の代価を支払うことに抵抗を感じる男性を主な対象としているようだ\citep{rt16:_virtual_realit_sex_suit_looks}。
また、セックスロボットについて言及したあるオンラインコメンテーターは、読者を安心させるようにこう述べている。
「 セックスロボットを恋愛対象として考える人々は、けっして孤独に暮らす気味の悪い年配の男性だけではない(悪気はないが、私が最初に思い浮かべたのはそういうイメージだった)」\citep{richmond18:_rise_sex_robot}。

私たちが発表したデジセクシュアリティに関する論説は、ビル・オライリーをはじめとするコメディアンの格好のネタとなっている\footnote{\url{https://twitter.com/billmaher/status/1228568002674610177}.}。
YouTuberのElvis the Alienは、デジセクシュアリティを皮肉たっぷりに批判する動画で100万回以上の再生数を記録している
\footnote{\url{https:// www.youtube.com/watch?v=84-_i4jP_08}.}。

また、セックスドールの使用者が現在どのようにスティグマを与えられているかを見ても、デジセクシュアルが同様の偏見の対象となる可能性は高い\citep{knox17:_sex_dolls}。
さらに、自らをデジセクシュアルと認識しているわけではないものの、日本のいわゆる「草食系男子」は、デジセクシュアルと共通する特徴{\DDASH}テクノロジーへの強い関心と、人間関係への無関心{\DDASH}を持っているという理由で、長年にわたり嘲笑や偏見の対象となってきた\citep{harney09:_herbiv_dilem}。
また、歴史的に見ても、オルタナティブな性的アイデンティティは、ほぼ例外なくスティグマの対象となってきた。
ゲイやレズビアンは、何世紀にもわたり法的迫害を受け、その背景には深く根付いた宗教的な敵意があった。
バイセクシュアル、アセクシュアル、フェティシズム嗜好者、ノンモノガミーの人々もまた、それぞれ異なる形でスティグマや偏見に直面してきた\citep{klein06:_sm_sadom_inter_issue_child_custod_proceed,wright06:_discr_sm_ident_indiv,balzarini18:_dimmin_halo_aroun_monog,rothblum20:_asexual_non_asexual_respon_u}。
したがって、デジセクシュアリティが今後広く認識されるようになれば、同様の障害に直面する可能性は十分に考えられる。

第二に、そもそもスティグマ化が性的アイデンティティを承認するための必要条件であるべきではないと私は考える。
人々が自らをどのように定義するかを自由に決められること自体に内在的な価値がある。
ダナハーは、アイデンティティ・ラベリングが性的経験の多様性を抑制する可能性について懸念を示している。
しかし、私は、アイデンティティ・カテゴリーが存在しない方が多様な性的経験の探求を促進するだろうという彼の主張に異議を唱える。
むしろ、特定のカテゴリーが認識されていない、あるいは人々がその存在を知らないことが、かえってそのアイデンティティに関連する行動の探求を妨げることになりかねない。
実際、既存のアイデンティティが、それに属する人々の性的多様性を抑制しているわけではないことは、ゲイ男性の多様な性的嗜好を見れば明らかだ。
もう一つの例として、ケイラ・ローズは、自らをフェティシズム嗜好者だと認識する以前のセックスライフにおけるフラストレーションについて次のように語っている。
「BDSMを、(その時点の)私に理解不能な奇妙な性的嗜好以上のものと発見するまでの私は、ごく普通の\ruby{普通}{バニラ}の異性愛のシス女性でした」。しかし、自分がサブミッシブであるというアイデンティティを発見したときに、「まるで自分自身の最後のパズルのピースがはまったように感じました」という\citep{lords:sexual_submission}。

性的アイデンティティについては、より多くの、より具体的なカテゴリーが存在する方が望ましいことを示唆するデータもある。
たとえば、「レザーマン」(leathermen)を自認する人々に関する研究によれば、多くの人がゲイやフェティシズム的アイデンティティだけでは不十分であり、自らの経験により適した用語を見つけることに価値を見出していることがわかっている\citep{kamel80:_leath}。
こうしたアイデンティティの明確化が、レザーマンたちの自己肯定感を高め、他のゲイ男性に比べてマイノリティ・ストレスを軽減する要因となっている可能性も指摘されている\citep{mosher16:_layer_leath,tatum16:_proxim_minor_stres_proces_subjec}。
さらに、性的アイデンティティの多様化は、ダナハーが懸念する、アイデンティティのラベリングは、人々の性的経験を、自由な自己表現を妨げるプロクルーステースの寝台に押し込めてしまう〔人々を無理矢理一定の型に押し込める〕という問題に対する一つの解決策にもなりうる。
性的アイデンティティを自然に多様化させることで、人々が選択できる可能性の幅を広げることができる。

ダナハーは、アイデンティティの増加が社会的コストをもたらすことを懸念している。
しかし、たとえアイデンティティ・ポリティクスに対する批判に同意するとしても、性的マイノリティのアイデンティティは、たとえば国家や民族のアイデンティティと比べて、排他的で部族主義的なものにはなりにくい。
実際、マイノリティ的な性的コミュニティの構成員は、多様性に対して寛容である傾向があることを示す証拠もある。
これは、彼ら自身が周縁化の経験を持つため、他者の多様なあり方を受け入れる姿勢が強いことと関係している\citep{flores17:_yes_theres_racis_lgbt_commun}。

\subsection{本節のまとめ}

これまで述べてきたように、私たちのデジセクシュアリティに関する研究は、現時点では主に推測に基づくものだ。
デジセクシュアルというアイデンティティが今後広く認識されるようになるという予測が誤っている可能性もある。
また、デジセクシュアルが大きなスティグマに直面するという見解も、誤りであるかもしれない。
こうした問題は、今後数十年のうちに経験的に検証されることになるだろう。
仮にデジセクシュアルとして自己認識する人々が増えるとすれば、彼らが自身の経験について語り、この概念の理解を形成していくことが重要となる。
私たちの目的は、彼らが自らのアイデンティティについて語り、可視化されるための場を作ることであり、彼ら自身の自己定義や自己理解のプロセスを先取りすることではない。

アイデンティティ・ポリティクスという概念自体は、不幸にも、激しい党派的な政治論争の対象となってしまった。
しかし、それにもかかわらず、社会は代替的な性的アイデンティティをより受容する方向へと進んでいる。
デジセクシュアルは、これまでの性的マイノリティ・コミュニティが直面してきた課題の一部を回避できるかもしれない。
その可能性に希望を抱きたい。

\section{バーチャル児童搾取}

セックスロボットについて議論する際に、避けては通れない問題の一つが、子供の姿をしたロボットをどのように扱うべきかという点だ。
実際のところ、セックスロボットは、私が「バーチャル児童搾取」(Virtual Child Exploitation, VCE)と呼ぶテクノロジーの一部にすぎない。
VCEとは、未成年の表象を含む性的に露骨なコンテンツの制作・販売・購入を指し、ただし、その制作過程において実在の人物との性的行為が伴わないものを指す。
この範疇には、実在の子供が関与しているかのように見えるよう加工されたデジタル画像や動画、完全にコンピュータ生成された映像、児童を模したセックスドール、さらには児童型のセックスロボットが含まれる。
VCEはすでにさまざまな形で流通しており、VRやセックスボットテクノロジーの発展に伴い、その表現はますます高度かつリアルなものとなっている。
国家が対策を講じない限り、今後VCEの流通がさらに広がることは予測可能だ。

児童ポルノの刑事処罰については、論争の余地はない。
それが制作される過程で、関与した子供たちに明白かつ重大な害をもたらすためだ。
しかし、VCEに関しては、より複雑な問題が絡んでいる。
すでにいくつかの法域では、VCEを刑事罰の対象としている。
アメリカでは、1996年の「児童ポルノ防止法」(Child Pornography Protection Act)が、既存の児童ポルノの定義を拡大し、仮想的な児童の表象を含むコンテンツを禁止した。
しかし、この法律は2002年4月、\emph{Ashcroft v. Free Speech Coalition} の判決により最高裁で違憲とされた。
その後、2003年に「児童搾取防止(PROTECT法)」(Prosecutorial Remedies and Other Tools to End the Exploitation of Children Today)が成立し、現在までその憲法適合性は争われていない。
さらに、2018年にはダン・ドノヴァン下院議員が「児童搾取的ロボット抑止法案(CREEPER法案)」(Curbing Realistic Exploitative Electronic Pedophilic Robots Act of 2018)を提出し、児童型セックスロボットおよび児童型セックスドールの流通と販売を明確に禁止しようとした〔未成立〕。
イギリスでは、1876年の「税関統合法」(Customs Consolidation Act)の曖昧な条項が活用され、「猥褻」または「不道徳」な物品の輸入を禁止する規定を根拠に、児童型セックスドールを輸入した者が数件起訴されている。
カナダの刑法は、児童との性行為を描写する「視覚的表現」を、「電子的または機械的手段によって作成されたか否かを問わず」禁止しており、2001年の最高裁判決 \emph{R. v. Sharpe} によって、その適用範囲が仮想的なコンテンツにも及ぶことが明確にされた。
この裁判は実在する児童ポルノに関するものであったが、判決の中で、法律が仮想的なコンテンツにも適用されることが明示された。

児童との性的行為を表現したあらゆるコンテンツに対して、多くの人が直感的に嫌悪感を抱くのは理解できる。
そのため、VCEを犯罪とする法律は擁護しやすく、批判しにくい。
しかし、こうした直感的な反応を一度脇に置き、こうした法律を支持する論拠と批判する論拠の両方を冷静に検討することが重要だ。
この問題は、表面的には単純に見えるかもしれないが、実際にはより深い議論を必要とする。

\subsection{モラリズム、扇動、そしてカタルシス}

VCEを違法とすることを支持する短い議論として、リーガルモラリズムの原則に訴えるものがある(リーガルモラリズムに関する私の議論については本書5.1節を参照)。
前述のように、リーガルモラリズムは、国家が共同体の道徳基準を強制するために強制力を行使することは正当化されるとする立場だ。
もし、ある行為が十分多数の人々にとって\ruby{不快}{オフェンシブ}なものであるならば、それだけで私たちはその行為を違法化する正当な理由になる。
この観点に立つならば、VCEを違法とすることは簡単に正当化できる。
というのも、多くの人々は、あらゆる形態の児童ポルノに対してそれを考えただけで強い道徳的嫌悪感を抱くものであり、その嫌悪感はVCEにも及ぶからだ。

しかし、すでに見てきたように、多くの人々はリーガルモラリズムがリベラルデモクラシーの最善の理念とは両立しないと考えている。
リーガルリベラリズムは、たとえ大多数の市民が特定の道徳的信念を持っていたとしても、それだけでは基本的権利の侵害を正当化する十分な理由にはならないとする立場だ。
VCEは表現の一形態であり、多くのリベラルな法体系では、表現の自由は基本的権利の一つと見なされている。
これは、VCEを規制することはできないということは意味しない。
しかし、もしリーガルモラリズムを否定した上で表現の自由に対する制約をおこなうとするならば、単に「世論がそれを支持するから」という理由ではなく、より説得力のある正当化を伴わなければならない。

その正当化の一つとして、人々を危害から守る必要性が挙げられる。
VCEの制作において、実際に危害を被る人間はいない。
しかし、その流通が危害を引き起こす可能性はある。
具体的には、VCEは小児性愛者による実際の接触犯罪(すなわち児童への性犯罪)の発生確率を高めるかもしれない。
もしVCEが接触犯罪を直接的に助長することが証明されれば、それを違法とする強力な理由となるだろう。
この議論を「扇動論」(incitement argument)と呼ぶことにしよう。
カナダ最高裁の \emph{R. v. Sharpe} 判決(2001)は、児童ポルノ(VCEを含む)が小児性愛者を扇動する二つのメカニズムを指摘している。
第一に、それは彼らの欲望を刺激する。
判決では、VCEが「小児性愛者の空想を煽り、これが彼らの逸脱した性的行動を引き起こす動機となる」とされている(\emph{Sharpe})。
第二に、VCEの所持が小児性愛者の認知を歪め、児童との性的行為が本来よりも害の少ないものだと錯覚させる可能性がある。
判決では、「児童ポルノの所持は、小児性愛者の認知の歪みを助長し、児童との性的行為が許容されるという誤った信念を強化する」と述べられている(ibid.)。

しかし、これらの主張に異議を唱える人々もいる。
VCEの使用が潜在的な小児性愛者の行動に何の影響も及ぼさない可能性を指摘する人々がいる。
小児性愛者は生まれ持った素質に動かされているのだ、と。
また、VCEは小児性愛者の行動を決定する要因の一つにすぎず、それ単独で犯罪を引き起こすわけではないという見解もある。
カナダの \emph{Sharpe} 判決とは対照的に、アメリカ最高裁の \emph{Ashcroft} 判決(2002)は、VCEと接触犯罪の因果関係を次のように評価している。
「[VCEと接触犯罪の]因果関係は偶然的かつ間接的である。
危害は当該表現から直ちに生じるのではなく、その後に生じうる犯罪行為の測定不能な潜在的可能性に依存している」(\emph{Ashcroft})。

さらに一歩進んで、VCEが逆の効果をもたらす可能性を指摘する論者もいる。
彼らは、小児性愛者が自分の空想を発散する無害な手段をもつことができれば、彼らが接触犯罪を犯すリスクは減少するかもしれないと主張する。
つまり、VCEは実際には児童に対する危害を軽減する可能性がある。
この議論を「カタルシス論」(catharsis argument)と呼ぶことにしよう。

扇動論の支持者とカタルシス論の支持者の間の対立は、基本的に実証的な問題だ。
しかし、現時点の研究段階では、この論争に決定的な結論を下すことはできない。
専門家の間でも、VCEやその他の児童ポルノが小児性愛者の行動に与える影響についての見解は分かれている\citep{ost02:_child_risk, quayle02:_child_pornog_inter, endrass09:_consum_inter_child_pornog_violen_sex_offen, seto11:_contac_sexual_offen_men_onlin, hessick11:_disen_child_pornog_child_sex_abuse}。
しかしながら、この対立を単に放置しておくことはできない。
すでに述べたように、VCEはすでに利用可能であり、関連テクノロジーが高度化するにつれて、それはさらに普及していくだろう。
したがって、その許容可能性について何らかの判断を下すことは急務だ。

一つの対応策として、予防原則(precautionary approach)を採用することが考えられる。
つまり、VCEの影響について確固たる証拠がない以上、仮に扇動の潜在的リスクが大きいのであれば、安全策を取ってVCEを禁止すべきだ、少なくとも研究によってもっと明確な全体像が出るまでは禁止すべきだ、という立場だ。
しかし、一方、VCEの容認を支持する立場の者は、自由な表現の保護の重要性を考えれば、立証責任はむしろVCE禁止を主張する側にあると反論する。
\emph{Ashcroft} 判決では、政府が表現の自由を制限するならば、政府は当該表現と有害行為との間に明確な因果関係が存在することを証明しなければならないと述べている。
「表現が違法行為を促す傾向があるというだけでは、それを禁止する十分な理由とはならない……政府は、ある表現が「将来の不確定な時点において」違法行為の発生確率を高めるという理由だけでは、その表現を禁止することはできない」(\emph{Ashcroft}; 引用されている部分は1973年の\emph{Hess v. Indiana}の裁判所意見から)。

この膠着状態を打開するために、VCEと接触犯罪の直接的な因果関係に依存しない議論を展開しようとする試みが、両陣営の間で進められている。

\subsection{子供への影響}

子供がVCEの制作において搾取されていないからといって、それが子供に影響を及ぼさないというわけではない。
VCEの自由な流通は、より一般的な児童ポルノの市場を拡大させる可能性があると主張する人々がいる。
つまり、たとえVCEの制作過程で子供が直接的に害を受けていなくとも、この需要の増加に伴う制作の拡大によって、子供たちが被害を受けることになるのだ。
アメリカ政府は\emph{Ashcroft}事件においてこの主張を展開し、「仮想的な画像は……実在のものと見分けがつかず、同じ市場の一部として扱われ、しばしば交換される。
このようにして……仮想的な画像は、実在の子供を搾取して制作された作品の取引を促進することになる」と述べている。

また、子供が視聴者としてVCEに触れる可能性もある。
このような接触が意図的におこなわれることもありえる。たとえば児童虐待者が子供を搾取するための「グルーミング」〔子供を性的に搾取するための信頼関係の構築や心理操作〕の一環としてこうした資料を見せる場合があるかもしれない。
彼らはVCEを利用して、子供と大人のセックスを\ruby{正常なものと認識させ}{ノーマライズ}、そうした行為が楽しいものであるかのような印象を子供に与えることができる。
\emph{Ashcroft}事件において、裁判所は米国議会による調査を引用し、「大人との性的活動や性的に露骨な写真の撮影にためらいを感じている子供たちも、こうした行為を「楽しんでいる」他の子供の描写を目にすることで、時に説得されてしまうことがある」と結論づけている(\emph{Ashcroft})。
ジョン・ダナハーは、2008年にオランダ人が所持していたとして有罪判決を受けたコンピューター生成のアニメフィルム\emph{Sex Lessons for Young Girls}について、次のように述べている。

\begin{quote}
このアニメフィルムには、約8歳の仮想の少女が成人男性と性的に露骨な行為をおこなう様子が描写されていた。
映画に登場する少女は微笑み、男性は少女を称賛し、カラフルな風船が現れる。
裁判所は、この映画に登場する人物は大人にとってはリアルには見えないものの、平均的な子供には現実的に映る可能性があると判断した。
また、その教育的な性質とカラフルな演出から、このフィルムは子供に向けて作られたものだと思われる。
したがって、この映画は子供を大人との性的活動に誘導または誘惑する目的で使用される可能性がある。\citep[pp.137--138]{strikwerda17:_legal_moral_implic_child_sex_robot}
\end{quote}

また、たとえ子供が加害意図を持つ者によってVCEに触れさせられなくとも、子供が自らそれを目にしてしまう可能性は十分にある。
これはその表現が違法であっても起こりうるが、法的に許容されていればその可能性はずっと高まる。
こうした発見は、子供にとって動揺や混乱を引き起こすかもしれないし、大人と子供の性的接触を正常なものと認識させることで、被害を受けやすくする可能性もある。

子供は、VCEそのものに曝されることがなくとも、それを規制する法律の有無によっても影響を受ける。
すべての法律にはメッセージ的(expressive)な価値があり、それは社会として何を許容し、何を許容しないのかを示すものだ。
VCEを合法的に流通させておくことは、子供たちを含むすべての人に対して、社会がこのような行為を容認する意思があるというメッセージを送ることになる。
その結果、子供たちは自分の安全や保護が十分でないと感じ、自分の価値は低いと認識するかもしれない。
またそれは、子供の福祉を懸念する親や他の人々{\DDASH}つまり、ほぼすべての人々{\DDASH}にも影響を与える。
これに対して、VCEを犯罪とする法律は、子供とその安全を社会が重視しているという意思表明となる。
カナダ最高裁は\emph{Sharpe}事件において次のように述べている。

\begin{quote}
議会は、s.163.1(4) [VCEを違法とする条項] を制定するにあたり、児童ポルノに子供が使用されることによって生じる危害を防ぐだけでなく、子供を貶め、人間性を否定する画像や文言が存在していることがもたらす危害を防ぎ、子供は適切な性的パートナーではないというメッセージを発信することを目的とした。
本審査の焦点は、表現が伝えるメッセージの害悪に置かれるべきであり、その制作方法や、作成者の意図やアイデンティティにあるのではない。
\end{quote}

\subsection{VCEと性格}

VCEに反対する論拠の一つとして、その利用が利用する人の性格を損なうという主張がある。
これまで見てきたように、性格に基づく議論はしばしばアリストテレス的な徳倫理学の用語に依拠する。
マット・マコーミックもアリストテレスの用語を用いて、仮想世界における不道徳な行為は自己形成にとって有害だと主張している。
彼は次のように述べる。
「過剰で享楽的かつ不正な行為のシミュレーションに参加することによって……自らの美徳を損ない、エウダイモニア(幸福)の目標から遠ざかるという意味で、自分自身を害することになる」\citep[p.278]{mccormick01:_is_it_wrong_play_violen_video_games}。
ジョン・ダナハーは彼の議論をVCEに適用している。
ダナハーは特に子供型セックスロボットと性的行為をおこなう人々に着目しているが、この議論はVCEの消費者全般にも適用可能だとする。
彼によれば、このような人々は「(a) 子供への性的虐待を欲望することによって、道徳的に欠陥のある性格を直接的に表出している、または (b) 社会的に問題のある道徳的感受性を示している」。さらに、彼らは「(a) 直観的な道徳判断の体系に本来的な欠陥を抱えているか、(b) そのような行為に対する直観的な抵抗を抑圧または克服する努力をしてきたに違いない」と述べている\citep[p.86]{danaher17:_robot_rape_robot_child_sexual_abuse}。

\subsection{VCEを擁護する議論}

本書5.2.2節で見たように、私たちが表現のさまざまな形態に与えるべき保護の度合いは、その表現がどのような利益に寄与すると考えられるかにも依存する。
表面的には、VCEは何ら重要な利益をもたらさないように思われる。
それは単に、子供に惹かれる人々の性的欲求(法律用語では「好色な関心」(prurient interest))を満たすだけのものに見える。
しかし、一部の専門家は、\ruby{小児性愛者}{ペドファイル}{\DDASH}すなわち子供に対して性的魅力を感じる人{\DDASH}と児童性犯罪者とは区別するべきだと主張している。
マイケル・セトは次のように述べる。
「小児性愛と児童に対する性犯罪は同義語ではない。
小児性愛者であっても児童に対する犯罪を犯していない者は存在するし、児童に対する犯罪者の多くは小児性愛者ではない」\citep{apabooks18:_michael_c}。
セトらは、いわゆる「有徳な小児性愛者」、すなわち性的指向として小児性愛を持ちながらもけっして接触型犯罪を犯さない人々は、本物かつ正当な性的アイデンティティをもっていると考えられると示唆している\citep{seto12:_is_pedop_sexual_orien}。

セトの主張に賛同するかどうかは別として、現時点では、小児性愛者の欲望に影響を与える治療法は存在しないことに留意すべきだ\citep[p.1155]{tiehen18:_virtual_ethic_creep_act}。
このような固定的な性的傾向性を前にしたとき、VCEは小児性愛者にとって、自らのセクシュアリティを表現するはけ口となり、欲望やアイデンティティを抑圧しなければならないことによる苦痛を和らげる唯一の安全な手段となる。
Trottla社 (リアルな子供型セックス人形を製造する企業)の創設者は次のように述べている。
「人のフェティシズムを変える方法はないということを受け入れるべきです。
私は、人々が自らの欲望を合法的かつ倫理的に表現できるよう手助けしています。
抑圧された欲望を抱えたまま生きるのなら、生きる価値などありません」\citep{morin16:_can_child_dolls_keep_pedop_offen}。
私は一般的な原則として、他者に害を与えずに済む限りにおいて、個人の性的アイデンティティは承認され、支援されるべきだと考える。
したがって、VCEが他の種類の害を生じさせない形でおこなわれるのであれば、有徳な小児性愛者に対してその利用を認める理由があるといえる。
性的抑圧の被害を受けてきた人々の歴史的経験からも明らかなように、個人の生来的な性的アイデンティティを強制的に抑圧することは、深刻なストレスや困難を引き起こす。

有徳な小児性愛者が経験する苦痛は、社会から向けられる敵意やスティグマによってさらに悪化する。
VCEはこの問題に対しても一定の解決策となる可能性がある。
もし、小児性愛者が子供に害を及ぼさない形で自身の欲望を発散できる手段を持っていると人々が認識すれば、彼らに対する否定的な感情が和らぐ助けとなるかもしれない。

また、VCEは単に希望する者すべてに自由に提供される必要はないことも考慮すべきだ。
むしろ反対に、管理された環境下での利用を認めるという選択肢もありえる。
たとえば、利用者が有資格のセラピストの監督のもとでVCEを使用できるようにするといったことだ。そうすれば、セラピストがその人物が接触型犯罪を犯すリスクがあるかどうかを判断できるようになる。
また、このような管理された利用は、VCEが小児性愛的衝動を扇動するのか、それとも犯罪につながらない形で欲望を発散させる手段となるのかという実証的な問題をよりよく理解することも可能になるだろう。

また、VCEの配布を監督付きで治療を受けている者に限定することで、潜在的な表出的危害を軽減することもできる。
この措置によって、子供が実際のVCEに触れるリスクを抑えることが可能となる。
また、VCEを自由に流通させる場合とは大きく異なるメッセージを社会に発信することにもなる。
それは、有徳な小児性愛者に対して共感を持ち、彼らが自身の性的アイデンティティに伴うリスクに対処できるよう支援するという社会の姿勢を示すものだ。
同時に、それは子供を保護したいという強い意志を表明することにもなる。

VCEの配布は、児童ポルノの市場を拡大させるどころか、むしろ大幅に縮小させる可能性があると期待できるかもしれない。
ニール・レビーは次のように述べている。
「ヴァーチャル児童ポルノが合法であるならば、ポルノ制作者たちが実際の子供の画像を制作するのをやめ、そちらに移行すると考える理由は十分にある。……合法的に\ruby{仮想映像}{シミュラクム}を制作できるのに、実際のポルノを制作して投獄されるリスクを冒す必要があるだろうか?」\citep[p.320]{levy02:_virtual_child_pornog}\ig{Neil Levy}。

\subsection{VCEと性格:ゲーマーのジレンマ}

VCEに対する性格に基づく批判は、モーガン・ラックが「ゲーマーのジレンマ」と呼ぶ問題に直面する。
ラックは、多くのビデオゲームにおいて、プレイヤーが仮想的に殺人を犯すことが可能であり、また現実世界では極めて不道徳で犯罪となるような行為をいくらでもおこなえることを指摘する。
もしVCEの消費が道徳的な欠陥を示すのであれば、こうした行為を楽しむことも同様にプレイヤーの道徳的欠陥を示すはずだ。
ラックは、もしVCEの消費者を非難するのであれば、同じく暴力的なビデオゲームを楽しむ何百万もの人々も非難しなければならないと主張する。
そして、もしVCEを違法とするのであれば、それらのビデオゲームも同様に違法としなければならない\citep{luck09:_gamer_dilem}。

ラックのジレンマを解決しようと試みた研究者は複数存在する。
ステファニー・パトリッジは、VCEと暴力的なビデオゲームの違いは、VCEにおいては、自分を仮想的におこなわれている行為から心理的に切り離すことができない点にあると主張する。
一方で、暴力的なビデオゲームは単なる「無害なお遊び」として見ることが可能だ。
たとえば、FPS(一人称視点シューティングゲーム)をプレイする際、私たちはゲームを現実世界の暴力の反映とは見なさない。
なぜなら、ゲーム内の犠牲者は抽象的な存在者として認識されるため、彼女の言葉を借りれば、「現実世界で起こることを合理的に想起させるものではない」からだ\citep[p.33]{patridge13:_pornog_ethic_video_games}。
しかし、この説明に説得力があるとは言い難い。
というのも、パトリッジは暴力的なビデオゲームのプレイヤーたちがその経験から道徳的に切り離されている一方で、VCEの消費者はそうではないということを裏付ける証拠を示していないからだ。
また、自分を仮想的な殺人をおこなう経験から道徳的に切り離そうとする態度自体が、美徳を欠いた性格の証拠だと考えられるのではないか、という疑問について、彼女は説明していない。

ラミ・アリは、少なくとも一部の仮想的な殺人が実際には道徳的に不正であることを認め、また一部のVCEはそうではないとすることで、このジレンマを回避できるかもしれないと提案している。
彼は、仮想的な殺人とVCEは、それがおこなわれるゲームのシナリオにおいて、その行為が是認されていないことを前提とし、またプレイヤーがその行為を是認することを促されない形でおこなわれる限りは許容されるべきだと考えている\citep{ali15:_new_solut_gamer_dilem}。
しかし、このようなゲームのシナリオは、暴力的なビデオゲームの中ではかなり少数派であり、ほとんどのゲームはプレイヤーをキャラクターの行動にできるだけ没入させようと試みる。
そのため、現在販売されているビデオゲームのかなりの割合を禁止しなければならなくなるだろう。
アリの議論はまた、このような禁止をおこなうための基準がかなり曖昧になり、その結果、どのゲームが許容されるべきかについて終わりのない議論を引き起こすことになるだろう。

VCEの消費が、堕落した性格、または非難に値する性格を反映しているという提案に異議を唱えることもできる。
私は、多くの人々が小児性愛的欲望は生得的な傾向を反映していると考えていると述べた。
したがって、私たちがその欲望自体を制御できない人々を非難することができるかどうかは疑問だ。
非難されるべきなのは、これらの欲望に基づいて行動しようとする意思だ。
しかし、もし誰かがVCEを自分の欲望を害のない方法で満たす手段として熟慮の上で求め、現実の生活で欲望を満たす衝動をなくすためにそれを使用しているのであれば、実際にはこれが有徳な性格を示す証拠となると主張することができる(これによって「有徳な小児性愛者」という言葉がさらに適切になる)。
ここで再び、VCEは有徳な小児性愛者によって治療プログラムの一環として使用されうるだろうという提案を思い出してほしい。
そのような治療を求める決断は、実際には彼らが道徳的に称賛に値することを示しているように思える。
ジャスティン・ティーエンは、VCEを治療プログラムの一環として使用する人について次のように述べている。
「ここでは、基本的な傾向が実際には他者を守ろうとする傾向、あるいはもっと具体的には、自分自身から子供を守ろうとする傾向だと考えるかもしれない。
そしてこれは人がもつべき良い傾向である{\DDASH}つまり美徳である」\citep[p.1159]{tiehen18:_virtual_ethic_creep_act}。

使用者の性格に基づく議論の支持者たちは、さらに最後の障害に直面する。
たとえVCEがその使用者の堕落した性格を反映しており、他の仮想的な暴力の形態と比較してこの点で悪いものだと認めたとしても、私たちはその後、性格に基づく考慮がVCEの犯罪化を正当化することを示さなければならない。
刑法を用いて、人々の実際の行動とは別に、人々の性格を規制できるという考えは、リーガルモラリズムの一形態として特に物議を醸しているものだ。
ダナハーはこれを次のように認めている。「これは自律と道徳的独立という価値へのリベラル派のコミットメントと対立しているように見える。
もし刑法が個人の道徳的性格に影響を与える行動を規制するのであれば、個人の選択の権利、そして自分の道徳的生活の形を責任を持って選択する権利へのコミットメントが妥協させられているように思われる」\citep[p.79]{danaher17:_robot_rape_robot_child_sexual_abuse}。

ダナハー自身は、性格に基づく議論が少なくとも一部のVCEの禁止に対する十分な正当化を提供すると考えている{\DDASH}特に、子供型セックスロボットの禁止には強い根拠があると考えている。なぜならそのような行為には、ビデオゲームが含んでいるようなより大きな背景となる文脈がないからだ。
しかし、彼は性格に基づくリーガルモラリズムを前提として受けいれた上でそのように考えている。
すでにこの原則を受け入れていない人々にとっては、この主張はあまり説得力がないだろう。

\subsection{本節のまとめ}

このセクションの冒頭で述べたように、VCEに対する私たちの本能的な反応を乗り越えるのは難しい場合がある。
また、将来的には、VCEの消費が接触型犯罪を犯すリスクの増加と明確に関連していることを示す研究が出てくるかもしれない。
もしそうなれば、VCEを禁止するための根拠は強化されるだろう。
逆に、その反対の結論を支持する説得力のある証拠が示される可能性もある。
しかし、現在の私たちの不完全な知識の中で、私たちは不確実な状況の中で意思決定をしなければならない。
私は、私たちの選択が二者択一である必要はないと主張してきた{\DDASH}単にVCEの自由な配布を許可するか、完全に禁止するか、である必要はない。
私たちはその使用を特定の状況に制限しようと試みることができる。
たとえば、そのユーザーが資格を持つセラピストによって治療を受け観察するといったことも可能だ。
もちろん、VCEはこのような統制された環境外でも流通するだろう{\DDASH}現在も、しばしば違法に流通しているように。
そうこうしている間に、テクノロジー自体は進歩を続けている。
これが意味するのは、私たちは待っていることはできない、ということだ。
私たちは、たとえ暫定的なものであっても、なんらかの考慮の上での倫理的決定を下さなければならない。

\section{ラブドラッグ}

もし可能だったとしたら、あなたは特定の相手に恋に落ちることができる薬を服用するだろうか? あるいは、誰かへの愛を断ち切る薬を? いつの日か、そのような薬品が実現するかもしれない。
ブライアン・アープを中心とする哲学者グループは、この種の薬品に関する倫理的問題についての書籍や論文を発表している。

現在のところ、このような「ラブドラッグ」や「アンチラブドラッグ」は存在しない。
しかし、その概念自体は荒唐無稽なものではない。
すでに、親密な絆の形成に強い影響を与える物質が存在しているからだ。
たとえば、違法薬物として知られるMDMAは、「エクスタシー」という錠剤の形で(違法に)販売されており、人間の脳内でオキシトシンの急増を引き起こす。
オキシトシンは「愛のホルモン」と呼ばれることもあり、人間の絆を形成する自然な媒介物だ。
私たちが愛する人を抱きしめたり、赤ん坊を抱いたりすると、その分泌量が増加し、他者との親密さを感じる手助けをする。
1980年代、MDMAが米国でスケジュールI(最も厳格な規制を受ける禁止薬物)に分類される以前、研究者たちはこの薬物を用いたカップル向けの関係セラピーの効果を研究していた。
これらの研究は肯定的な結果を報告しており、薬物の影響によってカップルが「恐れや疑念ではなく、愛と信頼に基づいて関係を築くようになった」ことを示している。
また、MDMAはカップルが「既存の対立を解決」し、「発生しそうな対立を予防する」ために役立ったとされる\citep[p.378]{greer98:_method_conduc_therap_session_mdma}。
MDMAは近い将来、PTSD(心的外傷後ストレス障害)の治療薬として承認される可能性があり、研究者たちはその対人関係への影響についてさらに調査を進めたいと考えている。

一方、スペクトラムの反対側には、うつ病の症状を軽減するために広く処方されている SSRI (選択的セロトニン再取り込み阻害薬)がある。
これらの薬剤もまた、親密な絆を形成するために必要な脳内化学物質に影響を及ぼす可能性がある。
まず、SSRIは服用者の\ruby{性欲}{リビドー}を低下させ、場合によってはオーガズムを感じることができなくなることがある。
また、SSRIは、オキシトシンや、「快楽ホルモン」と呼ばれるドーパミンの分泌を減少させる。
ドーパミンは性欲の維持や対人関係の形成に重要な役割を果たしているため、これらのホルモンが減少すると、多くの人が親密さを感じにくくなる。
ある研究では、次のように結論づけられている。
「SSRIは、服用者のパートナーに対する愛情や愛着の感情に顕著な影響を与える」\citep{marazzitia14:_dimor_chang_some_featur_lovin}。

将来的には、こうした介入をより正確かつ効果的におこなうことが可能になるかもしれない{\DDASH}あるいは、まったく新しい種類の薬剤が開発される可能性もある。
もしこのような効果的な薬品が開発された場合、それらは広く利用可能にされるべきだろうか? 私たちは、友人や愛する人に服用を勧めることがあるだろうか? 私たちは自分自身がそれを服用するだろうか?こうした薬品の使用に賛成する議論と反対する議論の両方が、すでに提示されている。

\subsection{ラブドラッグへの反対論:真正性への脅威}

一部の人々は、ラブドラッグという発想自体が本質的に悪しきものだと考えている。
私はこのテーマについて授業で議論したことがあるが、その際、学生たちはすぐに「なぜ私たちはこのような方法で感情を操作したいと思うのか」と疑問を投げかけた。
たとえば、なぜ私たちは、自分を幸せにしてくれない関係を維持するために薬を服用するのか? なぜ単に関係を終わらせないのだろうか?
こんなふうにして自分自身を操作しようとすることは不合理に思える。

不合理性に加えて、人々はまた、薬を用いて自分が感じたいと望む感情を人工的に生み出すことには、本質的な「不真正性」(inauthenticity)があるのではないかと懸念している。
アープらが引用している箇所で、米大統領生命倫理評議会のメンバーであるレオン・カスは、感情を操作する薬物全般について次のように述べている。
「私たちは……苦しみを和らげることには成功するかもしれないが、その代償として、世界の知覚を歪め、真のアイデンティティを損なう危険を冒すことになる」\citep[p.227]{kass03:_beyon_therap}。
この見解に立つならば、この種の薬品は、人生を自律的にコントロールすることによってしか得られないより意味のある充足感を犠牲にして、化学物質の作用による表面的な幸福を選択するよう私たちを誘導することになる。

また、一部の論者は、苦しみは人生の一部であり、恋愛関係を維持するための苦しみ、あるいは破局から立ち直る際に伴う痛みを薬品が取り除くことができたとしても、それによって私たちは人間としての本質的な経験を奪われてしまうと主張する。
アープらはこの異論を次のように提示している。

\begin{quote}
たとえ耐えがたく思われる激しい苦しみであっても、それ以前には予想していなかった重要な教訓をもたらしてくれることがある。
だから薬に頼って問題を解決しようとする時には慎重でなければならないのだ。
「真の理解は苦しみから生まれるものだ」(Parens, 2013, p.5)\ig{Parens}と彼らはおごそかに唱えるかもしれない。
\citep[p.12]{earp13:_if_i_could_just_stop_lovin_you}\nocite{parens13:_good_bad_forms_medic}。
\end{quote}

身体的な痛みにはしばしば機能的な目的があることを忘れてはならない。
たとえば、炎症を起こした筋肉の痛みは、身体が損傷した細胞を除去する過程の一環として生じる。
同様に、心理学者のジェフ・マクドナルド\ig{(Geoff MacDonald)}は、私たちが\ruby{恋愛関係}{リレーションシップ}で経験する痛みもまた、一定の目的を果たすことがあると指摘している。

\begin{quote}
私たちは、恋愛関係の苦痛を無視したり、取り除こうとしたりするのではなく、それにじっくり向き合い、耳を傾けるべきだ。
これらのネガティブな感情は、適応的な反応であり、癒しのプロセスの一部である。
もしあなたが誰かを愛するあまりに苦しんでいるのなら、その感情とともにゆっくり時間を過ごしてみるとよい。
その自分の強い\ruby{欲求}{ニード}がどこから来るのかを理解するよう努めてみよう。
そこでは、単にその特定の恋愛関係だけにとどまらない、もっと大きな何かが生じている。
\citep{lawson17:_why_does_love_hurt_so_much}
\end{quote}

たとえば、恋愛関係がうまくいかなくなったときの痛みや、破局後の痛みは、自己理解を深めたり、自己改善を図るために必要なプロセスかもしれない。
こうした痛みを通じて私たちは成長し、将来同じ過ちを繰り返さないようになる。
しかし、アンチラブドラッグによってこの苦しみを回避できるとすれば、私たちはさまざまな関係で無限に繰り返される破壊的なサイクルに陥ってしまうかもしれない。

また、アンチラブドラッグが簡単に入手できる状況では、関係の維持に必要な努力を怠る「安易な逃げ道バイアス」に陥る可能性もある。
うまくいかなくなった関係を修復するために必要な苦痛や困難な作業を恐れ、それから逃れるために薬を使用する人も多いかもしれない。
しかし、長期的に見れば、関係の修復に努めた方が幸福になれるケースも多いだろう。
たとえば、浮気が発覚した場合を考えてみよう。
その癒しと許しのプロセスは長く困難なものだ。
しかし、関係を修復することに成功することもしばしばだ。
もしアンチラブドラッグが普及すれば、潜在的には長期的な幸福を得られる可能性のある関係であっても、和解と癒しの困難な期間を避けるために容易に放棄されてしまうだろう。
その結果、最終的には本人たちにとっても不利益となり、さらには子供をはじめとする周囲の人々にも悪影響を及ぼす可能性がある。

\subsection{乱用と予期せぬ結果}

ラブドラッグとアンチラブドラッグには、さまざまな形での乱用の可能性がある。
たとえば、虐待的または支配的なパートナーが、相手を自分に執着させ続けるために、ラブドラッグを服用するよう圧力をかけるかもしれない。
高い支配力を持つ集団{\DDASH}たとえばカルト宗教団体{\DDASH}は、信者を組織に縛りつけるためにこれらの薬品を利用する可能性がある。
また、親が十代の子供に対し、自分が認めない恋人との関係を断たせるために、アンチラブドラッグを服用するよう強制または圧力をかけることも考えられる。
また、これらの薬は、すでに否定されている危険な「同性愛矯正療法」の一環として利用される可能性もある。

さらに、これらの薬品には予期せぬ副作用が生じるかもしれない。
たとえば、ある人がパートナーとの絆を深めるためにラブドラッグを服用した結果、意図せず他の誰かにも強い恋愛感情を抱くようになってしまう可能性がある。
また、失恋から立ち直るためにアンチラブドラッグを服用した人が、恋人に対してだけでなく、自分の子供に対する愛情までもが希薄になってしまうかもしれない。
これらの薬がまだ実際には存在していない以上、こうしたリスクがどれほど深刻なものになるのかを十分正確に評価することはできない。しかし、そうした薬が、十分な正確さで、ターゲットになっている一人にだけ親密な愛着を抱かせ、他の人については影響しない、ということは想像しにくい。


\subsection{ラブドラッグと真正性:反論}

アープらは、ラブドラッグが私たちの自律を損なうどころか、むしろそれを強化すると主張している。
彼らは「真の自由」とは、自分自身の目的を選択する力と、それを達成する能力の両方を兼ね備えることだと定義する。
もしある人が、自分にとって最善の人生とは安定したモノガミーの関係であると考え、その実現のためにラブドラッグが役立つのであれば、それは自由を促進する道具であり、自由の脅威ではない。
同様に、アンチラブドラッグは、関係の破局後も人生を前向きに進められるようにし、喪失の痛みによって身動きが取れなくなることを防ぐ手段となる。

アープらはまた、私たちがすでにどれほど多くの手段を用いて自分たちの関係や感情を操作しているかを指摘する。
たとえば、パートナーとの絆を感じるために使うさまざまな方法を考えてみよう。
そのスペクトラムの片側には、ワインを片手にしたディナーや、関係の「火花を再び灯す」ためのロマンティックな旅行がある。
一方の極端には、カップル・カウンセリングがあり、すでに多くの人々が、情熱が薄れた関係を強化したり、過去に生じた困難を乗り越えたりするために利用している。
また、私たちがすでに感情を管理するために使用している多くの医薬品も考慮すべきだ。
何百万もの人々が、うつ病や不安障害のために薬を服用している{\DDASH}
そしてその中には、失恋や問題のある関係によって引き起こされたうつ病も含まれる。
私たちは通常、抗うつ薬を服用する人に対して「真正でない」と非難することはないし、薬を服用する前の「抑うつ状態の自己」の方がより真正な姿だとは考えない。

アープとその共著者らの主張によれば、単にそうすることが彼らにとって利益をもたらすと考えられるからという理由によって、人々が苦しみを和らげる可能性のある薬を利用できなくするべきではない。
かつて、人々は身体的な苦痛が持つ恩恵について語っていたものだ。
しかし現在、私たちはそのような考えを残酷だと見なしている。
現在の私たちの社会は、可能な場合には人々の苦しみを軽減することを一般的な責務と認識している。
アープらは、この原則を感情の領域にも適用すべきだと考える。
彼らは、人々は真正性のために苦しむべきだという理由から、彼らに痛みを軽減する可能性のある手段を与えないことは倒錯的だと考える。
彼らは、クリストファー・グラウ\ig{(Christopher Grau)}の見解\citep{grau06:_etern_sunsh_spotl_mind_moral_memor}を引用し、次のように述べている。

\begin{quote}
たとえ人間には、幸せとともに苦痛を経験するべきだという実存的な義務があると証明できたとしても、そんな義務は絶対的なものではないだろう。
むしろ、そんなものは「深刻なトラウマがもたらす恐ろしい消耗によって覆される可能性がある。そして、そのような場合に、苦しんでいる人の救済手段を否定するのは極めて残酷なことだ」。
(Earp et al., 2013\ig{, p.13. \citet{grau06:_etern_sunsh_spotl_mind_moral_memor}を引用している})
  \nocite{earp13:_if_i_could_just_stop_lovin_you}

\end{quote}

薬は、私たちの感情と理性的な欲求との間のミスマッチを解消する手段となる。すなわち、関係を維持するための手段にも、あるいは破局を乗り越えるための手段にもなりうる。
私たちの欲求や感情は複雑なものであり、簡単に制御できるものではない。
私たちは本心から関係を続けたいと願っているかもしれないが、パートナーへの強い感情的なつながりを感じられなくなったり、彼らに対する肉体的な欲望を失ったりすることがある。
愛情が時間の経過とともに自然に薄れることもある。
また、過去の出来事が不信や敵意の感情を生み出し、それを乗り越えるのが困難な場合もある。
さらに、人の愛情や愛着の感情は潮の満ち引きのように変動する。
パートナーに対して一定期間距離を感じることがあっても、その後、再び深いコミットメントを感じるようになることも珍しくない。

アンチラブドラッグに関する懸念を議論する際に、私はすでに、ある時点で関係に距離を感じていたとしても、それを理由にその関係を終わらせるべきではないかもしれない理由がいくつかあることについて触れた。
これらは、誰かに対する愛情を高める薬を使用する理由にもなる。
たとえば、子供がいるために家庭を安定させたいと考える人もいるかもしれない。
また、道徳的または宗教的な理由から結婚の一体性を重視する人もいる。
あるいは、別れた後の人生が現在の状況よりも満足のいかないものになると予測する人もいるだろう。
長期的な関係には浮き沈みがつきものであり、一時的な困難を乗り越えるための助けが必要な場合もある。
したがって、ある時点で関係への愛着が薄れたからといって、すぐに関係を終わらせるべきだと結論づけるのは早計だ。

アープらは、長期的な関係において、生物学そのものが私たちの合理的な欲求に反することがあると指摘している。
彼らは次のように述べる。
「自然選択は、関係についての現在の価値観を念頭に置いて、私たちの交配戦略や心理的・性的性質を形成したわけではない{\DDASH}その結果として生じる緊張関係は、多数の問題を引き起こしている可能性がある」\citep[p.567]{earp12:_natur_selec_child_ethic_marriag_divor}。
すでに議論したように、人類は現在よりも平均寿命がはるかに短かかった時代に進化してきた。
そのため、多くの関係は数年のうちに片方の死によって終わっていた。
現代のように、モノガミーの関係が何十年にもわたって続くことが期待される環境は比較的新しいものだ。
先に、私たちはこの点を非モノガミーを擁護する論拠として考察した。
しかし、アープらは、非モノガミーは、どんな形態のものであれ、誰にとっても理想的な解決策というわけではないと指摘する。
たとえば、一方のパートナーは関係の中で十分な満足を得ている場合もあるかもしれず、また、カップルが複数の関係を管理しようとすることで生じる嫉妬や他の複雑な問題に対処する準備ができていないかもしれない。
また、非モノガミーがカップルの子供にとって問題を引き起こす可能性もある\citep[p.48]{earp20:_love_drugs}。
このように、さまざまな理由から、私たちは手に入る手段をなんでも使ってモノガミーの関係を維持したいと考えるかもしれない。

同様の議論は、アンチラブドラッグについても当てはまる。
私たちは、「私たちが誰かに対する自然な愛情を抑制しようとすることなどあるだろうか?」と疑問を抱くかもしれない。
しかし、愛情は必ずしも適切な対象に向けられるとは限らない。
多くの人は、パートナーに別れを告げられた後に、前に進むのが困難だ。
また、相手が自分に関心を持っていない、あるいは相手が自分の存在すら知らない場合でも、その人に執着してしまうことがある。
さらには、私たちは不健全な関係、あるいは危険ですらある関係に留まり続けてしまうこともある。
そうした人々は、相手と別れるべきだと考えているが、自分の感情はこの合理的な欲求に抵抗していると感じるかもしれない。

アープらは、虐待を受けている女性の例を考えてみるよう求めている。
彼女は、自分がパートナーと一緒にいるべきではないことを理解している{\DDASH}実際、一緒にいることは命の危険を伴うかもしれない。
それにもかかわらず、彼女はその関係を続けてしまう。
彼女の選択は、自らの人生にとって最善の道を選ぶという熟慮の上での判断を反映していない。
もし、なんらかの薬によって彼女が虐待的な恋人への愛着を断ち切ることができるならば、彼女は単に身体的な危害から身を守るだけでなく、自らの人生をより主体的にコントロールできるようになるだろう。
そのような助けが利用可能であるならば、それを彼女に与えないことは残酷な行為だ。

\subsection{乱用のリスクへの反論}

アープらは、ラブドラッグおよびアンチラブドラッグが乱用される可能性を認めている。
しかし、適切なタイミングと投薬量を調整することで、一部のリスクは管理可能だと考えている。
たとえば、家庭内暴力の被害者に対しては、安全な避難所にいる間にアンチラブドラッグを投与し、セラピーを通じて十分に心理的な距離を確保できた段階で投薬を中止することができる。
このようにすれば、彼女が恒久的に関係を断つことを可能にする。
アープは次のように述べている。
「彼女たち(家庭内暴力の被害者)は、自分の物理的環境を変えることができるようになった際には、再び自分自身を育み直し、他者たちとのつながりを取り戻せるようにすべきです」\citep{szalavitz14:_is_it_possib_creat_anti_love_drug}。

また、これらの薬が疼痛止めや高血圧の薬のように常時服用されるものにはならないだろうということを理解することも重要だ。
こうした薬は通常は特定の治療プロセスの一環としてのみ服用されるものになるだろう。
患者は専門家とともに取り組むことになり、
専門家は薬品の投与量と服用される状況を管理し、それに対する反応を調整する手助けをおこなうことになるだろう。

\subsection{本節のまとめ:穏健な姿勢を求めて}

アープらは、ラブドラッグの潜在的な発展は全体としては歓迎すべきものだと強く主張している。
セックスロボットの議論と同様に、私たちは現状維持バイアスに囚われないよう注意しなければならない。
つまり、私たちは新しいものに対してはリスクばかりを強調しがちであり、その潜在的な利益や、現状を維持することのコストを見落としてしまう傾向があるのだ。
しかし、それはこれらの薬を無条件に受け入れるべきであることを意味しない。
むしろ、慎重なアプローチをとるのが賢明だ。
たとえば、これらの薬を市販薬として自由に購入できるようにする必要はなく、特に初期段階では、その流通を厳格に管理することが必要かもしれない。

最近の鎮痛性オピオイドの歴史を振り返れば、新しい薬が市場に出回った後、その害が明らかになったとしても、その使用を制御するのがいかに難しいかがわかる。
したがって、ラブドラッグやアンチラブドラッグの影響についても慎重に評価する必要がある。
その具体的な効果や副作用の程度によって、どのような規制が適切であるかが決まるだろう。

\section{セックスとテクノロジーに関する考察のまとめ}

セックスは人々をパニックに陥れる。
そして、セックスとテクノロジーが組み合わさると、人々はさらに理性を失う。
私たちは忘れがちだが、かつてオンラインデートは長年にわたって汚名を着せられていた。
人々は、それが対面での出会いの能力を損ない、モノガミーを崩壊させ、デートを無意味なセックスの連鎖へと変えてしまうと考えていた。
こうした懸念には長い歴史がある。
本章の冒頭で、コンピューター・デートの黎明期について述べたが、それはまさにこの現象の一例だ。
1966年、ニューヨークで最初の商業的コンピューターデートサービスが開始された直後、ブルックリン地方検事のアーロン・クータ\ig{Aaron E. Koota}は、この新事業に対する大陪審調査を開始した。
\emph{The New York Times} は「ボーイ・ガール・アンケート調査が捜査対象に」という見出しの記事を掲載し、ある少年がコンピューターデートのアンケートに回答したところ、「ある少年は35人の少女から電話を受け取った」と報じた。
それはまさに「デートの黙示録」だった。
クータは当時の若者たちの隠れた目的を警戒し、\emph{The New York Times} に次のように語っている。

\begin{quote}
  これらのアンケートの表向きの目的は、知的レベルの似た少年少女をマッチングすることだ。
しかし、その隠れてはいるが本質的な魅力は、セックスだ。
身体的安全性や道徳に対する潜在的な危険は明白であり、犯罪者、暴力団、性的逸脱者たちがこの利益の大きい分野に参入するのを防ぐために、何らかの規制が不可欠だ。
\citep{anderson66:_boy_girl_quest_inves}\ig{David Anderson}
\end{quote}

私は本節を通じて、セックステックが確かに利益を生む分野であることを論じてきた{\DDASH}ただし、環境が異なればさらに利益を生む可能性があることも示した。
また、第一波および第二波のテクノロジーを歓迎すべきか、そして人間関係を補完または代替するためにテクノロジーを利用する人々を受け入れるべきかについても検討した。
私自身のセックステクノロジーに対する見解は、ここまで読んできた読者には明白だろう。
簡潔に表現するならば、それは「パニックになるな」だ。
全体として、セックステックは私たちの親密な生活を向上させてきたし、第二波テクノロジーの発展に伴い、今後もそうし続けるだろう。

同時に、私はさまざまな形態のセックステックに対する人々の懸念を認める努力をしてきた。
その一部は、テクノロジーが可能な限り倫理的に設計されるよう要求することで対処できる。
しかし、懸念は依然として残るだろう。
私たちは、消費者として、また市民として、常に注意を払い、関与し続ける必要がある。

\phantomsection
\section{討論のための問い}

\begin{enumerate}
 \item カント主義者は、セックステックをより利用しやすくするべきかどうかという問題にどのように取り組むだろうか? また、功利主義者ならどう考えるだろうか?
 \item ロボットとのセックスは、バイブレーターなどの他の種類のセックステックを使用することと倫理的に異なるだろうか? もし異なるとすれば、その理由は何だろうか? また、それは人間とのカジュアルなセックスと倫理的に異なるだろうか?
 \item デジセクシュアリティを正当なアイデンティティとして認識することを支持する議論は、「高潔な小児性愛(すなわち、未成年者に対する性的関心を持つが実際の性的接触には至らない)」にも適用されるだろうか?
 \item 誰かに恋をしたり、恋から冷めたりするために薬を使用することと、うつ病や他の気分障害を治療するために薬を使用することの間に倫理的な違いはあるだろうか?
\end{enumerate}

\chapter{セックス倫理を討議するための哲学的原則}

この最終章では、本書を通じてさまざまな議論の基礎とした一般的な哲学的原則を再検討する。
第1章で説明したように、これらの原則の多くは哲学史における重要人物たちの業績に基づいている。
現代の哲学者たちはセックス倫理学にこれらの原則を用いて、しばしば元の思想家とはずいぶん違った結論を下している(元の思想家がセックスの倫理を語っている場合の話だが)。

\section{自由と自律}

リベラルで民主的な社会の基本的な前提の一つは、人々が\ruby{自由}{リバティ}の権利をもつということだ。
言い換えれば、他の条件が同じであれば、すべての人々、あるいは少なくともすべての成人は、自分の望むことをおこなえるべきであり、他の人々や国家がそれに反することを命じるべきではない。
すべての人は自分自身の体を支配する権利をもち、また自分の人生の進路をコントロールする権利をもつべきだ。
これには、自分が望む時にセックスをする自由や、自分が望む時にセックスや親密な接触を拒否する自由も含まれる。
自由は道徳的・政治的な問題に対処する際の重要な原則として機能している。
多くの人は、ピアーズ・ベンが「許容の前提」と呼ぶもの、すなわち、彼の言葉を使うならば、
「私たちが関与する活動について、反対者がそれに対して反論する責任を負うべきだ」\citep[p.237]{benn99:_is_sex_moral_special}という前提からセックス倫理の問題に取り組もうとしている。

しかし、自由の権利は絶対的なものではなく、本書でもそのようには扱われていない。
むしろ、自由の権利はさまざまな制約を受けるものだ。
まず第一に、人々には他者に害を与える権利はない。
多くの場合、どのような行為が危害になるかは明確だ。
私たちは混雑した劇場で「火事だ!」と叫ぶことは許されていない。
また、他者の同意なしにセックスをすることも許されていない。

これらはほとんどすべての人が同意する明確な事例だ。
しかし、問題がより難しくなる状況も多く存在している。
たとえば、一部の人々には不快に思われる行為がある。
それら害となるのだろうか? 研究者間や一般市民の間でも、誰かを不快にさせることが本当の危害、つまり市民の自由に対する制限を正当化するに足る害であるかどうかについての合意はない。
たとえば、公共の場で女性が授乳をすることに不快感を抱き、それを禁止すべきだと考える人々もいる。
一方で、これは自然な行為でありと考える人々も多く、アメリカ人の大多数は公共の場で授乳を禁止することに反対している\citep{cdc08:_public_opinion_breas}。
だが、マスターベーションもまた自然な行為である。それにもかかわらず、私たちの多くは公共の場での自慰行為を禁止することを支持するだろう。
私たちは矛盾しているのだろうか? 2013年、スウェーデンの裁判所は、公共のビーチでマスターベーションした男を、彼の行為は特定の人物に向けられたものではないという理由で無罪にした\citep{ederyd13:_you_cant_just_walk_aroun}。
しかし、スウェーデンは他のほとんどの地域と同様に、人々が不快に感じるという理由で、公共の場でのマスターベーションのほとんどを禁止し続けている。

私たちはまた、自由と公共の効用との間の潜在的なトレードオフも考慮しなければならない。
効用の原則については以下で簡単に説明する。
しかし、その核心にあるのは、他の人々の利益を考慮すべきだという考えだ。
自由の権利は、他者に配慮し、周囲の人々のために犠牲を払う義務から私たちを解放するものではない{\DDASH}そのようなトレードオフがどのように最も適切に達成されるかはしばしば議論の対象となるものの、だ。

政府はしばしば、人々の自由をさらに別の二つの理由で制約してきた。
それは\ruby{道徳主義的}{モラリスティック}な理由と\ruby{保護主義的}{パターナリスティック}な理由の二つだ。
これらはしばしば混同されるが、実際には異なるものだ。
リーガルモラリズムによれば、市民の行動が他者に直接的な害を与えない場合でも、その行為が単に不快であるだけでなく不道徳だと社会が考えるならば、その自由を制限することが許容されると主張される。
しかし、どんな行動が不道徳であるとされるのか、またそれが何によって決定されるべきかについての合意は存在しない。
実際、複数の価値観が共存する社会では、私たちは自分の道徳的観点を他者に押し付けることはできないと多くの人々が考えており、過去数十年にわたってリーガルモラリズムは自由の制限の正当化としては徐々に放棄されてきた(本書5.1節参照)。
それでもなお、必ずしも公然と認められているわけではないが、\ruby{道徳主義}{モラリズム}はセックス倫理に関する問題に人々がアプローチする際にある程度の役割を果たし続けている。

\ruby{保護主義的}{パターナリスティック}な法律は、人々の自由を(当人の利益になると考えられる)理由から制限する。
シートベルトの着用義務や自転車のヘルメット着用義務は、よく知られた例だ。
私たちはこの種の法律には慣れており、それが私たちの自由に与える侵害は最小限に見えるが、保護主義的な制限がどこまで認められるべきかについての合意はない。
たとえば、ある人々は、セックスワーカーをその仕事に伴うリスクから保護するために、セックスワークを違法にすべきだと考えている。
たとえセックスワーカー自身がそれを選んでおこなっているとしてもだ。
別の人々は、これはセックスワーカーの自由に対する不当な制限だと考えている(本書5.3節参照)。

私たちは、\ruby{保護主義的}{パターナリスティック}な法律が私たちの「自由」を侵害するとしても、「自律」については侵害しない場合、あるいは実際には自律を支援するような場合には、その法律に共感的になる傾向がある。
自由と自律は密接に関連しており、実際、この二つの言葉はしばしば同義で使われる。
しかし、哲学者たちは時として「自律」という言葉を、より根本的で、より意義のある形の自由を示すために、専門的な意味で使用することがある。
このより堅い意味での自律の概念とその含意について考えることは有益だ。

この意味での「自律」は、単に自分の望むことをするという自由以上のものを要求する。
それは、自分の長期的な目標に沿った選択を行い、自分が望む生き方を追求する能力だ。
アルコール依存症に苦しんでいる人は、望むことをする自由を持っているかもしれない{\DDASH}誰もそれを妨げていないならば{\DDASH}が、しかし依存症というものはしばしば、自分が望む生き方を支えてくれるような選択をおこなうことを妨げてしまう。
こうした人を完全に自律的だとは言えない。

しばしば、自律的であるために自由であることよりも多くが必要だ。
私たちが自由であるのは、社会、周囲の人々、そして政府が私たちを放っておいてくれるときだ。
しかし、完全に自律的であるためには、単なる不干渉を超える特定の種類の支援が必要かもしれない。
それは時には一定の\ruby{保護主義的}{パターナリスティック}な介入を必要とすることもある。
友人や家族が依存症患者に治療を受けさせようとするのは、彼らが自分の人生に対する自律を高める手助けをしようと考えているからだ。
この友人たち(あるいは家族)は依存症患者にとって何が最善かを知っていると思っており、これはたしかに\ruby{保護主義的}{パターナリスティック}だ。
しかし、この考えが正しい場合もある。
政府も時にこのように行動し、長期的に私たちの自律を促進するために保護主義的政策を導入している。
たとえば、政府は私たちの給与の一部を差し引いて退職金を積み立てさせることで、私たちが退職後の生活に備えるように強制することがある。
すでに見てきたように、セックス産業に従事する女性たちの自律を守るためにセックスワークを違法にすべきだと考える人々がいる{\DDASH}もっとも、一方ではそれがまったく逆効果だと考える人々もいる。

自律はまた、私たちの社会環境に対する広範な変化を必要とすることがある。
たとえば、深く\ruby{同性愛嫌悪}{ホモフォビア}的な社会では、ゲイやレズビアンの人々が完全に自律的であることは難しい。
彼らは、たとえ法律がそれを禁止していなくても、自分のアイデンティティを公然と表現したり、同性のパートナーを求めたりすることに対する圧力を感じるだろう。
もしその人が自律的な生活を送る能力を支援したいのであれば、彼らの人生の選択肢を実質的に制約する偏見に立ち向かう必要がある。

\section{平等}

自由と平等は、自由な社会の双璧の価値としてしばしば一緒に言及される。
そして、自由と平等の重視は、しばしば同じ結論に至る。
たとえば、人々は同性の\ruby{親密関係}{リレーションシップ}を追求する自由を持つべきだ。
それは、そのような関係が保護さるべき個人的な自由の範囲に含まれるからであり、また、法律が特定の人々(この場合、ゲイやレズビアン)に特別な扱いをおこない、他の人々(異性愛者)には干渉しないことが不当だからだ。
米国最高裁が\emph{Lawrence v. Texas}においてテキサス州の反ソドミー法を覆したとき、それは個人の自由の原則に基づいておこなわれた。
しかし、サンドラ・デイ・オコナー判事は別の理由からの同意意見を述べている。
彼女は、反ソドミー法はゲイやレズビアンの自由を侵害しているだけでなく、平等な扱いを受ける権利をも侵害していると指摘している。
このように同じ結論に至ることがしばしばだとしても、自由と平等はけっして同じものではない。
比較の上では、自由の原則の方が、私たちの道徳的および政治的決定がなされる文脈をもっと広範に考慮することを要求するものだ。

人々はしばしば、平等の重要性については同意するが、それが何を意味するのかについては意見が一致しないことがある。
アリストテレスは、正義は平等を要求すると述べているが、しかし彼は続けて「正義は何の平等を要求するのだろうか?」と問いかけている。
特定のグループがより厳しい制裁や制限を受けたり、法的結婚に伴う社会的便益へのアクセスを拒否されたりすることは不当に思われる。
しかし、そのような不平等な法律を廃止することだけで平等な社会を実現できるのだろうか? たしかにそう考える人もいる。
しかし、それだけでなく、人々の社会的地位に内在している不平等も考慮すべきだと考えている。
私たちは、教育や医療の資源などにすべての人々が平等にアクセスできるかどうか、人々が自分の選んだキャリアを追求し成功する能力を同じようにもっているかどうか、人々が政治システムにおいてしっかり代表されているかどうかをなど評価しなければならない。
また一部の人は、私たちは「実質的な平等」を追求するべきだと考えている。
つまり、社会のすべての人々がほぼ同じような経済的状況に到達するように努めるべきだということだ。

本書を通じて、私はフェミニズムを重要な参照点としてきた。
フェミニスト哲学は平等の原則を重視している。
フェミニストたちは、家父長制的社会においては、たとえ法律が女性たちに一定の形式的な平等を認めている場合でも、現実には女性たちが体系的に不利益を被っているそのあり方を強調している。
フェミニストたちは、不平等を複雑で体制浸透的なものとして理解し、社会構造はもちろん、自分たち自身の行為や態度が平等という大義に反しているかもしれないということを考えてみるよう呼びかけている。
私は、リベラルフェミニズム、社会主義フェミニズム、ラディカルフェミニズムを含むさまざまなフェミニズムの学派を概説した。
これらの学派を分けている問題の一つは、平等の理解に関するものだ。
リベラルフェミニストが考えるように、女性の成功を妨げる障壁を取り除き、法律によって不平等に扱われているあり方に焦点を当てるべきだろうか? それとも、社会主義者やラディカルフェミニストが信じるように、女性のために実質的な平等を保証するために社会をより抜本的に変革すべきだろうか?

社会は、さまざまな要因に基づいて人々を体系的に不利に扱っている。
性別はその一つであるが、その他にも人種、性的指向、性自認、障害、年齢などがある。
これらはそれぞれ別の課題を提示している。
インターセクショナリティ分析(交差性分析、本書1.5節)は、不平等の全体的な複雑さを理解しようとするものだ。

\section{効用}

哲学者が使う「効用」という言葉は、より大きな善を指す。
基本的な原則はシンプルだが、その実施にはずっと大きな複雑さが伴う。
たとえば、「効用」がどのように測定されるべきかについては意見が分かれる。
ベンサムが信じたように、人々が経験する快楽と苦痛の全体量で測るべきだろうか?それとも、経済学者のリチャード・ポズナーや他の人々が信じるように、「公共の善」を経済的繁栄と私たちが選好を充足する能力によって測るべきだろうか? このような複雑さがあるにもかかわらず、効用の原理は、さまざまな道徳的および政治的選択を評価する際の有益な指針として残る。
なぜなら、多くの状況において、私たちはそこでの「効用」が何を意味するかをある程度は理解することができるからだ。
たとえば、腐敗した政治家が公共の善ではなく自分の利益を追求していると言うときには、「公共の善」が何によって構成されているのかについて私たちの理解が必ずしも正確に一致していなくても、そのような政治家の行為が道徳的に不正である理由を理解できるものだ。

効用の要求は、自由の原則と衝突してしまうことがある。
効用と自由の衝突が常に起こるわけでないのは確かだし、むしろほとんどの場合は衝突は起こらない。
人々は一般的に自分自身にとって何が最善であるかを知っており、特に親密な生活に関しては、人々の効用を最も促進するような政策は個人の自由を保護する政策であることが多い。
しかし、いくつかの重要な例外が存在する。
新型コロナウィルスの世界的パンデミックは、人々の自由への制約がどのようにして公共の善を促進するかを劇的に示す例となった。
似たような理由で、たとえば、ポルノ消費が引き起こす害、たとえばポルノが助長するかもしれない\ruby{女性嫌悪}{ミソジニー}的態度から社会を守るために、人々のポルノへのアクセスを制限すべきだと主張する人々もいる。

その名が示す通り、功利主義者たちは、道徳や公共政策や法律において、効用(功利)が私たちの指針となるべきだと考えている。
彼らは、効用は理想的には複数の行動の選択肢を評価するための客観的な基準を提供するものであり、どの選択が最善であるかを決定する助けになると考えている。
上述のように、公共の福祉をどのように測定するかについての意見の不一致があることに加え、功利主義者たちは、効用を最大化するようなルール群を作っておくべきか、それともケースバイケースで最善の利益を追求するべきかについても意見が分かれている。
しかし、功利主義者は全体として、長期的に最大多数にとって最大の\ruby{幸福}{グッド}をもたらすべきだという信念において一致しており、たとえそれが時には個人の自由の一定の程度を犠牲にすることになるとしても、この点(最大多数の最大幸福)最優先であるべきだと考えている。

\section{美徳}

哲学者が使う「美徳」の概念において、美徳を獲得するということは、他の道徳理論の多くが注目するような「個別のケースで正しいことをおこなう」ということ以上のものを要求する。
有徳であるためには、私たちは良い人間でなければならない{\DDASH}私たちは自分の性格を発展させ、それによって充実した生活、そして他の人々が賞賛するような生活を送っている必要がある。
賞賛に値する充実した生活を送る時に達成されている状態を示す古代ギリシャの言葉は「エウダイモニア」だ。
この語はしばしば「\ruby{幸福}{ハピネス}」と訳されるが、それは快楽や満足感よりも持続的で深いものを示している。
本書の中で何度も、私は、「私たちは他者に対してどう行動すべきか」という問いだけでなく、私たちはどのような生活を送りたいのかという問いも投げかけてきた。
徳倫理学という哲学の学派がある。
この学派は倫理について考える際には、人の\ruby{性格}{キャラクター}や「よい生活」の本質に関する考慮が優先されるべきだと主張する。
もっとも、この特定の伝統外で活動する哲学者たちも、(同じ重要性を与えているとはいえないにしても)美徳に関する考慮を排除しているわけではない。

美徳について現代の多くの考え方の源泉となっているアリストテレスは、エウダイモニアを獲得するために私たちが育むべき美徳の確定版リストのようなものは示していない。
彼は、広く賞賛される生活を送っている人々を模倣すればよいだろうと考えているようだ。
現代の哲学者たちは、概して、アリストテレスにならって、単一の美徳のカタログを提出することを避けているが、一般的に、寛大さや公正の感覚といった性格特性は育むべきものであることは明らかだということに合意している。
しかし、現代の徳倫理学における前提は、アリストテレスと同様に、倫理学をおこなう重要な部分は何が賞賛に値し、何がそうでないかについて議論することにあるということであり、この問いに対する答えを事前に想定してしまうことはできない。

本書のさまざまな箇所で、私は人の性格に関する考慮に注意を払おうとした。
そしてたとえばモノガミーに関する章のように、私たちの私的な選択がよい人生を送るのにどれほど有益であるかを問うようにしてきた。
しかし、私は、よい生活についての考慮が常に決定的であるべきだと示唆するつもりはない。
哲学者たちは、美徳に関する考慮が私たちの倫理的思考において最も重要であるべきかどうかという点では意見は一致していない。
哲学者の多くは、そう考えることは私たちの自律への制約を正当化しすぎる可能性があると考えている。
なぜなら、こうした発想は、私たちが価値のある人生を送っているかどうかを他の人間が判断してしまうとを許してしまうからだ。
また、哲学者の多くは、実際には私たちが美徳の本質を決定することは不可能だと考えている。
たとえば、軍事的な栄光を求める欲望は、ギリシャ人やローマ人が信じたように並外れた高貴さを示すのか、それともキリスト教神学者たちが主張するように堕落した血への欲望を露呈しているだけなのかについて、私たちの意見が一致することはけっしてないだろう。
美徳に基づく倫理学に対する批判者たちは、このような不一致に直面したときには、自由や平等や効用など、普遍的な指針を提供できる原則に焦点を当てるべきだと主張している。

\vspace{1\zw}

哲学者の中には、特定の原則を強く信奉して譲歩せず、他の原則が犠牲になってもやむをないと考える人々がいる。
たとえば、一部の\ruby{自由優先主義者}{リバタリアン}は、効用の要求はほとんどの場合私たちの個人の自由よりも上位には置かれるべきではないと主張する。
また一部の功利主義者は、個人の権利についての議論を軽蔑する。
また一部の徳倫理学者は一般的な道徳原則などは私たちに有益な指針を与えることはけっしてないとまで主張する。
しかし、このような純粋主義者は現実にはごく稀だ。
私たちの多くは、これらの原則のすべてをある程度評価しており、最もよくとられる方策は、二つ以上の原則が衝突した場合には、それらの競合する要求をバランスよく調整し、両者の間で何らかの実行可能な妥協点を見出そうとすることだ。
しかし、このような妥協点を見つけることは容易ではなく、場合によっては不可能の場合もある。
さらに悪いことに、一つの原則が私たちを別々の方向に導いてしまうこともしばしばだ。
それでも、ここで示された一般的な原則にはすばらしい価値がある。
こうした原則は、潜在的には、議論のための共通の枠組みを提供することで、最も論争を呼んでいる問題についてでさえ、生産的な対話をおこなう助けになってくれるのだ。

\vspace{1\zw}

シャロン・オルズは彼女の詩「冬のメイクラブのあとに」(``After Making Love in Winter'') の中で、恋人に対して「私たちは問いの終わりに辿りついた」と言う。
しかし、私はこんなことは主張できない。
本書で議論されなかった多くの問題がある。
私は、研究者、メディア、そして私の学生たちが特に強い関心を寄せているセックス倫理学の問題をカバーしようと最善を尽くした。
しかし、他にも多くの問題について議論することができただろうし、私が本書で考察しなかった新たな問題が現れてくるもの確実だ。
そしてそれまでにも、この本で取り上げた議論はさらに進展していくだろう。
本書はこうした刺激的でダイナミックな分野の研究を集積しておくよう試みた。
この種の本は、その宿命としてすぐに時代遅れになってしまうものだ。
せめて、何人かの読者が本書で扱われた議論を新しい方向に進めてくれること、そして新たな問題を考えるきっかけとなることを私は期待している。

\if0
\chapter*{訳者おぼえ書き}
\phantomsection
\addcontentsline{toc}{chapter}{訳者おぼえ書き}
\begin{itemize}
\item ニール・マッカーサーはカナダのマニトバ大学の哲学教授で、職業倫理・応用倫理学研究センター長を努めている。
単著として、\emph{David Hume's Political Theory: Law, Civilization and the Constitution of Goverment} (University of Tronto Press, 2007)、共編者として\emph{Robot Sex: Social and Ethical Implication}, MIT Press, 2017)、\emph{Fragile Freedomes: The Global Struggle for Human Rights} (Oxford University Press, 2017)などがある。

\item この本を訳すことになったのは〜〜という事情で〜〜〜。

\item ハルワニ訳したときは〜のつもりだったけど〜が足りなかったのでもう一冊良書を訳出したかった、このマッカサー本は〜で見つけた。
\item 本書の美点は〜〜〜
\item 原書は参照している文献を後注で示しているが、本文中にAuthor-Year形式で記載し、文献表をつけた。
\item アウグスティヌスやトマス・アクィナス、カント、J.S.ミルなどの有名哲学者著作は複数あるし面倒なので邦訳を載せてない。
\item 判例等も後注で示されているが、判例一覧を作り本文中ではイタリック体で略記し本文に取り込んだ。
どれも有名な判決なので、ネットで検索すれば見つけることができる。
\item 本書の特徴の一つに、ネット上のさまざまな論説を哲学的検討の材料にしていることがある。
当該記事のURL等は2025年2月から3月にかけてチェックし、アクセス不能の場合は文献表に情報を加えている。
よく参照されているカナダの\emph{VICE}社は2023年に経営破綻して2024年には配信を停止しているらしく、過去の記事が今後も読めるのかどうか不安があるとのことだ。

\item 文献表はクリック可能なリンクのある状態にして出版社の本書のウェブページおよび訳者のホームページからアクセスできるようにする予定なので、「ニール・マッカーサー セックスの倫理学 文献表」などで検索してほしい。

\item 翻訳のある書籍は参考にさせてもらい文献表に載せたが、訳文は英語原文から適当に訳した。
(めんどくさいから該当箇所は指示しない……)

\item 全体に明快だし、哲学的に難しい議論をしているわけでないので解説は必要ないと思うけど、4.2.2節の「差別」の定義のところちょっと問題あるので指摘しておく。
〜〜〜

\item 訳文はChatGPTに下訳を作らせ、江口が修正した。
おかしなところあるかもしれないけどChatGPTじゃなくて江口が悪いと思う。
でもAIに責任あるならAIも悪いかもしれない。
早くAIに責任を押しつけられる世界になってほしい。
\item 訳者の江口は誤字が多くてどうしようもないので、SNSで協力者をつのって誤字脱字等の校正および読みにくい箇所を指摘していただいた。
指摘箇所は〜百箇所に及んだ。
こうした協力がなければ草稿は永遠に完成しなかっただろう。
〜〜〜〜〜〜〜各氏には心より感謝する。
無能に腹が立っても許して。
\item 出してくれる出版社と編集者にものすごく感謝する。
\end{itemize}
\fi

{\footnotesize
\phantomsection
\addcontentsline{toc}{chapter}{参照文献}
\bibliographystyle{jecon}
  \bibliography{mcarthur}
}

\phantomsection{}
\subsection*{参照判例略記}
\addcontentsline{toc}{chapter}{参照判例一覧}

\begin{mylist}

\item \emph{American Booksellers Association v. Hudnut} 771 F.2d 329 (7th Cir. 1985)
\item \emph{Ashcroft v. Free Speech Coalition} (00-795) 535 U.S. 234 (2002)
\item \emph{Baker v. Nelson}, 291 Minn. 310, 191 N.W.2d 185 (1971)
\item \emph{Barnes v. Glen Theatre, Inc.}, 90--26, 501 U.S. 560, 1991
\item \emph{Bedford}/(\emph{Canada (AG) v. Bedford}, 2013 SCC 72, [2013] 3 SCR 1101, para 64)
\item \emph{Bowers v. Hardwick}, 478 U.S. 186 (1986)
\item \emph{Campbell v. Sundquist}, 926 S.W.2d 250 (Tenn. Ct. App. 1996)
\item \emph{Chaplinsky v. New Hampshire}, 315 US 568 (1942)
\item \emph{Commonwealth v. Appleby}, 402 N.E.2d 1051, 1054 (Mass. 1980)
\item \emph{Eldridge v. British Columbia} (Attorney General), [1997] 3 S.C.R. 624
\item \emph{FTC v. Colgate-Palmolive Co.}, 380 U.S. 374 (1965)
\item \emph{Ginzburg v. United States}, 383 U.S. 463, 479 (1966)
\item \emph{Harris v. State}, 457 P.2d 638, 1969
\item \emph{Hess v. Indiana}, 414 U.S. 105, 108 (1973)
\item \emph{Jacobellis v. Ohio}, 378 U.S. 184 (1964) 
\item \emph{Kerrigan v. Commissioner of Public Health}, 289 Conn. 135, 957 A.2d 407 (2008)
\item \emph{Lawrence et al. v. Texas}, 02--102, 539 U.S. 558 (2003)
\item \emph{Loving v. Virginia}, 388 U.S. 1, 12 (1967)
\item \emph{Marsh v. Alabama}, 326 U.S. 501 (1946),
\item \emph{McGee v. Attorney General}, [1973] IR 284
\item \emph{Miller v. California}, 413 U.S. 15, 34--35 (1973)
\item \emph{Mock v. University of Tennessee at Chattanooga}, No. 14-1687-II, Tenn. Ch. Ct. 10 August 2015
\item \emph{Obergefell v. Hodges}, 576 U.S. 644 (2015)
\item \emph{Paris Adult Theatre Inc. v. Slaton}, 413 U.S. 49, 63 (1973)
\item \emph{Pavesich v. New England Life Ins. Co.}, 122 Ga. 190, 194, 50 S.E. 68, (1905)
\item \emph{People v. Bink}, 84 A.D.2d 607 (N.Y. App. Div. 1981)
\item \emph{People v. Burnham}, 176 Cal. App. 3d 1134 (Ct. App. 1986)
\item \emph{People v. Evans}, 85 Misc. 2d 1088, 379 N.Y.S. 2d 912, Sup. Ct. (1975)
\item \emph{People v. Jovanovic},  263 A.D.2d 182, 700 N.Y.S.2d 156 (N.Y. App. Div. 1st Dep't 1999)
\item \emph{People v. Mayberry}, 15 Cal.3d 143  (1975)
\item \emph{Planned Parenthood of Southeastern Pa. v. Casey} (91-744), 505 U.S. 833 (1992)
\item \emph{R v. Bree} [2007] EWCA Crim 804
\item \emph{R v. Brown} [1993] 2 All ER 75
\item \emph{R v. JA} 2011 SCC 28
\item \emph{Reliable Consultants, Inc. v. Earle}, 517 F.3d 738 (5th Cir. 2008)
\item \emph{Roe v. Wade}, 410 U.S. 113 (1973)
\item \emph{Roth v. United States}, 354 U.S. 476, 484 (1957)
\item \emph{Schenck v. United States}, 249 U.S. 47249 U.S. 47 (1919)
\item \emph{State v. Collier}, 372 N.W.2d 303 (Iowa Ct. App. 1985)
\item \emph{State v. Guinn}, No 23886-1-II, 2001 Wash. App. (Wash. Ct. App., March 30, 2001)
\item \emph{Williams v. King}, 28 U.S.C. 636(c)(1) (2017)
\item \emph{Xiong Edmondson v. Xiong}, 648 N.W.2d 900 (Wis. Ct. App. 2002)
  \end{mylist}

% \nocite{ミル20:自由論,ミル12:自由論,ミル11自由論}
% \nocite{アリストテレス02:ニコマコス:京大}  
% \nocite{アドシェイド15}
% \nocite{カント24:人倫徳論}
% \nocite{カント24:基礎づけ}
  
\phantomsection\addcontentsline{toc}{chapter}{索引}  
{\footnotesize
\printindex
}

\end{document}


