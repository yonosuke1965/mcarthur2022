\documentclass[]{jlreq} \usepackage{mystyle}
\title{}
\author{江口聡}
\begin{document}
\maketitle




% ――とりわけセックス、同意、ポルノグラフィ、セックスワーク、自己決定、表現の自由、そして社会的規範と道徳的批判の関係――を、抽象論ではなく「現場の論点」として丁寧に扱う点に特徴がありま

% McArthurの議論は、単なる保守/リベラルの対立図式に回収されるものではなく、個人の自由と他者への害(harm)をめぐる境界をどこに引くべきか、そして倫理的判断が公共空間でどのように語られ、制度や文化に影響するのかを、理詰めで追い詰めていくタイプです。挑発的なテーマでも感情的な断罪や迎合に逃げず、理論と事例を往復しながら論点を切り分けていく姿勢は、応用倫理学の実践として非常に教育的でもあります。

セクシュアリティやセックスといったセンシティブな問題をメディアや学者が扱うことの難しさを感じさせられる。


私の人生と知的生活の半分は、ネットがなければ出会うことのなかったであろう実名・匿名の方々との議論や協力的な交流、あるいは通りすがりの軽い会話によって支えられてきた。本書は、そうした人々の生活と思索にわずかなりとも寄与し、建設的な議論の一助となることを願って、翻訳刊行するものである。

私の人生と知的生活の半分は、ネットがなければ知りあうことのなかったであろう実名・匿名の方々との議論や協力的な交流、あるいは通りすがりの軽い会話によって形づくられてきた。本書は、そうした人々の生活と思索に少しでも資することを願い、また建設的な議論の助けになりたいという思いから翻訳刊行したものである。

本書は、Neil McArthur, \emph{The Ethics of Sex}, Routledge, 2023の全訳である。ニール・マッカーサーは、カナダのマニトバ大学の哲学教授であり、専門職倫理・応用倫理センターのディレクターを務めている。彼はヒュームの道徳哲学研究から出発し、現在は現代社会が直面する争点、とりわけセックス倫理学(セックステクノロジー、性的同意、ポルノグラフィ、セックスワーク、表現の自由、社会的規範に対する道徳的批判など)の分野で業績を挙げている。本書のほか、単著として David Hume's Political Theory: Law, Commerce, and the Constitution of Government, University of Toronto Press、共編著として John Danaher and Neil McArthur (eds.), \emph{Robot Sex: Social and Ethical Issues}, MIT Press, 2017などがある。

本書は各テーマについて非常に平明な説明と議論を行っており、ここで内容を詳述する必要はないだろう。セックスの哲学および倫理学は近年の英語圏の応用哲学・応用倫理学においてホットなテーマであり、本書はとりわけ倫理学的論争の中心となっている多数の問題群──カジュアルセックス、オンラインデート、BDSM、性的同意、一夫一婦制度(モノガミー)、セックスワーク、ポルノグラフィ、さらには各種の新たなセックステクノロジーなど──について、詳細な検討を行っている。学術的価値に加え、社会的にも大きな価値をもつことは疑いない。

名古屋大学出版会からは、すでに2024年にラジャ・ハルワニの『愛・セックス・結婚の哲学』(江口聡・岡本慎平監訳)を翻訳刊行させていただいた。訳者としては、恋愛とセックス、そして結婚制度に関する哲学的議論の必要性と重要性について、研究者のみならず一般読者層からも一定の理解を得られたものと考えている。『愛・セックス・結婚の哲学』も多くの哲学的・倫理学的問題を扱っているが、セックス倫理学において中心的な位置を占める「性的同意」の問題については、紹介と掘り下げが十分でない面があったのは否めない(ハルワニは他にも多くの論点を扱っているため、紙幅の制約上やむをえない事情もある)。その点で本書は、性的同意をめぐる近年の議論を理解するための格好の導入であり、また他のテーマについても重要な検討を含んでいる。本書の翻訳出版が屋上屋を重ねるように見えるとしても、国内読者に紹介したいと考え、名古屋大学出版会に重ねて刊行をお願いすることになった。

本書とハルワニの著作とで重なるテーマも少なくない。しかし、哲学的色彩の強いハルワニの『愛・セックス・結婚の哲学』が、恋愛や結婚制度を含む幅広い論点を扱うのに対し、本書はより社会的に論争となっている具体的問題を射程に収め、規範倫理学の複数の立場から検討を加える。また同一の主題に対して異なる視点を提示するという点でも、翻訳刊行する意義は大きいと考えられる。

本書の美点の一つとして、大量の新聞記事やオンラインメディア記事を参照・言及している点を挙げたい。BDSM、ポルノグラフィ、セックスワーク、あるいはデジセクシュアリティ、そしてマッカーサーがVCE(バーチャル児童搾取)と呼ぶ各種のテクノロジーなどは、偏見にさらされやすい性的実践であり、研究者が自分たちの限られた経験だけをもとに議論できる問題群ではない。そもそも人々の実践と経験は、アカデミックな形で記録・表現されないまま埋もれていることも多い。そのため、公刊された学術論文だけで十分な事実的知見が得られるとは限らない。セックスに関する倫理的問題の多くは、現在では学術論文よりもオンラインメディア上で議論される傾向があり、そうしたメディア上での、人々の生に近い声を聞き取ることが必要だとマッカーサーは考えている。こうした態度は、これからの応用倫理学研究に必要なものだろう。

ちなみに、訳者は本書(〜頁)で言及されている日本のドールメーカーであるTROTTLA社に、記事内容の確認のためメールで連絡を取らせていただいた。同社によれば、海外メディアの取材内容には納得のいかない点が多く、記事にも偏見や誤解が含まれていたという。また当該記事をもとにした二次メディアの報道にも困惑し、その後は海外からの取材を受けていないとのことであった。セクシュアリティに関わる話題や問題を、メディアおよび研究者が扱う難しさを意識させられた一件だった。

名古屋大学出版会、とりわけ編集者の橘宗吾さん、堤亮介さんには、ハルワニ書に引き続き翻訳企画を許していただいたうえ、今回も精密なご指導をいただき、感謝にたえない。本書の翻訳にあたっては、訳者の知人・友人に加え、オンライン(特にX/Twitter)で淡い交流をもたせていただいている方々の協力を受けている。訳者の能力不足のために直すことのできない誤訳、誤字脱字、文章のヨレなどについて、オンライン上で厖大な量の指摘をいただいた。そうした協力がなければ、比較的短時間で翻訳を完成させることはできなかっただろう。お名前の掲載を許可いただいた山口流聖、柏木昭彦、上川智之、宮島柚果、椎名真大、大谷卓史、架神恭介の各氏(順不同)に加え、匿名のままでありつづけることを選択した他の多くの方々(訳者もどのようなアイデンティティの方であるか把握できていない人々も多い)にも心から感謝する。

私の人生と知的生活の半分は、ネットがなければ知りあうことのなかったであろう実名・匿名の方々との議論や協力的な交流、あるいは通りすがりの軽い会話によって形づくられてきた。本書は、そうした人々の生活と思索に少しでも資することを願い、また建設的な議論の一助となることを期して、翻訳刊行したものである。



%\renewcommand{\refname}{出典}
% \addcontentsline{toc}{section}{\refname}
\bibliographystyle{jecon}
\bibliography{BIB}
\end{document} %----------------------------------------------------------------


%%% Local Variables:
%%% mode: japanese-latex
%%% TeX-master: t
%%% coding: utf-8
%%% End:
